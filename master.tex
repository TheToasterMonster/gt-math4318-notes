\documentclass[12pt, letterpaper, oneside]{book}
\usepackage[margin={0.6in, 0.75in}]{geometry}
\usepackage{microtype}
% \usepackage{kpfonts}
\usepackage{amsmath, amssymb, amsthm}
\usepackage{hyperref}
\usepackage{parskip}
\usepackage[many]{tcolorbox}
\usepackage{footnote}
\usepackage{cancel}
\usepackage{titlesec}
\usepackage{pgffor}
\usepackage[shortlabels]{enumitem}

\renewcommand{\chaptername}{Lecture}
\newtheorem{axiom}{Axiom}[chapter]
\newtheorem{theorem}{Theorem}[chapter]
\newtheorem{prop}{Proposition}[chapter]
\newtheorem{corollary}{Corollary}[theorem]
\newtheorem{lemma}{Lemma}[chapter]
\theoremstyle{definition}
\newtheorem{definition}{Definition}[chapter]
\newtheorem{exercise}{Exercise}[chapter]
\newtheorem{example}{Example}[definition]
\newtheorem*{remark}{Remark}

\tcbset{sharp corners, breakable, enhanced, parbox=false}
\newtcolorbox{mybox}[3][]
{
  colframe = #2!150,
  colback  = #2!5,
  coltitle = #2!0!white,  
  title    = {#3},
  #1,
}

\titleformat{\chapter}[display]
    {\normalfont\huge\bfseries}{\chaptertitlename\ \thechapter}{20pt}{\Huge}
\titlespacing*{\chapter}{0pt}{0pt}{40pt}

\newcommand{\R}{\mathbb{R}}
\newcommand{\N}{\mathbb{N}}
\newcommand{\Z}{\mathbb{Z}}
\newcommand{\C}{\mathbb{C}}
\newcommand{\Q}{\mathbb{Q}}
\newcommand{\F}{\mathbb{F}}

\title{MATH 4318: Analysis II}
\author{Frank Qiang\\Instructor: Zhiwu Lin}
\date{Georgia Institute of Technology\\Spring 2024}

\begin{document}
  \maketitle

  \begingroup
  \let\cleardoublepage\clearpage
  \tableofcontents
  \endgroup

  % \foreach \i in {00, 01, 02, 03, 04, ..., 50} {%
  %   \edef\FileName{lectures/lecture\i.tex}%     The % here are necessary to eliminate any
  %   \IfFileExists{\FileName}{%  spurious spaces that may get inserted
  %      \input{\FileName}%       at these points
  %   }
  % }
  \chapter{Jan.~9 --- The Derivative}

\section{Defining the Derivative}
\begin{definition}
  Let $f$ be a real-valued function on an open interval
  $U \subseteq \R$. Let $x_0 \in U$, we say $f$ is
  \emph{differentiable} at $x_0$ if
  \[
    \lim_{x \to x_0} \frac{f(x) - f(x_0)}{x - x_0}
    = \lim_{h \to 0} \frac{f(x_0 + h) - f(x_0)}{h}
  \]
  exists. If it does, then this limit, denoted by
  $f'(x_0)$, is called the \emph{derivative} of $f$ at
  $x_0$.
\end{definition}

\begin{remark}
By definition, for any $\epsilon > 0$, there exists
$\delta > 0$ such that
\[
  \left|\frac{f(x) - f(x_0)}{x - x_0} - f'(x_0)\right| \le \epsilon
\]
if $|x - x_0| < \delta$ and $x \in U$. Multiplying
both sides by $|x - x_0|$ yields
\[
  |f(x) - f(x_0) - f'(x_0)(x - x_0)| \le \epsilon |x - x_0|.
\]
In other words,
\[|f(x) - \varphi(x)| \le \epsilon |x - x_0|\]
where $\varphi(x) = f(x_0) + f'(x_0)(x - x_0)$.
In other words, $\varphi(x)$ is a first-order
approximation of $f(x)$ near $x_0$.
Geometrically, this is approximating the graph of
$y = f(x)$ by the tangent line $y = \varphi(x)$.
\end{remark}

\section{Immediate Properties}
\begin{prop}
  Let $U \subseteq \R$ be an open set and $ f : U \to \R$.
  If $f$ is differentiable at $x_0 \in U$, then
  $f$ is continuous at $x_0$.
\end{prop}

\begin{proof}
  Pick any $\epsilon_0 > 0$. Then there exists
  $\delta_0 > 0$ such that whenever
  $|x - x_0| < \delta_0$ and $x \in U$,
  \[|f(x) - f(x_0) - f'(x_0)(x - x_0)| \le \epsilon_0 |x - x_0|.\]
  By the triangle inequality,
  \[
    |f(x) - f(x_0)| \le \epsilon_0 |x - x_0| + |f'(x_0)| |x - x_0| = (\epsilon_0 + |f'(x_0)|) |x - x_0|.
  \]
  Now for any $\epsilon > 0$, choose
  $\delta = \min\{\delta_0, \epsilon / (\epsilon_0 + |f'(x_0)|)\}$.
  Then
  \[
    |f(x) - f(x_0)| \le (\epsilon_0 + |f'(x_0)|) |x - x_0| = (\epsilon_0 + |f'(x_0)|) \delta < \epsilon
  \]
  whenever $|x - x_0| < \delta$ and $x \in U$.
  Thus $f$ is continuous at $x_0$.
\end{proof}

\begin{example}
  Take the function
  \[
    f(x) = \begin{cases}
      x \sin(1/x) & \text{if } x \ne 0 \\
      0 & \text{if } x = 0.
    \end{cases}
  \]
  Note that $f$ is continuous on $\R$. For $x \ne 0$,
  continuity is clear since both $x$ and $\sin(1/x)$
  are continuous. At $x = 0$, we have
  \[
    \lim_{x \to 0} f(x) = \lim_{x \to 0} x \sin(1/x) = 0 = f(0)
  \]
  since $|x \sin(1/x)| \le |x|$ for all $x \in \R$,
  so $f$ is also continuous at $x = 0$. However,
  $f$ is not differentiable at $x = 0$. Consider the
  limit
  \[
    \lim_{x \to 0} \frac{f(x) - f(0)}{x - 0} = \lim_{x \to 0} \sin(1/x),
  \]
  which does not exist since $\sin(1/x)$ oscillates. So
  $f$ is not differentiable at $x = 0$.
\end{example}

\begin{example}
  Take the function $f(x) = |x|$, which is continuous
  everywhere on $\R$. However, $f$ is not differentiable at
  $x = 0$, since
  \[
    \lim_{x \to 0} \frac{f(x) - f(0)}{x - 0} = \lim_{x \to 0} \frac{|x|}{x}.
  \]
  Note that
  \[
    \frac{|x|}{x} = \begin{cases}
      1 & \text{if } x > 0 \\
      -1 & \text{if } x < 0,
    \end{cases}
  \]
  so the limit does not exist as $x \to 0$. Thus
  $f$ is not differentiable at $x = 0$.
\end{example}

\begin{remark}
  For the previous example, we can however define
  the \emph{left (right) derivative} by
  \[
    f_-'(x_0) = \lim_{x \to x_0^-} \frac{f(x) - f(x_0)}{x - x_0}
  \quad \text{and} \quad f_+'(x_0) = \lim_{x \to x_0^+} \frac{f(x) - f(x_0)}{x - x_0}.
  \]
  If $f$ is differentiable, then
  $f_-'(x_0) = f_+'(x_0)$. In the previous example,
  $f_-'(0) = -1$ and $f_+'(0) = 1$. For the first
  example however, even $f_{\pm}'(0)$ does not exist.
\end{remark}

\begin{remark}
  In one dimension, the existence of the derivative
  implies that the function is differentiable (the
  function is approximated by a linear function). However,
  in multiple dimensions, the existence of partial
  derivatives does not imply differentiability.
\end{remark}

\section{Rules for Differentiation}
\begin{prop}
  Let $U \subseteq \R$ be open and $f, g : U \to \R$
  be differentiable. Then
  \begin{enumerate}
    \item $(f + g)'(x_0) = f'(x_0) + g'(x_0)$
    \item $(fg)'(x_0) = f'(x_0)g(x_0) + f(x_0)g'(x_0)$
    \item $(f/g)'(x_0) = (f'(x_0)g(x_0) - f(x_0)g'(x_0)) / (g(x_0)^2)$.
  \end{enumerate}
\end{prop}

\begin{proof}
  Find in textbook (Rosenlicht).
\end{proof}

\begin{prop}
  We have $\frac{d}{dx}(c) = 0$, $\frac{d}{dx}(x) = 1$,
  and $\frac{d}{dx}(x^n) = nx^{n - 1}$ for all $n \in \N$.
\end{prop}

\begin{proof}
  We prove the last claim (the power rule) for
  $n \ge 1$ by induction. The base case
  $n = 1$ is the first claim which is true. Now suppose
  that the result holds for any $n \le k \in \N$, and
  we show that it remains true for $n = k + 1$. By
  the product rule, we have
  \[
    \frac{d}{dx} (x^{k + 1})
    = \frac{d}{dx} (x \cdot x^k)
    = \frac{d}{dx}(x) \cdot x^k + x \cdot \frac{d}{dx}(x^k) = x^k + xkx^{k - 1} = (k + 1)x^k.
  \]
  Thus by induction this holds for all $n \ge 1$.
  We can do negative integers by the quotient rule.
\end{proof}

\begin{remark}
  The power rule actually holds for any $n \in \R$.
\end{remark}

\begin{prop}[Chain rule]
  Let $U$ and $V$ be open sets of $\R$ and let
  $f : U \to V, g : V \to \R$ be differentiable.
  Let $x_0 \in U$ be such that $f'(x_0)$ and
  $g'(f(x_0))$ exist. Then $(g \circ f)'(x_0)$ exists
  and
  \[(g \circ f)'(x_0) = g'(f(x_0)) f'(x_0).\]
\end{prop}

\begin{proof}
  For any fixed $y_0$ for which $g'(y_0)$ exists,
  set
  \[
    A(y, y_0) =
    \begin{cases}
      (g(y) - g(y_0)) / (y - y_0) & \text{if } y \in V \text{ and } y \ne y_0 \\
      g'(y_0) & \text{if } y = y_0.
    \end{cases}
  \]
  Then $A$ is continuous at $y_0$. To find
  $(g \circ f)'(x_0)$, observe that
  \begin{align*}
    \lim_{x \to x_0} \frac{g(f(x)) - g(f(x_0))}{x - x_0}
    &= \lim_{x \to x_0} \frac{A(f(x), f(x_0))(f(x) - f(x_0))}{x - x_0} \\
    &= \lim_{x \to x_0} A(f(x), f(x_0)) \lim_{x \to x_0}\frac{f(x) - f(x_0)}{x - x_0}
    = g'(f(x_0)) f'(x_0),
  \end{align*}
  by the continuity of $A$ at $f(x_0)$ and the
  differentiability of $f$ at $x_0$.
\end{proof}

\begin{remark}
  The rough idea of what we did here is
  \begin{align*}
    \lim_{x \to x_0} \frac{g(f(x)) - g(f(x_0))}{x - x_0}
    &= \lim_{x \to x_0} \frac{g(f(x)) - g(f(x_0))}{f(x) - f(x_0)} \frac{f(x) - f(x_0)}{x - x_0} \\
    &= \lim_{x \to x_0} \frac{g(f(x)) - g(f(x_0))}{f(x) - f(x_0)} \lim_{x \to x_0} \frac{f(x) - f(x_0)}{x - x_0}
    = g'(f(x_0)) f'(x_0).
  \end{align*}
  But does not quite work as stated since
  it might be that $f(x) = f(x_0)$ even if $x \ne x_0$.
  We can fix this by introducing the function $A$
  as we did in the proof, though the overall idea is
  the same.
\end{remark}

\begin{remark}
  If $f$ is monotone near $x_0$, then we can define
  the \emph{inverse function} $f^{-1}$ so that
  $(f^{-1} \circ f)(x) = x$ near $x_0$. If
  $f'(x_0)$ exists, then by the chain rule
  applied to $x = (f^{-1} \circ f)(x)$ at $x = x_0$
  we have
  \[
    1 = \frac{d}{dx}(f^{-1} \circ f)(x_0)
    = \frac{d}{dx} f^{-1}(f(x_0)) \cdot f'(x_0)
    \implies \frac{d}{dx} f^{-1}(f(x_0)) = \frac{1}{f'(x_0)}.
  \]
\end{remark}

\begin{example}
  Let $f(x) = e^x$ with $f^{-1}(x) = \ln(x)$. Since
  $f'(x) = f(x) = e^x$, we have
  \[\frac{d}{dx} f^{-1}(f(x_0)) = \frac{1}{f'(x_0)} \implies
    \frac{d}{dx} \ln(e^{x_0}) = \frac{1}{e^{x_0}}.
  \]
  Letting $e^{x_0} = h$, we have
  $\left.\frac{d}{dx} \ln(x)\right|_{x = h} = 1 / h$,
  which recovers the familiar formula.
\end{example}

  \chapter{Jan.~11 --- The Mean Value Theorem}

\section{The Mean Value Theorem}
\begin{lemma}
  Let $I \subseteq \R$ be open, $f : I \to \R$
  is differentiable at $x_0 \in I$ and $f'(x_0) \ne 0$.
  Suppose $f'(x_0) > 0$, then there exists $\delta > 0$
  such that for any $x \in (x_0 - \delta, x_0 + \delta)$,
  \begin{enumerate}
    \item if $x > x_0$, then $f(x) > f(x_0)$,
    \item if $x < x_0$, then $f(x) < f(x_0)$.
  \end{enumerate}
\end{lemma}

\begin{proof}
  Take $\epsilon = f'(x_0) / 2$. By the
  definition of the derivative, there exists
  $\delta > 0$ such that for ay $|x - x_0| < \delta$,
  we have
  \[
    \left| \frac{f(x) - f(x_0)}{x - x_0} - f'(x_0) \right|
    < \epsilon = \frac{1}{2} f'(x_0).
  \]
  By the triangle inequality,
  \[
    \frac{f(x) - f(x_0)}{x - x_0}
    > \frac{1}{2} f'(x_0) > 0.
  \]
  This quotient being positive immediately implies
  the desired results.
\end{proof}

\begin{theorem}
  If $f(x)$ is differentiable in an open interval
  $I$ and $f$ obtains its local maximum (or minimum)
  at $x_0 \in I$, then $f'(x_0) = 0$.
\end{theorem}

\begin{proof}
  Suppose otherwise that $f'(x_0) \ne 0$. Assume without
  loss of generality that $f'(x_0) > 0$. Then by the
  previous lemma, there exists $\delta > 0$ such that
  for $x \in (x_0 - \delta, x_0 + \delta)$, if $x > x_0$
  then $f(x) > f(x_0)$ and if $x < x_0$ then
  $f(x) < f(x_0)$. So $x_0$ cannot be a local maximum
  or minimum, which is a contradiction.
\end{proof}

\begin{theorem}[Rolle's middle value theorem]
  Let $f(x)$ be continuous on $[a, b]$ and
  differentiable in $(a, b)$. Suppose $f(a) = f(b)$,
  then there exists $x_0 \in (a, b)$ such that
  $f'(x_0) = 0$.
\end{theorem}

\begin{proof}
  Since $f$ is continuous on a compact set, it obtains
  both a maximum and minimum on $[a, b]$. Let
  $M$ be the maximum and $m$ be the minimum. If
  $M = m$, then $f(x) \equiv M$ and $f'(x) = 0$
  everywhere. If $M > m$, then at least one of the maximum
  or minimum must be obtained at an interior point
  $x_0 \in (a, b)$ since $f(a) = f(b)$. By the previous
  theorem, $f'(x_0) = 0$ at this point and we are done.
\end{proof}

\begin{example}
  Show that the equation
  $4ax^3 + 3bx^2 + 2cx = a + b + c$ has at least one
  root in $(0, 1)$.
\end{example}

\begin{proof}
  Consider the equation
  \[
    4ax^3 + 3bx^2 + 2cx - (a + b + c) = 0.
  \]
  Notice that the left hand side is the derivative
  of the function
  \[f(x) = ax^4 + bx^3 + cx^2 - (a + b + c)x.\]
  So we just need to show that $f'(x) = 0$ for some
  $x$. For this, we can check that $f(0) = f(1) = 0$,
  and thus by Rolle's theorem there exists
  $x_0 \in (0, 1)$ such that $f'(x_0) = 0$. So
  $x_0$ is a root.
\end{proof}

\begin{theorem}[Lagrange's middle value theorem]
  Let $f_9x)$ be continuous on $[a, b]$ and differentiable
  in $(a, b)$. Then there exists $x_0 \in (a, b)$
  such that
  \[f'(x_0) = \frac{f(b) - f(a)}{b - a}.\]
\end{theorem}

\begin{proof}
  Subtract the secant line through $(a, f(a))$ and
  $(b, f(b))$ from $f(x)$ to get
  \[g(x) = f(x) - \frac{f(b) - f(a)}{b - a} (x - a).\]
  Note that $g(a) = g(b) = f(a)$. So by Rolle's theorem,
  there exists $x_0 \in (a, b)$ such that $g'(x_0) = 0$.
  But
  \[
    0 = g'(x_0) = f'(x_0) - \frac{f(b) - f(a)}{b - a},
  \]
  which is the desired result.
\end{proof}

\begin{corollary}
  Suppose $f \in C([a, b])$, i.e. $f$ is continuous
  on $[a, b]$, and that $f$ is differentiable in
  $(a, b)$. Then the following statements are equivalent:
  \begin{enumerate}
    \item $f'(x) \ge 0$ in $(a, b)$,
    \item $f(x)$ is increasing, i.e. if $x_1 > x_2$,
      then $f(x_1) \ge f(x_2)$.
  \end{enumerate}
  In particular, if $f'(x) > 0$ in $(a, b)$, then
  $f(x)$ is strictly increasing, i.e. if $x_1 > x_2$,
  then $f(x_1) > f(x_2)$.
\end{corollary}

\begin{proof}
  $(2 \Rightarrow 1)$ For any $x_0 \in (a, b)$,
  \[
    f'(x_0) = \lim_{h \to 0} \frac{f(x_0 + h) - f(x_0)}{h} \ge 0
  \]
  since $f(x_0 + h) - f(x_0) \ge 0$ for $h > 0$
  as $f$ is increasing.

  $(1 \Rightarrow 2)$ Take $x_1 > x_2$, then
  by Lagrange's theorem there exists $\xi \in (x_2, x_1)$
  such that
  \[f(x_2) - f(x_1) = f'(\xi)(x_2 - x_1) \ge 0.\]
  So $f(x_1) \ge f(x_2)$. The strict version follows
  from changing the above inequality to a strict one.
\end{proof}

\section{Applications}
\begin{example}
  Show that
  \[\frac{2}{2x + 1} < \ln(1 + 1 / x)\]
  for any $x > 0$.
\end{example}

\begin{proof}
  Let $f(x) = 2 / (2x + 1) - \ln(1 + 1 / x)$. Taking
  the derivative yields
  \[
    f'(x) = \frac{1}{(2x + 1)^2 x (x + 1)} > 0,
  \]
  so $f$ is strictly increasing in $(0, \infty)$.
  Note that $f \to 0$ as $x \to \infty$, so $f(x) < 0$
  for all $x > 0$.
\end{proof}

\begin{example}
  Show that $b / a > b^a / a^b$
  when $b > a > 1$.
\end{example}

\begin{proof}
  Take log on both sides to get
  $\ln b - \ln a > a \ln b - b \ln a$. This gives
  \[
    (b - 1)\ln a > (a - 1) \ln b
    \quad \iff \quad \frac{\ln a}{a - 1} > \frac{\ln b}{b - 1}.
  \]
  Note that this is a monotonicity property. So
  let $f(x) = (\ln x) / (x - 1)$ for $x > 1$. Then
  \[
    f'(x) = \frac{x - 1 - x\ln x}{x(x - 1)^2} < 0
  \]
  when $x > 1$ because $x - 1 - x \ln x < 0$. To see
  the last claim, define $g(x) = x - 1 - x \ln x$
  and note that $g'(x) = -\ln x < 0$ for $x > 1$.
  But $g(0) = 0$, so $g(x) < 0$ for $x > 1$. So $f$
  is strictly decreasing.
\end{proof}

\begin{example}
  Show that
  \[
    \lim_{x \to 0} \frac{e^x - e^{\sin x}}{x - \sin x} = 1.
  \]
  Let $f(x) = e^x$. Then there exists $\xi$ between
  $x$ and $\sin x$ such that
  \[
    e^x - e^{\sin x} = (x - \sin x) e^{\xi(x)},
  \]
  where the choice of $\xi$ may vary for different $x$.
  Then
  \[
    \lim_{x \to 0} \frac{e^x - e^{\sin x}}{x - \sin x} = 
    \lim_{x \to 0} e^{\xi(x)}.
  \]
  Now note that $\xi(x)$ is always between $x$ and
  $\sin x$, which both tend to $0$ as $x \to 0$. So by
  the squeeze theorem we have $\xi(x) \to 0$ as $x \to 0$
  and thus $e^{\xi(x)} \to 1$ as $x \to 0$.
\end{example}

\section{Cauchy's Mean Value Theorem}
\begin{theorem}[Cauchy's middle value theorem]
  Let $f, g \in C([a, b])$ and $f, g$ be differentiable
  in $(a, b)$. Suppose $g'(x) \ne 0$ for any
  $x \in (a, b)$. Then there exists $x_0 \in (a, b)$
  such that
  \[
    \frac{f'(x_0)}{g'(x_0)} = \frac{f(b) - f(a)}{g(b) - g(a)}.
  \]
\end{theorem}

\begin{proof}
  Use a similar construction as before and let
  \[
    F(x) = f(x) - f(a) - \frac{f(b) - f(a)}{g(b) - g(a)} (g(x) - g(a)).
  \]
  Note that $F(b) = F(a) = 0$, so by Rolle's theorem
  there exists $x_0 \in (a, b)$ such that
  $F'(x_0) = 0$. Then
  \[0 = F'(x_0) = f'(x_0) - \frac{f(b) - f(a)}{g(b) - g(a)} g'(x_0),\]
  which implies the desired result.
\end{proof}

\begin{remark}
  The $g'(x) \ne 0$ condition guarantees that
  $g$ is monotone, even if $g'$ may fail to be
  continuous.
\end{remark}

\begin{remark}
  If $g$ is a monotonically increasing function,
  we can view $g$ as a mapping
  $g : [a, b] \to [g(a), g(b)]$, which we can view
  as a change of variables $x \mapsto u$. Since $g$ is
  monotone, we have an inverse $x = g^{-1}(u)$. Then
  \[f(x) = f(g^{-1}(u)) = (f \circ g^{-1})(u) = \widetilde{f}(u).\]
  By Lagrange's theorem,
  \[
  \frac{\widetilde{f}(g(b)) - \widetilde{f}(g(a))}{g(b) - g(a)}
    = \widetilde{f}'(u_0)
  \]
  for some $u_0 \in (g(a, g(b))$. Now note that
  \[
    \widetilde{f}(g(b)) = (f \circ g^{-1})(g(b)) = f(b),
    \quad \widetilde{f}(g(a)) = f(a).
  \]
  So the left-hand side is precisely
  \[
    \text{LHS} = \frac{f(b) - f(a)}{g(b) - g(a)}.
  \]
  By the chain rule, we have
  \[
    \text{RHS} = \widetilde{f}'(u_0) = (f \circ g^{-1})'(u_0)
    = f'(g^{-1}(u_0)) (g^{-1})'(u_0)
    = f'(x_0) \frac{1}{g'(x_0)}.
  \]
  This recovers Cauchy's mean value theorem. So they are
  equivalent even if Cauchy's seems stronger.
\end{remark}

  \chapter{Jan.~16 --- Taylor's Theorem}

\section{Darboux's Lemma}
\begin{lemma}[Darboux's lemma]
  If $f$ is differentiable in $(a, b)$, continuous
  on $[a, b]$ and $f'(a) < f'(b)$, then for any
  $c \in (f'(a), f'(b))$, there exists $x_0 \in (a, b)$
  such that $f'(x_0) = c$.
\end{lemma}

\begin{proof}
  See homework.
\end{proof}

\begin{remark}
  There exists an example of a differentiable function
  $f(x)$ but $f'(x)$ is not continuous, e.g.
  \[
    f(x) =
    \begin{cases}
      x^2 \sin(1/x) & \text{if } x \ne 0 \\
      0 & \text{if } x = 0.
    \end{cases}
  \]
  We can compute that
  \[
    f'(x) =
    \begin{cases}
      2x \sin(1/x) - \cos(1/x) & \text{if } x \ne 0 \\
      0 & \text{if } x = 0,
    \end{cases}
  \]
  and we can verify as an exercise that $f'(x)$ is not
  continuous at $x = 0$.
\end{remark}

\begin{remark}
  Darboux's lemma guarantees that $g'(x) \ne 0$ implies
  either $g'(x) > 0$ or $g'(x) < 0$ everywhere in
  the conditions for Cauchy's mean value theorem.
\end{remark}

\section{L'H\^opital's Rule}
\begin{theorem}[L'H\^opital's rule, $0 / 0$]
  Let $f, g$ be differentiable in $(a, b)$,
  $\lim_{x \to a^+} f(x) = \lim_{x \to a^+} g(x) = 0$,
  and $g'(x) \ne 0$ for any $x \in (a, b)$. Then
  if $\lim_{x \to a^+} f'(x)/g'(x)$ exists, we have
  \[
    \lim_{x \to a^+} \frac{f(x)}{g(x)} = \lim_{x \to a^+} \frac{f'(x)}{g'(x)}.
  \]
\end{theorem}

\begin{proof}
  By Cauchy's theorem, for any $x \in (a, b)$, there
  exists $\xi(x) \in (a, x)$ such that
  \[
    \frac{f(x)}{g(x)} = \frac{f(x) - f(a)}{g(x) - g(a)}
    = \frac{f'(\xi(x))}{g'(\xi(x))}.
  \]
  If $x \to a^+$, then $\xi(x) \to a^+$, so
  \[
    \lim_{x \to a^+} \frac{f(x)}{g(x)}
    = \lim_{x \to a^+} \frac{f'(\xi(x))}{g'(\xi(x))}
    = \lim_{x \to a^+} \frac{f'(x)}{g'(x)},
  \]
  as desired.
\end{proof}

\begin{corollary}
  Let $f, g$ be differentiable in $(a, \infty)$,
  $\lim_{x \to \infty} f(x) = \lim_{x \to \infty} g(x) = 0$,
  and $g'(x) \ne 0$ for any $x \in (a, \infty)$. Then
  if $\lim_{x \to \infty} f'(x)/g'(x)$ exists, we have
  \[
    \lim_{x \to \infty} \frac{f(x)}{g(x)} = \lim_{x \to \infty} \frac{f'(x)}{g'(x)}.
  \]
\end{corollary}

\begin{proof}
  Assume $a > 0$. Define $\widetilde{f}(y) = f(1/y)$ and
  $\widetilde{g}(y) = g(1/y)$ with $y \in (0, 1 / a)$.
  By L'H\^opital's rule,
  \[
    \lim_{y \to 0^+} \frac{\widetilde{f}(y)}{\widetilde{g}(y)}
    = \lim_{y \to 0^+} \frac{\widetilde{f}'(y)}{\widetilde{g}'(y)}
    = \lim_{y \to \infty} \frac{f'(1 / y) \cdot (-1 / y^2)}{g'(1 / y) \cdot (-1 / y^2)}
    = \lim_{x \to \infty} \frac{f'(x)}{g'(x)},
  \]
  as desired.
\end{proof}

\begin{theorem}[L'H\^opital, $\infty / \infty$]
  Let $f, g$ be differentiable in $(a, b)$,
  $\lim_{x \to a^+} |f(x)| = \lim_{x \to a^+} |g(x)| = \infty$,
  and $g'(x) \ne 0$ for any $x \in (a, b)$. Then
  if $\lim_{x \to a^+} f'(x)/g'(x)$ exists, we have
  \[
    \lim_{x \to a^+} \frac{f(x)}{g(x)} = \lim_{x \to a^+} \frac{f'(x)}{g'(x)}.
  \]
\end{theorem}

\begin{proof}
  Left as an exercise.
\end{proof}

\begin{remark}
  Saying that the absolute values of $f$ and $g$ go to
  infinity works, since the existence of the limit rules
  out oscillatory behavior.
\end{remark}

\begin{remark}
  These cases of $\infty / \infty$ and $0 / 0$ are
  are called \emph{indefinite types}. Other indefinite
  types include $0 \cdot \infty$, $0^0$, $\infty^0$
  $1^\infty$, $\infty - \infty$, etc. But we can try
  to reduce them to the cases we know. For example,
  if $f(x) \to 0^+$ and $g(x) \to 0^+$ when $x \to x_0$,
  then $\lim_{x \to x_0} f(x)^{g(x)}$ is $0^0$.
  Letting $y(x) = f(x)^{g(x)}$, we can take the log
  to get
  \[
    \ln y(x) = g(x) \ln f(x) = \frac{\ln f(x)}{1 / g(x)}
    = \frac{\infty}{\infty}.
  \]
\end{remark}

\begin{example}
  We can see that (this is a $\infty - \infty$ case)
  \[
    \lim_{x \to 0^+} \frac{1}{x^2} - \frac{\cot x}{x}
    = \lim_{x \to 0^+} \frac{1 + x \cot x}{x^2}
    = \lim_{x \to 0^+} \frac{-(\cot x - x \csc^2 x)}{2x \sin^2 x}.
  \]
  Note that $x \cot x = x \cos x / \sin x \to 1$ as
  $x \to 0$. Now note that $\sin x / x \to 1$ as $x \to 0$,
  so we continue with
  \[
    \lim_{x \to 0^+} \frac{-(\cot x - x \csc^2 x)}{2x \sin^2 x}
    = \lim_{x \to 0^+} \frac{x - \sin x \cos x}{2x^3} \frac{x^2}{\sin^2 x}
  \]
  Since $x^2 / \sin^2 x \to 1$ as $x \to 0$, we can
  look at the remaining part to get
  \[
    \lim_{x \to 0^+} \frac{x - \sin x \cos x}{2x^3}
    = \lim_{x \to 0^+} \frac{1 - \cos 2x}{6x^2}
    = \lim_{x \to 0^+} \frac{2 \sin 2x}{12x} = \frac{1}{3}.
  \]
  So $\lim_{x \to 0^+} (1 / x^2 - \cot x / x) = 1 / 3$.
\end{example}

\section{Taylor's Theorem}
\begin{theorem}[Peano remainder term]
  Let $f : [a, b] \to \R$ be differentiable at $x = a$
  up to $n$th order of derivatives, i.e.
  $f'(a), f''(a), \dots, f^{(n)}(a)$ exist. Then as
  $x \to a^+$, we have
  \[f(x) = \sum_{k = 0}^n \frac{f^(k)(a)}{k!}(x - a)^k + o((x - a)^n).\]
  Call the polynomial part of the above $P_n(x)$, which
  is also known as the \emph{Taylor polynomial} of order $n$.
\end{theorem}

\begin{proof}
  To show that the error term is $o((x - a)^n)$, we have
  \[
    \lim_{x \to a^+} \frac{f(x) - P_n(x)}{(x - a)^n}
    = \lim_{x \to a^+} \frac{f'(x) - P_n'(x)}{n(x - a)^{n - 1}}
    = \frac{1}{n!} \lim_{x \to a^+} \left[\frac{f^{n - 1}(x) - f^{n - 1}(a)}{x - a} - f^{(n)}(a)\right] = 0
  \]
  by L'H\^opital's rule, where we used the observation
  that
  $f^{(k)}(a) = P_n^{(k)}(a)$ for $1 \le k \le n$.
  The final step is a result of the existence of $f^{(n)}(a)$.
\end{proof}

\begin{lemma}[Rolle's theorem for higher order derivatives]
  Let $f \in C^([a, b])$ and differentiable to
  $(n + 1)$ order. If $f'(a) = \dots = f^{(n)}(a) = 0$
  and $f(a) = f(b)$, then there exists $x_0 \in (a, b)$
  such that $f^{(n + 1)}(x_0) = 0$.
\end{lemma}

\begin{proof}
  Since $f(a) = f(b)$, by the usual Rolle's theorem
  there exists $x_1 \in (a, b)$ such that $f'(x_1) = 0$.
  Then since $f'(a) = f'(x_1) = 0$, by
  Rolle's theorem again,
  there exists $x_2 \in (a, x_1)$ such that $f''(x_2) = 0$.
  Repeat this to get $x_{n + 1} \in (a, x_n) \subseteq (a, b)$ such that
  $f^{(n + 1)}(x_{n + 1}) = 0$. Take $x_0 = x_{n + 1}$
  to finish.
\end{proof}

\begin{theorem}[Lagrange remainder term]
  Let $f \in C^n([a, b])$, in particular,
  $f'(a), \dots, f^{(n)}(a)$ exist. Additionally,
  assume $f$ is $(n + 1)$-th differentiable in $(a, b)$.\footnote{Note that the $(n + 1)$-th derivative need not be continuous here.}
  Then
  \[
    f(x) = \sum_{k = 0}^n \frac{f^{(k)}(a)}{k!}(x - a)^k + R_n(x) \quad \text{where} \quad R_n(x) = \frac{f^{(n + 1)}(\xi)}{(n + 1)!}(x - a)^{n + 1}
  \]
  for some $\xi \in [a, x]$.
\end{theorem}

\begin{proof}
  Define $P(x) = P_n(x) + \lambda (x - a)^{n + 1}$, where
  we choose $\lambda \in \R$ such that $P(b) = f(b)$, i.e.
  \[
    \lambda = \frac{f(b) - P_n(b)}{(b - a)^{n + 1}}.
  \]
  Consider $g(x)  = f(x) - P(x)$, which satisfies
  $g(a) = g(b) = 0$ and $g'(a) = \dots = g^{(n)}(a) = 0$.
  Then by Rolle's theorem (higher order), there exists
  $\xi \in (a, b)$ such that $g^{(n + 1)}(\xi) = 0$.
  In other words,
  \[
    f^{(n + 1)} - P^{(n + 1)}(\xi) = 0
    \implies f^{(n + 1)}(\xi) - (n + 1)! \underbrace{\frac{f(b) - P_n(b)}{(b - a)^{n + 1}}}_{\lambda} = 0.
  \]
  This implies that
  \[
    f(b) = P_n(b) + \frac{1}{(n + 1)!} f^{(n + 1)}(\xi) (b - a)^{n + 1},
  \]
  and since we picked $b$ arbitrarily (kind of),
  we can take $b = x$ and we are done since
  $\xi \in [a, b]$.
\end{proof}

\begin{remark}
  The choice of $\xi$ in Lagrange's remainder term
  may (and likely does) vary for different $x$.
\end{remark}

\begin{remark}
  The Taylor polynomial is unique in the sense that
  if $f : [a, b] \to \R$ and $f'(a), \dots, f^{(n)}(a)$
  exist, then if
  \[f(x) = p(x) + o((x - a)^n) \text{as} x \to a^+\]
  as $x \to a^+$ for some polynomial $p(x)$ with
  $\deg p \le n$, then
  $p(x) = P_n(x) = \sum_{k = 0}^n \frac{f^{(k)}(a)}{k!} (x - a)^k$.
  This is because if $Q(x) = p(x) - P_n(x)$, then
  by Taylor's formula (Peano form), we get
  \[
    \lim_{x \to a^+} \frac{Q(x)}{(x - a)^n}
    = \lim_{x \to a^+} \frac{p(x) - \sum_{k = 0}^n \frac{f^{(k)}(a)}{k!} (x - a)^k}{(x - a)^n} = 0.
  \]
  From here this implies that $Q(x) = 0$ since $\deg Q \le n$.
\end{remark}

  \chapter{Jan.~18 --- Taylor Polynomials}

\section{Common Taylor Polynomials}
We have
\begin{align*}
  e^x &= 1 + x + \frac{x^2}{2!} + \dots + \frac{x^n}{n!} + o(x^n), \\
  \sin x &= x - \frac{x^3}{3!} + \frac{x^5}{5!} - \dots
  + \frac{(-1)^{n - 1} x^{2n - 1}}{(2n - 1)!} + o(x^{2n}), \\
  \cos &= 1 - \frac{x^2}{2!} + \frac{x^4}{4!} - \dots
  + \frac{(-1)^n x^{2n}}{(2n)!} + o(x^{2n + 1}), \\
  (1 + x)^\alpha &= 1 + \alpha x + \frac{\alpha(\alpha - 1)}{2!} x^2 + \dots + \frac{\alpha(\alpha - 1) \dots (\alpha - n + 1)}{n!} x^n + o(x^n), \\
  \ln (1 + x) &= x - \frac{x^2}{2} + \frac{x^3}{3}
  + \dots + (-1)^{n - 1} \frac{x^n}{n} + o(x^n).
\end{align*}

\section{Combining Taylor Polynomials}
\begin{remark}
  If $a = 0$ and $f(x)$ is even in $(-b, b)$, then
  \[
    f(x) = \sum_{k = 0}^{n / 2} a_k x^{2k} + o(x^n).
  \]
  Similarly if $f(x)$ is odd in $(-b, b)$, then
  \[
    f(x) = \sum_{k = 0}^{n / 2} a_k x^{2k + 1} + o(x^{n + 1}).
  \]
\end{remark}

\begin{remark}
To create new Taylor polynomials from known ones,
we can observe that if
$f(x) = P_n(x) + o((x - a)^n)$ and $g(x) = Q_n(x) + o((x - a)^n)$, then
\[
  f(x) + g(x) = (P_n(x) + Q_n(x)) + o((x - a)^n)
  \quad \text{and} \quad
  f(x) g(x) = \underbrace{(P_n(x) Q_n(x))}_{\text{take first $n$ terms}} + o((x - a)^n).
\]
If $P_n(x) = \sum_{k = 0}^n a_k (x - a)^k$ and
$Q_n(x) = \sum_{k = 0}^n b_k (x - a)^k$, then
$f(x)g(x)$ has Taylor polynomial
$\sum_{k = 0}^n c_k (x - a)^k$ where
\[c_k = \sum_{i = 0}^k a_i b_{k - i}.\]
If $h(x) = f(x) / g(x)$ and $g(x) \ne 0$ near $x = a$,
then $f(x) = h(x) g(x)$. Let $h(x) = \sum_{k = 0}^n c_k (x - a)^k + o((x - a)^n)$, then
\[
  a_k = \sum_{i = 0}^k c_i b_{k - i}
\]
for $0 \le k \le n$, after which we can solve for the
$c_k$.
\end{remark}

\begin{example}
Find the Taylor polynomial for $\tan x$ up to $n = 5$.
\end{example}

\begin{proof}
Note that $\tan x$ is odd, so we can write
\[
  \tan x = x + a_3 x^3 + a_5 x^5 + o(x^5).
\]
Now since $\tan x = \sin x / \cos x$, we have
$\sin x = \tan x \cos x$, so
\[
  x - \frac{x^3}{6} + \frac{x^5}{5!} + o(x^5)
  = (x + a_3 x^3 + a_5x^5)(1 - \frac{x^2}{2!} + \frac{x^4}{4!})
\]
We can solve to get
\[
  \begin{cases}
    -\frac{1}{6} = -\frac{1}{2} + a_3 \\
    \frac{1}{5!} = \frac{1}{4!} - \frac{a_3}{2!} + a_5
  \end{cases}
  \implies \quad
  a_3 = -\frac{1}{3}, \quad
  a_5 = \frac{2}{15}
\]
as the coefficients for the Taylor polynomial.
\end{proof}

\begin{remark}
  If
  \[
    f'(x) = \sum_{k = 0}^n b_k (x - a)^k + o((x - a)^n),
  \]
  then the anti-derivative of $f(x)$ has
  \[f(x) = f(x_0) + \sum_{k = 0}^n a_{k + 1} (x - a){k + 1} + o((x - a)^{n + 1}),\]
  where $a_{k + 1} = b_k / (k + 1)$ for $0 \le k \le n$.
  This is because
  \[
    b_k = \frac{(f')^{(k)}(a)}{k!} = \frac{f^{(k + 1)}(a)}{k!}
    \quad \text{and} \quad
    a_{k + 1} = \frac{f^{(k + 1)}(a)}{k + 1}
    = \frac{1}{k + 1} \frac{f^{(k + 1)}(a)}{k!}
    = \frac{b_k}{k + 1}.
  \]
\end{remark}

\begin{example}
  Find the Taylor polynomial for $f(x) = \arctan x$.
\end{example}

\begin{proof}
  Recall that
  \[f'(x) = \frac{1}{1 + x^2} = \sum_{k = 0}^n (-1)^k x^{2k}.\]
  Using the above we get
  \[
    f(x) = \arctan x = \sum_{k = 0}^n (-1)^k \frac{x^{2k + 1}}{2k + 1} + o(x^{2n + 2})
  \]
  as the Taylor polynomial.
\end{proof}

\section{Applications for Taylor Polynomials}
\subsection{Finding Limits}
\begin{remark}
Let $f(x) = ax^n + o(x^n)$ as $x \to 0$
and $g(x) = bx^n + o(x^n)$ where $b \ne 0$.
Then
\[
  \lim_{x \to 0} \frac{f(x)}{g(x)} = \frac{a}{b}.
\]
\end{remark}

\begin{remark}
  For the polynomial of $f(g(x))$, we can do
  \[
    f(u) = \sum_{k = 0}^n a_k (u - g(a))^k + o((u - g(a))^n),
    \quad \text{where} \quad
    u = g(x) = \sum_{k = 0}^n b_k(x - a)^k + o((x - a)^n).
  \]
  Then we can substitute  in $u = g(x)$ to find the
  overall polynomial.
\end{remark}

\begin{example}
  Find
  \[
    \lim_{x \to 0} \frac{\sqrt{1 + 2 \tan x} - e^x + x^2}{\arcsin x - \sin x}.
  \]
\end{example}

\begin{proof}
  Note that
  \begin{align*}
    \sqrt{1 + 2\tan x} - e^x + x^2 &= \frac{2x^3}{3} + o(x^3), \\
    \arcsin x - \sin x &= \frac{x^3}{3} + o(x^3).
  \end{align*}
  So the desired limit is $2$.
\end{proof}

\begin{remark}
  If $f(x) = ax^n + o(x^n)$ and $g(x) = bx^m + o(x^m)$
  for $a, b \ne 0$, then
  \[
    \lim_{x \to 0} \frac{f(x)}{g(x)} = \begin{cases}
      a / b & \text{if $m = n$}, \\
      0 & \text{if $m < n$}, \\
      \infty & \text{if $m > n$}.
    \end{cases}
  \]
\end{remark}

\begin{example}
  Assume $f(x) = 1 + ax^n + o(x^n)$ where
  $a \ne 0$ and
  \[
    g(x) = \frac{1}{bx^n + o(x^n)}, \quad \text{i.e.} \quad
    \frac{1}{g(x)} = bx^n + o(x^n).
  \]
  for $b \ne 0$. Then
  \[
    \lim_{x \to 0} f(x)^{g(x)} = e^{a / b}.
  \]
  Let $y(x) = f(x)^{g(x)}$, then
  $\ln y(x) = g(x) \ln f(x)$. Note that
  \[
    \ln f(x) = \ln(1 + ax^n + o(x^n))
    = ax^n + o(x^n),
  \]
  so that
  \[
    \frac{\ln f(x)}{1 / g(x)} = \frac{ax^n + o(x^n)}{bx^n + o(x^n)} \to \frac{a}{b}
  \]
  as $x \to 0$. Thus $\ln y(x) \to a / b$ and
  $y(x) \to e^{a / b}$ as $x \to 0$.
\end{example}

\begin{example}
  Find
  \[
    \lim_{x \to 0} \left[\cos(xe^x) - \ln (1 - x) - x\right]^{\cot x^3}.
  \]
\end{example}

\begin{proof}
  Here we have
  \[
    f(x) = \cos(xe^x) - \ln(1 - x) - x
    = 1 - \frac{2}{3} x^3 + o(x^3) \quad
    \text{and} \quad \frac{1}{g(x)} = \tan x^3 = x^3 + o(x^3).
  \]
  Thus the limit is $e^{-2 / 3}$.
\end{proof}

\subsection{Estimation}
\begin{example}
  Let $f(x)$ be twice differentiable in $[0, 1]$ and
  $f(0) = f(1)$. Further assume $|f''(x)| \le M$
  for $0 \le x \le 1$. Prove that
  $|f'(x)| \le M / 2$ for $0 \le x \le 1$.
\end{example}

\begin{proof}
  Recall that Lagrange's form of Taylor's theorem says
  \[
    f(x) = f(a) + f'(a) (x - a) + \frac{f''(\xi)}{2!}(x - a)^2
  \]
  for some $\xi$ between $a$ and $x$. Thus for any
  $x \in (0, 1)$, we have
  \[
    f(1) = f(x) + f'(x)(1 - x) + \frac{f''(\xi_1)}{2}(1 - x)^2.
  \]
  Similarly, we have
  \[
    f(0) = f(x) + f'(x)(-x) + \frac{f''(\xi_2)}{2}x^2.
  \]
  Here $x \le \xi_1 \le 1$ and $0 \le \xi_2 \le x$.
  Since $f(1) = f(2)$, we can solve for $f'(x)$ to get
  \[
    f'(x) = \frac{f''(\xi_2)x^2 - f''(\xi_1)(1 - x)^2}{2}.
  \]
  Then taking absolute values yields
  \[
    |f'(x)| \le M \left(\frac{x^2 + (1 - x)^2}{2}\right) \le \frac{M}{2} \max_{0 \le x \le 1} \left[x^2 + (1 - x)^2\right]
    = \frac{M}{2},
  \]
  as desired.
\end{proof}

\begin{example}
  Let $f(x)$ be twice differentiable in $[0, 1]$ and
  $f'(a) = f'(b) = 0$. Then there exists $\xi \in (a, b)$
  such that
  \[
    |f''(\xi)| \ge 4 \frac{|f(a) - f(b)|}{(b - a)^2}.
  \]
\end{example}

\begin{proof}
  Note that this is equivalent to
  \[
    |f(b) - f(a)| \le f''(\xi)\left(\frac{b - a}{2}\right)^2.
  \]
  Then we have
  \[
    f\left(\frac{b + a}{2}\right) = f(a) + \frac{f''(\xi_1)}{2}\left(\frac{b - a}{2}\right)^2
    = f(b) - \frac{f''(\xi_2)}{2}\left(\frac{b - a}{2}\right)^2,
  \]
  so that
  \[
    f(b) - f(a) = \frac{f''(\xi_2) + f''(\xi_1)}{2} \left(\frac{b - a}{2}\right)^2.
  \]
  From here we have
  \[
    |f(b) - f(a)| \le \underbrace{\frac{|f''(\xi_1)| + |f''(\xi_2)|}{2}}_{= |f''(\xi)|} \left(\frac{b - a}{2}\right)^2
  \]
  for some $\xi \in (a, b)$ by Darboux's lemma, as
  desired.
\end{proof}

  \chapter{Jan.~23 --- The Riemann Integral}

\section{The Anti-Derivative}

Recall the \emph{anti-derivative} from calculus:

\begin{definition}
  Let $f : U \to \R$ where $U$ is an interval in $\R$.
  If there exists a differentiable function $F : U \to \R$
  such that $F'(x) = f(x)$ for all $x \in U$, then
  $F(x)$ is an \emph{anti-derivative} of $f$, denoted
  \[F(x) = \int f(x)\, dx.\]
  This is also called the \emph{indefinite integral} of
  $f$.
\end{definition}

\begin{remark}
  The anti-derivatives of a function can differ by a
  constant.
\end{remark}

\begin{example}
  Find an anti-derivative of $f(x) = |x|$ for $x \in \R$.
\end{example}

\begin{proof}
  If $x > 0$, we have $f(x) = x$ and so
  $F(x) = x^2 / 2$. If $x < 0$, then $f(x) = -x$
  and so $F(x) = -x^2 / 2$. We can also write this as
  \[
    F(x) = x \cdot \frac{|x|}{2}.
  \]
  Clearly for $x \ne 0$, we have $F'(x) = f(x)$. At
  $x = 0$, we have
  \[
    \lim_{x \to 0} \frac{F(x) - f(0)}{x} = \lim_{x \to 0}
    \frac{1}{2} |x| = 0,
  \]
  so $F'(0) = f(0)$ and $F$ is an anti-derivative of $f$.
\end{proof}

\begin{remark}
  The eventual goal is to show that any continuous
  function $f : [a, b] \to \R$ has an anti-derivative.
\end{remark}

\begin{example}
  Find an anti-derivative for
  \[
    f(x) =
    \begin{cases}
      1 & \text{if } x > 0\\
      0 & \text{if } x \le 0 \\
      -1 & \text{if } x < 0.
    \end{cases}
  \]
\end{example}

\begin{proof}
  We can try to use $F(x) = |x|$, but recall that $F$ is
  not differentiable at $x = 0$. More generally, suppose
  that $f(x)$ has some anti-derivative $F(x)$, i.e.
  $f(x) = F'(x)$. By Darboux's theorem, $f(x)$ must
  take all values in $(-1, 1)$, which is a contradiction
  with the definition of $f$.
\end{proof}

\begin{remark}
  If $f(x)$ has a jump discontinuity, then it has no
  anti-derivative.
\end{remark}

\section{The Riemann Integral}
Recall from calculus that if $f(x)$ is defined in
$[a, b]$ and $F'(x) = f(x)$, then we have\footnote{This is the \emph{fundamental theorem of calculus}.}
\[
  \int_a^b f(x)\, dx = F(x) \Big|_a^b = F(b) - F(a).
\]
We called this the \emph{definite integral} of $f$ in
calculus, but we would like a more rigorous definition.

\begin{definition}
  Let $a, b \in \R$ and $a < b$. A \emph{partition} of
  the interval $[a, b]$ is a finite sequence of
  numbers $x_0, x_1, \dots, x_n$ such that
  $a = x_0 < x_1 < \dots < x_n = b$.
\end{definition}

\begin{definition}
  The \emph{width} of a partition $x_0, x_1 , \dots, x_n$ is
  $\max\{x_i - x_{i - 1} : i = 1, 2, \dots, n\}$.
\end{definition}

\begin{definition}
  For any partition $x_0, x_1, \dots, x_n$, define the
  \emph{Riemann sum} to be
  \[
    S = \sum_{i = 1}^n f(x_i') (x_i - x_{i - 1}),
  \]
  where $x_i'$ is any point between $x_{i - 1}$ and
  $x_i$, inclusive.\footnote{The geometric intuition of the Riemann sum is an approximation for the \emph{area} under the graph of $f$ by rectangles.}
\end{definition}

\begin{definition}
  Let $a, b \in \R$ with $a < b$ and $f : [a, b] \to \R$.
  We say $f$ is \emph{Riemann integrable} on $[a, b]$ if
  there exists $A \in \R$ such that for all
  $\epsilon > 0$, there exists $\delta > 0$ such that
  $|S - A| < \epsilon$ whenever $S$ is any Riemann
  sum for a partition of $[a, b]$ with width less than
  $\delta$. We call $A$ the \emph{Riemann integral} of
  $f$ on $[a, b]$ and denote it by
  \[
    A = \int_a^b f(x)\, dx.
  \]
\end{definition}

\begin{remark}
  If $f$ is Riemann integrable, then
  \[
    A = \int_a^b f(x)\, dx
  \]
  is unique. This is because if $A$ and $A'$ are two
  numbers for the Riemann integral, then for any
  $\epsilon > 0$, there exists $\delta > 0$ such that
  \[
    |A - S| < \epsilon \quad \text{and} \quad
    |A' - S| < \epsilon
  \]
  for any Riemann sum $S$ associated with a partition of
  width less than $\delta$. Then
  \[
    |A - A'| \le |A - S| + |A' - S| < 2\epsilon,
  \]
  so $A = A'$ and thus the Riemann integral is unique.
\end{remark}

\begin{example}
  Let $f(x) = c$ on $[a, b]$, a constant function. Then for any partition $x_0, x_1, \dots, x_n$,
  \[
    S = \sum_{i = 1}^n f(x_i') (x_i - x_{i - 1}) =
    \sum_{i = 1}^n c (x_i - x_{i - 1}) = c(b - a)
    \implies \int_a^b c\, dx = c(b - a).
  \]
\end{example}

\begin{example}
  Fix $\xi \in [a, b]$ and let $f : [a, b] \to \R$
  be defined by
  \[
    f(x) = \begin{cases}
      0 & \text{if } x \ne \xi \\
      c & \text{if } x = \xi.
    \end{cases}
  \]
  Check that
  \[
    A = \int_a^b f(x)\, dx = 0.
  \]
\end{example}

\begin{proof}
  For any partition $a = x_0 < x_1 < \dots < x_n = b$
  with width $\delta$, we have
  \[
    |S| = \left|\sum_{i = 1}^n f(x_i') (x_i - x_{i - 1})\right|
    \le |c| 2\delta
  \]
  since $\xi$ can be in at most two of the intervals of
  the partition. Then for any $\epsilon > 0$,
  choose $\delta = \epsilon / (2|c|)$, so that
  $|S| < \epsilon$ for any partition of width less
  than $\delta$. From this we can conclude that $A = 0$.
\end{proof}

\begin{example}
  Consider a step function.
  Let $\alpha, \beta \in [a, b]$ with $\alpha < \beta$.
  Define $f : [a, b] \to \R$ by
  \[
    f(x) = \begin{cases}
      1 & \text{if } x \in (\alpha, \beta) \\
      0 & \text{if } x \notin (\alpha, \beta) \text{ and } x \in [a, b].
    \end{cases}
  \]
  Note that $f$ has no anti-derivative, but it
  is Riemann integrable. In fact,
  \[
    \int_a^b f(x)\, dx = \beta - \alpha.
  \]
  To see this, take any partition
  $a = x_0 < x_1 < \dots < x_n = b$ with width less
  than $\delta$. Then
  \[
    S = \sum_{i = 1}^n f(x_i') (x_i - x_{i - 1}) =
    \sum_{[x_{i - 1}, x_i] \cap [\alpha, \beta] \ne \varnothing} f(x_i') (x_i - x_{i - 1}).
  \]
  Each partition is in two classes: Either (1) it
  only partially intersects $[\alpha, \beta]$ or (2)
  it is contained in $[\alpha, \beta]$. So
  \[
    S = \underbrace{1 (\text{total length of intervals of class 2})}_{I_1}
    + \underbrace{|f(x_i')| (\text{total length of intervals of class 1})}_{I_2}.
  \]
  We have $|I_1 - (\beta - \alpha)| < 2 \delta$ and
  $|I_2| < 2\delta$ since there are at most two intervals
  of class 1. So
  \[
    |S - (\beta - \alpha)| \le |I_1| + |I_2| < 4\delta.
  \]
  So $f(x)$ is Riemann integrable and
  \[
    \int_a^b f(x)\, dx = \beta - \alpha,
  \]
  as desired.
\end{example}

\begin{example}
  Define $f : [a, b] \to \R$ by
  \[
    f(x) = \begin{cases}
      1 & \text{if $x$ is rational} \\
      0 & \text{if $x$ is irrational}.
    \end{cases}
  \]
  Then $f(x)$ is not Riemann integrable. For
  any partition $a = x_0 < x_1 < \dots < x_n = b$,
  \[
    S = \sum_{i = 1}^n f(x_i') (x_i - x_{i - 1}) =
    \begin{cases}
      b - a & \text{if $x_i'$ are all rational} \\
      0 & \text{if $x_i'$ are all irrational}.
    \end{cases}
  \]
  We can always choose $x_i'$ to be in either case
  since the rationals and irrationals are both dense
  in $\R$. So there is no $A \in \R$ such that
  $|A - S| < \epsilon$, no matter how small we take
  $\delta$ to be.
\end{example}

\begin{remark}
  The function $f$ from the previous example is not
  Riemann integrable, but it is Lebesgue integrable.
  In fact,
  \[
    L = \int_a^b f(x)\, dx = 0
  \]
  with respect to the Lebesgue measure. This
  is because the set of rational numbers
  $\Q$ has measure zero.
\end{remark}

\section{Properties of the Riemann Integral}
\begin{prop}
We have the following linearity properties of the Riemann integral:
\begin{enumerate}
  \item If $f, g : [a, b] \to \R$ are Riemann integrable, then
  $f \pm g$ are also integrable and
  \[
    \int_a^b (f \pm g) \, dx = \int_a^b f(x)\, dx \pm \int_a^b g(x)\, dx
  \]
  \item For any $c \in \R$, $cf$ is integrable
    and
    \[
      \int_a^b cf \, dx = c \int_a^b f(x)\, dx.
    \]
\end{enumerate}
\end{prop}

\begin{proof}
  See textbook, fairly straightforward.
\end{proof}

\begin{remark}
  Since we only discuss Riemann integration in this
  class, we will sometimes simply say ``integrable''
  instead of ``Riemann integrable.''
\end{remark}

\begin{prop}
  If $f : [a, b] \to \R$ is integrable and $f(x) \ge 0$,
  then
  \[
    \int_a^b f(x)\, dx \ge 0.
  \]
\end{prop}

\begin{proof}
  Let
  \[A = \int_a^b f(x)\, dx.\]
  Then for any $\epsilon > 0$, there exists $\delta > 0$
  such that for any partition of width $< \delta$,
  we have $|A - S| < \epsilon$. But
  \[
    S = \sum_{i = 1}^n f(x_i') (x_i - x_{i - 1}) \ge 0,
  \]
  Then we have $A > S - \epsilon \ge -\epsilon$, so
  taking $\epsilon \to 0$ gives $A \ge 0$.
\end{proof}

\begin{corollary}
  If $f, g : [a, b] \to \R$ are integrable and
  $f(x) \ge g(x)$ for all $x \in [a, b]$, then
  \[
    \int_a^b f(x)\, dx \ge \int_a^b g(x)\, dx.
  \]
\end{corollary}

\begin{proof}
  By linearity,
  \[
    \int_a^b f(x)\, dx - \int_a^b g(x)\, dx =
    \int_a^b (f(x) - g(x))\, dx \ge 0
  \]
  since $f(x) - g(x) \ge 0$ by assumption.
\end{proof}

\begin{corollary}
  If $f : [a, b] \to \R$ is integrable and
  $m \le f(x) \le M$ for all $x \in [a, b]$, then
  \[
    m(b - a) \le \int_a^b f(x)\, dx \le M(b - a).
  \]
\end{corollary}

  \chapter{Jan.~25 --- Riemann Integrability}

\section{Conditions for Integrability}
\begin{lemma}
  \label{lem:first-integrable}
  A function $f : [a, b] \to \R$ is integrable if and only if
  for any $\epsilon > 0$, there exists $\delta > 0$
  such that $|S_1 - S_2| < \epsilon$ whenever $S_1$
  and $S_2$ are Riemann sums for partitions of
  width less than $\delta$.
\end{lemma}

\begin{proof}
  $(\Rightarrow)$ If $f$ is integrable, then for
  any $\epsilon > 0$, there exists $\delta > 0$ such that
  \[
    \left|S - \int_a^b f(x)\, dx\right| < \frac{\epsilon}{2}
  \]
  for any Riemann sum $S$ of a partition with width
  less than $\delta$. Then
  \[
    |S_1 - S_2| \le
    \left|S_1 - \int_a^b f(x)\, dx\right| +
    \left|S_2 - \int_a^b f(x)\, dx\right| <
    \frac{\epsilon}{2} + \frac{\epsilon}{2} = \epsilon,
  \]
  as desired.

  $(\Leftarrow)$ Take the special partition into intervals
  of equal length, with width $(a - b) / n$. Pick
  the middle point in each interval, and let
  \[
    S_n = \sum_{i = 1}^n f(x_i') (x_i - x_{i - 1})
  \]
  be the corresponding Riemann sum. Now we check that
  $\{S_n\}_{n = 1}^\infty$ is a Cauchy sequence. This is
  because for any $\epsilon > 0$, if $N$ is large
  enough, then for any $n, m \ge N$, we have
  $|S_n - S_m| < \epsilon$ if $1 / N < \delta$. Then
  $\{S_n\}_{n = 1}^\infty$ converges, so let
  $\lim_{n \to \infty} S_n = A$. Now for any
  $\epsilon > 0$, there exists $\delta > 0$ such that
  for any Riemann sum $S$ with width $< \delta$,
  if $1 / n < \delta$, then $|S_n - S| < \epsilon / 2$.
  So
  \[
    |S - A| \le |S_n - S| + |S_n - A|
    < \frac{\epsilon}{2} + \frac{\epsilon}{2} = \epsilon,
  \]
  if $n$ is large enough. Thus
  \[
    A = \int_a^b f(x)\, dx
  \]
  exists and is the Riemann integral of $f$.
\end{proof}

\begin{remark}
Recall the step function $f : [a, b] \to \R$ given by
\[
  f(x) =
  \begin{cases}
    1 & \text{if } x \in (\alpha, \beta) \subseteq [a, b] \\
    0 & \text{if } x \notin (\alpha, \beta).
  \end{cases}
\]
Last time we saw that $f$ is integrable and that
\[
  \int_a^b f(x)\, dx = \beta - \alpha.
\]
Now let us consider a more general step function. We
call $f$ a \emph{step function} on $[a, b]$ if there
exists
a partition $x_0 < x_1 < \dots < x_n$ of $[a, b]$ such
that $f(x)$ is constant on each subinterval
$(x_{i - 1}, x_i)$.
\end{remark}

\begin{lemma}
  If $f : [a, b] \to \R$ is a step function for
  a partition $x_0 < x_1 < \dots < x_n$ and
  $f(x) = c_i$ when $x \in (x_{i - 1}, x_i)$, then
  $f$ is integrable and
  \[
    \int_a^b f(x)\, dx = \sum_{i = 1}^n c_i (x_i - x_{i - 1}).
  \]
\end{lemma}

\begin{proof}
  Define
  \[
    \varphi_i(x) = \begin{cases}
      1 & \text{if } x \in (x_{i - 1}, x_i) \\
      0 & \text{otherwise}.
    \end{cases}
  \]
  Now let
  \[
    h = f - \sum_{i = 1}^n c_i \varphi_i.
  \]
  Then $h(x)$ is nonzero only at $\{x_i\}_{i = 0}^n$.
  Each $\varphi_i$ is integrable and $h$ is integrable
  with
  \[
    \int_a^b h(x)\, dx = 0,
  \]
  so $f$ is also integrable and
  \[
    \int_a^b f(x)\, dx
    = \sum_{i = 1}^{n} c_i \int_a^b \varphi_i(x)\, dx
    = \sum_{i = 1}^n c_i (x_i - x_{i - 1})
  \]
  by linearity and the integral of a simple
  step function that we calculated before.
\end{proof}

\begin{prop}
  A function $f : [a, b] \to \R$ is integrable if
  and only if for any $\epsilon > 0$, there exist
  step functions $f_1, f_2$ such that
  $f_1(x) \le f(x) \le f_2(x)$ for all $x \in [a, b]$
  and
  \[
    \int_a^b (f_2 - f_1)\, dx < \epsilon.
  \]
\end{prop}

\begin{proof}
  $(\Leftarrow)$ For any $\epsilon > 0$, choose
  step functions $f_1, f_2$ such that
  \[
    \int_a^b (f_2 - f_1)\, dx < \frac{\epsilon}{3}.
  \]
  Then there exists $\delta > 0$ such that for any
  partition with width $< \delta$, the Riemann
  sums $S_1, S_2$ for $f_1, f_2$ satisfy
  \[
    |S_1 - \int_a^b f_1(x)\, dx| < \frac{\epsilon}{3}
    \quad\text{and}\quad
    |S_2 - \int_a^b f_2(x)\, dx| < \frac{\epsilon}{3}.
  \]
  So for any partition width $< \delta$, the Riemann sum
  of $f$ is
  \[
    S = \sum_{i = 1}^n f(x_i') (x_i - x_{i - 1}),
  \]
  and $S_1 \le S \le S_2$ since
  \[
    S_1 = \sum_{i = 1}^n f_1(x_i') (x_i - x_{i - 1})
    \quad \text{and} \quad
    S_2 = \sum_{i = 1}^n f_2(x_i') (x_i - x_{i - 1}).
  \]
  So $S$ is in the interval $(S_1, S_2)$, which has
  length $< \epsilon$ by the triangle inequality on the
  previous results. For any two Riemann sums of
  $f$ with partitions of width $< \delta$, we have
  $|S' - S''| < \epsilon$. Thus $f$ is integrable.

  $(\Rightarrow)$ First we show that $f$ is bounded
  in $[a, b]$. This is because for any $\epsilon > 0$,
  there exists $\delta > 0$ such that any two Riemann
  sums $S_1, S_2$ corresponding to partitions of
  width $< \delta$.
  satisfy $|S_1 - S_2| < \epsilon$. Let
  \[
    S_1 = \sum_{i = 1}^n f(x_i') (x_i - x_{i - 1}),
  \]
  and replace $x_{i_0}' \in (x_{i_0 - 1}, x_{i_0})$
  with $x_{i_0}'' \in (x_{i_0 - 1}, x_{i_0})$. Keep
  $x_i'$ for $i \ne i_0$. Define this new Riemann sum
  to be $S_2$. Then
  \[
    |S_2 - S_1| \le |f(x_{i_0}'') - f(x_{i_0}')| |x_{i_0} - x_{i_0 - 1}|
    < \epsilon,
  \]
  so that
  \[
    |f(x_{i_0}'')| \le |f(x_{i_0}')| + \frac{\epsilon}{x_{i_0} - x_{i_0 - 1}},
  \]
  i.e. $f$ is bounded in $(x_{i_0 - 1}, x_{i_0})$ since
  $x_{i_0}''$ was arbitrary. Since we also picked $i_0$
  arbitrarily, we can repeat this for any
  interval to conclude that $f$ is bounded in $[a, b]$.

  Now for any partition $x_0 < x_1 < \dots < x_n$
  with width $< \delta$, define
  \[
    m_i = \inf \{f(x) : x \in (x_{i - 1}, x_i)\}
    \quad \text{and} \quad
    M_i = \sup \{f(x) : x \in (x_{i - 1}, x_i)\}.
  \]
  Define the step function
  \[
    f_1(x) =
    \begin{cases}
      m_i & \text{if } x \in (x_{i - 1}, x_i) \\
      \min\{m_1, \dots, m_n\} & \text{if } x = x_i \text{ for } i = 0, \dots, n.
    \end{cases}
  \]
  Similarly define
  \[
    f_2(x) =
    \begin{cases}
      M_i & \text{if } x \in (x_{i - 1}, x_i) \\
      \max\{M_1, \dots, M_n\} & \text{if } x = x_i \text{ for } i = 0, \dots, n.
    \end{cases}
  \]
  Observe that $f_1(x) \le f(x) \le f_2(x)$ for
  any $x \in [a, b]$ by construction. Now we verify
  that
  \[
    \int_a^b (f_2 - f_1)\, dx < \epsilon
  \]
  if $\delta > 0$ is small enough. This is because
  for any $\eta > 0$, there exists $x_i', x_i'' \in [x_{i - 1}, x_i]$
  such that $f(x_i') < m_i + \eta$ and
  $f(x_i'') > M_i - \eta$. Then
  \[
    \sum_{i = 1}^n (f(x_i'') - f(x_i'))(x_i - x_{i - 1})
    > \sum_{i = 1}^n (M_i - m_i - 2\eta)(x_i - x_{i - 1})
    = \int_a^b (f_2 - f_1)\, dx - 2\eta(b - a).
  \]
  If $\delta > 0$ is small enough, then
  \[
    \sum_{i = 1}^n (f(x_i'') - f(x_i'))(x_i - x_{i - 1})
    < \epsilon
  \]
  since this a difference of two Riemann sums with
  partitions of width $< \delta$. Thus
  \[
    \int_a^b (f_2 - f_1)\, dx < \epsilon + 2\eta(b - a).
  \]
  But $\eta$ was arbitrary, so taking $\eta \to 0$
  gives the desired result.
\end{proof}

\begin{corollary}
  If $f : [a, b] \to \R$ is integrable, then it is
  bounded.
\end{corollary}

\begin{proof}
  This was shown in the proof of the previous proposition.
\end{proof}

\begin{theorem}
  If $f : [a, b] \to \R$ is continuous, then $f$ is
  integrable.
\end{theorem}

\begin{proof}
  Since $f$ is continuous on the compact set $[a, b]$,
  it is uniformly continuous. So for any $\epsilon > 0$,
  there exists $\delta > 0$ such that for any
  $x', x'' \in [a, b]$, we have
  $|f(x') - f(x'')| < \epsilon$ whenever
  $|x' - x''| < \delta$. Now let $S_1, S_2$
  be two Riemann sums
  with partitions of width $< \delta$. Assume without
  loss of generality that $S_1, S_2$ are defined
  over the same partition (we can always combine two
  partitions to give a finer partition, if necessary).
  Let
  \[
    S_1 = \sum_{i = 1}^n f(x_i') (x_i - x_{i - 1})
    \quad \text{and} \quad
    S_2 = \sum_{i = 1}^n f(x_i'') (x_i - x_{i - 1}).
  \]
  Then
  \[
    |S_1 - S_2| \le \sum_{i = 1}^n |f(x_i') - f(x_i'')| (x_i - x_{i - 1})
    < \epsilon \sum_{i = 1}^n (x_i - x_{i - 1})
    = \epsilon (b - a).
  \]
  Since $\epsilon > 0$ was arbitrary, we conclude that
  that $f$ is integrable by Lemma \ref{lem:first-integrable}.
\end{proof}

\section{The Fundamental Theorem of Calculus}
\begin{theorem}[Fundamental theorem of calculus]
  If $f : [a, b] \to \R$ has anti-derivative
  $F : [a, b] \to \R$ and $f \in \mathcal{R}([a, b])$,\footnote{Here $\mathcal{R}([a, b])$ is the class of Riemann integrable functions on $[a, b]$.}
  then
  \[
    \int_a^b f(x)\, dx = F(b) - F(a).
  \]
\end{theorem}

\begin{proof}
  Since $f$ is integrable, let
  \[
    A = \int_a^b f(x)\, dx.
  \]
  For any $\epsilon > 0$, there exists $\delta > 0$
  such that for any Riemann sum $S$ with partition of
  width $< \delta$, we have $|S - A| < \epsilon$. Let
  $x_0 < x_1 < \dots < x_n$ be a partition of
  width $< \delta$. Then by telescoping,
  \[
    F(b) - F(a) = \sum_{i = 1}^n (F(x_i) - F(x_{i - 1}))
    = \sum_{i = 1}^n f(x_i') (x_i - x_{i - 1})
  \]
  by Lagrange's mean value theorem, where
  $x_i' \in (x_{i - 1}, x_i)$. Then
  \[
    |F(b) - F(a) - A| = |S - A| < \epsilon,
  \]
  so letting $\epsilon \to 0$ gives $F(b) - F(a) = A$.
\end{proof}

\begin{remark}
  The fundamental theorem of calculus requires both
  being Riemann integrable and having an anti-derivative,
  which do not always overlap. In fact, neither is
  a subset of the other.
\end{remark}

\begin{example}
  The step function
  \[
    f(x) =
    \begin{cases}
      -1 & \text{if } 0 \le x \le 1 \\
      1 & \text{if } 1 < x \le 2
    \end{cases}
  \]
  is integrable but has no anti-derivative.
\end{example}

\begin{example}
  Define
  \[
    F(x) =
    \begin{cases}
      0 & \text{if } x = 0 \\
      x^2 \sin(1/x) & \text{if } x \ne 0.
    \end{cases}
  \]
  Then we have
  \[
    F'(x) = f(x) =
    \begin{cases}
      0 & \text{if } x = 0 \\
      (-2 / x)(\cos(1 / x^2)) + 2x \sin(1 / x^2) & \text{if } x \ne 0.
    \end{cases}
  \]
  We can check that $F'(0) = 0$ via the definition
  of the derivative. Note that $f$ has an anti-derivative,
  namely $F$. However, $f$ is not integrable since
  it is not bounded near $x = 0$.
\end{example}

  \chapter{Jan.~30 --- More Integrability}

\section{Conditions for an Anti-Derivative}
\begin{lemma}
  \label{lem:split-integral}
  Let $c \in (a, b)$. Then $f \in \mathcal{R}([a, b])$ if and only if $f \in \mathcal{R}([a, c])$ and $f \in \mathcal{R}([c, b])$. Moreover,
  \[
    \int_a^b f(x) \, dx = \int_a^c f(x) \, dx + \int_c^b f(x) \, dx. \tag{$*$}
  \]
\end{lemma}

\begin{proof}
  $(\Rightarrow)$ If $f \in \mathcal{R}([a, b])$, then
  for any $\epsilon > 0$, there exist two step
  functions $f_1, f_2$ such that $f_1 \le f \le f_2$
  and
  \[
    \int_a^b (f_2 - f_1) \, dx < \epsilon.
  \]
  Let $f_1, f_2$ be the restrictions to $[a, c]$.
  Then still $f_1 \le f \le f_2$ on $[a, c]$ and
  \[
    \int_a^c (f_2 - f_1)\, \le
    \int_a^b (f_2 - f_1) < \epsilon
  \]
  since $f_2 - f_1$ is a nonnegative step function. (Note
  that the desired result is easy to verify for step
  functions.)
  So $f \in \mathcal{R}([a, c])$, and the same
  argument works to show that $f \in \mathcal{R}([c, b])$.

  $(\Leftarrow)$ If $f \in \mathcal{R}([a, c])$ and
  $f \in \mathcal{R}([c, b])$, then for any
  $\epsilon > 0$, there exist step functions
  $g_1, g_2, h_1, h_2$ such that
  $g_1 \le f \le g_2$ on $[a, c]$,
  $h_1 \le f \le h_2$ on $[c, b]$, and
  \[
    \int_a^c (g_2 - g_1)\, dx < \epsilon,
    \quad
    \int_c^b (h_2 - h_1)\, dx < \epsilon.
  \]
  Now define
  \[
    f_i = \begin{cases}
      g_i & \text{if } x \in [a, c) \\
      h_i & \text{if } x \in [c, b]
    \end{cases}
  \]
  for $i = 1, 2$. Then $f_1 \le f \le f_2$ on $[a, b]$,
  and
  \[
    \int_a^b (f_2 - f_1)\, dx
    = \int_a^c (g_2 - g_1)\, dx + \int_c^b (h_2 - h_1)\, dx
    < 2\epsilon,
  \]
  so $f \in \mathcal{R}([a, b])$. Now to prove $(*)$,
  note that $f \in \mathcal{R}([a, c])$, so for
  any $\epsilon > 0$ there exist Riemann sums $S_1$
  on $[a, c]$ and $S_2$ on $[c, b]$ such that
  \[
    |S_1 - \int_a^c f(x)\, dx| < \frac{\epsilon}{3},
    \quad
    |S_2 - \int_c^b f(x)\, dx| < \frac{\epsilon}{3}.
  \]
  Now choose $\delta > 0$ such that if the Riemann
  sum $S$ has partition with
  width $< \delta$, then
  \[
    |S - \int_a^c f(x)\, dx| < \frac{\epsilon}{3},
    \quad
    |S - \int_c^b f(x)\, dx| < \frac{\epsilon}{3},
    \quad
    |S - \int_a^b f(x)\, dx| < \frac{\epsilon}{3}.
  \]
  Now combine $S_1, S_2$ on $[a, b]$ to be a Riemann
  sum $S = S_1 + S_2$, so that
  \[
    |S - \int_a^b f(x)\, dx| < \frac{\epsilon}{3}.
  \]
  By the triangle inequality on the previous results,
  \[
    \left| \int_a^b f(x)\, dx - \left(\int_a^c f(x)\, dx + \int_c^b f(x)\, dx\right)\right| < \epsilon.
  \]
  Since $\epsilon$ is arbitrarily small, we conclude that
  \[
    \int_a^b f(x)\, dx = \int_a^c f(x)\, dx + \int_c^b f(x)\, dx
  \]
  as desired.
\end{proof}

\begin{remark}
  The formula $(*)$ is true for any three numbers
  $a, b, c$, as long as $f$ is integrable. This is
  because by convention, if $a > b$, then
  \[
    \int_a^b f(x)\, dx = -\int_b^a f(x)\, dx.
  \]
\end{remark}

\begin{theorem}
  If $f : [a, b] \to \R$ is continuous, then\footnote{Note that this integral is well-defined since any continuous function is integrable, and a continuous function restricted to a subset of its domain, i.e. $[a, x] \subseteq [a, b]$, remains continuous.}
  \[
    F(x) = \int_a^x f(\xi) \, d\xi
  \]
  is an anti-derivative of $f$.
\end{theorem}

\begin{proof}
  For any $x_0 \in (a, b)$, we check that
  $F'(x_0) = f(x_0)$. We can compute using Lemma
  \ref{lem:split-integral} that
  \begin{align*}
    \left| \frac{F(x_0 + h) - F(x_0)}{h} - f(x_0) \right|
    &= \left| \frac{1}{h} \left( \int_a^{x_0 + h} f(x)\, dx - \int_a^{x_0} f(x)\, dx \right) - f(x_0) \right| \\
    &= \left| \frac{1}{h} \int_{x_0}^{x_0 + h} f(x)\, dx - f(x_0) \right|
    = \left| \frac{1}{h} \int_{x_0}^{x_0 + h} (f(x) - f(x_0))\, dx \right|.
  \end{align*}
  The last step is from observing
  \[
    f(x_0) = \frac{1}{h} \int_{x_0}^{x_0 + h} f(x_0)\, dx.
  \]
  Since $f$ is continuous, for any $\epsilon > 0$, there
  exists $\delta$
  such that if $|x_0 - x| < \delta$, then
  $|f(x) - f(x_0)| < \epsilon$. This gives
  \[
    \left| \frac{1}{h} \int_{x_0}^{x_0 + h} (f(x) - f(x_0))\, dx \right|
    \le \frac{1}{h} \int_{x_0}^{x_0 + h} |f(x) - f(x_0)|\, dx
    \le \frac{\epsilon h}{h} = \epsilon
  \]
  if $|h| < \delta$. Thus,
  \[
    \lim_{h \to 0} \frac{F(x_0 + h) - F(x_0)}{h} - f(x_0) = f(x_0),
  \]
  so we indeed have $F'(x_0) = f(x_0)$.
\end{proof}

\section{More Conditions for Integrability}
\begin{definition}
  Let $f : [a, b] \to \R$ be bounded and
  $x_0 < x_1 < \cdots < x_n$ be a partition of $[a, b]$.
  Define
  \[
    \omega_i = \sup\{|f(x) - f(y)| : x, y \in [x_{i - 1}, x_i)\}
  \]
  for $i = 1, 2, \dots, n$.
\end{definition}

\begin{theorem}
  A function $f : [a, b] \to \R$ is integrable if and only
  if for any $\epsilon > 0$, there exists $\delta > 0$
  such that for any partition with width $< \delta$,
  we have
  \[
    \sum_{i = 1}^n \omega_i \Delta x_i < \epsilon,
  \]
  where $\Delta x_i = x_i - x_{i - 1}$.
\end{theorem}

\begin{proof}
  $(\Leftarrow)$ For any $\epsilon > 0$, choose any
  two Riemann sums $S_1, S_2$ over partitions with
  width $< \delta$. Assume without loss of generality
  that $S_1$ and $S_2$ are defined over the same
  (maybe refined) partition. Let
  \[
    S_1 = \sum_{i = 1}^n f(x_i') (x_i - x_{i - 1}),
    \quad
    S_2 = \sum_{i = 1}^n f(x_i'') (x_i - x_{i - 1}).
  \]
  Then we have
  \[
    |S_1 - S_2| \le \sum_{i = 1}^n |f(x_i') - f(x_i'')| \Delta x_i
    \le \sum_{i = 1}^n \omega_i \Delta x_i < \epsilon.
  \]
  Then by Lemma \ref{lem:first-integrable}, we conclude
  that $f$ is integrable.

  $(\Rightarrow)$ Since $f$ is integrable, by Lemma
  \ref{lem:first-integrable} we have that for any
  $\epsilon > 0$, there eixsts $\delta > 0$ such that
  for any two Riemann sums $S_1, S_2$ over partitions
  of with $< \delta$, we have $|S_1 - S_2| < \epsilon$.
  In the interval $[x_{i - 1}, x_i]$, let
  \[
    M_i = \sup_{x \in [x_{i - 1}, x_i]} |f(x)|, \quad
    m_i = \inf_{x \in [x_{i - 1}, x_i]} |f(x)|.
  \]
  In particular note that $\omega_i = M_i - m_i$.
  Now for any $\eta > 0$, there exist
  $x_i', x_i'' \in [x_{i - 1}, x_i]$ such that
  \[
    f(x_i') > M_i - \eta, \quad f(x_i'') < m_i + \eta.
  \]
  Let
  \[
    S_1 = \sum_{i = 1}^n f(x_i') \Delta x_i,
    \quad
    S_2 = \sum_{i = 1}^n f(x_i'') \Delta x_i.
  \]
  Then we have
  \[
    |S_1 - S_2| \le \left| \sum_{i = 1}^n (f(x_i') - f(x_i'')) \Delta x_i \right|
  \]
  Note that $f(x_i') - f(x_i'') \ge M_i - m_i - 2\eta$
  for $\eta$ sufficiently small. Thus
  \[
    |S_1 - S_2| \ge \sum_{i = 1}^n \omega_i \Delta x_i
    - 2\eta \sum_{i = 1}^n \Delta x_i,
  \]
  so that
  \[
    \sum_{i = 1}^n \omega_i \Delta x_i
    \le |S_1 - S_2| + 2\eta (b - a)
    < \epsilon + 2\eta (b - a).
  \]
  From here letting $\eta \to 0$ gives the desired result.
\end{proof}

\begin{theorem}
  A function $f : [a, b] \to \R$ is integrable if and only
  if for any $\epsilon > 0$, there exists a partition
  such that
  \[
    \sum_{i = 1}^n \omega_i \Delta x_i < \epsilon.
  \]
\end{theorem}

\begin{proof}
  $(\Rightarrow)$ This is immediate from the
  previous theorem.

  $(\Leftarrow)$ Let $S_1$ be the given sum
  and
  \[
    S_2 = \sum_{i = 1}^n \omega_i \Delta x_i
  \]
  be any other Riemann sum over a partition of
  width $< \delta$. Then $S_2 \le 2S_1 < 2\epsilon$ at
  least since we will have
  $\omega_i' \le \omega_i + \omega_{i - 1}$ if $\omega_i'$
  is the analogous value corresponding to $S_2$.
\end{proof}

  \chapter{Feb.~1 --- Riemann Integrability, Part 3}

\section{Even More Conditions for Integrability}
\begin{example}
  If $f(x)$ is monotone on $[a, b]$, then
  $f \in \mathcal{R}([a, b])$.
\end{example}

\begin{proof}
  Suppose $f(x)$ is monotone increasing on
  $[a, b]$ and $f(x)$ is not constant (since the result
  is trivial if $f$ is constant). Then
  $f(a) \le f(x) \le f(b)$. For any $\epsilon > 0$,
  for any partition $x_0 < \dots < x_n$ with width
  \[
    \delta < \frac{\epsilon}{f(b) - f(a)},
  \]
  we have on $[x_{i - 1}, x_i]$ that
  $M_i = f(x_i)$ and $f(x_{i - 1}) = m_i$ since
  $f$ is monotone. Then
  \[
    \omega_i(f) = f(x_i) - f(x_{i - 1}) = M_i - m_i.
  \]
  Thus
  \[
    \sum_{i = 1}^n \omega_i(f) \Delta x_i
    = \sum_{i = 1}^n (f(x_i) - f(x_{i - 1})) \Delta x_i
    < \frac{\epsilon}{f(b) - f(a)} \sum_{i = 1}^n (f(x_i) - f(x_{i - 1})) = \epsilon
  \]
  since the sum telescopes and comes out to
  $f(b) - f(a)$. Thus $f$ is integrable.
\end{proof}

\begin{theorem}[Du Bois-Reymond]
  Let $f$ be bounded on $[a, b]$. Then $f \in \mathcal{R}([a, b])$
  if and only if for any $\epsilon, a > 0$,
  there exists a partition such that
  the total length of subintervals with
  $\{\omega_i(f) \le \epsilon\}$ is $< a$.
\end{theorem}

\begin{proof}
  For any partition $x_0 < \dots < x_n$, split
  \[
    \sum_{i = 1}^n \omega_i(f) \Delta x_i =
    \sum_{(A)} \omega_i(f) \Delta x_i +
    \sum_{(B)} \omega_i(f) \Delta x_i
  \]
  where $(A)$ is over subintervals with width
  $\omega_i(f) < \epsilon$ and $(B)$ is over
  subintervals with width $\omega_i(f) \ge \epsilon$.

  $(\Rightarrow)$ Let
  \[
    \Omega = \sup_{x, y \in [a, b]} |f(x) - f(y)|.
  \]
  For any $\epsilon > 0$, for
  \[
    \epsilon_1 = \frac{\epsilon}{2(b - a)} \quad \text{and} \quad a = \frac{\epsilon}{2\Omega},
  \]
  by assumption there exists a partition
  $x_0 < \dots < x_n$ such that
  \begin{align*}
    \sum_{i = 1}^n \omega_i(f) \Delta x_i
    &= \sum_{(A)} \omega_i(f) \Delta x_i +
    \sum_{(B)} \omega_i(f) \Delta x_i \\
    &< \frac{\epsilon}{2(b - a)} \sum_{(a)} \Delta x_i
    + \Omega \sum_{(B)} \Delta x_i
    < \frac{\epsilon}{2(b - a)}(b - a) + \Omega \frac{\epsilon}{2\Omega} = \epsilon.
  \end{align*}
  So we see that $f \in \mathcal{R}([a, b])$ as desired.

  $(\Rightarrow)$ If $f \in \mathcal{R}([a, b])$, then
  for any $\epsilon , a > 0$, there exists a partition
  $x_0 < \dots < x_n$ such that
  \[
    \sum_{i = 1}^n \omega_i(f) \Delta x_i < a\epsilon.
  \]
  Then we have
  \[
    \epsilon \sum_{(B)} \Delta x_i
    \le \sum_{(B)} \omega_i(f) \Delta x_i
    < a\epsilon
    \implies \sum_{(B)} \Delta x_i < a,
  \]
  which shows the desired result.
\end{proof}

\begin{corollary}
  If $f : [a, b] \to \R$ is bounded and has
  only finitely many discontinuity points, then
  $f \in \mathcal{R}([a, b])$.
\end{corollary}

\begin{proof}
  Suppose $f(x)$ has $p$ discontinuity points on
  $[a, b]$ and $m \le f(x) \le M$ for all $x \in [a, b]$.
  Then for any $\epsilon > 0$, first (1) we construct $p$
  small open intervals on $[a, b]$ containing the $p$
  discontinuity points with
  \[
    \text{total length} < \frac{\epsilon}{2(M - m)}.
  \]
  Next (2) for any subintervals in $[a, b]$ excluding the above
  $p$ subintervals, $f$ is continuous on them, so
  there exists a partition such that
  \[
    \sum_{(2)} \omega_i(f) \Delta x_i < \frac{\epsilon}{2}.
  \]
  Now combine (1) and (2) to get
  \[
    \sum_{i = 1}^n \omega_i(f) \Delta
    = \sum_{(1)} \omega_i(f) \Delta x_i
    + \sum_{(2)} \omega_i(f) \Delta x_i
    < (M - m) \frac{\epsilon}{2(M - m)} + \frac{\epsilon}{2} = \epsilon.
  \]
  Thus $f \in \mathcal{R}([a, b])$, as desired.
\end{proof}

\begin{example}
  Consider
  \[
    f(x) =
    \begin{cases}
      \sin(1 / x) & \text{if } x \ne 0 \\
      A & \text{if } x = 0
    \end{cases}
  \]
  for any constant $A \in \R$. Then by the previous
  corollary, $f \in \mathcal{R}([0, 1])$.
\end{example}

\begin{theorem}
  If $f, g \in \mathcal{R}([a, b])$, then
  $fg \in \mathcal{R}([a, b])$.
\end{theorem}

\begin{proof}
  Since $f, g$ are integrable, they are bounded.
  So assume $|f|, |g| \le M$. Then for any $\epsilon > 0$,
  there exists $\delta > 0$ such that for any
  partition of width $< \delta$, we have
  \[
    \sum_{i = 1}^n \omega_i(f) \Delta x_i <
    \frac{\epsilon}{2M}, \quad
    \sum_{i = 1}^n \omega_i(g) \Delta x_i <
    \frac{\epsilon}{2M}.
  \]
  Notice
  \[
    \omega_i(fg) \le M(\omega_i(f) + \omega_i(g))
  \]
  because
  \begin{align*}
    |f(x) g(x) - f(y) g(y)
    &\le |g(x)| |f(x) - f(y)| + |f(y)| |g(x) - g(y)| \\
    &\le M(|f(x) - f(y)| + |g(x) - g(y)|).
  \end{align*}
  Taking suprememes over $x, y \in [x_{i - 1}, x_i]$ from
  here
  gives $\omega_i(fg) \le M(\omega_i(f) + \omega_i(g))$.
  Then
  \[
    \sum_{i = 1}^n \omega_i(fg) \Delta x_i
    \le M \left(\sum_{i = 1}^n \omega_i(f) \Delta x_i + \sum_{i = 1}^n \omega_i(g) \Delta x_i\right)
    < M\left(\frac{\epsilon}{2M} + \frac{\epsilon}{2M}\right) = \epsilon.
  \]
  Thus $fg \in \mathcal{R}([a, b])$ as desired.
\end{proof}

\begin{theorem}
  If $f \in \mathcal{R}([a, b])$, then
  $|f| \in \mathcal{R}([a, b])$ and
  \[
    \left| \int_a^b f(x) \, dx \right| \le \int_a^b |f(x)| \, dx.
  \]
\end{theorem}

\begin{proof}
  Since $f \in \mathcal{R}([a, b])$, for any
  $\epsilon > 0$ there exists a partition
  $x_0 < \dots < x_n$ such that
  \[
    \sum_{i = 1}^n \omega_i(f) \Delta x_i < \epsilon.
  \]
  Since
  \[
    \big||f(x)| - |f(y)|\big| \le |f(x) - f(y)|,
  \]
  taking supremums over $x, y \in [x_{i - 1}, x_i]$
  gives $\omega_i(|f|) \le \omega_i(f)$. Then
  \[
    \sum_{i = 1}^n \omega_i(|f|) \Delta x_i
    \le \sum_{i = 1}^n \omega_i(f) \Delta x_i < \epsilon.
  \]
  So we indeed have $|f| \in \mathcal{R}([a, b])$.
  Now observe that $-|f| \le f \le |f|$. After
  integrating, we get
  \[
    -\int_a^b |f(x)| \, dx \le \int_a^b f(x) \, dx \le \int_a^b |f(x)| \, dx.
  \]
  This immediately implies the desired result.
\end{proof}

\begin{example}[Cauchy-Schwarz]
  If $f, g \in \mathcal{R}([a, b])$, then
  \[
    \left| \int_a^b f(x) g(x) \, dx \right|
    \le \left( \int_a^b f(x)^2 \, dx \right)^{1/2}
    \left( \int_a^b g(x)^2 \, dx \right)^{1/2}. \tag{$*$}
  \]
\end{example}

\begin{proof}
  Let
  \[
    A = \int_a^b f^2 \, dx, \quad
    B = \int_a^b |fg| \, dx,\quad
    C = \int_a^b g^2 \, dx.
  \]
  Note that it suffices to
  show that $B^2 \le AC$, which will imply $(*)$
  by the previous theorem. Then
  \[
    0 \le \int_a^b (t|f| - |g|)^2\, dx
    = A t^2 - 2Bt + C
  \]
  for any $t \in \R$. So the discriminant must satisfy
  $(2B)^2 - 4AC \le 0$, which gives $B^2 \le AC$
  as desired.
\end{proof}

\begin{example}[Riemann-Lebesgue lemma]
  If $f \in \mathcal{R}([a, b])$, then
  \[
    \lim_{\lambda \to \infty} \int_a^b f(x) \sin(\lambda x) \, dx = 0.
  \]
\end{example}

\begin{proof}
  Since $f \in \mathcal{R}([a, b])$, for any
  $\epsilon > 0$ there exists a partition
  $x_0 < \dots < x_n$ of $[a, b]$ such that
  \[
    \sum_{i = 1}^n \omega_i(f) \Delta x_i < \frac{\epsilon}{2}.
  \]
  Also assume $|f| \le M$ on $[a, b]$ since $f$ is
  integrable. Then we choose
  \[
    \lambda > \frac{4nM}{\epsilon}.
  \]
  We can estimate
  \begin{align*}
    \left| \int_a^b f(x) \sin(\lambda x) \, dx \right|
    &= \left|\sum_{i = 1}^n \int_{x_{i - 1}}^{x_i} (f(x) - f(x_{i}) + f(x_i)) \sin(\lambda x)\, dx\right| \\
    &\le
    \sum_{i = 1}^n |f(x_i)| \left|\int_{x_{i - 1}}^{x_i} \sin(\lambda x)\, dx\right|
    + \sum_{i = 1}^n \int_{x_{i - 1}}^{x_i} \underbrace{|f(x) - f(x_i)|}_{\le \omega_i(f)} \underbrace{|\sin(\lambda x)|}_{\le 1}\, dx \\
    &\le
    M \sum_{i = 1}^n \frac{\overbrace{|\cos(\lambda x_{i}) - \cos(\lambda x_{i - 1})|}^{\le 2}}{\lambda}
    + \sum_{i = 1}^n \int_{x_{i - 1}}^{x_i} \omega_i(f)\, dx \\
    &\le M \frac{2n}{\lambda} + \sum_{i = 1}^n \omega_i(f) \Delta x_i < \frac{\epsilon}{2} + \frac{\epsilon}{2} = \epsilon.
  \end{align*}
  So as $\lambda \to \infty$, the integral goes to $0$.
\end{proof}

\begin{remark}
  Recall that
  \[
    f(x) =
    \begin{cases}
      0 & \text{if } x \text{ is irrational} \\
      1 & \text{if } x \text{ is rational}
    \end{cases}
  \]
  is not Riemann integrable, but we might expect that
  this
  should integrate to $0$. The Lebesgue integral
  will fix this, which was discovered much later.
\end{remark}

  \chapter{Feb.~6 --- Exchange of Limit Operations}

\section{Motivation}
If we have a sequence of functions $\{f_n\}$ where
$f_n \to f$ pointwise, then does
\[
  \int_a^b f_n\, dx \to \int_a^b f\, dx
\]
if each $f_n$ is integrable? Does
$f_n' \to f'$ if $f_n$ is differentiable?

\begin{example}
  Define
  \[
    f_n(x) =
    \begin{cases}
      4n^2 x & \text{if } x \in [0, 1 / 2n] \\
      4n - 4n^2 x & \text{if } x \in (1 / 2n, 1 / n) \\
      0 & \text{if } x \in [1 / n, 1],
    \end{cases}
  \]
  where the graph of $f_n$ looks like a triangle with
  peak at $x = 1 / 2n$ and height $2n$.
  When we let $n \to \infty$, we see that for any
  $x \in [0, 1]$, we have $f_n(x) \to 0$. But
  \[
    \int_0^1 f_n(x)\, dx = \text{area of triangle}
    = \frac{1}{2} (2n) \cdot \frac{1}{n} = 1.
  \]
  So we see that in this case,
  \[
    \lim_{n \to \infty} \int_0^1 f_n \, dx \ne \int_0^1 \lim_{n \to \infty} f_n \, dx.
  \]
\end{example}

\section{Exchange of the Limit and Integral}

\begin{theorem}
  Let $f_1, \dots, f_n, \dots$ be a uniformly
  convergent sequence of continuous functions
  on $[a, b]$. Then
  \[
    \int_a^b \lim_{n \to \infty} f_n(x)\, dx
    = \lim_{n \to \infty} \int_a^b f_n(x)\, dx.
  \]
\end{theorem}

\begin{proof}
  Suppose that $f_n \to f$ uniformly.
  By definition of uniform convergence, for any
  $\epsilon > 0$ there exists $N$ such that if
  $n \ge N$, then
  \[
    \max_{x \in [a, b]}|f_n(x) - f(x)| < \frac{\epsilon}{b - a}
  \]
  Each $f_n \to f$ uniformly and each $f_n$ is continuous,
  $f$ is also continuous. In particular, $f$ is
  integrable and
  \[
    -\frac{\epsilon}{b - a} < f_n(x) - f(x) < \frac{\epsilon}{b - a},
  \]
  so integrating on both sides gives
  \[
    -\epsilon < \int_a^b f_n(x)\, dx - \int_a^b f(x)\, dx < \epsilon \implies
    \left| \int_a^b f_n(x)\, dx - \int_a^b f(x)\, dx \right| < \epsilon.
  \]
  Then this implies
  \[
    \lim_{n \to \infty} \int_a^b f_n(x)\, dx = \int_a^b f(x)\, dx,
  \]
  as desired.
\end{proof}

\begin{remark}
  The previous theorem still holds even if each $f_n$
  is only Riemann integrable. The only thing we need to
  check is that the limit function $f$ is
  also Riemann integrable. This is because for
  any $\epsilon > 0$, if $n$ is large enough,
  \[
    -\frac{\epsilon}{3(b - a)} + f_n(x) \le f(n) \le f_n(x) + \frac{\epsilon}{3(b - a)}.
  \]
  Since $f_n \in \mathcal{R}([a, b])$, there exist
  two step functions $g_1, g_2$ satisfying
  $g_1 \le f_n \le g_2$, and
  \[
    \int_a^b (g_2 - g_1) < \frac{\epsilon}{3}.
  \]
  Now note that
  \[
    g_1(x) - \frac{\epsilon}{3(b - a)} \le f(x) \le g_2(x) + \frac{\epsilon}{3(b - a)},
  \]
  so we see
  \[
    \int_a^b \left[\left(g_2(x) + \frac{\epsilon}{3(b - a)}\right) - \left(g_1(x) - \frac{\epsilon}{3(b - a)}\right)\right] \, dx
    = \frac{\epsilon}{3} + \frac{2\epsilon}{3} = \epsilon.
  \]
  This gives $f \in \mathcal{R}([a, b])$, so we
  can carry through the rest of the previous proof.
\end{remark}

\section{Exchange of the Limit and Derivative}
\begin{theorem}
  Let $f_1, \dots, f_n, \dots$ be a sequence of functions
  on an open interval $U$ in $\R$ and that each $f_n$ has
  a continuous derivative. Suppose $\{f_n'\}$ converges
  uniformly on $U$ and for some $a \in U$,
  $\{f_n'(a)\}$ converges. Then
  \[
    \lim_{n \to \infty} f_n(x) = f(x)
  \]
  exists and $f(x)$ is differentiable. Furthermore,
  we have
  \[
    f' = \lim_{n \to \infty} f_n'.
  \]
\end{theorem}

\begin{proof}
  By the fundamental theorem of calculus, we have
  \[
    \int_a^x f_n'(t)\, dt = f_n(x) - f_n(a). \tag{$*$}
  \]
  Let $\lim_{n \to \infty} f_n' = g$, where $g$ is
  continuous since $f_n' \to g$ uniformly and
  each $f_n'$ is continuous. Then take $n \to \infty$
  in $(*)$, where
  \[
    \text{LHS} \to \int_a^x g(t)\, dt.
  \]
  Let $\lim_{n \to \infty} f_n(x) = f(x)$, which
  exists by $(*)$. Then
  $\text{RHS} \to f(x) - f(a)$,
  so we see that
  \[
    f(x) - f(a) = \int_a^x g(t)\, dt.
  \]
  Then $f$ is an anti-derivative of $g$, or in other
  words, $f' = g$ as desired.
\end{proof}

\section{Infinite Series}
\begin{definition}
Suppose we have a sequence of numbers
$a_1, a_2, a_3, \dots, a_n, \dots$. Then
\[
  a_1 + a_2 + a_3 + \dots + a_n + \dots
  = \sum_{n = 1}^\infty a_n
\]
is called an \emph{infinite series}. We say the
infinite series \emph{converges} to $A$ if
the \emph{partial sums}
\[
  S_m = \sum_{n = 1}^m a_m
\]
converge to $A$ as $m \to \infty$.
\end{definition}

\begin{example}[Geometric series]
  For a fixed $a$, the series
  \[
    \sum_{n = 0}^\infty a^n = 1 + a + a^2 + \dots + a^n + \dots
  \]
  converges if and only
  if $|a| < 1$, and the limit is $1 / (1 - a)$. This is
  because
  \[
    S_m = 1 + a + \dots + a^m
    = \frac{1 - a^{m + 1}}{1 - a}.
  \]
  If $|a| < 1$, then $a^{m + 1} \to 0$ as $m \to \infty$,
  so $S_m \to 1 / (1 - a)$. On the other hand,
  if $|a| > 1$, then $|a^{m + 1}| \to \infty$ as $m \to \infty$.
  If $a = 1$, then
  \[
    S_m = 1 + 1 + \dots + 1 = m,
  \]
  so $S_m \to \infty$. If $a = -1$, then
  \[
    \sum_{n = 0}^\infty (-1)^n = 1 - 1 + 1 - 1 + \dots,
  \]
  which diverges since its partial sums
  oscillate. So the condition
  is indeed necessary and sufficient.
\end{example}

\begin{prop}
  A series
  $\sum_{n = 1}^\infty a_n$
  converges if and only if for every $\epsilon > 0$,
  there exists integer $N$ such that if $n > m \ge N$,
  then
  \[|a_{m + 1} + a_{m + 2} + \dots + a_n| < \epsilon.\]
\end{prop}

\begin{proof}
  Let $S_m = \sum_{n = 1}^m a_n$ be the partial
  sums. Then $\sum_{n = 1}^\infty a_n$ converges
  if and only if $\{S_m\}$ is Cauchy. This is equivalent
  to say that for
  all $\epsilon > 0$, there exists $N$ such that if
  $n > m \ge N$, then
  \[
    |a_{m + 1} + a_{m + 2} + \dots + a_n|
    = |S_n - S_m| < \epsilon.
  \]
  This is precisely the desired result.
\end{proof}

\begin{corollary}
  If $\sum_{n = 1}^\infty a_n$ converges, then
  $a_n \to 0$ as $n \to \infty$.
\end{corollary}

\begin{proof}
  Take $m = n - 1$ in the previous proposition, which
  gives $|a_n| < \epsilon$ for $n \ge N + 1$.
\end{proof}

\begin{corollary}
  If $\sum_{n = 1}^\infty a_n$ and $\sum_{n = 1}^\infty b_n$
  differs in only finitely many terms, then the
  two series have the same convergence properties.
\end{corollary}

\begin{proof}
  Simply take $N$ larger than the last spot where
  the two series differ. Then the difference of
  partial sums in the previous proposition are the same
  for both series.
\end{proof}

\begin{example}[Harmonic series]
  The series
  \[
    \sum_{n = 1}^\infty \frac{1}{n} = 1 + \frac{1}{2} + \frac{1}{3} + \dots + \frac{1}{n} + \dots.
  \]
  diverges. Two see this, choose $n = 2m$ in the
  previous proposition and
  \[
    a_{m + 1} + a_{m + 2} + \dots + a_{2m}
    = \frac{1}{m + 1} + \frac{1}{m + 2} + \dots + \frac{1}{2m}
    \ge \frac{1}{2m} m = \frac{1}{2}.
  \]
  So the series must diverge.
\end{example}

\begin{prop}
  If $a_n \ge 0$, then
  $\sum_{n = 1}^\infty a_n$ either converges or
  has arbitrarily large partial sums, i.e. diverges to $\infty$.
\end{prop}

\begin{proof}
  Let $S_m = \sum_{n = 1}^m a_n$. Since $a_n \ge 0$,
  we see that $S_m$ is an increasing nonnegative sequence.
  Then by the monotone convergence theorem,
  $\{S_m\}$ converges if and only if it is bounded above.
\end{proof}

\begin{prop}[Comparison test]
  If $\sum_{n = 1}^\infty a_n$ and $\sum_{n = 1}^\infty b_n$
  are two infinite series such that
  $|a_n| \le b_n$ and $\sum_{n = 1}^\infty b_n$ converges,
  then $\sum_{n = 1}^\infty a_n$ converges and
  \[
    \left| \sum_{n = 1}^\infty a_n \right| \le \sum_{n = 1}^\infty b_n.
  \]
\end{prop}

\begin{proof}
  If $\sum_{n = 1}^\infty b_n$ converges, then
  for any $\epsilon > 0$, there exists $N$ such that
  if $n > m \ge N$, we have
  \[
    b_{m + 1} + b_{m + 2} + \dots + b_n < \epsilon.
  \]
  Then by the triangle inequality, we have
  \[
    |a_{m + 1} + a_{m + 2} + \dots + a_n|
    \le |a_{m + 1}| + |a_{m + 2}| + \dots + |a_n|
    \le b_{m + 1} + b_{m + 2} + \dots + b_n < \epsilon.
  \]
  Thus $\sum_{n = 1}^\infty a_n$ also converges.
  The last part is left as an exercise.
\end{proof}

\begin{example}[$p$-series]
  The series
  \[
    \sum_{n = 1}^\infty \frac{1}{n^p}
  \]
  converges if and only if $p > 1$.
\end{example}

\begin{prop}[Ratio test]
  If $\sum_{n = 1}^\infty a_n$ is a nonzero infinite
  series and
  there exists $\rho < 1$ such that
  \[
    \left| \frac{a_{n + 1}}{a_n} \right| \le \rho
  \]
  for all $n$ sufficiently large, then
  $\sum_{n = 1}^\infty a_n$ converges. If
  \[
    \left| \frac{a_{n + 1}}{a_n} \right| \ge 1
  \]
  for all $n$ large enough, then the series diverges.
\end{prop}

\begin{proof}
  First we show the second part. If
  $|a_{n + 1}| \ge |a_n|$ for $n \ge N$, then
  \[|a_n| \ge |a_{n - 1}| \ge \dots \ge |a_N|.\]
  Then $\{a_n\}$ does not converge to $0$, so
  $\sum_{n = 1}^\infty a_n$ diverges.
  First part left for next class.
\end{proof}

  \chapter{Feb.~8 --- Infinite Series}

\section{Lots of Convergence Tests}
\begin{theorem}[Comparison test, second version]
  Let $\sum_{n = 1}^\infty a_n$ and $\sum_{n = 1}^\infty b_n$ be two infinite series satisfying
  $0 \le a_n \le b_n$. Then
  \begin{enumerate}
    \item If $\sum_{n = 1}^\infty b_n$ converges, then $\sum_{n = 1}^\infty a_n$ converges.
    \item If $\sum_{n = 1}^\infty a_n$ diverges,
      then $\sum_{n = 1}^\infty b_n$ diverges.
  \end{enumerate}
\end{theorem}

\begin{proof}
  (1) Let $A_n$ and $B_n$ be the partial sums of
  $\sum_{n = 1}^\infty a_n$ and $\sum_{n = 1}^\infty b_n$, respectively.
  Then $B_n$ is bounded above since $\sum_{n = 1}^\infty b_n$ converges.
  But $A_n \le B_n$ since $0 \le a_n \le b_n$,
  so $A_n$ is also bounded above.
  Now note that $A_n$ is increasing since $a_n \ge 0$,
  so by the monotone convergence theorem,
  $A_n$ must converge.

  (2) Since $A_n$ is increasing,
  $\sum_{n = 1}^\infty a_n$ must diverge to $\infty$,
  i.e. $A_n$ is unbounded. But $A_n \le B_n$,
  so $B_n$ is also unbounded and thus we see that
  $\sum_{n = 1}^\infty b_n$ diverges.
\end{proof}

\begin{remark}
  In the above theorem, (1) remains true if
  \begin{itemize}
    \item $0 \le a_n \le b_n$ when $n \ge n_0$
    \item $0 \le a_n \le Mb_n$ for some $M > 0$,
    \item or there exists $0 < d_n < M$ such that
      $0 \le a_n \le d_n b_n$.
  \end{itemize}
\end{remark}

\begin{corollary}
  If $a_n, b_n > 0$ and
  \[
    \frac{a_{n + 1}}{a_n} \le \frac{b_{n + 1}}{b_n},
  \]
  then $\sum_{n = 1}^\infty b_n$ converges implies
  that $\sum_{n = 1}^\infty a_n$ converges.
\end{corollary}

\begin{proof}
  Let $d_n = a_n / b_n > 0$. Then
  \[
    d_{n + 1} = \frac{a_{n + 1}}{b_{n + 1}} \le \frac{a_n}{b_n} = d_n,
  \]
  which we can extended to $d_n \le \dots \le d_1$.
  Then $\{d_n\}$ is a bounded sequence, so
  $a_n = b_n d_n$, which implies the desired conclusion
  by the above remark.
\end{proof}

\begin{remark}
  If $n$ large, $e^{a n} \gg n^b \gg (\ln n)^c$
  for any $a, b, c > 0$. In particular,
  $e^{an} / n^b \to \infty$ when $n \to \infty$.
\end{remark}

\begin{example}
  Determine the convergence of
  \begin{enumerate}
    \item $\displaystyle \sum_{n = 2}^\infty \frac{1}{(\ln n)^p}$
      for $p > 0$,
    \item $\displaystyle \sum_{n = 2}^\infty \frac{\ln(n!)}{n^p}$ for $p > 0$,
    \item and $\displaystyle \sum_{n = 1}^\infty \frac{n^{n - 2}}{e^n n!}$.
  \end{enumerate}
\end{example}

\begin{proof}
  (1) We have
  \[
    \frac{1}{(\ln n)^p} > \frac{1}{n}
  \]
  for $n$ large, so the sum diverges by comparison to
  the harmonic series.

  (2) Note that
  \[
    \ln(n!) = \sum_{k = 1}^n \ln k > \frac{n \ln 2}{2},
  \]
  so we have
  \[
    \frac{\ln(n!)}{n^p} > \frac{\ln 2}{2} \frac{1}{n^{p - 1}}.
  \]
  By comparing to the $p$-series, we see that the
  series diverges when $p \le 2$. Also we have
  \[
    \frac{\ln(n!)}{n^p} < \frac{n \ln n}{n^p}
    = \frac{\ln n}{n^{p - 1}},
  \]
  so when $p > 2$, we get convergence.

  (3) Let $a_n = n^{n - 2} / (e^n n!)$. Recall that
  $(1 + 1 / n)^n \to e$ as $n \to \infty$ and also
  $(1 + 1 / n)^n$ is increasing. Then
  \[
    \frac{a_{n + 1}}{a_n}
    = \frac{(n + 1)^{n - 1} e^n n!}{e^{n + 1}(n + 1)! n^{n - 2}}
    = \frac{\left(1 + \frac{1}{n}\right)^{n - 2}}{e}
    = \underbrace{\frac{\left(1 + \frac{1}{n}\right)^n}{e}}_{< 1} \left(1 + \frac{1}{n}\right)^{-2}
    < \left(\frac{n}{n + 1}\right)^2 = \frac{\frac{1}{(n + 1)^2}}{\frac{1}{n^2}}.
  \]
  Let $b_n = 1 / n^2$, then $a_{n + 1} / a_n \le b_{n + 1} / b_n$,
  so by the previous corollary, we
  see that $\sum_{n = 1}^\infty a_n$ converges.
\end{proof}

\begin{theorem}[Comparison test, third version]
  Let $(A) \sim \sum_{n = 1}^\infty a_n$ and
  $(B) \sim \sum_{n = 1}^\infty b_n$ be two positive
  series and suppose that
  \[
    \lim_{n \to \infty} \frac{a_n}{b_n} = \ell > 0.
  \]
  Then $(A)$ converges if and only if $(B)$ converges.
\end{theorem}

\begin{proof}
  Let $\epsilon = \ell / 2 > 0$. Then there exists $N$
  such that when $n \ge N$, we have
  \[
    \frac{\ell}{2} < \frac{a_n}{b_n} < \frac{3\ell}{2}
    \implies \frac{\ell}{2} b_n < a_n < \frac{3\ell}{2} b_n.
  \]
  Thus $(A)$ converges if and only if $(B)$ converges.
\end{proof}

\begin{example}
  Determine the convergence of
  \begin{enumerate}
    \item $\displaystyle \sum_{n = 1}^\infty \frac{2n^2 + 5n + 1}{\sqrt{n^6 - 3n^2 + 1}}$,
    \item $\displaystyle \sum_{n = 1}^\infty \frac{1}{n^{1 + \frac{1}{n}}}$,
    \item and $\displaystyle \sum_{n = 1}^\infty \left[1 - \sqrt[3]{\frac{n - 1}{n + 1}}\right]^p$ for $p > 0$.
  \end{enumerate}
\end{example}

\begin{proof}
  (1) Let
  \[
    a_n = \frac{2n^2 + 5n + 1}{\sqrt{n^6 - 3n^2 + 1}}
    \sim \frac{2n^2}{\sqrt{n^6}} = \frac{2}{n}.
  \]
  Then $a_n / (2 / n) \to 1$ as $n \to \infty$, so
  $\sum a_n$ diverges since the harmonic series
  diverges.

  (2) Let
  \[
    a_n = \frac{1}{n^{1 + \frac{1}{n}}}
    \quad \text{and} \quad
    b_n = \frac{1}{n}.
  \]
  Then $a_n / b_n = 1 / (n^{1 / n}) \to 1$ as $n \to \infty$,
  so $\sum a_n$ diverges since the harmonic series
  diverges.

  (3) Write
  \begin{align*}
    \sqrt[3]{\frac{n - 1}{n + 1}}
    = \sqrt[3]{\frac{1 - \frac{1}{n}}{1 + \frac{1}{n}}}
    &= \left(1 - \frac{1}{n}\right)^{1 / 3} \left(1 + \frac{1}{n}\right)^{-1 / 3} \\
    &= \left(1 - \frac{1}{3n} + o(1 / n)\right) \left(1 - \frac{1}{3n} + o(1 / n)\right)
    = 1 - \frac{2}{3n} + o(1 / n),
  \end{align*}
  Then we see that
  \[
    a_n \sim \left(\frac{2}{3n}\right)^p,
  \]
  so $\sum a_n$ converges if and only if $p > 1$.
\end{proof}

\begin{theorem}
  Let $\sum_{n = 1}^\infty$ be a positive series and
  suppose that
  \[\limsup_{n \to \infty} \sqrt[n]{a_n} = \ell.\]
  Then
  \begin{enumerate}
    \item if $\ell < 1$, then $\sum_{n = 1}^\infty a_n$ converges,
    \item and if $\ell > 1$, then $\sum_{n = 1}^\infty a_n$ diverges.
  \end{enumerate}
\end{theorem}

\begin{proof}
  (1) When $\ell < 1$, then there exists $q$ with
  $\ell < q < 1$ and $N$ such that when $n \ge N$,
  $\sqrt[n]{a_n} < q$. Then $a_n < q^n$,
  so $\sum a_n$ converges by comparing to the
  geometric series.

  (2) When $\ell > 1$, there exists $\ell > q > 1$ and
  a subsequence $\{n_k\}$ such that
  $(a_{n_k})^{1 / n_k} > q$. This implies that
  $a_{n_k} > q^{n_k}$, so $a_{n_k} \to \infty$ as
  $n_k \to \infty$. Thus $\sum a_n$ diverges since
  we do not have $a_n \to 0$.
\end{proof}

\begin{example}
  Determine the convergence of
  \begin{enumerate}
    \item $\displaystyle \sum_{n =1 }^\infty \left[1 + \frac{1}{\sqrt{n}}\right]^{-n^{3 / 2}}$,
    \item and $\displaystyle \sum_{n = 1}^\infty \left(\frac{3n}{n + 5}\right)^n \left(\frac{n + 2}{n + 3}\right)^{n^2}$.
  \end{enumerate}
\end{example}

\begin{proof}
  (1) Let $a_n$ be the $n$th term in the sum and we
  see that
  \[
    \sqrt[n]{a_n} =
    \left[1 + \frac{1}{\sqrt{n}}\right]^{-\sqrt{n}}.
  \]
  Since $\sqrt{n} \to \infty$ as $n \to \infty$, we
  may replace $\sqrt{n}$ with $n$ to see that
  $\sqrt[n]{a_n} \to 1 / e < 1$, so
  $\sum a_n$ converges.

  (2) Let
  \[
    \sqrt[n]{a_n} = \frac{3n}{n + 5} \left(\frac{n + 2}{n + 3}\right)^n.
  \]
  For the second term, we see that
  \[
    \left(\frac{n + 2}{n + 3}\right)^n
    = \left(\frac{1 + 2 / n}{1 + 3 / n}\right)^n
    \longrightarrow \frac{e^2}{e^3} = \frac{1}{e}
  \]
  as $n \to \infty$.
  Then $\sqrt[n]{a_n} \to 3 / e > 1$, so
  $\sum a_n$ diverges.
\end{proof}

\end{document}
