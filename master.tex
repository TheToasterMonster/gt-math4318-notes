\documentclass[12pt, letterpaper, oneside]{book}
\usepackage[margin={0.6in, 0.75in}]{geometry}
\usepackage{microtype}
% \usepackage{kpfonts}
\usepackage{amsmath, amssymb, amsthm}
\usepackage{parskip}
\usepackage[many]{tcolorbox}
\usepackage{footnote}
\usepackage{cancel}
\usepackage{titlesec}
\usepackage{pgffor}
\usepackage[shortlabels]{enumitem}
\usepackage{hyperref}

\usepackage[overload]{textcase}

\renewcommand{\chaptername}{Lecture}
\newtheorem{axiom}{Axiom}[chapter]
\newtheorem{theorem}{Theorem}[chapter]
\newtheorem{prop}{Proposition}[chapter]
\newtheorem{corollary}{Corollary}[theorem]
\newtheorem{lemma}{Lemma}[chapter]
\theoremstyle{definition}
\newtheorem{definition}{Definition}[chapter]
\newtheorem{exercise}{Exercise}[chapter]
\newtheorem{example}{Example}[definition]
\newtheorem*{remark}{Remark}

\tcbset{sharp corners, breakable, enhanced, parbox=false}
\newtcolorbox{mybox}[3][]
{
  colframe = #2!150,
  colback  = #2!5,
  coltitle = #2!0!white,  
  title    = {#3},
  #1,
}

\titleformat{\chapter}[display]
    {\normalfont\huge\bfseries}{\chaptertitlename\ \thechapter}{20pt}{\Huge}
\titlespacing*{\chapter}{0pt}{0pt}{40pt}

\newcommand{\R}{\mathbb{R}}
\newcommand{\N}{\mathbb{N}}
\newcommand{\Z}{\mathbb{Z}}
\newcommand{\C}{\mathbb{C}}
\newcommand{\Q}{\mathbb{Q}}
\newcommand{\F}{\mathbb{F}}

\DeclareMathOperator{\Vol}{Vol}
\DeclareMathOperator{\Int}{int}
\DeclareMathOperator{\area}{area}

\title{MATH 4318: Analysis II}
\author{Frank Qiang\\Instructor: Zhiwu Lin}
\date{Georgia Institute of Technology\\Spring 2024}

\begin{document}
  \maketitle

  \begingroup
  \let\cleardoublepage\clearpage
  \tableofcontents
  \endgroup

  % \foreach \i in {00, 01, 02, 03, 04, ..., 50} {%
  %   \edef\FileName{lectures/lecture\i.tex}%     The % here are necessary to eliminate any
  %   \IfFileExists{\FileName}{%  spurious spaces that may get inserted
  %      \input{\FileName}%       at these points
  %   }
  % }
  \chapter{Jan.~9 --- The Derivative}

\section{Defining the Derivative}
\begin{definition}
  Let $f$ be a real-valued function on an open interval
  $U \subseteq \R$. Let $x_0 \in U$, we say $f$ is
  \emph{differentiable} at $x_0$ if
  \[
    \lim_{x \to x_0} \frac{f(x) - f(x_0)}{x - x_0}
    = \lim_{h \to 0} \frac{f(x_0 + h) - f(x_0)}{h}
  \]
  exists. If it does, then this limit, denoted by
  $f'(x_0)$, is called the \emph{derivative} of $f$ at
  $x_0$.
\end{definition}

\begin{remark}
By definition, for any $\epsilon > 0$, there exists
$\delta > 0$ such that
\[
  \left|\frac{f(x) - f(x_0)}{x - x_0} - f'(x_0)\right| \le \epsilon
\]
if $|x - x_0| < \delta$ and $x \in U$. Multiplying
both sides by $|x - x_0|$ yields
\[
  |f(x) - f(x_0) - f'(x_0)(x - x_0)| \le \epsilon |x - x_0|.
\]
In other words,
\[|f(x) - \varphi(x)| \le \epsilon |x - x_0|\]
where $\varphi(x) = f(x_0) + f'(x_0)(x - x_0)$.
In other words, $\varphi(x)$ is a first-order
approximation of $f(x)$ near $x_0$.
Geometrically, this is approximating the graph of
$y = f(x)$ by the tangent line $y = \varphi(x)$.
\end{remark}

\section{Immediate Properties}
\begin{prop}
  Let $U \subseteq \R$ be an open set and $ f : U \to \R$.
  If $f$ is differentiable at $x_0 \in U$, then
  $f$ is continuous at $x_0$.
\end{prop}

\begin{proof}
  Pick any $\epsilon_0 > 0$. Then there exists
  $\delta_0 > 0$ such that whenever
  $|x - x_0| < \delta_0$ and $x \in U$,
  \[|f(x) - f(x_0) - f'(x_0)(x - x_0)| \le \epsilon_0 |x - x_0|.\]
  By the triangle inequality,
  \[
    |f(x) - f(x_0)| \le \epsilon_0 |x - x_0| + |f'(x_0)| |x - x_0| = (\epsilon_0 + |f'(x_0)|) |x - x_0|.
  \]
  Now for any $\epsilon > 0$, choose
  $\delta = \min\{\delta_0, \epsilon / (\epsilon_0 + |f'(x_0)|)\}$.
  Then
  \[
    |f(x) - f(x_0)| \le (\epsilon_0 + |f'(x_0)|) |x - x_0| = (\epsilon_0 + |f'(x_0)|) \delta < \epsilon
  \]
  whenever $|x - x_0| < \delta$ and $x \in U$.
  Thus $f$ is continuous at $x_0$.
\end{proof}

\begin{example}
  Take the function
  \[
    f(x) = \begin{cases}
      x \sin(1/x) & \text{if } x \ne 0 \\
      0 & \text{if } x = 0.
    \end{cases}
  \]
  Note that $f$ is continuous on $\R$. For $x \ne 0$,
  continuity is clear since both $x$ and $\sin(1/x)$
  are continuous. At $x = 0$, we have
  \[
    \lim_{x \to 0} f(x) = \lim_{x \to 0} x \sin(1/x) = 0 = f(0)
  \]
  since $|x \sin(1/x)| \le |x|$ for all $x \in \R$,
  so $f$ is also continuous at $x = 0$. However,
  $f$ is not differentiable at $x = 0$. Consider the
  limit
  \[
    \lim_{x \to 0} \frac{f(x) - f(0)}{x - 0} = \lim_{x \to 0} \sin(1/x),
  \]
  which does not exist since $\sin(1/x)$ oscillates. So
  $f$ is not differentiable at $x = 0$.
\end{example}

\begin{example}
  Take the function $f(x) = |x|$, which is continuous
  everywhere on $\R$. However, $f$ is not differentiable at
  $x = 0$, since
  \[
    \lim_{x \to 0} \frac{f(x) - f(0)}{x - 0} = \lim_{x \to 0} \frac{|x|}{x}.
  \]
  Note that
  \[
    \frac{|x|}{x} = \begin{cases}
      1 & \text{if } x > 0 \\
      -1 & \text{if } x < 0,
    \end{cases}
  \]
  so the limit does not exist as $x \to 0$. Thus
  $f$ is not differentiable at $x = 0$.
\end{example}

\begin{remark}
  For the previous example, we can however define
  the \emph{left (right) derivative} by
  \[
    f_-'(x_0) = \lim_{x \to x_0^-} \frac{f(x) - f(x_0)}{x - x_0}
  \quad \text{and} \quad f_+'(x_0) = \lim_{x \to x_0^+} \frac{f(x) - f(x_0)}{x - x_0}.
  \]
  If $f$ is differentiable, then
  $f_-'(x_0) = f_+'(x_0)$. In the previous example,
  $f_-'(0) = -1$ and $f_+'(0) = 1$. For the first
  example however, even $f_{\pm}'(0)$ does not exist.
\end{remark}

\begin{remark}
  In one dimension, the existence of the derivative
  implies that the function is differentiable (the
  function is approximated by a linear function). However,
  in multiple dimensions, the existence of partial
  derivatives does not imply differentiability.
\end{remark}

\section{Rules for Differentiation}
\begin{prop}
  Let $U \subseteq \R$ be open and $f, g : U \to \R$
  be differentiable. Then
  \begin{enumerate}
    \item $(f + g)'(x_0) = f'(x_0) + g'(x_0)$
    \item $(fg)'(x_0) = f'(x_0)g(x_0) + f(x_0)g'(x_0)$
    \item $(f/g)'(x_0) = (f'(x_0)g(x_0) - f(x_0)g'(x_0)) / (g(x_0)^2)$.
  \end{enumerate}
\end{prop}

\begin{proof}
  Find in textbook (Rosenlicht).
\end{proof}

\begin{prop}
  We have $\frac{d}{dx}(c) = 0$, $\frac{d}{dx}(x) = 1$,
  and $\frac{d}{dx}(x^n) = nx^{n - 1}$ for all $n \in \N$.
\end{prop}

\begin{proof}
  We prove the last claim (the power rule) for
  $n \ge 1$ by induction. The base case
  $n = 1$ is the first claim which is true. Now suppose
  that the result holds for any $n \le k \in \N$, and
  we show that it remains true for $n = k + 1$. By
  the product rule, we have
  \[
    \frac{d}{dx} (x^{k + 1})
    = \frac{d}{dx} (x \cdot x^k)
    = \frac{d}{dx}(x) \cdot x^k + x \cdot \frac{d}{dx}(x^k) = x^k + xkx^{k - 1} = (k + 1)x^k.
  \]
  Thus by induction this holds for all $n \ge 1$.
  We can do negative integers by the quotient rule.
\end{proof}

\begin{remark}
  The power rule actually holds for any $n \in \R$.
\end{remark}

\begin{prop}[Chain rule]
  Let $U$ and $V$ be open sets of $\R$ and let
  $f : U \to V, g : V \to \R$ be differentiable.
  Let $x_0 \in U$ be such that $f'(x_0)$ and
  $g'(f(x_0))$ exist. Then $(g \circ f)'(x_0)$ exists
  and
  \[(g \circ f)'(x_0) = g'(f(x_0)) f'(x_0).\]
\end{prop}

\begin{proof}
  For any fixed $y_0$ for which $g'(y_0)$ exists,
  set
  \[
    A(y, y_0) =
    \begin{cases}
      (g(y) - g(y_0)) / (y - y_0) & \text{if } y \in V \text{ and } y \ne y_0 \\
      g'(y_0) & \text{if } y = y_0.
    \end{cases}
  \]
  Then $A$ is continuous at $y_0$. To find
  $(g \circ f)'(x_0)$, observe that
  \begin{align*}
    \lim_{x \to x_0} \frac{g(f(x)) - g(f(x_0))}{x - x_0}
    &= \lim_{x \to x_0} \frac{A(f(x), f(x_0))(f(x) - f(x_0))}{x - x_0} \\
    &= \lim_{x \to x_0} A(f(x), f(x_0)) \lim_{x \to x_0}\frac{f(x) - f(x_0)}{x - x_0}
    = g'(f(x_0)) f'(x_0),
  \end{align*}
  by the continuity of $A$ at $f(x_0)$ and the
  differentiability of $f$ at $x_0$.
\end{proof}

\begin{remark}
  The rough idea of what we did here is
  \begin{align*}
    \lim_{x \to x_0} \frac{g(f(x)) - g(f(x_0))}{x - x_0}
    &= \lim_{x \to x_0} \frac{g(f(x)) - g(f(x_0))}{f(x) - f(x_0)} \frac{f(x) - f(x_0)}{x - x_0} \\
    &= \lim_{x \to x_0} \frac{g(f(x)) - g(f(x_0))}{f(x) - f(x_0)} \lim_{x \to x_0} \frac{f(x) - f(x_0)}{x - x_0}
    = g'(f(x_0)) f'(x_0).
  \end{align*}
  But does not quite work as stated since
  it might be that $f(x) = f(x_0)$ even if $x \ne x_0$.
  We can fix this by introducing the function $A$
  as we did in the proof, though the overall idea is
  the same.
\end{remark}

\begin{remark}
  If $f$ is monotone near $x_0$, then we can define
  the \emph{inverse function} $f^{-1}$ so that
  $(f^{-1} \circ f)(x) = x$ near $x_0$. If
  $f'(x_0)$ exists, then by the chain rule
  applied to $x = (f^{-1} \circ f)(x)$ at $x = x_0$
  we have
  \[
    1 = \frac{d}{dx}(f^{-1} \circ f)(x_0)
    = \frac{d}{dx} f^{-1}(f(x_0)) \cdot f'(x_0)
    \implies \frac{d}{dx} f^{-1}(f(x_0)) = \frac{1}{f'(x_0)}.
  \]
\end{remark}

\begin{example}
  Let $f(x) = e^x$ with $f^{-1}(x) = \ln(x)$. Since
  $f'(x) = f(x) = e^x$, we have
  \[\frac{d}{dx} f^{-1}(f(x_0)) = \frac{1}{f'(x_0)} \implies
    \frac{d}{dx} \ln(e^{x_0}) = \frac{1}{e^{x_0}}.
  \]
  Letting $e^{x_0} = h$, we have
  $\left.\frac{d}{dx} \ln(x)\right|_{x = h} = 1 / h$,
  which recovers the familiar formula.
\end{example}

  \chapter{Jan.~11 --- The Mean Value Theorem}

\section{The Mean Value Theorem}
\begin{lemma}
  Let $I \subseteq \R$ be open, $f : I \to \R$
  is differentiable at $x_0 \in I$ and $f'(x_0) \ne 0$.
  Suppose $f'(x_0) > 0$, then there exists $\delta > 0$
  such that for any $x \in (x_0 - \delta, x_0 + \delta)$,
  \begin{enumerate}
    \item if $x > x_0$, then $f(x) > f(x_0)$,
    \item if $x < x_0$, then $f(x) < f(x_0)$.
  \end{enumerate}
\end{lemma}

\begin{proof}
  Take $\epsilon = f'(x_0) / 2$. By the
  definition of the derivative, there exists
  $\delta > 0$ such that for ay $|x - x_0| < \delta$,
  we have
  \[
    \left| \frac{f(x) - f(x_0)}{x - x_0} - f'(x_0) \right|
    < \epsilon = \frac{1}{2} f'(x_0).
  \]
  By the triangle inequality,
  \[
    \frac{f(x) - f(x_0)}{x - x_0}
    > \frac{1}{2} f'(x_0) > 0.
  \]
  This quotient being positive immediately implies
  the desired results.
\end{proof}

\begin{theorem}
  If $f(x)$ is differentiable in an open interval
  $I$ and $f$ obtains its local maximum (or minimum)
  at $x_0 \in I$, then $f'(x_0) = 0$.
\end{theorem}

\begin{proof}
  Suppose otherwise that $f'(x_0) \ne 0$. Assume without
  loss of generality that $f'(x_0) > 0$. Then by the
  previous lemma, there exists $\delta > 0$ such that
  for $x \in (x_0 - \delta, x_0 + \delta)$, if $x > x_0$
  then $f(x) > f(x_0)$ and if $x < x_0$ then
  $f(x) < f(x_0)$. So $x_0$ cannot be a local maximum
  or minimum, which is a contradiction.
\end{proof}

\begin{theorem}[Rolle's middle value theorem]
  Let $f(x)$ be continuous on $[a, b]$ and
  differentiable in $(a, b)$. Suppose $f(a) = f(b)$,
  then there exists $x_0 \in (a, b)$ such that
  $f'(x_0) = 0$.
\end{theorem}

\begin{proof}
  Since $f$ is continuous on a compact set, it obtains
  both a maximum and minimum on $[a, b]$. Let
  $M$ be the maximum and $m$ be the minimum. If
  $M = m$, then $f(x) \equiv M$ and $f'(x) = 0$
  everywhere. If $M > m$, then at least one of the maximum
  or minimum must be obtained at an interior point
  $x_0 \in (a, b)$ since $f(a) = f(b)$. By the previous
  theorem, $f'(x_0) = 0$ at this point and we are done.
\end{proof}

\begin{example}
  Show that the equation
  $4ax^3 + 3bx^2 + 2cx = a + b + c$ has at least one
  root in $(0, 1)$.
\end{example}

\begin{proof}
  Consider the equation
  \[
    4ax^3 + 3bx^2 + 2cx - (a + b + c) = 0.
  \]
  Notice that the left hand side is the derivative
  of the function
  \[f(x) = ax^4 + bx^3 + cx^2 - (a + b + c)x.\]
  So we just need to show that $f'(x) = 0$ for some
  $x$. For this, we can check that $f(0) = f(1) = 0$,
  and thus by Rolle's theorem there exists
  $x_0 \in (0, 1)$ such that $f'(x_0) = 0$. So
  $x_0$ is a root.
\end{proof}

\begin{theorem}[Lagrange's middle value theorem]
  Let $f_9x)$ be continuous on $[a, b]$ and differentiable
  in $(a, b)$. Then there exists $x_0 \in (a, b)$
  such that
  \[f'(x_0) = \frac{f(b) - f(a)}{b - a}.\]
\end{theorem}

\begin{proof}
  Subtract the secant line through $(a, f(a))$ and
  $(b, f(b))$ from $f(x)$ to get
  \[g(x) = f(x) - \frac{f(b) - f(a)}{b - a} (x - a).\]
  Note that $g(a) = g(b) = f(a)$. So by Rolle's theorem,
  there exists $x_0 \in (a, b)$ such that $g'(x_0) = 0$.
  But
  \[
    0 = g'(x_0) = f'(x_0) - \frac{f(b) - f(a)}{b - a},
  \]
  which is the desired result.
\end{proof}

\begin{corollary}
  Suppose $f \in C([a, b])$, i.e. $f$ is continuous
  on $[a, b]$, and that $f$ is differentiable in
  $(a, b)$. Then the following statements are equivalent:
  \begin{enumerate}
    \item $f'(x) \ge 0$ in $(a, b)$,
    \item $f(x)$ is increasing, i.e. if $x_1 > x_2$,
      then $f(x_1) \ge f(x_2)$.
  \end{enumerate}
  In particular, if $f'(x) > 0$ in $(a, b)$, then
  $f(x)$ is strictly increasing, i.e. if $x_1 > x_2$,
  then $f(x_1) > f(x_2)$.
\end{corollary}

\begin{proof}
  $(2 \Rightarrow 1)$ For any $x_0 \in (a, b)$,
  \[
    f'(x_0) = \lim_{h \to 0} \frac{f(x_0 + h) - f(x_0)}{h} \ge 0
  \]
  since $f(x_0 + h) - f(x_0) \ge 0$ for $h > 0$
  as $f$ is increasing.

  $(1 \Rightarrow 2)$ Take $x_1 > x_2$, then
  by Lagrange's theorem there exists $\xi \in (x_2, x_1)$
  such that
  \[f(x_2) - f(x_1) = f'(\xi)(x_2 - x_1) \ge 0.\]
  So $f(x_1) \ge f(x_2)$. The strict version follows
  from changing the above inequality to a strict one.
\end{proof}

\section{Applications}
\begin{example}
  Show that
  \[\frac{2}{2x + 1} < \ln(1 + 1 / x)\]
  for any $x > 0$.
\end{example}

\begin{proof}
  Let $f(x) = 2 / (2x + 1) - \ln(1 + 1 / x)$. Taking
  the derivative yields
  \[
    f'(x) = \frac{1}{(2x + 1)^2 x (x + 1)} > 0,
  \]
  so $f$ is strictly increasing in $(0, \infty)$.
  Note that $f \to 0$ as $x \to \infty$, so $f(x) < 0$
  for all $x > 0$.
\end{proof}

\begin{example}
  Show that $b / a > b^a / a^b$
  when $b > a > 1$.
\end{example}

\begin{proof}
  Take log on both sides to get
  $\ln b - \ln a > a \ln b - b \ln a$. This gives
  \[
    (b - 1)\ln a > (a - 1) \ln b
    \quad \iff \quad \frac{\ln a}{a - 1} > \frac{\ln b}{b - 1}.
  \]
  Note that this is a monotonicity property. So
  let $f(x) = (\ln x) / (x - 1)$ for $x > 1$. Then
  \[
    f'(x) = \frac{x - 1 - x\ln x}{x(x - 1)^2} < 0
  \]
  when $x > 1$ because $x - 1 - x \ln x < 0$. To see
  the last claim, define $g(x) = x - 1 - x \ln x$
  and note that $g'(x) = -\ln x < 0$ for $x > 1$.
  But $g(0) = 0$, so $g(x) < 0$ for $x > 1$. So $f$
  is strictly decreasing.
\end{proof}

\begin{example}
  Show that
  \[
    \lim_{x \to 0} \frac{e^x - e^{\sin x}}{x - \sin x} = 1.
  \]
  Let $f(x) = e^x$. Then there exists $\xi$ between
  $x$ and $\sin x$ such that
  \[
    e^x - e^{\sin x} = (x - \sin x) e^{\xi(x)},
  \]
  where the choice of $\xi$ may vary for different $x$.
  Then
  \[
    \lim_{x \to 0} \frac{e^x - e^{\sin x}}{x - \sin x} = 
    \lim_{x \to 0} e^{\xi(x)}.
  \]
  Now note that $\xi(x)$ is always between $x$ and
  $\sin x$, which both tend to $0$ as $x \to 0$. So by
  the squeeze theorem we have $\xi(x) \to 0$ as $x \to 0$
  and thus $e^{\xi(x)} \to 1$ as $x \to 0$.
\end{example}

\section{Cauchy's Mean Value Theorem}
\begin{theorem}[Cauchy's middle value theorem]
  Let $f, g \in C([a, b])$ and $f, g$ be differentiable
  in $(a, b)$. Suppose $g'(x) \ne 0$ for any
  $x \in (a, b)$. Then there exists $x_0 \in (a, b)$
  such that
  \[
    \frac{f'(x_0)}{g'(x_0)} = \frac{f(b) - f(a)}{g(b) - g(a)}.
  \]
\end{theorem}

\begin{proof}
  Use a similar construction as before and let
  \[
    F(x) = f(x) - f(a) - \frac{f(b) - f(a)}{g(b) - g(a)} (g(x) - g(a)).
  \]
  Note that $F(b) = F(a) = 0$, so by Rolle's theorem
  there exists $x_0 \in (a, b)$ such that
  $F'(x_0) = 0$. Then
  \[0 = F'(x_0) = f'(x_0) - \frac{f(b) - f(a)}{g(b) - g(a)} g'(x_0),\]
  which implies the desired result.
\end{proof}

\begin{remark}
  The $g'(x) \ne 0$ condition guarantees that
  $g$ is monotone, even if $g'$ may fail to be
  continuous.
\end{remark}

\begin{remark}
  If $g$ is a monotonically increasing function,
  we can view $g$ as a mapping
  $g : [a, b] \to [g(a), g(b)]$, which we can view
  as a change of variables $x \mapsto u$. Since $g$ is
  monotone, we have an inverse $x = g^{-1}(u)$. Then
  \[f(x) = f(g^{-1}(u)) = (f \circ g^{-1})(u) = \widetilde{f}(u).\]
  By Lagrange's theorem,
  \[
  \frac{\widetilde{f}(g(b)) - \widetilde{f}(g(a))}{g(b) - g(a)}
    = \widetilde{f}'(u_0)
  \]
  for some $u_0 \in (g(a, g(b))$. Now note that
  \[
    \widetilde{f}(g(b)) = (f \circ g^{-1})(g(b)) = f(b),
    \quad \widetilde{f}(g(a)) = f(a).
  \]
  So the left-hand side is precisely
  \[
    \text{LHS} = \frac{f(b) - f(a)}{g(b) - g(a)}.
  \]
  By the chain rule, we have
  \[
    \text{RHS} = \widetilde{f}'(u_0) = (f \circ g^{-1})'(u_0)
    = f'(g^{-1}(u_0)) (g^{-1})'(u_0)
    = f'(x_0) \frac{1}{g'(x_0)}.
  \]
  This recovers Cauchy's mean value theorem. So they are
  equivalent even if Cauchy's seems stronger.
\end{remark}

  \chapter{Jan.~16 --- Taylor's Theorem}

\section{Darboux's Lemma}
\begin{lemma}[Darboux's lemma]
  If $f$ is differentiable in $(a, b)$, continuous
  on $[a, b]$ and $f'(a) < f'(b)$, then for any
  $c \in (f'(a), f'(b))$, there exists $x_0 \in (a, b)$
  such that $f'(x_0) = c$.
\end{lemma}

\begin{proof}
  See homework.
\end{proof}

\begin{remark}
  There exists an example of a differentiable function
  $f(x)$ but $f'(x)$ is not continuous, e.g.
  \[
    f(x) =
    \begin{cases}
      x^2 \sin(1/x) & \text{if } x \ne 0 \\
      0 & \text{if } x = 0.
    \end{cases}
  \]
  We can compute that
  \[
    f'(x) =
    \begin{cases}
      2x \sin(1/x) - \cos(1/x) & \text{if } x \ne 0 \\
      0 & \text{if } x = 0,
    \end{cases}
  \]
  and we can verify as an exercise that $f'(x)$ is not
  continuous at $x = 0$.
\end{remark}

\begin{remark}
  Darboux's lemma guarantees that $g'(x) \ne 0$ implies
  either $g'(x) > 0$ or $g'(x) < 0$ everywhere in
  the conditions for Cauchy's mean value theorem.
\end{remark}

\section{L'H\^opital's Rule}
\begin{theorem}[L'H\^opital's rule, $0 / 0$]
  Let $f, g$ be differentiable in $(a, b)$,
  $\lim_{x \to a^+} f(x) = \lim_{x \to a^+} g(x) = 0$,
  and $g'(x) \ne 0$ for any $x \in (a, b)$. Then
  if $\lim_{x \to a^+} f'(x)/g'(x)$ exists, we have
  \[
    \lim_{x \to a^+} \frac{f(x)}{g(x)} = \lim_{x \to a^+} \frac{f'(x)}{g'(x)}.
  \]
\end{theorem}

\begin{proof}
  By Cauchy's theorem, for any $x \in (a, b)$, there
  exists $\xi(x) \in (a, x)$ such that
  \[
    \frac{f(x)}{g(x)} = \frac{f(x) - f(a)}{g(x) - g(a)}
    = \frac{f'(\xi(x))}{g'(\xi(x))}.
  \]
  If $x \to a^+$, then $\xi(x) \to a^+$, so
  \[
    \lim_{x \to a^+} \frac{f(x)}{g(x)}
    = \lim_{x \to a^+} \frac{f'(\xi(x))}{g'(\xi(x))}
    = \lim_{x \to a^+} \frac{f'(x)}{g'(x)},
  \]
  as desired.
\end{proof}

\begin{corollary}
  Let $f, g$ be differentiable in $(a, \infty)$,
  $\lim_{x \to \infty} f(x) = \lim_{x \to \infty} g(x) = 0$,
  and $g'(x) \ne 0$ for any $x \in (a, \infty)$. Then
  if $\lim_{x \to \infty} f'(x)/g'(x)$ exists, we have
  \[
    \lim_{x \to \infty} \frac{f(x)}{g(x)} = \lim_{x \to \infty} \frac{f'(x)}{g'(x)}.
  \]
\end{corollary}

\begin{proof}
  Assume $a > 0$. Define $\widetilde{f}(y) = f(1/y)$ and
  $\widetilde{g}(y) = g(1/y)$ with $y \in (0, 1 / a)$.
  By L'H\^opital's rule,
  \[
    \lim_{y \to 0^+} \frac{\widetilde{f}(y)}{\widetilde{g}(y)}
    = \lim_{y \to 0^+} \frac{\widetilde{f}'(y)}{\widetilde{g}'(y)}
    = \lim_{y \to \infty} \frac{f'(1 / y) \cdot (-1 / y^2)}{g'(1 / y) \cdot (-1 / y^2)}
    = \lim_{x \to \infty} \frac{f'(x)}{g'(x)},
  \]
  as desired.
\end{proof}

\begin{theorem}[L'H\^opital, $\infty / \infty$]
  Let $f, g$ be differentiable in $(a, b)$,
  $\lim_{x \to a^+} |f(x)| = \lim_{x \to a^+} |g(x)| = \infty$,
  and $g'(x) \ne 0$ for any $x \in (a, b)$. Then
  if $\lim_{x \to a^+} f'(x)/g'(x)$ exists, we have
  \[
    \lim_{x \to a^+} \frac{f(x)}{g(x)} = \lim_{x \to a^+} \frac{f'(x)}{g'(x)}.
  \]
\end{theorem}

\begin{proof}
  Left as an exercise.
\end{proof}

\begin{remark}
  Saying that the absolute values of $f$ and $g$ go to
  infinity works, since the existence of the limit rules
  out oscillatory behavior.
\end{remark}

\begin{remark}
  These cases of $\infty / \infty$ and $0 / 0$ are
  are called \emph{indefinite types}. Other indefinite
  types include $0 \cdot \infty$, $0^0$, $\infty^0$
  $1^\infty$, $\infty - \infty$, etc. But we can try
  to reduce them to the cases we know. For example,
  if $f(x) \to 0^+$ and $g(x) \to 0^+$ when $x \to x_0$,
  then $\lim_{x \to x_0} f(x)^{g(x)}$ is $0^0$.
  Letting $y(x) = f(x)^{g(x)}$, we can take the log
  to get
  \[
    \ln y(x) = g(x) \ln f(x) = \frac{\ln f(x)}{1 / g(x)}
    = \frac{\infty}{\infty}.
  \]
\end{remark}

\begin{example}
  We can see that (this is a $\infty - \infty$ case)
  \[
    \lim_{x \to 0^+} \frac{1}{x^2} - \frac{\cot x}{x}
    = \lim_{x \to 0^+} \frac{1 + x \cot x}{x^2}
    = \lim_{x \to 0^+} \frac{-(\cot x - x \csc^2 x)}{2x \sin^2 x}.
  \]
  Note that $x \cot x = x \cos x / \sin x \to 1$ as
  $x \to 0$. Now note that $\sin x / x \to 1$ as $x \to 0$,
  so we continue with
  \[
    \lim_{x \to 0^+} \frac{-(\cot x - x \csc^2 x)}{2x \sin^2 x}
    = \lim_{x \to 0^+} \frac{x - \sin x \cos x}{2x^3} \frac{x^2}{\sin^2 x}
  \]
  Since $x^2 / \sin^2 x \to 1$ as $x \to 0$, we can
  look at the remaining part to get
  \[
    \lim_{x \to 0^+} \frac{x - \sin x \cos x}{2x^3}
    = \lim_{x \to 0^+} \frac{1 - \cos 2x}{6x^2}
    = \lim_{x \to 0^+} \frac{2 \sin 2x}{12x} = \frac{1}{3}.
  \]
  So $\lim_{x \to 0^+} (1 / x^2 - \cot x / x) = 1 / 3$.
\end{example}

\section{Taylor's Theorem}
\begin{theorem}[Peano remainder term]
  Let $f : [a, b] \to \R$ be differentiable at $x = a$
  up to $n$th order of derivatives, i.e.
  $f'(a), f''(a), \dots, f^{(n)}(a)$ exist. Then as
  $x \to a^+$, we have
  \[f(x) = \sum_{k = 0}^n \frac{f^(k)(a)}{k!}(x - a)^k + o((x - a)^n).\]
  Call the polynomial part of the above $P_n(x)$, which
  is also known as the \emph{Taylor polynomial} of order $n$.
\end{theorem}

\begin{proof}
  To show that the error term is $o((x - a)^n)$, we have
  \[
    \lim_{x \to a^+} \frac{f(x) - P_n(x)}{(x - a)^n}
    = \lim_{x \to a^+} \frac{f'(x) - P_n'(x)}{n(x - a)^{n - 1}}
    = \frac{1}{n!} \lim_{x \to a^+} \left[\frac{f^{n - 1}(x) - f^{n - 1}(a)}{x - a} - f^{(n)}(a)\right] = 0
  \]
  by L'H\^opital's rule, where we used the observation
  that
  $f^{(k)}(a) = P_n^{(k)}(a)$ for $1 \le k \le n$.
  The final step is a result of the existence of $f^{(n)}(a)$.
\end{proof}

\begin{lemma}[Rolle's theorem for higher order derivatives]
  Let $f \in C^([a, b])$ and differentiable to
  $(n + 1)$ order. If $f'(a) = \dots = f^{(n)}(a) = 0$
  and $f(a) = f(b)$, then there exists $x_0 \in (a, b)$
  such that $f^{(n + 1)}(x_0) = 0$.
\end{lemma}

\begin{proof}
  Since $f(a) = f(b)$, by the usual Rolle's theorem
  there exists $x_1 \in (a, b)$ such that $f'(x_1) = 0$.
  Then since $f'(a) = f'(x_1) = 0$, by
  Rolle's theorem again,
  there exists $x_2 \in (a, x_1)$ such that $f''(x_2) = 0$.
  Repeat this to get $x_{n + 1} \in (a, x_n) \subseteq (a, b)$ such that
  $f^{(n + 1)}(x_{n + 1}) = 0$. Take $x_0 = x_{n + 1}$
  to finish.
\end{proof}

\begin{theorem}[Lagrange remainder term]
  Let $f \in C^n([a, b])$, in particular,
  $f'(a), \dots, f^{(n)}(a)$ exist. Additionally,
  assume $f$ is $(n + 1)$-th differentiable in $(a, b)$.\footnote{Note that the $(n + 1)$-th derivative need not be continuous here.}
  Then
  \[
    f(x) = \sum_{k = 0}^n \frac{f^{(k)}(a)}{k!}(x - a)^k + R_n(x) \quad \text{where} \quad R_n(x) = \frac{f^{(n + 1)}(\xi)}{(n + 1)!}(x - a)^{n + 1}
  \]
  for some $\xi \in [a, x]$.
\end{theorem}

\begin{proof}
  Define $P(x) = P_n(x) + \lambda (x - a)^{n + 1}$, where
  we choose $\lambda \in \R$ such that $P(b) = f(b)$, i.e.
  \[
    \lambda = \frac{f(b) - P_n(b)}{(b - a)^{n + 1}}.
  \]
  Consider $g(x)  = f(x) - P(x)$, which satisfies
  $g(a) = g(b) = 0$ and $g'(a) = \dots = g^{(n)}(a) = 0$.
  Then by Rolle's theorem (higher order), there exists
  $\xi \in (a, b)$ such that $g^{(n + 1)}(\xi) = 0$.
  In other words,
  \[
    f^{(n + 1)} - P^{(n + 1)}(\xi) = 0
    \implies f^{(n + 1)}(\xi) - (n + 1)! \underbrace{\frac{f(b) - P_n(b)}{(b - a)^{n + 1}}}_{\lambda} = 0.
  \]
  This implies that
  \[
    f(b) = P_n(b) + \frac{1}{(n + 1)!} f^{(n + 1)}(\xi) (b - a)^{n + 1},
  \]
  and since we picked $b$ arbitrarily (kind of),
  we can take $b = x$ and we are done since
  $\xi \in [a, b]$.
\end{proof}

\begin{remark}
  The choice of $\xi$ in Lagrange's remainder term
  may (and likely does) vary for different $x$.
\end{remark}

\begin{remark}
  The Taylor polynomial is unique in the sense that
  if $f : [a, b] \to \R$ and $f'(a), \dots, f^{(n)}(a)$
  exist, then if
  \[f(x) = p(x) + o((x - a)^n) \text{as} x \to a^+\]
  as $x \to a^+$ for some polynomial $p(x)$ with
  $\deg p \le n$, then
  $p(x) = P_n(x) = \sum_{k = 0}^n \frac{f^{(k)}(a)}{k!} (x - a)^k$.
  This is because if $Q(x) = p(x) - P_n(x)$, then
  by Taylor's formula (Peano form), we get
  \[
    \lim_{x \to a^+} \frac{Q(x)}{(x - a)^n}
    = \lim_{x \to a^+} \frac{p(x) - \sum_{k = 0}^n \frac{f^{(k)}(a)}{k!} (x - a)^k}{(x - a)^n} = 0.
  \]
  From here this implies that $Q(x) = 0$ since $\deg Q \le n$.
\end{remark}

  \chapter{Jan.~18 --- Taylor Polynomials}

\section{Common Taylor Polynomials}
We have
\begin{align*}
  e^x &= 1 + x + \frac{x^2}{2!} + \dots + \frac{x^n}{n!} + o(x^n), \\
  \sin x &= x - \frac{x^3}{3!} + \frac{x^5}{5!} - \dots
  + \frac{(-1)^{n - 1} x^{2n - 1}}{(2n - 1)!} + o(x^{2n}), \\
  \cos &= 1 - \frac{x^2}{2!} + \frac{x^4}{4!} - \dots
  + \frac{(-1)^n x^{2n}}{(2n)!} + o(x^{2n + 1}), \\
  (1 + x)^\alpha &= 1 + \alpha x + \frac{\alpha(\alpha - 1)}{2!} x^2 + \dots + \frac{\alpha(\alpha - 1) \dots (\alpha - n + 1)}{n!} x^n + o(x^n), \\
  \ln (1 + x) &= x - \frac{x^2}{2} + \frac{x^3}{3}
  + \dots + (-1)^{n - 1} \frac{x^n}{n} + o(x^n).
\end{align*}

\section{Combining Taylor Polynomials}
\begin{remark}
  If $a = 0$ and $f(x)$ is even in $(-b, b)$, then
  \[
    f(x) = \sum_{k = 0}^{n / 2} a_k x^{2k} + o(x^n).
  \]
  Similarly if $f(x)$ is odd in $(-b, b)$, then
  \[
    f(x) = \sum_{k = 0}^{n / 2} a_k x^{2k + 1} + o(x^{n + 1}).
  \]
\end{remark}

\begin{remark}
To create new Taylor polynomials from known ones,
we can observe that if
$f(x) = P_n(x) + o((x - a)^n)$ and $g(x) = Q_n(x) + o((x - a)^n)$, then
\[
  f(x) + g(x) = (P_n(x) + Q_n(x)) + o((x - a)^n)
  \quad \text{and} \quad
  f(x) g(x) = \underbrace{(P_n(x) Q_n(x))}_{\text{take first $n$ terms}} + o((x - a)^n).
\]
If $P_n(x) = \sum_{k = 0}^n a_k (x - a)^k$ and
$Q_n(x) = \sum_{k = 0}^n b_k (x - a)^k$, then
$f(x)g(x)$ has Taylor polynomial
$\sum_{k = 0}^n c_k (x - a)^k$ where
\[c_k = \sum_{i = 0}^k a_i b_{k - i}.\]
If $h(x) = f(x) / g(x)$ and $g(x) \ne 0$ near $x = a$,
then $f(x) = h(x) g(x)$. Let $h(x) = \sum_{k = 0}^n c_k (x - a)^k + o((x - a)^n)$, then
\[
  a_k = \sum_{i = 0}^k c_i b_{k - i}
\]
for $0 \le k \le n$, after which we can solve for the
$c_k$.
\end{remark}

\begin{example}
Find the Taylor polynomial for $\tan x$ up to $n = 5$.
\end{example}

\begin{proof}
Note that $\tan x$ is odd, so we can write
\[
  \tan x = x + a_3 x^3 + a_5 x^5 + o(x^5).
\]
Now since $\tan x = \sin x / \cos x$, we have
$\sin x = \tan x \cos x$, so
\[
  x - \frac{x^3}{6} + \frac{x^5}{5!} + o(x^5)
  = (x + a_3 x^3 + a_5x^5)(1 - \frac{x^2}{2!} + \frac{x^4}{4!})
\]
We can solve to get
\[
  \begin{cases}
    -\frac{1}{6} = -\frac{1}{2} + a_3 \\
    \frac{1}{5!} = \frac{1}{4!} - \frac{a_3}{2!} + a_5
  \end{cases}
  \implies \quad
  a_3 = -\frac{1}{3}, \quad
  a_5 = \frac{2}{15}
\]
as the coefficients for the Taylor polynomial.
\end{proof}

\begin{remark}
  If
  \[
    f'(x) = \sum_{k = 0}^n b_k (x - a)^k + o((x - a)^n),
  \]
  then the anti-derivative of $f(x)$ has
  \[f(x) = f(x_0) + \sum_{k = 0}^n a_{k + 1} (x - a){k + 1} + o((x - a)^{n + 1}),\]
  where $a_{k + 1} = b_k / (k + 1)$ for $0 \le k \le n$.
  This is because
  \[
    b_k = \frac{(f')^{(k)}(a)}{k!} = \frac{f^{(k + 1)}(a)}{k!}
    \quad \text{and} \quad
    a_{k + 1} = \frac{f^{(k + 1)}(a)}{k + 1}
    = \frac{1}{k + 1} \frac{f^{(k + 1)}(a)}{k!}
    = \frac{b_k}{k + 1}.
  \]
\end{remark}

\begin{example}
  Find the Taylor polynomial for $f(x) = \arctan x$.
\end{example}

\begin{proof}
  Recall that
  \[f'(x) = \frac{1}{1 + x^2} = \sum_{k = 0}^n (-1)^k x^{2k}.\]
  Using the above we get
  \[
    f(x) = \arctan x = \sum_{k = 0}^n (-1)^k \frac{x^{2k + 1}}{2k + 1} + o(x^{2n + 2})
  \]
  as the Taylor polynomial.
\end{proof}

\section{Applications for Taylor Polynomials}
\subsection{Finding Limits}
\begin{remark}
Let $f(x) = ax^n + o(x^n)$ as $x \to 0$
and $g(x) = bx^n + o(x^n)$ where $b \ne 0$.
Then
\[
  \lim_{x \to 0} \frac{f(x)}{g(x)} = \frac{a}{b}.
\]
\end{remark}

\begin{remark}
  For the polynomial of $f(g(x))$, we can do
  \[
    f(u) = \sum_{k = 0}^n a_k (u - g(a))^k + o((u - g(a))^n),
    \quad \text{where} \quad
    u = g(x) = \sum_{k = 0}^n b_k(x - a)^k + o((x - a)^n).
  \]
  Then we can substitute  in $u = g(x)$ to find the
  overall polynomial.
\end{remark}

\begin{example}
  Find
  \[
    \lim_{x \to 0} \frac{\sqrt{1 + 2 \tan x} - e^x + x^2}{\arcsin x - \sin x}.
  \]
\end{example}

\begin{proof}
  Note that
  \begin{align*}
    \sqrt{1 + 2\tan x} - e^x + x^2 &= \frac{2x^3}{3} + o(x^3), \\
    \arcsin x - \sin x &= \frac{x^3}{3} + o(x^3).
  \end{align*}
  So the desired limit is $2$.
\end{proof}

\begin{remark}
  If $f(x) = ax^n + o(x^n)$ and $g(x) = bx^m + o(x^m)$
  for $a, b \ne 0$, then
  \[
    \lim_{x \to 0} \frac{f(x)}{g(x)} = \begin{cases}
      a / b & \text{if $m = n$}, \\
      0 & \text{if $m < n$}, \\
      \infty & \text{if $m > n$}.
    \end{cases}
  \]
\end{remark}

\begin{example}
  Assume $f(x) = 1 + ax^n + o(x^n)$ where
  $a \ne 0$ and
  \[
    g(x) = \frac{1}{bx^n + o(x^n)}, \quad \text{i.e.} \quad
    \frac{1}{g(x)} = bx^n + o(x^n).
  \]
  for $b \ne 0$. Then
  \[
    \lim_{x \to 0} f(x)^{g(x)} = e^{a / b}.
  \]
  Let $y(x) = f(x)^{g(x)}$, then
  $\ln y(x) = g(x) \ln f(x)$. Note that
  \[
    \ln f(x) = \ln(1 + ax^n + o(x^n))
    = ax^n + o(x^n),
  \]
  so that
  \[
    \frac{\ln f(x)}{1 / g(x)} = \frac{ax^n + o(x^n)}{bx^n + o(x^n)} \to \frac{a}{b}
  \]
  as $x \to 0$. Thus $\ln y(x) \to a / b$ and
  $y(x) \to e^{a / b}$ as $x \to 0$.
\end{example}

\begin{example}
  Find
  \[
    \lim_{x \to 0} \left[\cos(xe^x) - \ln (1 - x) - x\right]^{\cot x^3}.
  \]
\end{example}

\begin{proof}
  Here we have
  \[
    f(x) = \cos(xe^x) - \ln(1 - x) - x
    = 1 - \frac{2}{3} x^3 + o(x^3) \quad
    \text{and} \quad \frac{1}{g(x)} = \tan x^3 = x^3 + o(x^3).
  \]
  Thus the limit is $e^{-2 / 3}$.
\end{proof}

\subsection{Estimation}
\begin{example}
  Let $f(x)$ be twice differentiable in $[0, 1]$ and
  $f(0) = f(1)$. Further assume $|f''(x)| \le M$
  for $0 \le x \le 1$. Prove that
  $|f'(x)| \le M / 2$ for $0 \le x \le 1$.
\end{example}

\begin{proof}
  Recall that Lagrange's form of Taylor's theorem says
  \[
    f(x) = f(a) + f'(a) (x - a) + \frac{f''(\xi)}{2!}(x - a)^2
  \]
  for some $\xi$ between $a$ and $x$. Thus for any
  $x \in (0, 1)$, we have
  \[
    f(1) = f(x) + f'(x)(1 - x) + \frac{f''(\xi_1)}{2}(1 - x)^2.
  \]
  Similarly, we have
  \[
    f(0) = f(x) + f'(x)(-x) + \frac{f''(\xi_2)}{2}x^2.
  \]
  Here $x \le \xi_1 \le 1$ and $0 \le \xi_2 \le x$.
  Since $f(1) = f(2)$, we can solve for $f'(x)$ to get
  \[
    f'(x) = \frac{f''(\xi_2)x^2 - f''(\xi_1)(1 - x)^2}{2}.
  \]
  Then taking absolute values yields
  \[
    |f'(x)| \le M \left(\frac{x^2 + (1 - x)^2}{2}\right) \le \frac{M}{2} \max_{0 \le x \le 1} \left[x^2 + (1 - x)^2\right]
    = \frac{M}{2},
  \]
  as desired.
\end{proof}

\begin{example}
  Let $f(x)$ be twice differentiable in $[0, 1]$ and
  $f'(a) = f'(b) = 0$. Then there exists $\xi \in (a, b)$
  such that
  \[
    |f''(\xi)| \ge 4 \frac{|f(a) - f(b)|}{(b - a)^2}.
  \]
\end{example}

\begin{proof}
  Note that this is equivalent to
  \[
    |f(b) - f(a)| \le f''(\xi)\left(\frac{b - a}{2}\right)^2.
  \]
  Then we have
  \[
    f\left(\frac{b + a}{2}\right) = f(a) + \frac{f''(\xi_1)}{2}\left(\frac{b - a}{2}\right)^2
    = f(b) - \frac{f''(\xi_2)}{2}\left(\frac{b - a}{2}\right)^2,
  \]
  so that
  \[
    f(b) - f(a) = \frac{f''(\xi_2) + f''(\xi_1)}{2} \left(\frac{b - a}{2}\right)^2.
  \]
  From here we have
  \[
    |f(b) - f(a)| \le \underbrace{\frac{|f''(\xi_1)| + |f''(\xi_2)|}{2}}_{= |f''(\xi)|} \left(\frac{b - a}{2}\right)^2
  \]
  for some $\xi \in (a, b)$ by Darboux's lemma, as
  desired.
\end{proof}

  \chapter{Jan.~23 --- The Riemann Integral}

\section{The Anti-Derivative}

Recall the \emph{anti-derivative} from calculus:

\begin{definition}
  Let $f : U \to \R$ where $U$ is an interval in $\R$.
  If there exists a differentiable function $F : U \to \R$
  such that $F'(x) = f(x)$ for all $x \in U$, then
  $F(x)$ is an \emph{anti-derivative} of $f$, denoted
  \[F(x) = \int f(x)\, dx.\]
  This is also called the \emph{indefinite integral} of
  $f$.
\end{definition}

\begin{remark}
  The anti-derivatives of a function can differ by a
  constant.
\end{remark}

\begin{example}
  Find an anti-derivative of $f(x) = |x|$ for $x \in \R$.
\end{example}

\begin{proof}
  If $x > 0$, we have $f(x) = x$ and so
  $F(x) = x^2 / 2$. If $x < 0$, then $f(x) = -x$
  and so $F(x) = -x^2 / 2$. We can also write this as
  \[
    F(x) = x \cdot \frac{|x|}{2}.
  \]
  Clearly for $x \ne 0$, we have $F'(x) = f(x)$. At
  $x = 0$, we have
  \[
    \lim_{x \to 0} \frac{F(x) - f(0)}{x} = \lim_{x \to 0}
    \frac{1}{2} |x| = 0,
  \]
  so $F'(0) = f(0)$ and $F$ is an anti-derivative of $f$.
\end{proof}

\begin{remark}
  The eventual goal is to show that any continuous
  function $f : [a, b] \to \R$ has an anti-derivative.
\end{remark}

\begin{example}
  Find an anti-derivative for
  \[
    f(x) =
    \begin{cases}
      1 & \text{if } x > 0\\
      0 & \text{if } x \le 0 \\
      -1 & \text{if } x < 0.
    \end{cases}
  \]
\end{example}

\begin{proof}
  We can try to use $F(x) = |x|$, but recall that $F$ is
  not differentiable at $x = 0$. More generally, suppose
  that $f(x)$ has some anti-derivative $F(x)$, i.e.
  $f(x) = F'(x)$. By Darboux's theorem, $f(x)$ must
  take all values in $(-1, 1)$, which is a contradiction
  with the definition of $f$.
\end{proof}

\begin{remark}
  If $f(x)$ has a jump discontinuity, then it has no
  anti-derivative.
\end{remark}

\section{The Riemann Integral}
Recall from calculus that if $f(x)$ is defined in
$[a, b]$ and $F'(x) = f(x)$, then we have\footnote{This is the \emph{fundamental theorem of calculus}.}
\[
  \int_a^b f(x)\, dx = F(x) \Big|_a^b = F(b) - F(a).
\]
We called this the \emph{definite integral} of $f$ in
calculus, but we would like a more rigorous definition.

\begin{definition}
  Let $a, b \in \R$ and $a < b$. A \emph{partition} of
  the interval $[a, b]$ is a finite sequence of
  numbers $x_0, x_1, \dots, x_n$ such that
  $a = x_0 < x_1 < \dots < x_n = b$.
\end{definition}

\begin{definition}
  The \emph{width} of a partition $x_0, x_1 , \dots, x_n$ is
  $\max\{x_i - x_{i - 1} : i = 1, 2, \dots, n\}$.
\end{definition}

\begin{definition}
  For any partition $x_0, x_1, \dots, x_n$, define the
  \emph{Riemann sum} to be
  \[
    S = \sum_{i = 1}^n f(x_i') (x_i - x_{i - 1}),
  \]
  where $x_i'$ is any point between $x_{i - 1}$ and
  $x_i$, inclusive.\footnote{The geometric intuition of the Riemann sum is an approximation for the \emph{area} under the graph of $f$ by rectangles.}
\end{definition}

\begin{definition}
  Let $a, b \in \R$ with $a < b$ and $f : [a, b] \to \R$.
  We say $f$ is \emph{Riemann integrable} on $[a, b]$ if
  there exists $A \in \R$ such that for all
  $\epsilon > 0$, there exists $\delta > 0$ such that
  $|S - A| < \epsilon$ whenever $S$ is any Riemann
  sum for a partition of $[a, b]$ with width less than
  $\delta$. We call $A$ the \emph{Riemann integral} of
  $f$ on $[a, b]$ and denote it by
  \[
    A = \int_a^b f(x)\, dx.
  \]
\end{definition}

\begin{remark}
  If $f$ is Riemann integrable, then
  \[
    A = \int_a^b f(x)\, dx
  \]
  is unique. This is because if $A$ and $A'$ are two
  numbers for the Riemann integral, then for any
  $\epsilon > 0$, there exists $\delta > 0$ such that
  \[
    |A - S| < \epsilon \quad \text{and} \quad
    |A' - S| < \epsilon
  \]
  for any Riemann sum $S$ associated with a partition of
  width less than $\delta$. Then
  \[
    |A - A'| \le |A - S| + |A' - S| < 2\epsilon,
  \]
  so $A = A'$ and thus the Riemann integral is unique.
\end{remark}

\begin{example}
  Let $f(x) = c$ on $[a, b]$, a constant function. Then for any partition $x_0, x_1, \dots, x_n$,
  \[
    S = \sum_{i = 1}^n f(x_i') (x_i - x_{i - 1}) =
    \sum_{i = 1}^n c (x_i - x_{i - 1}) = c(b - a)
    \implies \int_a^b c\, dx = c(b - a).
  \]
\end{example}

\begin{example}
  Fix $\xi \in [a, b]$ and let $f : [a, b] \to \R$
  be defined by
  \[
    f(x) = \begin{cases}
      0 & \text{if } x \ne \xi \\
      c & \text{if } x = \xi.
    \end{cases}
  \]
  Check that
  \[
    A = \int_a^b f(x)\, dx = 0.
  \]
\end{example}

\begin{proof}
  For any partition $a = x_0 < x_1 < \dots < x_n = b$
  with width $\delta$, we have
  \[
    |S| = \left|\sum_{i = 1}^n f(x_i') (x_i - x_{i - 1})\right|
    \le |c| 2\delta
  \]
  since $\xi$ can be in at most two of the intervals of
  the partition. Then for any $\epsilon > 0$,
  choose $\delta = \epsilon / (2|c|)$, so that
  $|S| < \epsilon$ for any partition of width less
  than $\delta$. From this we can conclude that $A = 0$.
\end{proof}

\begin{example}
  Consider a step function.
  Let $\alpha, \beta \in [a, b]$ with $\alpha < \beta$.
  Define $f : [a, b] \to \R$ by
  \[
    f(x) = \begin{cases}
      1 & \text{if } x \in (\alpha, \beta) \\
      0 & \text{if } x \notin (\alpha, \beta) \text{ and } x \in [a, b].
    \end{cases}
  \]
  Note that $f$ has no anti-derivative, but it
  is Riemann integrable. In fact,
  \[
    \int_a^b f(x)\, dx = \beta - \alpha.
  \]
  To see this, take any partition
  $a = x_0 < x_1 < \dots < x_n = b$ with width less
  than $\delta$. Then
  \[
    S = \sum_{i = 1}^n f(x_i') (x_i - x_{i - 1}) =
    \sum_{[x_{i - 1}, x_i] \cap [\alpha, \beta] \ne \varnothing} f(x_i') (x_i - x_{i - 1}).
  \]
  Each partition is in two classes: Either (1) it
  only partially intersects $[\alpha, \beta]$ or (2)
  it is contained in $[\alpha, \beta]$. So
  \[
    S = \underbrace{1 (\text{total length of intervals of class 2})}_{I_1}
    + \underbrace{|f(x_i')| (\text{total length of intervals of class 1})}_{I_2}.
  \]
  We have $|I_1 - (\beta - \alpha)| < 2 \delta$ and
  $|I_2| < 2\delta$ since there are at most two intervals
  of class 1. So
  \[
    |S - (\beta - \alpha)| \le |I_1| + |I_2| < 4\delta.
  \]
  So $f(x)$ is Riemann integrable and
  \[
    \int_a^b f(x)\, dx = \beta - \alpha,
  \]
  as desired.
\end{example}

\begin{example}
  Define $f : [a, b] \to \R$ by
  \[
    f(x) = \begin{cases}
      1 & \text{if $x$ is rational} \\
      0 & \text{if $x$ is irrational}.
    \end{cases}
  \]
  Then $f(x)$ is not Riemann integrable. For
  any partition $a = x_0 < x_1 < \dots < x_n = b$,
  \[
    S = \sum_{i = 1}^n f(x_i') (x_i - x_{i - 1}) =
    \begin{cases}
      b - a & \text{if $x_i'$ are all rational} \\
      0 & \text{if $x_i'$ are all irrational}.
    \end{cases}
  \]
  We can always choose $x_i'$ to be in either case
  since the rationals and irrationals are both dense
  in $\R$. So there is no $A \in \R$ such that
  $|A - S| < \epsilon$, no matter how small we take
  $\delta$ to be.
\end{example}

\begin{remark}
  The function $f$ from the previous example is not
  Riemann integrable, but it is Lebesgue integrable.
  In fact,
  \[
    L = \int_a^b f(x)\, dx = 0
  \]
  with respect to the Lebesgue measure. This
  is because the set of rational numbers
  $\Q$ has measure zero.
\end{remark}

\section{Properties of the Riemann Integral}
\begin{prop}
We have the following linearity properties of the Riemann integral:
\begin{enumerate}
  \item If $f, g : [a, b] \to \R$ are Riemann integrable, then
  $f \pm g$ are also integrable and
  \[
    \int_a^b (f \pm g) \, dx = \int_a^b f(x)\, dx \pm \int_a^b g(x)\, dx
  \]
  \item For any $c \in \R$, $cf$ is integrable
    and
    \[
      \int_a^b cf \, dx = c \int_a^b f(x)\, dx.
    \]
\end{enumerate}
\end{prop}

\begin{proof}
  See textbook, fairly straightforward.
\end{proof}

\begin{remark}
  Since we only discuss Riemann integration in this
  class, we will sometimes simply say ``integrable''
  instead of ``Riemann integrable.''
\end{remark}

\begin{prop}
  If $f : [a, b] \to \R$ is integrable and $f(x) \ge 0$,
  then
  \[
    \int_a^b f(x)\, dx \ge 0.
  \]
\end{prop}

\begin{proof}
  Let
  \[A = \int_a^b f(x)\, dx.\]
  Then for any $\epsilon > 0$, there exists $\delta > 0$
  such that for any partition of width $< \delta$,
  we have $|A - S| < \epsilon$. But
  \[
    S = \sum_{i = 1}^n f(x_i') (x_i - x_{i - 1}) \ge 0,
  \]
  Then we have $A > S - \epsilon \ge -\epsilon$, so
  taking $\epsilon \to 0$ gives $A \ge 0$.
\end{proof}

\begin{corollary}
  If $f, g : [a, b] \to \R$ are integrable and
  $f(x) \ge g(x)$ for all $x \in [a, b]$, then
  \[
    \int_a^b f(x)\, dx \ge \int_a^b g(x)\, dx.
  \]
\end{corollary}

\begin{proof}
  By linearity,
  \[
    \int_a^b f(x)\, dx - \int_a^b g(x)\, dx =
    \int_a^b (f(x) - g(x))\, dx \ge 0
  \]
  since $f(x) - g(x) \ge 0$ by assumption.
\end{proof}

\begin{corollary}
  If $f : [a, b] \to \R$ is integrable and
  $m \le f(x) \le M$ for all $x \in [a, b]$, then
  \[
    m(b - a) \le \int_a^b f(x)\, dx \le M(b - a).
  \]
\end{corollary}

  \chapter{Jan.~25 --- Riemann Integrability}

\section{Conditions for Integrability}
\begin{lemma}
  \label{lem:first-integrable}
  A function $f : [a, b] \to \R$ is integrable if and only if
  for any $\epsilon > 0$, there exists $\delta > 0$
  such that $|S_1 - S_2| < \epsilon$ whenever $S_1$
  and $S_2$ are Riemann sums for partitions of
  width less than $\delta$.
\end{lemma}

\begin{proof}
  $(\Rightarrow)$ If $f$ is integrable, then for
  any $\epsilon > 0$, there exists $\delta > 0$ such that
  \[
    \left|S - \int_a^b f(x)\, dx\right| < \frac{\epsilon}{2}
  \]
  for any Riemann sum $S$ of a partition with width
  less than $\delta$. Then
  \[
    |S_1 - S_2| \le
    \left|S_1 - \int_a^b f(x)\, dx\right| +
    \left|S_2 - \int_a^b f(x)\, dx\right| <
    \frac{\epsilon}{2} + \frac{\epsilon}{2} = \epsilon,
  \]
  as desired.

  $(\Leftarrow)$ Take the special partition into intervals
  of equal length, with width $(a - b) / n$. Pick
  the middle point in each interval, and let
  \[
    S_n = \sum_{i = 1}^n f(x_i') (x_i - x_{i - 1})
  \]
  be the corresponding Riemann sum. Now we check that
  $\{S_n\}_{n = 1}^\infty$ is a Cauchy sequence. This is
  because for any $\epsilon > 0$, if $N$ is large
  enough, then for any $n, m \ge N$, we have
  $|S_n - S_m| < \epsilon$ if $1 / N < \delta$. Then
  $\{S_n\}_{n = 1}^\infty$ converges, so let
  $\lim_{n \to \infty} S_n = A$. Now for any
  $\epsilon > 0$, there exists $\delta > 0$ such that
  for any Riemann sum $S$ with width $< \delta$,
  if $1 / n < \delta$, then $|S_n - S| < \epsilon / 2$.
  So
  \[
    |S - A| \le |S_n - S| + |S_n - A|
    < \frac{\epsilon}{2} + \frac{\epsilon}{2} = \epsilon,
  \]
  if $n$ is large enough. Thus
  \[
    A = \int_a^b f(x)\, dx
  \]
  exists and is the Riemann integral of $f$.
\end{proof}

\begin{remark}
Recall the step function $f : [a, b] \to \R$ given by
\[
  f(x) =
  \begin{cases}
    1 & \text{if } x \in (\alpha, \beta) \subseteq [a, b] \\
    0 & \text{if } x \notin (\alpha, \beta).
  \end{cases}
\]
Last time we saw that $f$ is integrable and that
\[
  \int_a^b f(x)\, dx = \beta - \alpha.
\]
Now let us consider a more general step function. We
call $f$ a \emph{step function} on $[a, b]$ if there
exists
a partition $x_0 < x_1 < \dots < x_n$ of $[a, b]$ such
that $f(x)$ is constant on each subinterval
$(x_{i - 1}, x_i)$.
\end{remark}

\begin{lemma}
  If $f : [a, b] \to \R$ is a step function for
  a partition $x_0 < x_1 < \dots < x_n$ and
  $f(x) = c_i$ when $x \in (x_{i - 1}, x_i)$, then
  $f$ is integrable and
  \[
    \int_a^b f(x)\, dx = \sum_{i = 1}^n c_i (x_i - x_{i - 1}).
  \]
\end{lemma}

\begin{proof}
  Define
  \[
    \varphi_i(x) = \begin{cases}
      1 & \text{if } x \in (x_{i - 1}, x_i) \\
      0 & \text{otherwise}.
    \end{cases}
  \]
  Now let
  \[
    h = f - \sum_{i = 1}^n c_i \varphi_i.
  \]
  Then $h(x)$ is nonzero only at $\{x_i\}_{i = 0}^n$.
  Each $\varphi_i$ is integrable and $h$ is integrable
  with
  \[
    \int_a^b h(x)\, dx = 0,
  \]
  so $f$ is also integrable and
  \[
    \int_a^b f(x)\, dx
    = \sum_{i = 1}^{n} c_i \int_a^b \varphi_i(x)\, dx
    = \sum_{i = 1}^n c_i (x_i - x_{i - 1})
  \]
  by linearity and the integral of a simple
  step function that we calculated before.
\end{proof}

\begin{prop}
  A function $f : [a, b] \to \R$ is integrable if
  and only if for any $\epsilon > 0$, there exist
  step functions $f_1, f_2$ such that
  $f_1(x) \le f(x) \le f_2(x)$ for all $x \in [a, b]$
  and
  \[
    \int_a^b (f_2 - f_1)\, dx < \epsilon.
  \]
\end{prop}

\begin{proof}
  $(\Leftarrow)$ For any $\epsilon > 0$, choose
  step functions $f_1, f_2$ such that
  \[
    \int_a^b (f_2 - f_1)\, dx < \frac{\epsilon}{3}.
  \]
  Then there exists $\delta > 0$ such that for any
  partition with width $< \delta$, the Riemann
  sums $S_1, S_2$ for $f_1, f_2$ satisfy
  \[
    |S_1 - \int_a^b f_1(x)\, dx| < \frac{\epsilon}{3}
    \quad\text{and}\quad
    |S_2 - \int_a^b f_2(x)\, dx| < \frac{\epsilon}{3}.
  \]
  So for any partition width $< \delta$, the Riemann sum
  of $f$ is
  \[
    S = \sum_{i = 1}^n f(x_i') (x_i - x_{i - 1}),
  \]
  and $S_1 \le S \le S_2$ since
  \[
    S_1 = \sum_{i = 1}^n f_1(x_i') (x_i - x_{i - 1})
    \quad \text{and} \quad
    S_2 = \sum_{i = 1}^n f_2(x_i') (x_i - x_{i - 1}).
  \]
  So $S$ is in the interval $(S_1, S_2)$, which has
  length $< \epsilon$ by the triangle inequality on the
  previous results. For any two Riemann sums of
  $f$ with partitions of width $< \delta$, we have
  $|S' - S''| < \epsilon$. Thus $f$ is integrable.

  $(\Rightarrow)$ First we show that $f$ is bounded
  in $[a, b]$. This is because for any $\epsilon > 0$,
  there exists $\delta > 0$ such that any two Riemann
  sums $S_1, S_2$ corresponding to partitions of
  width $< \delta$.
  satisfy $|S_1 - S_2| < \epsilon$. Let
  \[
    S_1 = \sum_{i = 1}^n f(x_i') (x_i - x_{i - 1}),
  \]
  and replace $x_{i_0}' \in (x_{i_0 - 1}, x_{i_0})$
  with $x_{i_0}'' \in (x_{i_0 - 1}, x_{i_0})$. Keep
  $x_i'$ for $i \ne i_0$. Define this new Riemann sum
  to be $S_2$. Then
  \[
    |S_2 - S_1| \le |f(x_{i_0}'') - f(x_{i_0}')| |x_{i_0} - x_{i_0 - 1}|
    < \epsilon,
  \]
  so that
  \[
    |f(x_{i_0}'')| \le |f(x_{i_0}')| + \frac{\epsilon}{x_{i_0} - x_{i_0 - 1}},
  \]
  i.e. $f$ is bounded in $(x_{i_0 - 1}, x_{i_0})$ since
  $x_{i_0}''$ was arbitrary. Since we also picked $i_0$
  arbitrarily, we can repeat this for any
  interval to conclude that $f$ is bounded in $[a, b]$.

  Now for any partition $x_0 < x_1 < \dots < x_n$
  with width $< \delta$, define
  \[
    m_i = \inf \{f(x) : x \in (x_{i - 1}, x_i)\}
    \quad \text{and} \quad
    M_i = \sup \{f(x) : x \in (x_{i - 1}, x_i)\}.
  \]
  Define the step function
  \[
    f_1(x) =
    \begin{cases}
      m_i & \text{if } x \in (x_{i - 1}, x_i) \\
      \min\{m_1, \dots, m_n\} & \text{if } x = x_i \text{ for } i = 0, \dots, n.
    \end{cases}
  \]
  Similarly define
  \[
    f_2(x) =
    \begin{cases}
      M_i & \text{if } x \in (x_{i - 1}, x_i) \\
      \max\{M_1, \dots, M_n\} & \text{if } x = x_i \text{ for } i = 0, \dots, n.
    \end{cases}
  \]
  Observe that $f_1(x) \le f(x) \le f_2(x)$ for
  any $x \in [a, b]$ by construction. Now we verify
  that
  \[
    \int_a^b (f_2 - f_1)\, dx < \epsilon
  \]
  if $\delta > 0$ is small enough. This is because
  for any $\eta > 0$, there exists $x_i', x_i'' \in [x_{i - 1}, x_i]$
  such that $f(x_i') < m_i + \eta$ and
  $f(x_i'') > M_i - \eta$. Then
  \[
    \sum_{i = 1}^n (f(x_i'') - f(x_i'))(x_i - x_{i - 1})
    > \sum_{i = 1}^n (M_i - m_i - 2\eta)(x_i - x_{i - 1})
    = \int_a^b (f_2 - f_1)\, dx - 2\eta(b - a).
  \]
  If $\delta > 0$ is small enough, then
  \[
    \sum_{i = 1}^n (f(x_i'') - f(x_i'))(x_i - x_{i - 1})
    < \epsilon
  \]
  since this a difference of two Riemann sums with
  partitions of width $< \delta$. Thus
  \[
    \int_a^b (f_2 - f_1)\, dx < \epsilon + 2\eta(b - a).
  \]
  But $\eta$ was arbitrary, so taking $\eta \to 0$
  gives the desired result.
\end{proof}

\begin{corollary}
  If $f : [a, b] \to \R$ is integrable, then it is
  bounded.
\end{corollary}

\begin{proof}
  This was shown in the proof of the previous proposition.
\end{proof}

\begin{theorem}
  If $f : [a, b] \to \R$ is continuous, then $f$ is
  integrable.
\end{theorem}

\begin{proof}
  Since $f$ is continuous on the compact set $[a, b]$,
  it is uniformly continuous. So for any $\epsilon > 0$,
  there exists $\delta > 0$ such that for any
  $x', x'' \in [a, b]$, we have
  $|f(x') - f(x'')| < \epsilon$ whenever
  $|x' - x''| < \delta$. Now let $S_1, S_2$
  be two Riemann sums
  with partitions of width $< \delta$. Assume without
  loss of generality that $S_1, S_2$ are defined
  over the same partition (we can always combine two
  partitions to give a finer partition, if necessary).
  Let
  \[
    S_1 = \sum_{i = 1}^n f(x_i') (x_i - x_{i - 1})
    \quad \text{and} \quad
    S_2 = \sum_{i = 1}^n f(x_i'') (x_i - x_{i - 1}).
  \]
  Then
  \[
    |S_1 - S_2| \le \sum_{i = 1}^n |f(x_i') - f(x_i'')| (x_i - x_{i - 1})
    < \epsilon \sum_{i = 1}^n (x_i - x_{i - 1})
    = \epsilon (b - a).
  \]
  Since $\epsilon > 0$ was arbitrary, we conclude that
  that $f$ is integrable by Lemma \ref{lem:first-integrable}.
\end{proof}

\section{The Fundamental Theorem of Calculus}
\begin{theorem}[Fundamental theorem of calculus]
  If $f : [a, b] \to \R$ has anti-derivative
  $F : [a, b] \to \R$ and $f \in \mathcal{R}([a, b])$,\footnote{Here $\mathcal{R}([a, b])$ is the class of Riemann integrable functions on $[a, b]$.}
  then
  \[
    \int_a^b f(x)\, dx = F(b) - F(a).
  \]
\end{theorem}

\begin{proof}
  Since $f$ is integrable, let
  \[
    A = \int_a^b f(x)\, dx.
  \]
  For any $\epsilon > 0$, there exists $\delta > 0$
  such that for any Riemann sum $S$ with partition of
  width $< \delta$, we have $|S - A| < \epsilon$. Let
  $x_0 < x_1 < \dots < x_n$ be a partition of
  width $< \delta$. Then by telescoping,
  \[
    F(b) - F(a) = \sum_{i = 1}^n (F(x_i) - F(x_{i - 1}))
    = \sum_{i = 1}^n f(x_i') (x_i - x_{i - 1})
  \]
  by Lagrange's mean value theorem, where
  $x_i' \in (x_{i - 1}, x_i)$. Then
  \[
    |F(b) - F(a) - A| = |S - A| < \epsilon,
  \]
  so letting $\epsilon \to 0$ gives $F(b) - F(a) = A$.
\end{proof}

\begin{remark}
  The fundamental theorem of calculus requires both
  being Riemann integrable and having an anti-derivative,
  which do not always overlap. In fact, neither is
  a subset of the other.
\end{remark}

\begin{example}
  The step function
  \[
    f(x) =
    \begin{cases}
      -1 & \text{if } 0 \le x \le 1 \\
      1 & \text{if } 1 < x \le 2
    \end{cases}
  \]
  is integrable but has no anti-derivative.
\end{example}

\begin{example}
  Define
  \[
    F(x) =
    \begin{cases}
      0 & \text{if } x = 0 \\
      x^2 \sin(1/x) & \text{if } x \ne 0.
    \end{cases}
  \]
  Then we have
  \[
    F'(x) = f(x) =
    \begin{cases}
      0 & \text{if } x = 0 \\
      (-2 / x)(\cos(1 / x^2)) + 2x \sin(1 / x^2) & \text{if } x \ne 0.
    \end{cases}
  \]
  We can check that $F'(0) = 0$ via the definition
  of the derivative. Note that $f$ has an anti-derivative,
  namely $F$. However, $f$ is not integrable since
  it is not bounded near $x = 0$.
\end{example}

  \chapter{Jan.~30 --- More Integrability}

\section{Conditions for an Anti-Derivative}
\begin{lemma}
  \label{lem:split-integral}
  Let $c \in (a, b)$. Then $f \in \mathcal{R}([a, b])$ if and only if $f \in \mathcal{R}([a, c])$ and $f \in \mathcal{R}([c, b])$. Moreover,
  \[
    \int_a^b f(x) \, dx = \int_a^c f(x) \, dx + \int_c^b f(x) \, dx. \tag{$*$}
  \]
\end{lemma}

\begin{proof}
  $(\Rightarrow)$ If $f \in \mathcal{R}([a, b])$, then
  for any $\epsilon > 0$, there exist two step
  functions $f_1, f_2$ such that $f_1 \le f \le f_2$
  and
  \[
    \int_a^b (f_2 - f_1) \, dx < \epsilon.
  \]
  Let $f_1, f_2$ be the restrictions to $[a, c]$.
  Then still $f_1 \le f \le f_2$ on $[a, c]$ and
  \[
    \int_a^c (f_2 - f_1)\, \le
    \int_a^b (f_2 - f_1) < \epsilon
  \]
  since $f_2 - f_1$ is a nonnegative step function. (Note
  that the desired result is easy to verify for step
  functions.)
  So $f \in \mathcal{R}([a, c])$, and the same
  argument works to show that $f \in \mathcal{R}([c, b])$.

  $(\Leftarrow)$ If $f \in \mathcal{R}([a, c])$ and
  $f \in \mathcal{R}([c, b])$, then for any
  $\epsilon > 0$, there exist step functions
  $g_1, g_2, h_1, h_2$ such that
  $g_1 \le f \le g_2$ on $[a, c]$,
  $h_1 \le f \le h_2$ on $[c, b]$, and
  \[
    \int_a^c (g_2 - g_1)\, dx < \epsilon,
    \quad
    \int_c^b (h_2 - h_1)\, dx < \epsilon.
  \]
  Now define
  \[
    f_i = \begin{cases}
      g_i & \text{if } x \in [a, c) \\
      h_i & \text{if } x \in [c, b]
    \end{cases}
  \]
  for $i = 1, 2$. Then $f_1 \le f \le f_2$ on $[a, b]$,
  and
  \[
    \int_a^b (f_2 - f_1)\, dx
    = \int_a^c (g_2 - g_1)\, dx + \int_c^b (h_2 - h_1)\, dx
    < 2\epsilon,
  \]
  so $f \in \mathcal{R}([a, b])$. Now to prove $(*)$,
  note that $f \in \mathcal{R}([a, c])$, so for
  any $\epsilon > 0$ there exist Riemann sums $S_1$
  on $[a, c]$ and $S_2$ on $[c, b]$ such that
  \[
    |S_1 - \int_a^c f(x)\, dx| < \frac{\epsilon}{3},
    \quad
    |S_2 - \int_c^b f(x)\, dx| < \frac{\epsilon}{3}.
  \]
  Now choose $\delta > 0$ such that if the Riemann
  sum $S$ has partition with
  width $< \delta$, then
  \[
    |S - \int_a^c f(x)\, dx| < \frac{\epsilon}{3},
    \quad
    |S - \int_c^b f(x)\, dx| < \frac{\epsilon}{3},
    \quad
    |S - \int_a^b f(x)\, dx| < \frac{\epsilon}{3}.
  \]
  Now combine $S_1, S_2$ on $[a, b]$ to be a Riemann
  sum $S = S_1 + S_2$, so that
  \[
    |S - \int_a^b f(x)\, dx| < \frac{\epsilon}{3}.
  \]
  By the triangle inequality on the previous results,
  \[
    \left| \int_a^b f(x)\, dx - \left(\int_a^c f(x)\, dx + \int_c^b f(x)\, dx\right)\right| < \epsilon.
  \]
  Since $\epsilon$ is arbitrarily small, we conclude that
  \[
    \int_a^b f(x)\, dx = \int_a^c f(x)\, dx + \int_c^b f(x)\, dx
  \]
  as desired.
\end{proof}

\begin{remark}
  The formula $(*)$ is true for any three numbers
  $a, b, c$, as long as $f$ is integrable. This is
  because by convention, if $a > b$, then
  \[
    \int_a^b f(x)\, dx = -\int_b^a f(x)\, dx.
  \]
\end{remark}

\begin{theorem}
  If $f : [a, b] \to \R$ is continuous, then\footnote{Note that this integral is well-defined since any continuous function is integrable, and a continuous function restricted to a subset of its domain, i.e. $[a, x] \subseteq [a, b]$, remains continuous.}
  \[
    F(x) = \int_a^x f(\xi) \, d\xi
  \]
  is an anti-derivative of $f$.
\end{theorem}

\begin{proof}
  For any $x_0 \in (a, b)$, we check that
  $F'(x_0) = f(x_0)$. We can compute using Lemma
  \ref{lem:split-integral} that
  \begin{align*}
    \left| \frac{F(x_0 + h) - F(x_0)}{h} - f(x_0) \right|
    &= \left| \frac{1}{h} \left( \int_a^{x_0 + h} f(x)\, dx - \int_a^{x_0} f(x)\, dx \right) - f(x_0) \right| \\
    &= \left| \frac{1}{h} \int_{x_0}^{x_0 + h} f(x)\, dx - f(x_0) \right|
    = \left| \frac{1}{h} \int_{x_0}^{x_0 + h} (f(x) - f(x_0))\, dx \right|.
  \end{align*}
  The last step is from observing
  \[
    f(x_0) = \frac{1}{h} \int_{x_0}^{x_0 + h} f(x_0)\, dx.
  \]
  Since $f$ is continuous, for any $\epsilon > 0$, there
  exists $\delta$
  such that if $|x_0 - x| < \delta$, then
  $|f(x) - f(x_0)| < \epsilon$. This gives
  \[
    \left| \frac{1}{h} \int_{x_0}^{x_0 + h} (f(x) - f(x_0))\, dx \right|
    \le \frac{1}{h} \int_{x_0}^{x_0 + h} |f(x) - f(x_0)|\, dx
    \le \frac{\epsilon h}{h} = \epsilon
  \]
  if $|h| < \delta$. Thus,
  \[
    \lim_{h \to 0} \frac{F(x_0 + h) - F(x_0)}{h} - f(x_0) = f(x_0),
  \]
  so we indeed have $F'(x_0) = f(x_0)$.
\end{proof}

\section{More Conditions for Integrability}
\begin{definition}
  Let $f : [a, b] \to \R$ be bounded and
  $x_0 < x_1 < \cdots < x_n$ be a partition of $[a, b]$.
  Define
  \[
    \omega_i = \sup\{|f(x) - f(y)| : x, y \in [x_{i - 1}, x_i)\}
  \]
  for $i = 1, 2, \dots, n$.
\end{definition}

\begin{theorem}
  A function $f : [a, b] \to \R$ is integrable if and only
  if for any $\epsilon > 0$, there exists $\delta > 0$
  such that for any partition with width $< \delta$,
  we have
  \[
    \sum_{i = 1}^n \omega_i \Delta x_i < \epsilon,
  \]
  where $\Delta x_i = x_i - x_{i - 1}$.
\end{theorem}

\begin{proof}
  $(\Leftarrow)$ For any $\epsilon > 0$, choose any
  two Riemann sums $S_1, S_2$ over partitions with
  width $< \delta$. Assume without loss of generality
  that $S_1$ and $S_2$ are defined over the same
  (maybe refined) partition. Let
  \[
    S_1 = \sum_{i = 1}^n f(x_i') (x_i - x_{i - 1}),
    \quad
    S_2 = \sum_{i = 1}^n f(x_i'') (x_i - x_{i - 1}).
  \]
  Then we have
  \[
    |S_1 - S_2| \le \sum_{i = 1}^n |f(x_i') - f(x_i'')| \Delta x_i
    \le \sum_{i = 1}^n \omega_i \Delta x_i < \epsilon.
  \]
  Then by Lemma \ref{lem:first-integrable}, we conclude
  that $f$ is integrable.

  $(\Rightarrow)$ Since $f$ is integrable, by Lemma
  \ref{lem:first-integrable} we have that for any
  $\epsilon > 0$, there eixsts $\delta > 0$ such that
  for any two Riemann sums $S_1, S_2$ over partitions
  of with $< \delta$, we have $|S_1 - S_2| < \epsilon$.
  In the interval $[x_{i - 1}, x_i]$, let
  \[
    M_i = \sup_{x \in [x_{i - 1}, x_i]} |f(x)|, \quad
    m_i = \inf_{x \in [x_{i - 1}, x_i]} |f(x)|.
  \]
  In particular note that $\omega_i = M_i - m_i$.
  Now for any $\eta > 0$, there exist
  $x_i', x_i'' \in [x_{i - 1}, x_i]$ such that
  \[
    f(x_i') > M_i - \eta, \quad f(x_i'') < m_i + \eta.
  \]
  Let
  \[
    S_1 = \sum_{i = 1}^n f(x_i') \Delta x_i,
    \quad
    S_2 = \sum_{i = 1}^n f(x_i'') \Delta x_i.
  \]
  Then we have
  \[
    |S_1 - S_2| \le \left| \sum_{i = 1}^n (f(x_i') - f(x_i'')) \Delta x_i \right|
  \]
  Note that $f(x_i') - f(x_i'') \ge M_i - m_i - 2\eta$
  for $\eta$ sufficiently small. Thus
  \[
    |S_1 - S_2| \ge \sum_{i = 1}^n \omega_i \Delta x_i
    - 2\eta \sum_{i = 1}^n \Delta x_i,
  \]
  so that
  \[
    \sum_{i = 1}^n \omega_i \Delta x_i
    \le |S_1 - S_2| + 2\eta (b - a)
    < \epsilon + 2\eta (b - a).
  \]
  From here letting $\eta \to 0$ gives the desired result.
\end{proof}

\begin{theorem}
  A function $f : [a, b] \to \R$ is integrable if and only
  if for any $\epsilon > 0$, there exists a partition
  such that
  \[
    \sum_{i = 1}^n \omega_i \Delta x_i < \epsilon.
  \]
\end{theorem}

\begin{proof}
  $(\Rightarrow)$ This is immediate from the
  previous theorem.

  $(\Leftarrow)$ Let $S_1$ be the given sum
  and
  \[
    S_2 = \sum_{i = 1}^n \omega_i \Delta x_i
  \]
  be any other Riemann sum over a partition of
  width $< \delta$. Then $S_2 \le 2S_1 < 2\epsilon$ at
  least since we will have
  $\omega_i' \le \omega_i + \omega_{i - 1}$ if $\omega_i'$
  is the analogous value corresponding to $S_2$.
\end{proof}

  \chapter{Feb.~1 --- Riemann Integrability, Part 3}

\section{Even More Conditions for Integrability}
\begin{example}
  If $f(x)$ is monotone on $[a, b]$, then
  $f \in \mathcal{R}([a, b])$.
\end{example}

\begin{proof}
  Suppose $f(x)$ is monotone increasing on
  $[a, b]$ and $f(x)$ is not constant (since the result
  is trivial if $f$ is constant). Then
  $f(a) \le f(x) \le f(b)$. For any $\epsilon > 0$,
  for any partition $x_0 < \dots < x_n$ with width
  \[
    \delta < \frac{\epsilon}{f(b) - f(a)},
  \]
  we have on $[x_{i - 1}, x_i]$ that
  $M_i = f(x_i)$ and $f(x_{i - 1}) = m_i$ since
  $f$ is monotone. Then
  \[
    \omega_i(f) = f(x_i) - f(x_{i - 1}) = M_i - m_i.
  \]
  Thus
  \[
    \sum_{i = 1}^n \omega_i(f) \Delta x_i
    = \sum_{i = 1}^n (f(x_i) - f(x_{i - 1})) \Delta x_i
    < \frac{\epsilon}{f(b) - f(a)} \sum_{i = 1}^n (f(x_i) - f(x_{i - 1})) = \epsilon
  \]
  since the sum telescopes and comes out to
  $f(b) - f(a)$. Thus $f$ is integrable.
\end{proof}

\begin{theorem}[Du Bois-Reymond]
  Let $f$ be bounded on $[a, b]$. Then $f \in \mathcal{R}([a, b])$
  if and only if for any $\epsilon, a > 0$,
  there exists a partition such that
  the total length of subintervals with
  $\{\omega_i(f) \le \epsilon\}$ is $< a$.
\end{theorem}

\begin{proof}
  For any partition $x_0 < \dots < x_n$, split
  \[
    \sum_{i = 1}^n \omega_i(f) \Delta x_i =
    \sum_{(A)} \omega_i(f) \Delta x_i +
    \sum_{(B)} \omega_i(f) \Delta x_i
  \]
  where $(A)$ is over subintervals with width
  $\omega_i(f) < \epsilon$ and $(B)$ is over
  subintervals with width $\omega_i(f) \ge \epsilon$.

  $(\Rightarrow)$ Let
  \[
    \Omega = \sup_{x, y \in [a, b]} |f(x) - f(y)|.
  \]
  For any $\epsilon > 0$, for
  \[
    \epsilon_1 = \frac{\epsilon}{2(b - a)} \quad \text{and} \quad a = \frac{\epsilon}{2\Omega},
  \]
  by assumption there exists a partition
  $x_0 < \dots < x_n$ such that
  \begin{align*}
    \sum_{i = 1}^n \omega_i(f) \Delta x_i
    &= \sum_{(A)} \omega_i(f) \Delta x_i +
    \sum_{(B)} \omega_i(f) \Delta x_i \\
    &< \frac{\epsilon}{2(b - a)} \sum_{(a)} \Delta x_i
    + \Omega \sum_{(B)} \Delta x_i
    < \frac{\epsilon}{2(b - a)}(b - a) + \Omega \frac{\epsilon}{2\Omega} = \epsilon.
  \end{align*}
  So we see that $f \in \mathcal{R}([a, b])$ as desired.

  $(\Rightarrow)$ If $f \in \mathcal{R}([a, b])$, then
  for any $\epsilon , a > 0$, there exists a partition
  $x_0 < \dots < x_n$ such that
  \[
    \sum_{i = 1}^n \omega_i(f) \Delta x_i < a\epsilon.
  \]
  Then we have
  \[
    \epsilon \sum_{(B)} \Delta x_i
    \le \sum_{(B)} \omega_i(f) \Delta x_i
    < a\epsilon
    \implies \sum_{(B)} \Delta x_i < a,
  \]
  which shows the desired result.
\end{proof}

\begin{corollary}
  If $f : [a, b] \to \R$ is bounded and has
  only finitely many discontinuity points, then
  $f \in \mathcal{R}([a, b])$.
\end{corollary}

\begin{proof}
  Suppose $f(x)$ has $p$ discontinuity points on
  $[a, b]$ and $m \le f(x) \le M$ for all $x \in [a, b]$.
  Then for any $\epsilon > 0$, first (1) we construct $p$
  small open intervals on $[a, b]$ containing the $p$
  discontinuity points with
  \[
    \text{total length} < \frac{\epsilon}{2(M - m)}.
  \]
  Next (2) for any subintervals in $[a, b]$ excluding the above
  $p$ subintervals, $f$ is continuous on them, so
  there exists a partition such that
  \[
    \sum_{(2)} \omega_i(f) \Delta x_i < \frac{\epsilon}{2}.
  \]
  Now combine (1) and (2) to get
  \[
    \sum_{i = 1}^n \omega_i(f) \Delta
    = \sum_{(1)} \omega_i(f) \Delta x_i
    + \sum_{(2)} \omega_i(f) \Delta x_i
    < (M - m) \frac{\epsilon}{2(M - m)} + \frac{\epsilon}{2} = \epsilon.
  \]
  Thus $f \in \mathcal{R}([a, b])$, as desired.
\end{proof}

\begin{example}
  Consider
  \[
    f(x) =
    \begin{cases}
      \sin(1 / x) & \text{if } x \ne 0 \\
      A & \text{if } x = 0
    \end{cases}
  \]
  for any constant $A \in \R$. Then by the previous
  corollary, $f \in \mathcal{R}([0, 1])$.
\end{example}

\begin{theorem}
  If $f, g \in \mathcal{R}([a, b])$, then
  $fg \in \mathcal{R}([a, b])$.
\end{theorem}

\begin{proof}
  Since $f, g$ are integrable, they are bounded.
  So assume $|f|, |g| \le M$. Then for any $\epsilon > 0$,
  there exists $\delta > 0$ such that for any
  partition of width $< \delta$, we have
  \[
    \sum_{i = 1}^n \omega_i(f) \Delta x_i <
    \frac{\epsilon}{2M}, \quad
    \sum_{i = 1}^n \omega_i(g) \Delta x_i <
    \frac{\epsilon}{2M}.
  \]
  Notice
  \[
    \omega_i(fg) \le M(\omega_i(f) + \omega_i(g))
  \]
  because
  \begin{align*}
    |f(x) g(x) - f(y) g(y)
    &\le |g(x)| |f(x) - f(y)| + |f(y)| |g(x) - g(y)| \\
    &\le M(|f(x) - f(y)| + |g(x) - g(y)|).
  \end{align*}
  Taking suprememes over $x, y \in [x_{i - 1}, x_i]$ from
  here
  gives $\omega_i(fg) \le M(\omega_i(f) + \omega_i(g))$.
  Then
  \[
    \sum_{i = 1}^n \omega_i(fg) \Delta x_i
    \le M \left(\sum_{i = 1}^n \omega_i(f) \Delta x_i + \sum_{i = 1}^n \omega_i(g) \Delta x_i\right)
    < M\left(\frac{\epsilon}{2M} + \frac{\epsilon}{2M}\right) = \epsilon.
  \]
  Thus $fg \in \mathcal{R}([a, b])$ as desired.
\end{proof}

\begin{theorem}
  If $f \in \mathcal{R}([a, b])$, then
  $|f| \in \mathcal{R}([a, b])$ and
  \[
    \left| \int_a^b f(x) \, dx \right| \le \int_a^b |f(x)| \, dx.
  \]
\end{theorem}

\begin{proof}
  Since $f \in \mathcal{R}([a, b])$, for any
  $\epsilon > 0$ there exists a partition
  $x_0 < \dots < x_n$ such that
  \[
    \sum_{i = 1}^n \omega_i(f) \Delta x_i < \epsilon.
  \]
  Since
  \[
    \big||f(x)| - |f(y)|\big| \le |f(x) - f(y)|,
  \]
  taking supremums over $x, y \in [x_{i - 1}, x_i]$
  gives $\omega_i(|f|) \le \omega_i(f)$. Then
  \[
    \sum_{i = 1}^n \omega_i(|f|) \Delta x_i
    \le \sum_{i = 1}^n \omega_i(f) \Delta x_i < \epsilon.
  \]
  So we indeed have $|f| \in \mathcal{R}([a, b])$.
  Now observe that $-|f| \le f \le |f|$. After
  integrating, we get
  \[
    -\int_a^b |f(x)| \, dx \le \int_a^b f(x) \, dx \le \int_a^b |f(x)| \, dx.
  \]
  This immediately implies the desired result.
\end{proof}

\begin{example}[Cauchy-Schwarz]
  If $f, g \in \mathcal{R}([a, b])$, then
  \[
    \left| \int_a^b f(x) g(x) \, dx \right|
    \le \left( \int_a^b f(x)^2 \, dx \right)^{1/2}
    \left( \int_a^b g(x)^2 \, dx \right)^{1/2}. \tag{$*$}
  \]
\end{example}

\begin{proof}
  Let
  \[
    A = \int_a^b f^2 \, dx, \quad
    B = \int_a^b |fg| \, dx,\quad
    C = \int_a^b g^2 \, dx.
  \]
  Note that it suffices to
  show that $B^2 \le AC$, which will imply $(*)$
  by the previous theorem. Then
  \[
    0 \le \int_a^b (t|f| - |g|)^2\, dx
    = A t^2 - 2Bt + C
  \]
  for any $t \in \R$. So the discriminant must satisfy
  $(2B)^2 - 4AC \le 0$, which gives $B^2 \le AC$
  as desired.
\end{proof}

\begin{example}[Riemann-Lebesgue lemma]
  If $f \in \mathcal{R}([a, b])$, then
  \[
    \lim_{\lambda \to \infty} \int_a^b f(x) \sin(\lambda x) \, dx = 0.
  \]
\end{example}

\begin{proof}
  Since $f \in \mathcal{R}([a, b])$, for any
  $\epsilon > 0$ there exists a partition
  $x_0 < \dots < x_n$ of $[a, b]$ such that
  \[
    \sum_{i = 1}^n \omega_i(f) \Delta x_i < \frac{\epsilon}{2}.
  \]
  Also assume $|f| \le M$ on $[a, b]$ since $f$ is
  integrable. Then we choose
  \[
    \lambda > \frac{4nM}{\epsilon}.
  \]
  We can estimate
  \begin{align*}
    \left| \int_a^b f(x) \sin(\lambda x) \, dx \right|
    &= \left|\sum_{i = 1}^n \int_{x_{i - 1}}^{x_i} (f(x) - f(x_{i}) + f(x_i)) \sin(\lambda x)\, dx\right| \\
    &\le
    \sum_{i = 1}^n |f(x_i)| \left|\int_{x_{i - 1}}^{x_i} \sin(\lambda x)\, dx\right|
    + \sum_{i = 1}^n \int_{x_{i - 1}}^{x_i} \underbrace{|f(x) - f(x_i)|}_{\le \omega_i(f)} \underbrace{|\sin(\lambda x)|}_{\le 1}\, dx \\
    &\le
    M \sum_{i = 1}^n \frac{\overbrace{|\cos(\lambda x_{i}) - \cos(\lambda x_{i - 1})|}^{\le 2}}{\lambda}
    + \sum_{i = 1}^n \int_{x_{i - 1}}^{x_i} \omega_i(f)\, dx \\
    &\le M \frac{2n}{\lambda} + \sum_{i = 1}^n \omega_i(f) \Delta x_i < \frac{\epsilon}{2} + \frac{\epsilon}{2} = \epsilon.
  \end{align*}
  So as $\lambda \to \infty$, the integral goes to $0$.
\end{proof}

\begin{remark}
  Recall that
  \[
    f(x) =
    \begin{cases}
      0 & \text{if } x \text{ is irrational} \\
      1 & \text{if } x \text{ is rational}
    \end{cases}
  \]
  is not Riemann integrable, but we might expect that
  this
  should integrate to $0$. The Lebesgue integral
  will fix this, which was discovered much later.
\end{remark}

  \chapter{Feb.~6 --- Exchange of Limit Operations}

\section{Motivation}
If we have a sequence of functions $\{f_n\}$ where
$f_n \to f$ pointwise, then does
\[
  \int_a^b f_n\, dx \to \int_a^b f\, dx
\]
if each $f_n$ is integrable? Does
$f_n' \to f'$ if $f_n$ is differentiable?

\begin{example}
  Define
  \[
    f_n(x) =
    \begin{cases}
      4n^2 x & \text{if } x \in [0, 1 / 2n] \\
      4n - 4n^2 x & \text{if } x \in (1 / 2n, 1 / n) \\
      0 & \text{if } x \in [1 / n, 1],
    \end{cases}
  \]
  where the graph of $f_n$ looks like a triangle with
  peak at $x = 1 / 2n$ and height $2n$.
  When we let $n \to \infty$, we see that for any
  $x \in [0, 1]$, we have $f_n(x) \to 0$. But
  \[
    \int_0^1 f_n(x)\, dx = \text{area of triangle}
    = \frac{1}{2} (2n) \cdot \frac{1}{n} = 1.
  \]
  So we see that in this case,
  \[
    \lim_{n \to \infty} \int_0^1 f_n \, dx \ne \int_0^1 \lim_{n \to \infty} f_n \, dx.
  \]
\end{example}

\section{Exchange of the Limit and Integral}

\begin{theorem}
  Let $f_1, \dots, f_n, \dots$ be a uniformly
  convergent sequence of continuous functions
  on $[a, b]$. Then
  \[
    \int_a^b \lim_{n \to \infty} f_n(x)\, dx
    = \lim_{n \to \infty} \int_a^b f_n(x)\, dx.
  \]
\end{theorem}

\begin{proof}
  Suppose that $f_n \to f$ uniformly.
  By definition of uniform convergence, for any
  $\epsilon > 0$ there exists $N$ such that if
  $n \ge N$, then
  \[
    \max_{x \in [a, b]}|f_n(x) - f(x)| < \frac{\epsilon}{b - a}
  \]
  Each $f_n \to f$ uniformly and each $f_n$ is continuous,
  $f$ is also continuous. In particular, $f$ is
  integrable and
  \[
    -\frac{\epsilon}{b - a} < f_n(x) - f(x) < \frac{\epsilon}{b - a},
  \]
  so integrating on both sides gives
  \[
    -\epsilon < \int_a^b f_n(x)\, dx - \int_a^b f(x)\, dx < \epsilon \implies
    \left| \int_a^b f_n(x)\, dx - \int_a^b f(x)\, dx \right| < \epsilon.
  \]
  Then this implies
  \[
    \lim_{n \to \infty} \int_a^b f_n(x)\, dx = \int_a^b f(x)\, dx,
  \]
  as desired.
\end{proof}

\begin{remark}
  The previous theorem still holds even if each $f_n$
  is only Riemann integrable. The only thing we need to
  check is that the limit function $f$ is
  also Riemann integrable. This is because for
  any $\epsilon > 0$, if $n$ is large enough,
  \[
    -\frac{\epsilon}{3(b - a)} + f_n(x) \le f(n) \le f_n(x) + \frac{\epsilon}{3(b - a)}.
  \]
  Since $f_n \in \mathcal{R}([a, b])$, there exist
  two step functions $g_1, g_2$ satisfying
  $g_1 \le f_n \le g_2$, and
  \[
    \int_a^b (g_2 - g_1) < \frac{\epsilon}{3}.
  \]
  Now note that
  \[
    g_1(x) - \frac{\epsilon}{3(b - a)} \le f(x) \le g_2(x) + \frac{\epsilon}{3(b - a)},
  \]
  so we see
  \[
    \int_a^b \left[\left(g_2(x) + \frac{\epsilon}{3(b - a)}\right) - \left(g_1(x) - \frac{\epsilon}{3(b - a)}\right)\right] \, dx
    = \frac{\epsilon}{3} + \frac{2\epsilon}{3} = \epsilon.
  \]
  This gives $f \in \mathcal{R}([a, b])$, so we
  can carry through the rest of the previous proof.
\end{remark}

\section{Exchange of the Limit and Derivative}
\begin{theorem}
  Let $f_1, \dots, f_n, \dots$ be a sequence of functions
  on an open interval $U$ in $\R$ and that each $f_n$ has
  a continuous derivative. Suppose $\{f_n'\}$ converges
  uniformly on $U$ and for some $a \in U$,
  $\{f_n'(a)\}$ converges. Then
  \[
    \lim_{n \to \infty} f_n(x) = f(x)
  \]
  exists and $f(x)$ is differentiable. Furthermore,
  we have
  \[
    f' = \lim_{n \to \infty} f_n'.
  \]
\end{theorem}

\begin{proof}
  By the fundamental theorem of calculus, we have
  \[
    \int_a^x f_n'(t)\, dt = f_n(x) - f_n(a). \tag{$*$}
  \]
  Let $\lim_{n \to \infty} f_n' = g$, where $g$ is
  continuous since $f_n' \to g$ uniformly and
  each $f_n'$ is continuous. Then take $n \to \infty$
  in $(*)$, where
  \[
    \text{LHS} \to \int_a^x g(t)\, dt.
  \]
  Let $\lim_{n \to \infty} f_n(x) = f(x)$, which
  exists by $(*)$. Then
  $\text{RHS} \to f(x) - f(a)$,
  so we see that
  \[
    f(x) - f(a) = \int_a^x g(t)\, dt.
  \]
  Then $f$ is an anti-derivative of $g$, or in other
  words, $f' = g$ as desired.
\end{proof}

\section{Infinite Series}
\begin{definition}
Suppose we have a sequence of numbers
$a_1, a_2, a_3, \dots, a_n, \dots$. Then
\[
  a_1 + a_2 + a_3 + \dots + a_n + \dots
  = \sum_{n = 1}^\infty a_n
\]
is called an \emph{infinite series}. We say the
infinite series \emph{converges} to $A$ if
the \emph{partial sums}
\[
  S_m = \sum_{n = 1}^m a_m
\]
converge to $A$ as $m \to \infty$.
\end{definition}

\begin{example}[Geometric series]
  For a fixed $a$, the series
  \[
    \sum_{n = 0}^\infty a^n = 1 + a + a^2 + \dots + a^n + \dots
  \]
  converges if and only
  if $|a| < 1$, and the limit is $1 / (1 - a)$. This is
  because
  \[
    S_m = 1 + a + \dots + a^m
    = \frac{1 - a^{m + 1}}{1 - a}.
  \]
  If $|a| < 1$, then $a^{m + 1} \to 0$ as $m \to \infty$,
  so $S_m \to 1 / (1 - a)$. On the other hand,
  if $|a| > 1$, then $|a^{m + 1}| \to \infty$ as $m \to \infty$.
  If $a = 1$, then
  \[
    S_m = 1 + 1 + \dots + 1 = m,
  \]
  so $S_m \to \infty$. If $a = -1$, then
  \[
    \sum_{n = 0}^\infty (-1)^n = 1 - 1 + 1 - 1 + \dots,
  \]
  which diverges since its partial sums
  oscillate. So the condition
  is indeed necessary and sufficient.
\end{example}

\begin{prop}
  A series
  $\sum_{n = 1}^\infty a_n$
  converges if and only if for every $\epsilon > 0$,
  there exists integer $N$ such that if $n > m \ge N$,
  then
  \[|a_{m + 1} + a_{m + 2} + \dots + a_n| < \epsilon.\]
\end{prop}

\begin{proof}
  Let $S_m = \sum_{n = 1}^m a_n$ be the partial
  sums. Then $\sum_{n = 1}^\infty a_n$ converges
  if and only if $\{S_m\}$ is Cauchy. This is equivalent
  to say that for
  all $\epsilon > 0$, there exists $N$ such that if
  $n > m \ge N$, then
  \[
    |a_{m + 1} + a_{m + 2} + \dots + a_n|
    = |S_n - S_m| < \epsilon.
  \]
  This is precisely the desired result.
\end{proof}

\begin{corollary}
  If $\sum_{n = 1}^\infty a_n$ converges, then
  $a_n \to 0$ as $n \to \infty$.
\end{corollary}

\begin{proof}
  Take $m = n - 1$ in the previous proposition, which
  gives $|a_n| < \epsilon$ for $n \ge N + 1$.
\end{proof}

\begin{corollary}
  If $\sum_{n = 1}^\infty a_n$ and $\sum_{n = 1}^\infty b_n$
  differs in only finitely many terms, then the
  two series have the same convergence properties.
\end{corollary}

\begin{proof}
  Simply take $N$ larger than the last spot where
  the two series differ. Then the difference of
  partial sums in the previous proposition are the same
  for both series.
\end{proof}

\begin{example}[Harmonic series]
  The series
  \[
    \sum_{n = 1}^\infty \frac{1}{n} = 1 + \frac{1}{2} + \frac{1}{3} + \dots + \frac{1}{n} + \dots.
  \]
  diverges. Two see this, choose $n = 2m$ in the
  previous proposition and
  \[
    a_{m + 1} + a_{m + 2} + \dots + a_{2m}
    = \frac{1}{m + 1} + \frac{1}{m + 2} + \dots + \frac{1}{2m}
    \ge \frac{1}{2m} m = \frac{1}{2}.
  \]
  So the series must diverge.
\end{example}

\begin{prop}
  If $a_n \ge 0$, then
  $\sum_{n = 1}^\infty a_n$ either converges or
  has arbitrarily large partial sums, i.e. diverges to $\infty$.
\end{prop}

\begin{proof}
  Let $S_m = \sum_{n = 1}^m a_n$. Since $a_n \ge 0$,
  we see that $S_m$ is an increasing nonnegative sequence.
  Then by the monotone convergence theorem,
  $\{S_m\}$ converges if and only if it is bounded above.
\end{proof}

\begin{prop}[Comparison test]
  If $\sum_{n = 1}^\infty a_n$ and $\sum_{n = 1}^\infty b_n$
  are two infinite series such that
  $|a_n| \le b_n$ and $\sum_{n = 1}^\infty b_n$ converges,
  then $\sum_{n = 1}^\infty a_n$ converges and
  \[
    \left| \sum_{n = 1}^\infty a_n \right| \le \sum_{n = 1}^\infty b_n.
  \]
\end{prop}

\begin{proof}
  If $\sum_{n = 1}^\infty b_n$ converges, then
  for any $\epsilon > 0$, there exists $N$ such that
  if $n > m \ge N$, we have
  \[
    b_{m + 1} + b_{m + 2} + \dots + b_n < \epsilon.
  \]
  Then by the triangle inequality, we have
  \[
    |a_{m + 1} + a_{m + 2} + \dots + a_n|
    \le |a_{m + 1}| + |a_{m + 2}| + \dots + |a_n|
    \le b_{m + 1} + b_{m + 2} + \dots + b_n < \epsilon.
  \]
  Thus $\sum_{n = 1}^\infty a_n$ also converges.
  The last part is left as an exercise.
\end{proof}

\begin{example}[$p$-series]
  The series
  \[
    \sum_{n = 1}^\infty \frac{1}{n^p}
  \]
  converges if and only if $p > 1$.
\end{example}

\begin{prop}[Ratio test]
  If $\sum_{n = 1}^\infty a_n$ is a nonzero infinite
  series and
  there exists $\rho < 1$ such that
  \[
    \left| \frac{a_{n + 1}}{a_n} \right| \le \rho
  \]
  for all $n$ sufficiently large, then
  $\sum_{n = 1}^\infty a_n$ converges. If
  \[
    \left| \frac{a_{n + 1}}{a_n} \right| \ge 1
  \]
  for all $n$ large enough, then the series diverges.
\end{prop}

\begin{proof}
  First we show the second part. If
  $|a_{n + 1}| \ge |a_n|$ for $n \ge N$, then
  \[|a_n| \ge |a_{n - 1}| \ge \dots \ge |a_N|.\]
  Then $\{a_n\}$ does not converge to $0$, so
  $\sum_{n = 1}^\infty a_n$ diverges.
  First part left for next class.
\end{proof}

  \chapter{Feb.~8 --- Infinite Series}

\section{Lots of Convergence Tests}
\begin{theorem}[Comparison test, second version]
  Let $\sum_{n = 1}^\infty a_n$ and $\sum_{n = 1}^\infty b_n$ be two infinite series satisfying
  $0 \le a_n \le b_n$. Then
  \begin{enumerate}
    \item If $\sum_{n = 1}^\infty b_n$ converges, then $\sum_{n = 1}^\infty a_n$ converges.
    \item If $\sum_{n = 1}^\infty a_n$ diverges,
      then $\sum_{n = 1}^\infty b_n$ diverges.
  \end{enumerate}
\end{theorem}

\begin{proof}
  (1) Let $A_n$ and $B_n$ be the partial sums of
  $\sum_{n = 1}^\infty a_n$ and $\sum_{n = 1}^\infty b_n$, respectively.
  Then $B_n$ is bounded above since $\sum_{n = 1}^\infty b_n$ converges.
  But $A_n \le B_n$ since $0 \le a_n \le b_n$,
  so $A_n$ is also bounded above.
  Now note that $A_n$ is increasing since $a_n \ge 0$,
  so by the monotone convergence theorem,
  $A_n$ must converge.

  (2) Since $A_n$ is increasing,
  $\sum_{n = 1}^\infty a_n$ must diverge to $\infty$,
  i.e. $A_n$ is unbounded. But $A_n \le B_n$,
  so $B_n$ is also unbounded and thus we see that
  $\sum_{n = 1}^\infty b_n$ diverges.
\end{proof}

\begin{remark}
  In the above theorem, (1) remains true if
  \begin{itemize}
    \item $0 \le a_n \le b_n$ when $n \ge n_0$
    \item $0 \le a_n \le Mb_n$ for some $M > 0$,
    \item or there exists $0 < d_n < M$ such that
      $0 \le a_n \le d_n b_n$.
  \end{itemize}
\end{remark}

\begin{corollary}
  If $a_n, b_n > 0$ and
  \[
    \frac{a_{n + 1}}{a_n} \le \frac{b_{n + 1}}{b_n},
  \]
  then $\sum_{n = 1}^\infty b_n$ converges implies
  that $\sum_{n = 1}^\infty a_n$ converges.
\end{corollary}

\begin{proof}
  Let $d_n = a_n / b_n > 0$. Then
  \[
    d_{n + 1} = \frac{a_{n + 1}}{b_{n + 1}} \le \frac{a_n}{b_n} = d_n,
  \]
  which we can extended to $d_n \le \dots \le d_1$.
  Then $\{d_n\}$ is a bounded sequence, so
  $a_n = b_n d_n$, which implies the desired conclusion
  by the above remark.
\end{proof}

\begin{remark}
  If $n$ large, $e^{a n} \gg n^b \gg (\ln n)^c$
  for any $a, b, c > 0$. In particular,
  $e^{an} / n^b \to \infty$ when $n \to \infty$.
\end{remark}

\begin{example}
  Determine the convergence of
  \begin{enumerate}
    \item $\displaystyle \sum_{n = 2}^\infty \frac{1}{(\ln n)^p}$
      for $p > 0$,
    \item $\displaystyle \sum_{n = 2}^\infty \frac{\ln(n!)}{n^p}$ for $p > 0$,
    \item and $\displaystyle \sum_{n = 1}^\infty \frac{n^{n - 2}}{e^n n!}$.
  \end{enumerate}
\end{example}

\begin{proof}
  (1) We have
  \[
    \frac{1}{(\ln n)^p} > \frac{1}{n}
  \]
  for $n$ large, so the sum diverges by comparison to
  the harmonic series.

  (2) Note that
  \[
    \ln(n!) = \sum_{k = 1}^n \ln k > \frac{n \ln 2}{2},
  \]
  so we have
  \[
    \frac{\ln(n!)}{n^p} > \frac{\ln 2}{2} \frac{1}{n^{p - 1}}.
  \]
  By comparing to the $p$-series, we see that the
  series diverges when $p \le 2$. Also we have
  \[
    \frac{\ln(n!)}{n^p} < \frac{n \ln n}{n^p}
    = \frac{\ln n}{n^{p - 1}},
  \]
  so when $p > 2$, we get convergence.

  (3) Let $a_n = n^{n - 2} / (e^n n!)$. Recall that
  $(1 + 1 / n)^n \to e$ as $n \to \infty$ and also
  $(1 + 1 / n)^n$ is increasing. Then
  \[
    \frac{a_{n + 1}}{a_n}
    = \frac{(n + 1)^{n - 1} e^n n!}{e^{n + 1}(n + 1)! n^{n - 2}}
    = \frac{\left(1 + \frac{1}{n}\right)^{n - 2}}{e}
    = \underbrace{\frac{\left(1 + \frac{1}{n}\right)^n}{e}}_{< 1} \left(1 + \frac{1}{n}\right)^{-2}
    < \left(\frac{n}{n + 1}\right)^2 = \frac{\frac{1}{(n + 1)^2}}{\frac{1}{n^2}}.
  \]
  Let $b_n = 1 / n^2$, then $a_{n + 1} / a_n \le b_{n + 1} / b_n$,
  so by the previous corollary, we
  see that $\sum_{n = 1}^\infty a_n$ converges.
\end{proof}

\begin{theorem}[Comparison test, third version]
  Let $(A) \sim \sum_{n = 1}^\infty a_n$ and
  $(B) \sim \sum_{n = 1}^\infty b_n$ be two positive
  series and suppose that
  \[
    \lim_{n \to \infty} \frac{a_n}{b_n} = \ell > 0.
  \]
  Then $(A)$ converges if and only if $(B)$ converges.
\end{theorem}

\begin{proof}
  Let $\epsilon = \ell / 2 > 0$. Then there exists $N$
  such that when $n \ge N$, we have
  \[
    \frac{\ell}{2} < \frac{a_n}{b_n} < \frac{3\ell}{2}
    \implies \frac{\ell}{2} b_n < a_n < \frac{3\ell}{2} b_n.
  \]
  Thus $(A)$ converges if and only if $(B)$ converges.
\end{proof}

\begin{example}
  Determine the convergence of
  \begin{enumerate}
    \item $\displaystyle \sum_{n = 1}^\infty \frac{2n^2 + 5n + 1}{\sqrt{n^6 - 3n^2 + 1}}$,
    \item $\displaystyle \sum_{n = 1}^\infty \frac{1}{n^{1 + \frac{1}{n}}}$,
    \item and $\displaystyle \sum_{n = 1}^\infty \left[1 - \sqrt[3]{\frac{n - 1}{n + 1}}\right]^p$ for $p > 0$.
  \end{enumerate}
\end{example}

\begin{proof}
  (1) Let
  \[
    a_n = \frac{2n^2 + 5n + 1}{\sqrt{n^6 - 3n^2 + 1}}
    \sim \frac{2n^2}{\sqrt{n^6}} = \frac{2}{n}.
  \]
  Then $a_n / (2 / n) \to 1$ as $n \to \infty$, so
  $\sum a_n$ diverges since the harmonic series
  diverges.

  (2) Let
  \[
    a_n = \frac{1}{n^{1 + \frac{1}{n}}}
    \quad \text{and} \quad
    b_n = \frac{1}{n}.
  \]
  Then $a_n / b_n = 1 / (n^{1 / n}) \to 1$ as $n \to \infty$,
  so $\sum a_n$ diverges since the harmonic series
  diverges.

  (3) Write
  \begin{align*}
    \sqrt[3]{\frac{n - 1}{n + 1}}
    = \sqrt[3]{\frac{1 - \frac{1}{n}}{1 + \frac{1}{n}}}
    &= \left(1 - \frac{1}{n}\right)^{1 / 3} \left(1 + \frac{1}{n}\right)^{-1 / 3} \\
    &= \left(1 - \frac{1}{3n} + o(1 / n)\right) \left(1 - \frac{1}{3n} + o(1 / n)\right)
    = 1 - \frac{2}{3n} + o(1 / n),
  \end{align*}
  Then we see that
  \[
    a_n \sim \left(\frac{2}{3n}\right)^p,
  \]
  so $\sum a_n$ converges if and only if $p > 1$.
\end{proof}

\begin{theorem}
  Let $\sum_{n = 1}^\infty$ be a positive series and
  suppose that
  \[\limsup_{n \to \infty} \sqrt[n]{a_n} = \ell.\]
  Then
  \begin{enumerate}
    \item if $\ell < 1$, then $\sum_{n = 1}^\infty a_n$ converges,
    \item and if $\ell > 1$, then $\sum_{n = 1}^\infty a_n$ diverges.
  \end{enumerate}
\end{theorem}

\begin{proof}
  (1) When $\ell < 1$, then there exists $q$ with
  $\ell < q < 1$ and $N$ such that when $n \ge N$,
  $\sqrt[n]{a_n} < q$. Then $a_n < q^n$,
  so $\sum a_n$ converges by comparing to the
  geometric series.

  (2) When $\ell > 1$, there exists $\ell > q > 1$ and
  a subsequence $\{n_k\}$ such that
  $(a_{n_k})^{1 / n_k} > q$. This implies that
  $a_{n_k} > q^{n_k}$, so $a_{n_k} \to \infty$ as
  $n_k \to \infty$. Thus $\sum a_n$ diverges since
  we do not have $a_n \to 0$.
\end{proof}

\begin{example}
  Determine the convergence of
  \begin{enumerate}
    \item $\displaystyle \sum_{n =1 }^\infty \left[1 + \frac{1}{\sqrt{n}}\right]^{-n^{3 / 2}}$,
    \item and $\displaystyle \sum_{n = 1}^\infty \left(\frac{3n}{n + 5}\right)^n \left(\frac{n + 2}{n + 3}\right)^{n^2}$.
  \end{enumerate}
\end{example}

\begin{proof}
  (1) Let $a_n$ be the $n$th term in the sum and we
  see that
  \[
    \sqrt[n]{a_n} =
    \left[1 + \frac{1}{\sqrt{n}}\right]^{-\sqrt{n}}.
  \]
  Since $\sqrt{n} \to \infty$ as $n \to \infty$, we
  may replace $\sqrt{n}$ with $n$ to see that
  $\sqrt[n]{a_n} \to 1 / e < 1$, so
  $\sum a_n$ converges.

  (2) Let
  \[
    \sqrt[n]{a_n} = \frac{3n}{n + 5} \left(\frac{n + 2}{n + 3}\right)^n.
  \]
  For the second term, we see that
  \[
    \left(\frac{n + 2}{n + 3}\right)^n
    = \left(\frac{1 + 2 / n}{1 + 3 / n}\right)^n
    \longrightarrow \frac{e^2}{e^3} = \frac{1}{e}
  \]
  as $n \to \infty$.
  Then $\sqrt[n]{a_n} \to 3 / e > 1$, so
  $\sum a_n$ diverges.
\end{proof}

  \chapter{Feb.~15 --- Absolute Convergence}

\section{Example of the Root Test}
\begin{example}
  Determine the converge of the series
  \[
    \sum_{n = 1}^\infty \frac{n!}{n^{\sqrt{n}}}.
  \]
\end{example}

\begin{proof}
  By Stirling's formula,
  \[
    n! \sim \left(\frac{n}{e}\right)^n \sqrt{2\pi n},
  \]
  in the sense that their ratio tends to $1$ as
  $n \to \infty$. Then
  \[
    \sqrt[n]{a_n} \sim \frac{n}{e} (2\pi)^{1 / 2n} n^{1 / 2n - 1 / \sqrt{n}},
  \]
  Note that $(2\pi)^{1 / 2n} \to 1$ as $n \to \infty$
  since $2\pi$ is a constant, and
  \[
    \ln(n^{1 / 2n - 1 / \sqrt{n}})
    = \left(\frac{1}{2n} - \frac{1}{\sqrt{n}}\right) \ln n
    = \frac{\ln n}{\frac{1}{1 / 2n - 1 / \sqrt{n}}}
    = \frac{\ln n}{2n / \sqrt{1 - 2\sqrt{n}}}
    \sim \frac{\ln n}{2n} = \frac{1 / n}{2} \to 0
  \]
  by L'H\^opital's rule, so $n^{1 / 2n - 1 / \sqrt{n}} \to 1$
  as $n \to \infty$. Then
  $\sqrt[n]{a_n} \sim n / e \gg 1$,
  so this series diverges.
\end{proof}

\section{The Integral Test}
\begin{theorem}[Integral test]
  Let $\{a_n\}$  be a positive decreasing sequence. If
  there exists a continuous decreasing $f(x)$ on
  $[1, \infty)$ such that $a_n = f(n)$, then
  \[
    \sum_{n = 1}^\infty a_n \text{ converges} \quad \text{if and only if} \quad
    \int_1^\infty f(x) \, dx \text{ converges}.
  \]
\end{theorem}

\begin{proof}
  $(\Leftarrow)$ Suppose that
  \[
    \int_1^\infty f(x) \, dx.
  \]
  converges, i.e. the limit
  \[
    \lim_{A \to \infty} \int_1^A f(x) \, dx
  \]
  exists. Then
  \[
    a_k = f(k) = f(k)[k - (k - 1)] \le \int_{k - 1}^k f(x) \, dx
  \]
  since $f$ is decreasing. So
  \[
    \sum_{k = 2}^n a_k \le \sum_{k = 2}^n \int_{k - 1}^k f(x) \, dx
    = \int_1^n f(x)\, dx.
  \]
  Since the integral converges as $n \to \infty$, the
  partial sums of $\sum_{n = 1}^\infty a_n$ are bounded,
  so $\sum_{n = 1}^\infty a_n$ converges by the
  monotone convergence theorem since $a_n \ge 0$.

  $(\Rightarrow)$ Suppose
  \[
    \sum_{n = 1}^\infty a_n
  \]
  converges. Then
  \[
    a_k = f(k)[(k + 1) - k] \ge \int_k^{k + 1} f(x) \, dx
  \]
  since $f$ is decreasing. So
  \[
    \sum_{k = 1}^n a_k \ge \sum_{k = 1}^n \int_k^{k + 1} f(x) \, dx
    = \int_1^{n + 1} f(x) \, dx.
  \]
  Since the sum converges as $n \to \infty$, the
  integral is bounded and thus converges as $f(x) \ge 0$.
\end{proof}

\begin{example}
  For $p > 1$, show that
  \[
    \sum_{n = 1}^\infty \frac{1}{n^p}
  \]
  converges if and only if $p > 1$.
\end{example}

\begin{proof}
  Let $a_n = 1 / n^p$ and choose
  \[
    f(x) = \frac{1}{x^p}
  \]
  for $x > 0$ and note that $f(n) = a_n$.
  Then look at the integral
  \[
    \int_1^\infty \frac{1}{x^p}\, dx,
  \]
  where we can appeal to integration rules to see
  that this integral converges if and only if
  $p > 1$.
\end{proof}

\begin{example}
  Show that
  \[
    \sum_{n = 2}^\infty \frac{1}{n (\ln n)^p}
  \]
  converges if and only if $p > 1$.
\end{example}

\begin{proof}
  Look at
  \[
    \int_2^\infty \frac{1}{x (\ln x)^p}\, dx
    = \left[\frac{1}{-p + 1} (\ln x)^{-p + 1}\right]_{x = 2}^{x = \infty},
  \]
  which converges if and only if $p > 1$.
\end{proof}

\begin{example}
  Suppose $a_n > 0$ and let $S_n = \sum_{k = 1}^n a_k$.
  Then
  \begin{enumerate}
    \item $\displaystyle \sum_{n = 1}^\infty \frac{a_{n + 1}}{S_n \ln S_n}$ diverges if $\displaystyle \sum_{n = 1}^\infty a_n$ diverges,
    \item and $\displaystyle \sum_{n = 1}^\infty \frac{a_{n + 1}}{S_n (\ln S_n)^2}$ converges if $\displaystyle \sum_{n = 1}^\infty a_n$ converges.
  \end{enumerate}
\end{example}

\begin{proof}
  (1) Notice $a_n = S_n - S_{n - 1} > 0$, so
  \[
    \frac{a_{n + 1}}{S_n \ln S_n}
    = \frac{S_{n + 1} - S_n}{S_n \ln S_n}
    \ge \int_{S_n}^{S_{n + 1}} \frac{1}{x \ln x}\, dx
    = \ln(\ln S_{n + 1}) - \ln(\ln S_n).
  \]
  So we see that
  \[
    \sum_{k = 1}^n \frac{a_{k + 1}}{S_k \ln S_k}
    \ge \ln(\ln S_{n + 1}) - \ln(\ln a_1)
  \]
  since the sum telescopes. But
  $\ln(\ln S_{n + 1}) \to \infty$ as $n \to \infty$,
  so this sum diverges.

  The proof for (2) is left as an exercise, but the
  idea is similar.
\end{proof}

\begin{remark}
  If $a_n = 1$, then $S_n = n$ for all $n$, which implies
  that
  \[
    \sum_{n = 1}^\infty \frac{1}{n \ln n} \text{ diverges}
    \quad \text{and} \quad
    \sum_{n = 1}^\infty \frac{1}{n (\ln n)^2} \text{ converges},
  \]
  which matches the previous example.
\end{remark}

\begin{example}
  Let $f(x_0)$ be positive and decreasing no
  $[0, \infty)$. Then
  \[
    (A) = \int_a^\infty f(x)\, dx \text{ converges} \quad
  \text{if and only if} \quad
  (B) = \int_a^\infty f(x) \sin^2 x\, dx \text{ converges}.
  \]
\end{example}

\begin{proof}
  $(\Rightarrow)$ This direction is obvious since
  $0 \le f(x) \sin^2 x \le f(x)$ at every point.

  $(\Leftarrow)$ Suppose otherwise that $(A)$ diverges. Then
  \[
    \infty = \int_a^\infty f(x)\, dx
    = \sum_{n = 0}^\infty \int_{a + n\pi}^{a + (n + 1)\pi}
    f(x)\, dx
    \le \pi \sum_{n = 0}^\infty f(a + n\pi)
  \]
  since $f$ is decreasing. This implies that
  $\sum_{n = 0}^\infty f(a + n\pi)$ diverges. But then
  \[
    \int_a^\infty f(x) \sin^2 x\, dx
    \ge \sum_{n = 0}^\infty f(a + (n + 1)\pi) \int_{a + n\pi}^{a + (n + 1)\pi} \sin^2 x\, dx
    = \frac{\pi}{2}\sum_{n = 0}^\infty f(a + (n + 1)\pi),
  \]
  so we see that
  \[
    \int_a^\infty f(x) \sin^2 x\, dx
  \]
  diverges since $\sum_{n = 1}^\infty f(a + n\pi)$ diverges. Contradiction.
\end{proof}

\section{Absolute and Conditional Convergence}

\begin{definition}
  For a series $(A) = \sum_{n = 1}^\infty a_n$, if
  $\sum_{n = 1}^\infty |a_n|$ converges, then we say that
  $(A)$ \emph{converges absolutely}. If $(A)$ converges
  but $\sum_{n = 1}^\infty |a_n|$ diverges, then we say
  that $(A)$ \emph{converges conditionally}.
\end{definition}

\begin{example}
  The series
  \[
    \sum_{n = 1}^\infty (-1)^{n + 1} \frac{1}{n}
    = 1 - \frac{1}{2} + \frac{1}{3} - \frac{1}{4} + \dots
  \]
  converges conditionally. This is because the series
  itself converges by the alternating series test (which
  we will see later), but
  taking absolute values gives the harmonic series,
  which diverges.
\end{example}

\begin{example}
  For $p > 1$, the series
  \[
    \sum_{n = 1}^\infty (-1)^{n + 1} \frac{1}{n^p}
  \]
  converges absolutely. Unlike before, taking absolute
  values gives a $p$-series, which converges for $p > 1$.
\end{example}

\begin{definition}
  For any $a_n \in \R$, define the \emph{positive} and
  \emph{negative parts} of $a_n$ by
  \[
    a_n^+ =
    \begin{cases}
      a_n & \text{if } a_n \ge 0, \\
      0 & \text{if } a_n < 0,
    \end{cases}
    \quad \text{and} \quad
    a_n^- =
    \begin{cases}
      -a_n & \text{if } a_n \le 0, \\
      0 & \text{if } a_n > 0.
    \end{cases}
  \]
  Note that $a_n^+, a_n^- \ge 0$ and
  in particular, $a_n = a_n^+ - a_n^-$ and
  $|a_n| = a_n^+ + a_n^-$.
\end{definition}

\begin{theorem}
  Suppose $\sum_{n = 1}^\infty a_n$ converges absolutely.
  Then
  \begin{enumerate}
    \item $\sum_{n = 1}^\infty a_n^+$ and
    $\sum_{n = 1}^\infty a_n^-$ converge absolutely,
  \item and $\left|\sum_{n = 1}^\infty a_n\right| \le \sum_{n = 1}^\infty |a_n|$.
  \end{enumerate}
\end{theorem}

\begin{proof}
  (1) Write
  \[
    \sum_{n = 1}^\infty |a_n|
    = \sum_{n = 1}^\infty (a_n^+ + a_n^-).
  \]
  Since $a_n^+, a_n^- \ge 0$, this implies that
  $\sum_{n = 1}^\infty a_n^+$ and
  $\sum_{n = 1}^\infty a_n^-$ converge since they
  are bounded.

  (2) Note that
  \[
    \left|\sum_{k = 1}^n a_k\right|
    \le \sum_{k = 1}^n |a_k|
  \]
  in the finite case. Then let $n \to \infty$
  to get
  \[
    \left|\sum_{k = 1}^\infty a_k\right|
    \le \sum_{k = 1}^\infty |a_k|,
  \]
  which is the desired result.
\end{proof}

\begin{example}
  Does the series
  \[
    \sum_{n = 1}^\infty
    \left[\frac{\cos n}{\sqrt[3]{n^2}} - \sin\left(\frac{\cos n}{\sqrt[3]{n^2}}\right)\right]
  \]
  converge absolutely?
\end{example}

\begin{proof}
  Let
  \[
    a_n = 
    \frac{\cos n}{\sqrt[3]{n^2}} - \sin\left(\frac{\cos n}{\sqrt[3]{n^2}}\right)
  \]
  and Taylor expand to get
  \[
    a_n = \frac{\cos n}{\sqrt[3]{n^2}} - \left(\frac{\cos n}{\sqrt[3]{n^2}} + O(1 / n^2)\right)
    = O(1 / n^2).
  \]
  Since $\sum_{n = 1}^\infty 1 / n^2$ converges, we see
  that $\sum_{n = 1}^\infty a_n$ converges absolutely.
\end{proof}

\subsection{The Alternating Series Test}
\begin{definition}
  A series of the form
  \[
    \sum_{n = 1}^\infty (-1)^n a_n
    = a_1 - a_2 + a_3 - a_4 + \dots
  \]
  where $a_n > 0$ is called an \emph{alternating series}.
\end{definition}

\begin{theorem}[Leibniz test]
  Suppose $\{a_n\}_{n = 1}^\infty$ is a decreasing
  sequence and $\lim_{n \to \infty} a_n = 0$. Then
  the infinite series $\sum_{n = 1}^\infty (-1)^{n + 1} a_n$
  converges.
\end{theorem}

\begin{proof}
  Let
  \[
    S_n = \sum_{k = 1}^n (-1)^{k + 1} a_k.
  \]
  Then we have
  \[
    S_{2n} = (a_1 - a_2) + (a_3 - a_4) + \dots + (a_{2n - 1} - a_{2n})
    \ge S_{2n - 2} \ge 0
  \]
  and also that
  \[
    S_{2n} = a_1 - (a_2 - a_3) - (a_4 - a_5) - \dots - (a_{2n - 2} - a_{2n - 1}) - a_{2n} \le a_1
  \]
  since $\{a_n\}$ is decreasing. In particular,
  $S_{2n}$ is bounded and increasing, so
  \[
    S = \lim_{n \to \infty} S_{2n}
  \]
  exists by the monotone convergence theorem. Now
  observe that
  \[
    S_{2n + 1} = S_{2n} + a_{2n + 1},
  \]
  so taking $n \to \infty$ gives
  $S_{2n + 1} \to S$ since $S_{2n} \to S$ and
  $a_{2n + 1} \to 0$. Thus $\lim_{n \to \infty} S_n = S$.
\end{proof}

\begin{remark}
  This shows that the series
  \[
    \sum_{n = 1}^\infty (-1)^{n + 1} \frac{1}{n}
  \]
  from earlier indeed converges.
\end{remark}

\begin{remark}
  The proof also gives an error estimate for an
  alternating series. We have
  \[
    |S_n - S| \le a_{n + 1}
  \]
  since
  $|S_n - S| = a_{n + 1} - (a_{n + 2} - a_{n + 3}) + \dots
    \le a_{n + 1}$.
  We get this since the tail is still an alternating
  series, and an alternating series is bounded by its
  first term, as shown in the proof.
\end{remark}

  \chapter{Feb.~20 --- Series of Functions}

\section{Rearrangements of Series}

\begin{definition}
  Let $\sigma : \N \to \N$ be a bijection, i.e. it is
  one-to-one and onto. Given an
  infinite series $\sum_{n = 1}^\infty a_n$, we define
  the \emph{rearranged series} to be $\sum_{n = 1}^\infty a_{\sigma(n)}$.
\end{definition}

\begin{theorem}[Dirichlet]
  If $(A) = \sum_{n = 1}^\infty a_n$ converges absolutely
  and $(B) = \sum_{n = 1}^\infty a_{\sigma(n)}$ is any
  rearrangement of $(A)$, then $(B)$ also converges and
  \[
    \sum_{n = 1}^\infty a_{\sigma(n)} = \sum_{n = 1}^\infty a_n.
  \]
\end{theorem}

\begin{proof}
  First assume $a_n \ge 0$ and let $S = \sum_{n = 1}^\infty a_n$
  with partial sums $S_n = \sum_{k = 1}^n a_k$. Let
  $\sigma_n$ be the partial sums of $(B)$. Then
  \[
    \sigma_m \le \sum_{n = 1}^\infty a_n = S \tag{$*$}) \]
  since $a_n \ge 0$. Then $\sigma_m$ is bounded,
  so $(B)$ converges by the monotone convergence theorem.
  Also by $(*)$, we can take $m \to \infty$ to get
  \[
    \lim_{m \to \infty} \sigma_m \le S
    \implies \sum_{n = 1}^\infty a_{\sigma(n)} \le S.
  \]
  We also get the reverse inequality by thinking
  of $(A)$ as a rearrangement of $(B)$, so the two
  are equal.

  Now for the general case, suppose that
  $\sum_{n = 1}^\infty a_n$ is a series with differing
  signs. Let $a_n^+, a_n^- \ge 0$ be the positive
  and negative parts of $a_n$, respectively, so
  that $|a_n| = a_n^+ + a_n^-$. Now $b_n = a_{\sigma(n)}^+$,
  and $|b_n| = b_n^+ + b_n^-$, so
  $b_n^{\pm} = a_{\sigma(n)}^{\pm}$. By the previous
  part, $\sum_{n = 1}^\infty b_n^{\pm}$ converges and
  \[
    \sum_{n = 1}^\infty b_n^{\pm} = \sum_{n = 1}^\infty a_n^{\pm}.
  \]
  Then we have
  \[
    \sum_{n = 1}^\infty b_n
    = \sum_{n = 1}^\infty (b_n^+ - b_n^-)
    = \sum_{n = 1}^\infty (a_n^+ - a_n^-)
    = \sum_{n = 1}^\infty a_n,
  \]
  which is the desired conclusion.
\end{proof}

\begin{theorem}[Riemann]
  If $(A) = \sum_{n = 1}^\infty a_n$ converges
  conditionally, then there exists a rearrangement
  $(B) = \sum_{n = 1}^\infty a_{\sigma(n)}$ such that
  any one of the following occurs:
  \begin{enumerate}
    \item $(B)$ converges to any number $\sigma$,
    \item $(B)$ diverges to $\infty$,
    \item $(B)$ diverges to $-\infty$,
    \item $(B)$ oscillates in an unbounded manner,
    \item or $(B)$ oscillates in a bounded manner.
  \end{enumerate}
\end{theorem}

\begin{proof}
  Look this up.
\end{proof}

\begin{remark}
  This shows that when a series only converges
  conditionally, then pretty much anything can happen
  when rearranging, which is not the case for
  absolutely convergent series.
\end{remark}

\section{Series of Functions}
\begin{definition}
  A \emph{series of functions} is a series of the form
  $\sum_{n = 1}^\infty f_n(x)$, where each $f_n(x)$
  is defined on an interval $I$. Define its
  \emph{partial sums} to be $S_n(x) = \sum_{k = 1}^n f_k(x)$.
  We say that
  \[
    \sum_{n = 1}^\infty f_n(x) = S(x),
  \]
  i.e. the series \emph{converges} to $S(x)$, if
  $\lim_{n \to \infty} S_n(x) = S(x)$, which can be
  \emph{pointwise} or \emph{uniform}.
\end{definition}

\begin{theorem}[Cauchy criterion for uniform convergence]
  A series $\sum_{n = 1}^\infty f_n(x)$ converges
  uniformly on $I$ if and only if for every
  $\epsilon > 0$, there exists $N$ such that whenever
  $n \ge N$, for any $x \in I$ and integer $p$ we have
  \[
    \left| \sum_{k = n + 1}^{n + p} f_k(x) \right| < \epsilon.
  \]
\end{theorem}

\begin{proof}
  This is because $\{S_n(x)\}$ is a Cauchy sequence, so
  for any $\epsilon > 0$, there exists $N$ such that
  when $n \ge N$,
  \[
    \sup_{x \in I} |S_{n + p}(x) - S_{n}(x)| < \epsilon,
  \]
  which is precisely the given condition.
\end{proof}

\begin{corollary}
  If $\sum_{n = 1}^\infty f_n(x)$ converges uniformly
  on $I$, then $f_n(x)$ converges uniformly to $0$
  on $I$.
\end{corollary}

\begin{corollary}
  Let $f_n(x) \in C([a, b])$. If $\sum_{n = 1}^\infty f_n(x)$
  converges uniformly on $(a, b)$, then
  it converges uniformly on $[a, b]$.
\end{corollary}

\begin{proof}
  Since $\sum_{n = 1}^\infty f_n(x)$ converges uniformly,
  for any $\epsilon > 0$ there exists $N$ such that
  when $n \ge N$, we have
  \[
    \left| \sum_{k = n + 1}^{n + p} f_k(x) \right| < \epsilon
  \]
  for all $x \in (a, b)$. Let $x \to a^+$ and
  $x \to b^-$ to get
  \[
    \left| \sum_{k = n + 1}^{n + p} f_k(a) \right| \le \epsilon
    \quad \text{and} \quad
    \left| \sum_{k = n + 1}^{n + p} f_k(b) \right| \le \epsilon.
  \]
  So for all $x \in [a, b]$, we have
  \[
    \left| \sum_{k = n + 1}^{n + p} f_k(x) \right| \le \epsilon,
  \]
  which gives uniform convergence on $[a, b]$.
\end{proof}

\begin{example}
  The series
  \[
    \sum_{n = 1}^\infty \frac{1}{n^x}
  \]
  does not converge uniformly on $(1, \infty)$.
\end{example}

\begin{proof}
  Note that each $1 / n^x$ is continuous
  but at $x = 1$, the series becomes the harmonic
  series
  $\sum_{n = 1}^\infty 1 / n$, which diverges.
  So by the previous
  corollary, this series cannot converge uniformly on
  $(1, \infty)$.
\end{proof}

\begin{theorem}[Weierstrass $M$-test]
  If there exists a nonnegative and convergent
  series $\sum_{n = 1}^\infty M_n$ such that for all
  $x \in I$, we have
  \[
    |f_n(x)| \le M_n,
  \]
  then $\sum_{n = 1}^\infty f_n(x)$ converges
  uniformly on $I$.
\end{theorem}

\begin{proof}
  Since $\sum_{n = 1}^\infty M_n$ converges uniformly,
  for any $\epsilon > 0$ there exists $N$ such that
  \[
    \left| \sum_{k = n + 1}^{n + p} M_k \right| < \epsilon.
  \]
  This directly implies
  \[
    \left| \sum_{k = n + 1}^{n + p} f_k(x) \right|
    \le \sum_{k = n + 1}^{n + p} |f_k(x)|
    < \epsilon
  \]
  since each $M_k$ is nonnegative. Then
  $\sum_{n = 1}^\infty f_n(x)$ converges uniformly on $I$
  by the Cauchy criterion.
\end{proof}

\begin{example}
  If $\sum_{n = 1}^\infty a_n$ converges absolutely,
  then
  \[
    \sum_{n = 1}^\infty \frac{a_n x^n}{1 + x^{2n}}
  \]
  converges uniformly on $(-\infty, \infty)$.
\end{example}

\begin{proof}
  This is because
  \[
    \left| \frac{a_n x^n}{1 + x^{2n}} \right|
    \le |a_n| \underbrace{\frac{|x|^n}{1 + |x|^{2n}}}_{\le 1} \le |a_n|,
  \]
  so by the Weierstrass $M$-test, we get uniform
  convergence on $(-\infty, \infty)$.
\end{proof}

\section{Exchange of the Limit and Infinite Series}

\begin{theorem}
  Let $f_n \in \mathcal{R}([a, b])$ for each $n = 1, 2, \dots$.
  If $\sum_{n = 1}^\infty f_n(x)$ converges uniformly
  to $S(x)$ on $[a, b]$, then $S(x) \in R([a, b])$ and
  \[
    \sum_{n = 1}^\infty \int_a^b f_n(x) \, dx
    = \int_a^b S(x)\, dx
    = \int_a^b \sum_{n = 1}^\infty f_n(x) \, dx.
  \]
\end{theorem}

\begin{proof}
  Let $S_n(x) = \sum_{k = 1}^n f_k(x)$. Then
  $S_n \to S$ uniformly on $[a, b]$, so we can apply
  Theorem \ref{thm:exchange-limit-integral} to get
  $S(x) \in \mathcal{R}([a, b])$ and
  \[
    \sum_{n = 1}^\infty \int_a^b f_n(x) \, dx
    = \lim_{n \to \infty} \int_a^b S_n(x) \, dx
    = \int_a^b S(x) \, dx,
  \]
  which is the desired conclusion.\footnote{This is just applying the previous theorem on exchanging the limit and the integral.}
\end{proof}

\begin{theorem}
  Let $f_n(x)$ be continuously differentiable on
  $[a, b]$. Suppose that
  \[
    \sum_{n = 1}^\infty f_n(x) = S(x)
  \]
  pointwise on $[a, b]$, and $\sum_{n = 1}^\infty f_n'(x)$
  converges uniformly to $G(x)$ on $[a, b]$. Then
  $\sum_{n = 1}^\infty f_n(x)$ converges uniformly
  to $S(x)$ and $S'(x) = G(x)$. In other words, in this
  case we have
  \[
    \left(\sum_{n = 1}^\infty f_n(x)\right)'
    = \sum_{n = 1}^\infty f_n'(x).
  \]
\end{theorem}

\begin{proof}
  Similarly apply Theorem \ref{thm:exchange-limit-derivative},
  our previous theorem on
  exchanging the limit and the derivative.
\end{proof}

\begin{example}
  Show that
  \[
    S(x) = \sum_{n = 1}^\infty \frac{\sin nx}{n^3}
  \]
  is continuously differentiable on $(-\infty, \infty)$.
\end{example}

\begin{proof}
  For each fixed $n$, we see that
  $(\sin nx) / n^3$ is continuously
  differentiable, and we at least have pointwise
  convergence since
  \[
    \left| \frac{\sin nx}{n^3} \right| \le \frac{1}{n^3}.
  \]
  Now the series of derivatives
  \[
    \sum_{n = 1}^\infty \frac{n\cos nx}{n^3}
    = \sum_{n = 1}^\infty \frac{\cos nx}{n^2}
  \]
  converges uniformly on $(-\infty, \infty)$ by the
  Weierstrass $M$-test since
  \[
    \left| \frac{\cos nx}{n^2} \right| \le \frac{1}{n^2}.
  \]
  So we can apply the previous theorem to see that
  $S$ is continuously differentiable on
  $(-\infty, \infty)$.
\end{proof}

\begin{remark}
  Changing the $n^3$ to $n^2$ in the previous example
  will cause this argument to fail, since we get
  \[
    \left| \frac{\cos nx}{n} \right| \le \frac{1}{n},
  \]
  which does not converge. So we cannot use the
  Weierstrass $M$-test.
\end{remark}

\begin{example}
  Let
  \[
    S(x) = \sum_{n = 1}^\infty \frac{|x|}{n^2 + x^2}.
  \]
  Study its differentiability on $(-\infty, \infty)$.
\end{example}

\begin{proof}
  Observe that
  \[
    \frac{|x|}{n^2 + x^2} \le \frac{|x|}{n^2},
  \]
  so the series converges pointwise on
  $(-\infty, \infty)$. Now let $f_n(x) = |x| / (n^2 + x^2)$,
  and we see that
  \[
    \sum_{n = 1}^\infty f_n'(x)
    = \sum_{n = 1}^\infty \frac{\frac{(n^2 + x^2)|x|}{x} - 2x |x|}{(n^2 + x^2)^2}
    = \sum_{n = 1}^\infty \frac{\frac{n^2|x|}{x} - x |x|}{(n^2 + x^2)^2}
  \]
  for any $0 < |x| < A$.
  Now
  \[
    \left| \frac{n^2 |x|}{x} - x |x| \right|
    \le n^2 + A^2,
  \]
  so $\sum_{n = 1}^\infty f_n'(x)$ converges uniformly
  for any $x$ away from $0$. So $S(x)$ is differentiable
  when $x \ne 0$.
  At $x = 0$, we use the definition of the derivative
  to see that
  \[
    \lim_{\Delta x \to 0} \frac{S(\Delta x) - S(0)}{\Delta x}
    = \lim_{\Delta x \to 0} \frac{|\Delta x|}{\Delta x} \sum_{n = 1}^\infty \frac{1}{n^2 + (\Delta x)^2}.
  \]
  From this we get
  \[
    S'_-(0) = -\sum_{n = 1}^\infty \frac{1}{n^2}
    \ne \sum_{n = 1}^\infty \frac{1}{n^2}
    = S'_+(0),
  \]
  so $S(x)$ is not differentiable at $x = 0$.
\end{proof}

  \chapter{Feb.~22 --- Power Series}

\section{Power Series}

\begin{definition}
  Given a point $x_0$, a \emph{power series} around
  $x_0$ is a series of the form
  \[
    \sum_{n=0}^\infty a_n (x - x_0)^n
    = a_0 + a_1(x - x_0) + a_2(x - x_0)^2 + \dots.
  \]
\end{definition}

\begin{remark}
  The question is: For what $x$ does the series converge?
\end{remark}

\begin{example}
  Consider the series\footnote{Later we will see that this is the Taylor series for the exponential function $e^x$.}
  \[
    \sum_{n = 0}^\infty \frac{x^n}{n!}.
  \]
  In fact this series converges for all $x \in \R$. Let
  $a_n = x^n / n!$. By
  the ratio test, for $x \ne 0$, we have
  \[
    \lim_{n \to \infty} \left|\frac{a_{n + 1}}{a_n}\right|
    = \lim_{n \to \infty} \frac{|x|}{n + 1} = 0,
  \]
  so the series converges at $x \ne 0$. Of course, if
  $x = 0$, then every term past $n = 0$ is zero, so the
  series also converges there. Thus the series
  converges everywhere on $\R$.
\end{example}

\begin{example}
  The series
  \[
    \sum_{n = 0}^\infty n! x^n
  \]
  converges only at $x = 0$. For $x \ne 0$, we
  can similarly apply the ratio test to find that
  \[
    \lim_{n \to \infty} \left|\frac{a_{n + 1}}{a_n}\right|
    = \lim_{n \to \infty} |n + 1| |x| = \infty,
  \]
  so the series diverges for $x \ne 0$.
\end{example}

\begin{example}
  Recall that the geometric series
  \[
    \sum_{n = 0}^\infty x^n
  \]
  converges if and only if $|x| < 1$.
\end{example}

\section{Radius of Convergence}
\begin{lemma}
  \label{lem:converge-at}
  We have the following:
  \begin{enumerate}
    \item If $(A) = \sum_{n = 0}^\infty a_n x^n$ converges
      at $x = x_1 \ne 0$, then it converges absolutely
      for all $x$ with $|x| < |x_1|$.
    \item If $(A)$ diverges at $x = x_2 \ne 0$, then it
      diverges for all $x$ with $|x| > |x_2|$.
  \end{enumerate}
\end{lemma}

\begin{proof}
  (1) If $\sum_{n = 0}^\infty a_n x_1^n$ converges, then
  $a_n x_1^n \to 0$ as $n \to \infty$. In particular,
  the terms are bounded, so there exists $M > 0$
  such that $|a_n x_1^n| \le M$ for every $n \in \N$.
  For any $|x| < |x_1|$, let
  \[
    q = \left|\frac{x}{x_1}\right| < 1.
  \]
  Then we have
  \[
    |a_n x^n| = \left|a_n x_1^n \left(\frac{x}{x_1}\right)^n\right|
    \le Mq^n.
  \]
  Comparing with the geometric series, we get that
  $\sum_{n = 0}^\infty a_n x^n$ converges absolutely
  when $|x| < |x_1|$.

  (2) Suppose otherwise that there exists $x_3$ such that
  $|x_3| > |x_2|$ and $\sum_{n = 0}^\infty a_n x_3^n$
  converges. Then by (1), we see that the power series
  converges at $x = x_2$. Contradiction.
\end{proof}

\begin{corollary}
  \label{cor:radius-of-convergence}
  If $(A) = \sum_{n = 0}^\infty a_n x^n$ converges at
  some $x_1 \ne 0$ and diverges at $x_2 \ne 0$, then
  there exists $R > 0$ such that $(A)$ converges
  for $|x| < R$ and diverges for $|x| > R$.
\end{corollary}

\begin{proof}
  Let $E$ be the set of all convergence points of $(A)$.
  By Lemma \ref{lem:converge-at}, we have
  $E \subseteq \{|x| \le x_2\}$ since $(A)$ diverges
  for all $|x| > |x_2|$. Let
  \[R = \sup \{|x| : x \in E\},\]
  which exists since
  $E$ is nonempty and bounded above by $|x_2|$.
  Also $R > 0$ since $x_1 \in E$ and $x_1 \ne 0$. Now if
  $|x| < R$, then
  there exists $x_1 \in E$ such that
  $|x| < |x_1| < R$ and $x = x_1$ is a convergence point.\footnote{This is by definition of the supremum.}
  By Lemma \ref{lem:converge-at}, we get that
  $(A)$ converges at $x$. If $|x| > R$, then there
  exists $x_2$ such that $|x| > |x_2| > R$ and
  $(A)$ diverges at $x = x_2$. Then by Lemma
  \ref{lem:converge-at}, $(A)$ diverges at $x$.
\end{proof}

\begin{remark}
  The $R$ in Corollary \ref{cor:radius-of-convergence}
  is called the \emph{radius of convergence} of the
  power series. If $(A) = \sum_{n = 0}^\infty a_n x^n$
  converges for all $x \in \R$, then by convention
  we use $R = \infty$. If $(A)$ converges only
  at $x = 0$, we use $R = 0$. At $x = \pm R$, the
  convergence or divergence of $(A)$ needs to be checked
  separately.
\end{remark}

\begin{theorem}
  For a power series $\sum_{n = 0}^\infty a_n x^n$, let
  \[
    \limsup_{n \to \infty} \sqrt[n]{|a_n|} = \rho.
  \]
  Then
  \begin{enumerate}
    \item if $\rho = 0$, then $R = \infty$,
    \item if $\rho = \infty$, then $R = 0$,
    \item and if $\rho \in (0, \infty)$, then $R = 1 / \rho$.
  \end{enumerate}
\end{theorem}

\begin{proof}
  We use the root test for
  $\sum_{n = 0}^\infty |a_n x^n|$. Let $A_n = |a_n x^n|$.
  Then
  \[
    \limsup_{n \to \infty} \sqrt[n]{A_n}
    = \limsup_{n \to \infty} \sqrt[n]{|a_n|} \cdot |x|
    = \rho |x|.
  \]
  First suppose $0 < \rho < \infty$. Then $\sum_{n = 0}^\infty |a_n x^n|$
  converges if $|x| \rho < 1$ and diverges if
  $|x| \rho > 1$. This gives $R = 1 / \rho$. Now
  if $\rho = 0$, then
  \[
    \limsup_{n \to \infty} \sqrt[n]{A_n} = 0 < 1,
  \]
  so $\sum_{n = 0}^\infty |a_n x^n|$ regardless of the
  choice of $x$, i.e. $R = \infty$. Finally if $\rho = \infty$, then
  for $x_0 \ne 0$,
  \[
    \limsup_{n \to \infty} \sqrt[n]{A_n} = \infty > 1.
  \]
  By the root test, this implies that $\sum_{n = 0}^\infty |a_n x^n|$
  diverges for all $x \ne 0$, i.e. $R = 0$.
\end{proof}

\begin{corollary}
  For $\sum_{n = 0}^\infty a_n x^n$ with $a_n \ne 0$,
  if
  \[
    \lim_{n \to \infty} \left|\frac{a_{n + 1}}{a_n}\right|
    = \rho,
  \]
  then the radius of convergence is $R = 1 / \rho$.
\end{corollary}

\begin{proof}
  Left as an exercise.
\end{proof}

\begin{example}
  Find the convergence intervals for
  \begin{enumerate}
    \item $\displaystyle \sum_{n = 1}^\infty \frac{2^n(x + 1)^n}{n \ln^2(n + 1)}$,
    \item and $\displaystyle \sum_{n = 1}^\infty n^n x^{n^2}$.
  \end{enumerate}
\end{example}

\begin{proof}
  (1) Let $a_n$ be the summand. Then
  \[
    \lim_{n \to \infty} \left|\frac{a_{n + 1}}{a_n}\right|
    = \lim_{n \to \infty} \frac{2n \ln^2(n + 1)}{(n + 1)\ln^2(n + 2)}
    = 2,
  \]
  so $R = 1 / 2$. So we get convergence for
  $|x + 1| < 1 / 2$, i.e. $x \in (-3 / 2, -1 / 2)$.
  At $x = -3 / 2, -1 / 2$, we get
  \[
    \sum_{n = 1}^\infty \frac{1}{n \ln^2(n + 1)}
  \]
  after taking absolute values, which converges by
  the integral test. So the interval
  is $[-3 / 2, -1 / 2]$.

  (2) Let the general term be
  \[
    a_k =
    \begin{cases}
      n^n & \text{if $k = n^2$ for some $n \in \N$},\\
      0 & \text{otherwise}.
    \end{cases}
  \]
  Then we see that
  \[
    \limsup_{k \to \infty} \sqrt[k]{|a_k|}
    = \lim_{n \to \infty} \sqrt[n^2]{n^n}
    = \lim_{n \to \infty} \sqrt[n]{n} = 1,
  \]
  so $R = 1$. At $x = \pm 1$, the series diverges since
  the general term
  $n^n x^{n^2}$ does not go to $0$ when $|x| = 1$. So
  we conclude that the interval of convergence is
  $(-1, 1)$.
\end{proof}

\begin{theorem}
  If $(A) = \sum_{n = 0}^\infty a_n x^n$ has radius of
  convergence $R > 0$ (including $R = \infty$), then
  for any $0 < r < R$, the power series $(A)$ converges
  uniformly on $[-r, r]$. Moreover, if $(A)$ converges
  at $x = R < \infty$ (or $x = -R$), then $(A)$
  converges uniformly on $[0, R]$ (or $[-R, 0]$).
\end{theorem}

\begin{proof}
  For $x \in [-r, r]$, we have $|a_n x^n| \le |a_n| r^n$,
  and $\sum_{n = 0}^\infty |a_n| r^n$ converges since
  $r < R$. So by the Weierstrass $M$-test, we get
  uniform convergence on $[-r, r]$. Second part left
  as an exercise.
\end{proof}

\begin{corollary}
  \label{thm:power-series-analytic}
  For a power series $(A) = \sum_{n = 0}^\infty a_n x^n$,
  we have the following:
  \begin{enumerate}
    \item if $(A)$ has
      radius of convergence $R > 0$, then
      $f(x) = \sum_{n = 0}^\infty a_n x^n$ is continuous
      on $(-R, R)$,
    \item $f(x)$ is differentiable on $(-R, R)$,
    \item and $f(x) \in C^\infty(-R, R)$, i.e. it is
      infinitely differentiable.
  \end{enumerate}
\end{corollary}

\begin{proof}
  (1) We get
  \[
    \lim_{x \to x_0} f(x)
    = \lim_{x \to x_0} \sum_{n = 0}^\infty a_n x^n
    = \sum_{n = 0}^\infty \lim_{x \to x_0} a_n x^n
    = f(x_0)
  \]
  since we have uniform convergence.

  (2) One can verify that $\sum_{n = 0}^\infty a_n n x^{n - 1}$
  also has radius of convergence $R$. In particular, the
  derivative series also converges uniformly, so
  by Theorem \ref{thm:exchange-limit-derivative}, we can
  differentiate term by term.

  (3) We can repeat (2) as many times as we want.
\end{proof}

\begin{theorem}
  Suppose $f(x) = \sum_{n = 0}^\infty a_n x^n$ has radius
  of convergence $R > 0$. Then for any $x \in (-R, R)$,
  we have $f \in \mathcal{R}([0, x])$ and
  \[
    \int_0^x f(t)\, dt = \int_0^x \sum_{n = 0}^\infty a_n t^n\, dt
    = \sum_{n = 0}^\infty a_n \int_0^x t^n\, dt
    = \sum_{n = 0}^\infty \frac{a_n}{n + 1} x^{n + 1}.
  \]
\end{theorem}

\begin{proof}
  As $\sum_{n = 0}^\infty a_n x^n$ converges
  uniformly in $[-r, r]$, by Theorem \ref{thm:exchange-limit-integral}
  we can integrate term by term.
\end{proof}

\begin{example}
  Show that
  \[
    \sum_{n = 1}^\infty \frac{x^n}{n} = -\ln(1 - x)
  \]
  for $-1 < x < 1$.
\end{example}

\begin{proof}
  By the previous theorem, we can write
  \[
    \sum_{n = 1}^\infty \frac{x^n}{n}
    = \int_0^x \sum_{n = 1}^\infty t^{n - 1}\, dt
    = \int_0^x \frac{1}{1 - t} \, dt
    = -\ln(1 - x),
  \]
  as desired. Note that we used the geometric
  series in the second step.
\end{proof}

\begin{example}
  Find the sum
  \[
    \sum_{n = 0}^\infty \frac{(-1)^n}{3n + 1}.
  \]
\end{example}

\begin{proof}
  Let
  \[
    S(x) = \sum_{n = 0}^\infty \frac{(-1)^n}{3n + 1} x^{3n + 1}.
  \]
  In particular, $S(1)$ is the desired sum, so $S$
  converges at $x = 1$
  by the alternating series test. Then
  \begin{align*}
    S(1)
    &= \sum_{n = 0}^\infty (-1)^n \int_0^1 x^{3n}\, dx
    = \int_0^1 \sum_{n = 0}^\infty (-x^3)^n \, dt
    = \int_0^1 \frac{1}{1 + x^3}\, dx \\
    &= \frac{1}{3} \left[\ln \frac{1 + x}{\sqrt{1 - x + x^2}} + \sqrt{3} \arctan \frac{2x - 1}{\sqrt{3}}\right]_{x = 0}^{x = 1}
    = \frac{1}{3} \left(\ln 2 + \frac{\pi}{\sqrt{3}}\right)
  \end{align*}
  since we have uniform convergence on $[0, 1]$.
\end{proof}

  \chapter{Feb.~27 --- Taylor Series}

\section{Taylor Series}

Recall from Theorem \ref{thm:power-series-analytic} that $f(x) = \sum_{n = 0}^\infty a_n x^n$
is $C^\infty$ on $(-R, R)$, where $R$ is its radius
of convergence. Here we can differentiate term by term,
i.e. we have
\begin{align*}
  f(x) &= a_0 + a_1 x + a_2 x^2 + a_3 x^3 \cdots, \\
  f'(x) &= a_1 + 2a_2 x + 3a_3 x^2 + \cdots, \\
  f''(x) &= 2a_2 + 6a_3 x + \cdots.
\end{align*}
In particular, if we let $x = 0$, then
\[a_0 = f(0), \quad a_1 = f'(0), \quad a_2 = \frac{f''(0)}{2!}, \quad \dots, \quad a_n = \frac{f^{(n)}(0)}{n!}.\]
Here we call
\[
  f(x) = \sum_{n = 0}^\infty \frac{f^{(n)}(0)}{n!} x^n
\]
the \emph{Taylor series} of $f$ at $0$.

\begin{corollary}
  If $f(x)$ has the power series expansion
  $\sum_{n = 0}^\infty c_n (x - a)^n$ on an open interval
  $I$ containing $a$, then $f \in C^\infty(I)$ and
  and
  \[
    c_n = \frac{f^{(n)}(a)}{n!}.
  \]
  for all $n \ge 0$. In particular, if $f$ has a power
  series expansion
  \[
    f = \sum_{n = 0}^\infty c_n (x - a)^n,
  \]
  then $c_n$ is unique.
\end{corollary}

\begin{remark}
  Since Taylor series are unique, we may use any
  way we want to find its coefficients.
\end{remark}

\begin{example}
  Suppose we would like to find the Taylor series of
  $\ln(1 - x)$. It is easier to observe
  \[
    (\ln(1 - x))' = \frac{1}{1 - x}
  \]
  and find the Taylor series of its derivative to be
  \[
    \frac{1}{1 - x} = 1 + x + x^2 + \dots + x^n + \dots.
  \]
  Then we can integrate term by term to find
  the Taylor series of $\ln(1 - x)$.
\end{example}

\section{Convergence of Taylor Series}
Given a function $f \in C^\infty(I)$, do we always have
\[
  f(x) = \sum_{n = 0}^\infty \frac{f^{(n)}(0)}{n!} x^n
\]
in the interval $I$?

\begin{example}
  Consider
  \[
    f(x) =
    \begin{cases}
      e^{-1/x^2} & \text{if } x \ne 0, \\
      0 & \text{if } x = 0.
    \end{cases}
  \]
  Away from $x = 0$, we have
  \[
    f' = \frac{2}{x^3} e^{-1/x^2}, \quad
    f'' = e^{-1 / x^2} \left[-\frac{6}{x^4} + \frac{4}{x^6}\right], \quad
    f''' = e^{-1 / x^2} P_7(1 / x), \quad \dots.
  \]
  So $f$ is $C^\infty$ away from $x = 0$.
  In particular, $f^{(n)}(x) \to 0$ as $x \to 0$
  for all $n \ge 0$ since
  \[
    \lim_{x \to 0} \frac{1}{x^m} e^{-1 / x^2} = 0
  \]
  for all $m \ge 0$. Now also
  \[
    f'(0) = \lim_{x \to 0} \frac{e^{-1/x^2}}{x} = 0,
  \]
  so $f'$ is continuous on $\R$. We can continue
  this to see that each $f^{(n)}$ is continuous on $\R$.
  So $f \in C^\infty(\R)$ and
  $f^{(n)}(0) = 0$ for all $n \ge 0$. Thus its Taylor
  series is identically zero, but $f(x)$ is not the
  zero function.
\end{example}

\begin{remark}
Recall Lagrange's remainder term for the Taylor
polynomial, which says
\[
  f(x) - \sum_{k = 0}^n \frac{f^{(k)}(x_0)}{k!} (x - x_0)^k
  = R_n(x),
\]
where
\[
  R_n(x) = \frac{1}{(n + 1)!} f^{(n + 1)}(\xi) (x - x_0)^{n + 1}
\]
and $\xi$ is between $x$ and $x_0$. We can use this
to justify the convergence of Taylor series.
\end{remark}

\begin{theorem}
  Let $R \in (0, \infty)$ and $f \in C^\infty(x_0 - R, x_0 + R)$.
  If there exists $M > 0$ such that for all
  $x \in (x_0 - R, x_0 + R)$, we have
  \[
    |f^{(n)}(x)| \le M^n
  \]
  for each $n = 1, 2, \dots$, then
  \[
    f(x) = \sum_{n = 0}^\infty \frac{f^{(n)}(x_0)}{n!} (x - x_0)^n
  \]
  for all $x \in (x_0 - R, x_0 + R)$.
\end{theorem}

\begin{proof}
  By Lagrange's remainder term formula, we have
  \[
    |R_n(x)|
    = \frac{1}{(n + 1)!} |f^{(n + 1)}(\xi)| |x - x_0|^{n + 1}
    \le \frac{1}{(n + 1)} M^{n + 1} R^{n + 1}
    = \frac{(MR)^{n + 1}}{(n + 1)!} \to 0
  \]
  as $n \to \infty$ since $M, R$ are fixed. Thus we
  get the desired equality, since the error term goes
  to zero.
\end{proof}

\begin{example}
  For $f(x) = e^x$, we have $f^{(n)}(x) = e^x$. Then
  for $x \in (-R, R)$, we have $|f^{(n)}(x)| \le e^R$.
  So by the previous theorem, we get
  \[
    e^x = \sum_{n = 0}^\infty \frac{1}{n!} x^n
  \]
  for all $x \in \R$, since we can take $R$ as large as
  we would like.
\end{example}

\section{Metric Spaces}

\begin{definition}
  We call a pair $(X, \rho)$ a \emph{metric space} if
  $X$ is nonempty and
  $\rho : X \times X \to \R^+$
  satisfies:
  \begin{enumerate}
    \item positive-definiteness: $\rho(x, y) \ge 0$ and $\rho(x, y) = 0$
      if and only if $x = y$,
    \item symmetry: $\rho(x, y) = \rho(y, x)$,
    \item and the triangle inequality: $\rho(x, y) \le \rho(x, z) + \rho(z, y)$ for
      all $x, y, z \in X$.
  \end{enumerate}
  We say $\rho$ is a \emph{distance function} if it
  satisfies the above properties.
\end{definition}

\begin{example}
  For $X = \R^3$, we may take
  $\rho(\vec{x}, \vec{y}) = \|\vec{x} - \vec{y}\|$.
\end{example}

\begin{example}
  For $X = C([a, b])$, the set of all
  continuous functions on $[a, b]$, we can define
  \[
    \rho(x, y) = \max_{t \in [a, b]} |x(t) - y(t)|
  \]
  for any two $x(t), y(t) \in C([a, b])$.
  This is called the \emph{maximum norm} or
  the $\ell^\infty$ \emph{norm}.
\end{example}

\begin{definition}[Convergence]
  We say that $x_n \to x_0$ in $(X, \rho)$ if
  $\rho(x_n, x_0) \to 0$ as $n \to \infty$. We write
  \[
    \lim_{n \to \infty} x_n = x_0.
  \]
\end{definition}

\begin{example}
  If $x_n(t) \in C([a, b])$, then $x_n \to x_0$
  means $x_n(t) \to x_0(t)$ uniformly in $[a, b]$.\footnote{This is if we use the maximum norm from earlier.}
\end{example}

\begin{definition}
  We say that $\{x_n\}$ is a \emph{Cauchy sequence} in
  $(X, \rho)$ if $\varphi(x_n, x_m) \to 0$
  when $n, m \to \infty$, i.e. for any $\epsilon > 0$,
  there exists $N$ such that if $n, m \ge N$, then
  $\rho(x_n, x_m) < \epsilon$.
\end{definition}

\begin{definition}
  If every Cauchy sequence in $(X, \rho)$
  has a limit $x_n \to x^*$, then we say that
  $(X, \rho)$ is a \emph{complete} metric space.
\end{definition}

\begin{example}
  The continuous functions $C([a, b])$ with the maximum
  norm is complete.
\end{example}

\begin{definition}
  Let $(X, \rho), (Y, r)$ be
  two metric spaces. Then we say $T : X \to Y$ is
  \emph{continuous} if for any $\{x_n\}$ and $x_0 \in X$,
  we have
  $\rho(x_0, x_n) \to 0$ implies
  $r(T(x_0), T(x_n)) \to 0$.
\end{definition}

\begin{theorem}
  A function $T : (X, \rho) \to (Y, r)$ is continuous
  if and only if for all $\epsilon > 0$ and $x_0 \in X$,
  there exists $\delta = \delta(x_0, \epsilon) > 0$
  such that $\rho(x, x_0) < \delta$ implies
  $r(T(x), T(x_0)) < \epsilon$.
\end{theorem}

\begin{proof}
  Check this as an exercise
\end{proof}

\section{Existence and Uniqueness Problem for ODEs}

Consider the ordinary differential equation (ODE) problem
\[
  \begin{cases}
    \frac{dx}{dt} = F(t, x), \\
    x(0) = \xi.
  \end{cases} \tag{1}
\]
We would like to show that this ODE has a local solution.\footnote{The solution may not exist for longer time periods, e.g. it might blow up at some point.}
To do this, we can transform the ODE into\footnote{This process is called \emph{Picard iteration}.}
\[
  x(t) = \xi + \int_0^t f(\tau, x(\tau))\, d\tau.
\]
Now we would like to find a fixed point of this map.
Let $h > 0$ and consider $X = C([-h, h])$ with the
maximum norm. Define the mapping $T : X \to X$ by
\[
  (Tx)(t) = \xi + \int_0^t f(\tau, x(\tau))\, d\tau
\]
for any $x \in X$. Then solving $(1)$ is equivalent
to finding a point $x \in X$ such that $x = Tx$, i.e.
$x$ is a fixed point of $T$. We can do this via
contraction mapping.

\section{The Contraction Mapping Principle}
\begin{definition}
  We say $T : (X, \rho) \to (X, \rho)$ is a
  \emph{contraction mapping} if there exists
  $a \in (0, 1)$ such that
  \[
    \rho(Tx, Ty) \le a \rho(x, y)
  \]
  for any $x, y \in X$.
\end{definition}

\begin{example}
  Let $X = [0, 1]$ and $T(x)$ be differentiable on
  $[0, 1]$ with $T(x) \in [0, 1]$ and $|T'(x)| \le a < 1$.
  Then $T : X \to X$ is a contraction mapping since
  \[
    \rho(Tx, Ty) = |T(x) - T(y)|
    = |T'(\xi)(x - y)| \le a|x - y|
  \]
  by the mean value theorem.
\end{example}

\begin{theorem}[Contraction mapping principle\footnote{This is also known as the \emph{Banach fixed-point theorem}.}]
  Let $(X, \rho)$ be a complete metric space
  and $T : X \to X$ be a contraction mapping. Then
  $T$ has a unique fixed point on $X$.
\end{theorem}

\begin{proof}
  For any $x_0 \in X$, define a sequence recursively
  by $x_{n + 1} = Tx_n$, i.e. $x_n = T^n x_0$. Now we
  show that $\{x_n\}$ is a Cauchy sequence. To do this,
  observe that
  \[
    \rho(x_{n + 1}, x_n)
    = \rho(Tx_n, Tx_{n - 1})
    \le a \rho(x_n, x_{n - 1})
    \le a^2 \rho(x_{n - 1}, x_{n - 2})
    \le \dots
    \le a^n \rho(x_1, x_0)
  \]
  for some $0 < a < 1$ since $\rho$ is a contraction
  mapping.
  So for any integer $p > 0$, by the triangle
  inequality we have
  \[
    \rho(x_{n + p}, x_n) \le
    \sum_{i = 1}^p \rho(x_{n + i}, x_{n + i - 1})
    \le \sum_{i = 0}^{p - 1} a^{n + i} \rho(x_0, x_1)
    \le \sum_{i = 0}^\infty a^{n + i} \rho(x_0, x_1)
    = \frac{a^n}{1 - a} \rho(x_0, x_1).
  \]
  Since $\rho(x_0, x_1)$ is fixed and $0 < a < 1$, we
  see that $\{x_n\}$ is a Cauchy sequence. Since
  $X$ is complete, we have $x_n \to x^* \in X$. Now
  we show that $x^*$ is a fixed point of $T$. This is
  because $x_{n + 1} = Tx_n$, and letting $n \to \infty$
  gives $x^* = Tx^*$, since $T$ is continuous.\footnote{One can show that any contraction mapping is continuous.}
  Next, we show that $x^*$ is the only fixed point.
  To do this, suppose otherwise that
  $x^{**}$ is also a fixed point. Then
  \[
    \rho(x^*, x^{**}) = \rho(Tx^*, Tx^{**})
    \le a \rho(x^*, x^{**}).
  \]
  Since $0 < a < 1$, we must have $\rho(x^*, x^{**}) = 0$,
  i.e. $x^* = x^{**}$.
\end{proof}

  \chapter{Feb.~29 --- hi}

\section{Newton's Method}
Suppose we want to find a solution of $f(x) = 0$, for
some differentiable function $f$. We can use
\emph{Newton's method}, which starts with some initial
guess $x_0$, and iteratively computes
better guesses. More precisely, we look at $x_i$,
find the tangent line to $f$ at $x_i$, and then
take $x_{i + 1}$ to be the $x$-intercept of this
line. The tangent line is given by
\[
  y = f(x_n) + f'(x_n)(x - x_n).
\]
Setting $y = 0$ gives the recurrence
\[
  x_{n + 1} = x_n - \frac{f(x_n)}{f'(x_n)}.
\]
There are some nuances as to when Newton's method
converges, since the $x_i$ may end up oscillating or
have some other problem if $x_0$ is badly chosen (e.g.
if it is far away from the actual zero $\hat{x}$).

\begin{example}
  Let $f \in C^2([a, b])$ and $\hat{x} \in (a, b)$ such
  that $f(\hat{x}) = 0$ and $f'(\hat{x}) \ne 0$. Then
  there exists a neighborhood of $\hat{x}$, denoted
  by $U(\hat{x})$, such that for all $x_0 \in U(\hat{x})$,
  the sequence
  \[
    x_{n + 1} = x_n - \frac{f(x_n)}{f'(x_n)}
  \]
  converges to $\hat{x}$ as $n \to \infty$.
\end{example}

\begin{proof}
  Define the mapping
  \[
    Tx = x - \frac{f(x)}{f'(x)}
  \]
  in an interval containing $\hat{x}$, e.g.
  $I_{\delta} = [\hat{x} - \delta, \hat{x} + \delta]$.
  First we check that if $\delta$ is small, then
  $f : I_\delta \to I_\delta$. For this, let
  $x_1 = Tx$. If $|x - \hat{x}| < \delta$, then we have
  \[
    |x_1 - \hat{x}| = |Tx - T\hat{x}|
    \le a|x - \hat{x}| < \delta,
  \]
  for some $0 < a < 1$. This is because for all
  $x_1, x_2 \in I_\delta$, letting $g(x) = Tx$, we have
  \[
    |Tx_1 - Tx_2| =
    |g(x_1) - g(x_2)|
    = |g'(\xi)| |x_1 - x_2|
  \]
  for some $\xi \in I_\delta$ by the mean value
  theorem. Now notice that
  \[
    |g'(\xi)| = \left| 1 - \frac{f'(x)}{f'(x)}
    + \frac{f(x)f''(x)}{f'(x)^2} \right|
    = \left| \frac{f(x)f''(x)}{f'(x)^2} \right|
    < a < 1
  \]
  if $\delta$ is small enough, since $f(x)$ is small
  in $I_\delta$, and $f''(x)$ is bounded
  and $f'(x)$ is bounded away from zero. Then by the
  contraction mapping principle, we get the unique
  fixed point $\hat{x}$.
\end{proof}

\begin{remark}
  The rough idea is that $f'(x)$ is bounded away from
  zero in some open neighborhood $U$ of $\hat{x}$ since
  $f'(\hat{x}) \ne 0$, and $f'$ is continuous since
  $f \in C^2$. Now pick some compact interval
  $I \subseteq U$, and we get that $f'$ must
  be bounded on $I$. This gives that $f$ is
  Lipschitz, i.e. $|f(x) - f(y)| \le L|x - y|$ for
  all $x, y \in I$ and some constant $L \in \R$. Then
  we just need to show that $0 < L < 1$.
\end{remark}

\section{Existence and Uniqueness for ODEs}
Recall the ODE
\[
  \begin{cases}
    \frac{dx}{dt} = F(t, x). \\
    x(0) = \xi.
  \end{cases}
\]
We rewrite this into the integral form
\[
  x(t) = \xi + \int_0^t F(\tau, x(\tau)) \, d\tau. \tag{$*$}
\]
Now let $h > 0$ be some fixed constant that we
choose later. Define $X = C([-h, h])$ and $T : X \to X$ by
\[
  (Tx)(t) = \xi + \int_0^t F(\tau, x(\tau)) \, d\tau.
\]
Then $(*)$ is equivalent to finding a fixed point of $T$.
Assume $F(t, x)$ satisfies a \emph{local Lipschitz condition}, i.e.
there exist $\delta > 0$ and $L > 0$ such that when
$|t| \le h$, $|x_1 - \xi| \le \delta$, and
$|x_2 - \xi| \le \delta$, we have
\[
  |F(t, x_1) - F(t, x_2)| \le L|x_1 - x_2|.
\]
Define
\[
  B(\xi, \delta) := \{
    x(t) \in C([-h, h]) \mid \max_{|t| \le h} |x(t) - \xi| \le \delta
  \}.
\]
Think of this as the ball of radius $\delta$ around the
constant function $\xi$ in $C([-h, h])$. We want to check
\begin{enumerate}[(i)]
  \item if $h, \delta$ are small enough, then
    $T : B(\xi, \delta) \to B(\xi, \delta)$,
  \item and $T$ is a contraction.
\end{enumerate}
Once we do, then $T$ has a unique fixed point
$\hat{x}(t)$ by the contraction mapping principle,
which is the solution of the ODE for $t \in [-h, h]$.
For the first property, let
\[
  M = \max\{|F(t, x)| \mid t \in [-h, h] \times [\xi - \delta, \xi + \delta]\},
\]
which exists since $F$ is continuous and $[-h, h] \times [\xi - \delta, \xi + \delta]$ is compact.
If $h$ is small, then
\[
  \max_{t \in [-h, h]} |(Tx)(t) - \xi|
  = \max_{t \in [-h, h]} \left| \int_0^t F(\tau, x(\tau)) \, d\tau \right|
  \le Mh \le \delta
\]
since $t \le h$. So
$T : B(\xi, \delta) \to B(\xi, \delta)$. Now we show that
$T$ is a contraction. For any
$x(t), y(t) \in B(\xi, \delta)$,
\begin{align*}
  \rho(Tx, Ty)
  = \max_{t \in [-h, h]} \left| \int_0^t F(\tau, x(\tau)) - F(\tau, y(\tau)) \, d\tau \right|
  &\le h \max_{|t| \le h} |F(t, x(t)) - F(t, y(t))| \\
  &= L h \max_{|t| \le h} |x(t) - y(t)|
  = Lh \rho(x, y)
\end{align*}
by the Lipschitz condition. Now we can choose $h$ such
that
$Lh < 1$, which ensures that $T$ is a contraction.

\begin{remark}
  This is called the \emph{Picard-Lindel\"of theorem},
  or the \emph{existence and uniqueness theorem for ODEs}.
\end{remark}

\section{Implicit Function Theorem}
\begin{example}[Implicit function theorem]
  Let $f : \R \times \R \to \R$ and $U \times V \subseteq \R \times \R$ a
  neighborhood of $(x_0, y_0)$. Assume $f$ and
  $\partial f / \partial y$ are continuous on
  $U \times V$, and $f(x_0, y_0) = 0$ and
  \[
    \frac{\partial f}{\partial y}(x_0, y_0) \ne 0.
  \]
  Then there exists a neighborhood
  $U_0 \times V_0 \subseteq U \times V$ and a unique
  continuous function
  $\varphi : U_0 \to V_0$ satisfying
  \[
    f(x, \varphi(x)) = 0, \quad \varphi(x_0) = y_0.
  \]
\end{example}

\begin{proof}
  We want to solve $f(x, y) = 0$ in a neighborhood
  of $(x_0, y_0)$. Define the mapping $\varphi \mapsto T\varphi$,
  where $\varphi \in C([x_0 - r, x_0 + r])$ for
  some $r > 0$. We take the definition
  \[
    (T\varphi)(x) = \varphi(x) - \left(\frac{\partial f}{\partial y}(x_0, y_0)\right)^{-1} f(x, \varphi(x)).
  \]
  Observe that if $\varphi$ is a fixed point of $T$,
  then $f(x, \varphi(x)) = 0$. Let
  $X = C([x_0 - r, x_0 + r])$, and we will choose $r$
  later. For any $\varphi, \psi \in X$, define the
  distance
  \[
    \rho(\varphi, \psi) = \max_{x \in [x_0 - r, x_0 + r]} |\varphi(x) - \psi(x)|.
  \]
  Now we see that
  \begin{align*}
    \rho(T\varphi, T\psi)
    &= \max_{x \in [x_0 - r, x_0 + r]} \left| \varphi(x) - \psi(x) - \left(\frac{\partial f}{\partial y}(x_0, y_0)\right)^{-1} \left(f(x, \varphi(x)) - f(x, \psi(x))\right) \right| \\
    &= \max_{x \in [x_0 - r, x_0 + r]} \left| \left(1 - \frac{\partial f}{\partial y}(x_0, y_0)^{-1} \frac{\partial f}{\partial y}(x, \hat{y})\right) (\varphi(x) - \psi(x)) \right|
  \end{align*}
  for some $\hat{y}$ between $\varphi(x)$ and $\psi(x)$
  by the mean value theorem. Now define the ball
  \[
    B(y_0, \delta) = \{\varphi \in C(x_0 - r, x_0 + r) \mid \rho(\varphi, y_0) \le \delta\}.
  \]
  If $\varphi, \psi \in B(y_0, \delta)$ and
  $r, \delta$ are small enough, then
  \[
    \left|1 - \frac{\partial f}{\partial y}(x_0, y_0)^{-1} \frac{\partial f}{\partial y}(x, \hat{y})\right| < \frac{1}{2},
  \]
  which ensures a contraction.
  So it only remains to check that
  $T : B(y_0, \delta) \to B(y_0, \delta)$. For
  $\varphi \in B(y_0, \delta)$,
  \begin{align*}
    \rho(T\varphi, y_0)
    &\le \rho(T\varphi, Ty_0) + \varphi(Ty_0, y_0) \\
    &\le \frac{1}{2} \rho(\varphi, y_0) + \max_{x \in [x_0 - r, x_0 + r]} \left| \frac{\partial f}{\partial y}(x_0, y_0)^{-1} f(x, y_0) \right|
    \le \frac{1}{2} \delta + \frac{1}{2} \delta
    = \delta
  \end{align*}
  if $r$ is small enough since $f(x_0, y_0) = 0$.
  So $T\varphi \in B(y_0, \delta)$.
  Thus $T$ is a contraction mapping on $B(y_0, \delta)$,
  so it has a unique fixed point by the
  contraction mapping principle.
\end{proof}

\begin{remark}
  The idea is that if $(\partial f / \partial y)(x_0, y_0) \ne 0$, then the solution set to
  $f(x, y) = 0$
  in a neighborhood of $(x_0, y_0)$ is given by the
  curve $\varphi$.
\end{remark}

  \chapter{Mar.~5 --- The Derivative in \texorpdfstring{$\R^n$}{Rn}}

\section{Partial Derivatives}
\begin{definition}
  Let $f$ be a real-valued function defined on an
  open set $U \subseteq \R^n$. For a fixed point
  $a = (a_1, \dots, a_n) \in U$, the
  \emph{partial derivative} of $f$ at $a$ with
  respect to $x_i$ is
  \begin{align*}
    \frac{\partial f}{\partial x_i}(a)
    = \lim_{x_i \to a_i} \frac{f(a_1, \dots, a_{i-1}, x_i, a_{i+1}, \dots, a_n) - f(a_1, \dots, a_n)}{x_i - a_i}
    = \lim_{h \to 0} \frac{f(a + h \vec{e}_i) - f(a)}{h},
  \end{align*}
  when this limit exists. Here $\vec{e}_i$ is the
  $i$th standard basis vector in $\R^n$.
\end{definition}

\begin{remark}
  The following are equivalent notations for partial
  derivatives:
  \[
    \frac{\partial f}{\partial x_i}(a), \quad f_i'(a),
    \quad f_{x_i}, \quad D_i f
  .\]
\end{remark}

\begin{definition}
  The
  \emph{gradient} of $f$ at $a$ is a vector of the
  partial derivatives, i.e.
  \[
    \nabla f(a) = \left( \frac{\partial f}{\partial x_1}(a), \dots, \frac{\partial f}{\partial x_n}(a) \right).
  \]
\end{definition}

\begin{remark}
  In one dimension, if $f(x)$ is differentiable at $x_0$,
  then $f(x)$ is continuous at $x_0$, because
  \[
    \lim_{x \to x_0} \frac{f(x) - f(x_0)}{x - x_0} = f'(x_0),
  \]
  which implies
  \[
    f(x) - f(x_0) = f'(x_0)(x - x_0) + o(x - x_0).
  \]
  As $x \to x_0$, the RHS goes to zero, so
  $f(x) \to f(x_0)$. However, this need not hold
  in higher dimensions. For $n \ge 2$, even if
  $\partial f / \partial x_i (a)$ exists for all
  $i = 1, 2, \dots, n$, $f(x_0)$ might not be continuous
  at $x = a$.
\end{remark}

\begin{example}
  Consider the function
  \[
    f(x, y) =
    \begin{cases}
      xy / (x^2 + y^2) & \text{if } (x, y) \neq (0, 0), \\
      0 & \text{if } (x, y) = (0, 0).
    \end{cases}
  \]
  At $(0, 0)$, we have
  \[
    \frac{\partial f}{\partial x}(0, 0)
    = \lim_{x \to 0} \frac{f(x, 0) - 0}{x} = \lim_{x \to 0} \frac{0 - 0}{x} = 0.
  \]
  Similarly, $\partial f / \partial y (0, 0) = 0$.
  But $f(x, y)$ is not continuous at $(0, 0)$. For
  continuity, we need
  \[
    \lim_{(x, y) \to (0, 0)} f(x, y) = 0
  \]
  when $(x, y) \to (0, 0)$ along any path $\Gamma$
  from $(x, y)$ to $(0, 0)$. So it suffices to find two
  paths $\Gamma_1$ and $\Gamma_2$ with different limits
  to show that the limit does not exist. So choose
  $\Gamma : y = mx$. Then
  \[
    \lim_{\substack{(x, y) \to (0, 0) \\ \text{along } \Gamma}} \frac{x(mx)}{x^2 + m^2 x^2} = \frac{m}{1 + m^2}.
  \]
  This limit clearly depends on $m$, so we can
  simply choose
  two different values of $m$ to get differing limits.
  Hence we see that $f(x, y)$ is not continuous at
  $(0, 0)$.
\end{example}

\section{Differentiability}
\begin{definition}
  Let $f : U \to \R$ where $U \subseteq \R^n$ and
  $a = (a_1, \dots, a_n) \in U$. Then $f$ is
  \emph{differentiable} at $a$ if there exist
  constants $c_1, \dots, c_n \in \R$ such that
  \[
    \lim_{h \to 0} \frac{f(x) - (f(a) + c_1(x_1 - a_1) + \dots + c_n(x_n - a_n))}{d(x, a)} = 0.
  \]
  Here the distance is
  \[
    d(x, a) = \sqrt{(x_1 - a_1)^2 + \dots + (x_n - a_n)^2}.
  \]
\end{definition}

\begin{remark}
  If we let $\vec{c} = (c_1, \dots, c_n)$, then the
  definition of differentiability becomes
  \[
    f(x) = f(a) + \vec{c} \cdot (x - a) + o(d(x, a)).
  \]
  We will soon show that in fact $\vec{c} = \nabla f(a)$.
\end{remark}

\begin{remark}
  In the one-dimensional case, for $f(x)$ where
  $x \in I \subseteq \R$, we defined the derivative
  as
  \[
    \lim_{x \to a} \frac{f(x) - f(a)}{x - a} = f'(a).
  \]
  This is equivalent to
  \[
    \lim_{x \to a} \frac{f(x) - f(a) - f'(a)(x - a)}{|x - a|} = 0,
  \]
  which is the same as
  our new definition for differentiability.
\end{remark}

\begin{prop}
  If $f(x)$ is differentiable at $a$, then
  $\partial f / \partial x_i (a)$ exists and
  $c_i = \partial f / \partial x_i (a)$.
\end{prop}

\begin{proof}
  We can compute that
  \[
    \frac{\partial f}{\partial x_i}(a)
    = \lim_{h \to 0} \frac{f(a + h \vec{e}_i) - f(a)}{h}
    = \lim_{h \to 0} \frac{f(a) + \vec{c} \cdot h \vec{e}_i + o(|h|) - f(a)}{h}
    = \lim_{h \to 0} \frac{\vec{c} \cdot h \vec{e}_i + o(|h|)}{h}
    = c_i,
  \]
  which is the desired result.
\end{proof}

\begin{prop}
  If $f(x)$ is differentiable at $a$, then $f(x)$ is
  continuous at $a$.
\end{prop}

\begin{proof}
  From the definition, we have
  \[
    f(x) = f(a) + \vec{c} \cdot (x - a) + o(d(x, a)).
  \]
  As we take $x \to a$, we get $f(x) \to f(a)$ just
  as before.
\end{proof}

\begin{remark}
  If $f(x)$ is differentiable, then
  $f(x)$ has all partial derivatives $\partial f / \partial x_i$ at $x = a$.
  But the converse is not true in general. So
  differentiability is a strictly stronger condition
  in $\R^n$.
\end{remark}

\begin{lemma}
  Let $f : U \subseteq \R^n \to \R$, where $U$ is an open
  set. Then $f$ is differentiable at $x = a$ if and only
  if there exist functions $A_1, \dots, A_n$ on $U$,
  continuous at $x = a$, such that
  \[
    f(x) - f(a) = A_1(x)(x_1 - a_1) + A_2(x)(x_2 - a_2) + \dots + A_n(x)(x_n - a_n)
  \]
  for all $x \in U$. In this case,
  $\partial f / \partial x_i (a) = A_i(a)$.
\end{lemma}

\begin{proof}
  $(\Rightarrow)$ Assume $f(x)$ is differentiable
  at $x = a$. Then
  \[
    \lim_{x \to a} \frac{f(x) - (f(a) + f_1'(a)(x_1 - a_1) + \dots + f_n'(a)(x_n - a_n))}{d(x, a)} = 0.
  \]
  Define the function $e : U \to \R$ via
  \[
    e(x) = \frac{f(x) - (f(a) + f_1'(a)(x_1 - a_1) + \dots + f_n'(a)(x_n - a_n))}{|x_1 - a_1| + \dots + |x_n - a_n|}.
  \]
  Observe that the denominator is a distance
  equivalent to $d(x, a)$.\footnote{Two metrics (distances) $d_1, d_2 : X \times X \to \R$ on a set $X$ are \emph{equivalent} if there exist constants $c, C \in \R$ such that $cd_1(x, y) \le d_2(x, y) \le Cd_1(x, y)$ for all $x, y \in X$.} Then $\lim_{x \to a} e(x) = 0$, since
  the distances are equivalent. Then we get
  \begin{align*}
    f(x)
    &= f(a) + f_1'(a)(x_1 - a_1) + \dots + f_n'(a)(x_n - a_n) + e(x)(|x_1 - a_1| + \dots + |x_n - a_n|) \\
    &= f(a) + A_1(x)(x_1 - a_1) + \dots + A_n(x)(x_n - a_n),
  \end{align*}
  where we define
  \[
    A_i(x) =
    \begin{cases}
      f_i'(a) + e(x) & \text{if } x_i - a_i \ge 0 \\
      f_i'(a) - e(x) & \text{if } x_i - a_i < 0.
    \end{cases}
  \]
  Each $A_i$ is continuous and satisfies
  $A_i(a) = f_i'(a)$, as desired.

  $(\Leftarrow)$ Suppose $A_1, \dots, A_n$ exist
  such that
  \[
    f(x) - f(a) = A_1(x)(x_1 - a_1) + \dots + A_n(x)(x_n - a_n).
  \]
  We check that $f$ is differentiable at $x = a$.
  Choose $c_i = A_i(a)$. Then
  \[
    \left|\frac{f(x) - (f(a) - \sum_{i = 1}^n c_i (x_i - a_i))}{d(x, a)}\right|
    = \left|\frac{\sum_{i = 1}^n (A_i(x) - A_i(a))(x_i - a_i)}{d(x, a)}\right|
    \le \sum_{i = 1}^n |A_i(x) - A_i(a)| \to 0
  \]
  as $x \to a$, since each $A_i$ is continuous at $a$.
  Thus $f$ is differentiable at $x = a$.
\end{proof}

\begin{theorem}
  Let $U$ be an open set in $\R^n$ and suppose
  that $f : U \to \R$ has partial derivatives
  $f_1', \dots, f_n'$ on $U$ which are continuous
  at $x = a$. Then $f$ is differentiable at $x = a$.
\end{theorem}

\begin{proof}
  We change $x_i$ to $a_i$ one by one to get
  \begin{align*}
    f(x) - f(a)
    &= (f(x_1, \dots, x_n) - f(a_1, x_2, \dots, x_n)) \\
    &\quad + (f(a_1, x_2, \dots, x_n) - f(a_1, a_2, x_3, \dots, x_n)) \\
    & \quad + (f(a_1, a_2, x_3, \dots, x_n) - f(a_1, a_2, a_3, x_4, \dots, x_n)) \\
    & \quad \quad \quad \vdots \\
    &\quad + (f(a_1, \dots, a_{n-1}, x_n) - f(a_1, \dots, a_n)).
  \end{align*}
  Each of these terms differ in only one variable, so
  we can apply the mean value theorem to get
  \begin{align*}
    f(x) - f(a)
    &= f_1'(\xi_1, a_2, \dots, a_n)(x_1 - a_1) \\
    &\quad + f_2'(a_1, \xi_2, a_3, \dots, a_n)(x_2 - a_2) \\
    & \quad \quad \quad \vdots \\
    &\quad + f_n'(a_1, \dots, a_{n-1}, \xi_n)(x_n - a_n).
  \end{align*}
  So we can set
  \begin{align*}
    A_1(x) &= f_1'(\xi_1, a_2, \dots, a_n) (x_n - a_n) \\
    A_2(x) &= f_2'(a_1, \xi_2, a_3, \dots, a_n) (x_n - a_n) \\
    \vdots \\
    A_n(x) &= f_n'(a_1, \dots, a_{n-1}, \xi_n) (x_n - a_n).
  \end{align*}
  where $\xi_i$ is between $a_i$ and $x_i$ for each
  $1 \le i \le n$. Each $A_i$ is continuous since
  each $f_i'$ is continuous, so we get $A_i(x) \to A_i(a)$
  when $x \to a$. So $f(x)$ is differentiable at $x = a$.
\end{proof}

\begin{remark}
  In functional analysis, there is an analogous theorem
  that relates the G\^ateaux derivative (similar to
  a partial derivative) and the Fr\'echet derivative
  (similar to differentiability).
\end{remark}

\begin{remark}
  The existence of continuous partial derivatives at
  $x = a$ is
  a sufficient condition for differentiability at
  $x = a$, but it is not necessary.
\end{remark}

\begin{example}
  Consider the function
  \[
    f(x, y) =
    \begin{cases}
      (x^2 + y^2) \sin(1 / (x^2 + y^2)) & \text{if } x^2 + y^2 \ne 0, \\
      0 & \text{if } x^2 + y^2 = 0.
    \end{cases}
  \]
  First we verify that $f$ is differentiable at
  $(x, y) = (0, 0)$. We can compute
  \[
    |f(x, y) - f(0, 0)| = (x^2 + y^2) \left|\sin \frac{1}{x^2 + y^2}\right|
    = O((\sqrt{x^2 + y^2})^2),
  \]
  so $f$ is differentiable at $(0, 0)$, with the
  zero linear approximation $f(x, y) \approx 0x + 0y$.
  This also gives $f_1'(0, 0) = f_2'(0, 0) = 0$. However,
  if $(x, y) \ne (0, 0)$, then
  \[
    f_x(x, y) =
    2x \sin \frac{1}{x^2 + y^2} - \frac{2x}{x^2 + y^2} \cos \frac{1}{x^2 + y^2}
  \]
  by the product rule. Letting $(x, y) \to 0$, we see that
  the second term on the RHS has no limit. For
  example, along $y = 0$, we get
  \[
    \left.\frac{2x}{x^2 + y^2} \cos \frac{1}{x^2 + y^2} \right|_{y = 0}
      = \frac{2}{x} \cos \frac{1}{x^2},
  \]
  which has no limit as $x \to 0$. So $f_x(x, y)$ is
  not continuous at $(0, 0)$. By symmetry,
  $f_y(x, y)$ is also
  not continuous at $(0, 0)$. So a function $f$ can
  be differentiable at $x = a$ while its partial
  derivatives $f_i'(x)$ are
  not continuous at $x = a$.
\end{example}

  \chapter{Mar.~7 --- Vector-Valued Functions}

\section{The Derivative for Vector-Valued Functions}

\begin{definition}
  Let $f : U \subseteq \R^n \to \R^m$, where $U$ is an
  open set. We say $f = (f_1, \dots, f_m)$ is
  \emph{differentiable} if each $f_i$ is differentiable.
  In other words, for $x, a \in U$,
  \[
    f(x) - f(a)
    =
    \begin{pmatrix}
      \partial f_1 / \partial x_1 (a) & \dots & \partial f_1 / \partial x_n (a) \\
      \vdots & \ddots & \vdots \\
      \partial f_m / \partial x_1 (a) & \dots & \partial f_m / \partial x_n (a)
    \end{pmatrix}
    (x - a)
    + o(|x - a|).
  \]
  We may call the matrix of partial derivatives
  $f'(a)$ or $\partial f / \partial x$.
\end{definition}

\begin{theorem}[Chain rule]
  Let $U, V$ be open sets in $\R^n$ and $\R^m$,
  respectively, and $f : U \to V$, $g : V \to \R$ be
  functions. Let $a \in U$ such that $f$ is
  differentiable at $a$ and $g$ is differentiable at
  $f(a)$. Then $g \circ f(x) = g(f(x))$ is
  differentiable at $a$, and
  \[
    (g \circ f)_j' (a) = \sum_{i=1}^m g_i'(f(a)) (f_i)_j'(a),
  \]
  where $(f_i)_j'$ denotes $\partial f_i / \partial x_j$.
\end{theorem}

\begin{proof}
  See textbook (Rosenlicht).
\end{proof}

\begin{remark}
  In matrix notation, we can write this via
  \[
    \nabla (g \circ f) = \nabla g \frac{\partial f}{\partial x},
  \]
  where $\nabla g$ is a vector in $\R^m$ and
  $\partial f / \partial x$ is an $m \times n$ matrix.
\end{remark}

\begin{remark}
  When $f : U \subseteq \R^n \to \R^n$, we get
  \[
    \frac{\partial f}{\partial x}(a)
    =
    \begin{pmatrix}
      \partial f_1 / \partial x_1 (a) & \dots & \partial f_1 / \partial x_n (a) \\
      \vdots & \ddots & \vdots \\
      \partial f_n / \partial x_1 (a) & \dots & \partial f_n / \partial x_n (a)
    \end{pmatrix},
  \]
  an $n \times n$ matrix. So we can take its
  determinant, and we call $\det(\partial f / \partial x)$
  the \emph{Jacobian} of $f$.
\end{remark}

\section{Implicit Function Theorem in \texorpdfstring{$\R^n$}{Rn}}

\begin{remark}
  Recall Theorem \ref{thm:implicit}, the implicit function
  theorem on $\R$. Note that all solutions to $f(x, y) = 0$
  are in fact given by the continuous curve $y = \varphi(x)$
  in a neighborhood of $(x_0, y_0)$. To see this,
  suppose $f(x, y) - f(x, \varphi(x)) = 0$ for some $y$.
  Fixing $x$, by the mean value theorem we get
  \[
    \frac{\partial f}{\partial y}(x, \xi) (y - \varphi(x))
    = 0.
  \]
  Since $\partial f / \partial y(x, \xi) \ne 0$, this implies $y = \varphi(x)$.
  So there are no solutions off of the curve.
\end{remark}

\begin{theorem}
  \label{thm:implicit-higher-dim}
  Let $f : \R^n \times \R^m \to \R^m$ and
  $U \times V \subseteq \R^n \times \R^m$ be a neighborhood
  of $(x_0, y_0)$. Suppose $f$ and $\partial f / \partial y$
  are continuous on $U \times V$, and $f(x_0, y_0) = 0$ and
  \[
    \det \left( \frac{\partial f}{\partial y}(x_0, y_0) \right) \ne 0.
  \]
  Then there exists a neighborhood $U_0 \times V_0 \subseteq U \times V$
  of $(x_0, y_0)$ and a unique continuous function
  $\varphi : U_0 \to V_0$ satisfying
  \[
  \begin{cases}
    f(x, \varphi(x)) = 0, \\
    \varphi(x_0) = y_0.
  \end{cases}
  \]
\end{theorem}

\begin{proof}
  Take $T : \varphi \mapsto T\varphi$, where
  $\varphi \in C(\overline{B(x_0, r)}, \R^m)$ and\footnote{Recall that $\overline{A}$ is the \emph{closure} of a set $A$, i.e. the smallest closed set containing $A$.}
  \[
    \rho(\varphi, \psi) = \max_{\substack{x \in \overline{B(x_0, r)} \\ 1 \le i \le m}} |\varphi_i(x) - \psi_i(x)|
  \]
  is the metric.
  Define
  \[
    (T\varphi)(x) = \varphi(x) - \left( \frac{\partial f}{\partial y}(x_0, y_0) \right)^{-1} f(x, \varphi(x)).
  \]
  Note that this is a matrix inverse.
  Define
  \[
    X = \{
      \varphi \in C(\overline{B(x_0, r)}, \R^m)
      : \rho(\varphi, y_0) \le \delta
    \}
  \]
  Now it suffices to show that
  \begin{enumerate}
    \item $\rho(T\varphi, T\psi) \le \frac{1}{2} \rho(\varphi, \psi)$ for $r, \delta$ small enough,
    \item and $T : X \to X$
  \end{enumerate}
  in order to apply the contraction mapping theorem and
  finish.\footnote{The continuity of $\varphi$ is implied since it is a fixed point in a function space of continuous functions.} Check these two details as an exercise.
\end{proof}

\begin{corollary}[Inverse function theorem]
  If $f : U \subseteq \R^n \to \R^n$ such that
  $f, \partial f / \partial y$ are continuous in a
  neighborhood of $y_0 \in U$ and $\det(\partial f / \partial y(y_0)) \ne 0$,
  then there exists $g : V \to \R^n$, where $V$ is a
  neighborhood of $f(y_0)$, such that $g \circ f = \mathrm{id}$,
  i.e. $g(f(x)) = x$.
\end{corollary}

\begin{proof}
  Let $F(x, y) = x - f(y)$, and we want to solve
  $F(x, y) = 0$ near the point $(f(y_0), y_0)$. Apply the
  implicit function theorem to $F$ to finish. See
  textbook (Rosenlicht) for details.
\end{proof}

\section{Higher Order Derivatives}

\begin{definition}
  If $f : U \subseteq \R^n \to \R$ and $\partial f / \partial x_i$
  is differentiable with respect to $x_j$, then we can
  define\footnote{Note the indices, we have $f_{x_j x_i} = (f_i')_j'$.}
  \[
    \frac{\partial}{\partial x_j} \left(\frac{\partial f}{\partial x_i}\right)
    = \frac{\partial^2 f}{\partial x_j \partial x_i}
    = f_{x_j x_i}.
  \]
  These are the \emph{second order partial derivatives}.
  Similarly, we can define higher order partial
  derivatives.
\end{definition}

\begin{remark}
  We want to know: When is $f_{x_j x_i} = f_{x_i x_j}$
  (i.e. $(f_i')_j' = (f_j')_i'$)?
\end{remark}

\begin{theorem}
  Let $f : U \subseteq \R^n \to \R$ and $a \in U$.
  If $(f_i')_j'$ and $(f_j')_i'$ exist and are
  continuous in a neighborhood of $a$, then
  \[
    (f_i')_j'(a) = (f_j')_i'(a).
  \]
\end{theorem}

\begin{proof}
  Take a point $x = (x_1, x_2) \ne a$ in a small neighborhood
  of $a = (a_1, a_2)$. Define
  \[
    \Delta(x) = \frac{f(x_1, x_2) - f(x_1, a_2) - f(a_1, x_2) + f(a_1, a_2)}{(x_1 - a_1)(x_2 - a_2)}.
  \]
  Let $\varphi(x_1, x_2) = f(x_1, x_2) - f(x_1, a_2)$, so
  that
  \[
    \Delta x = \frac{\varphi(x_1, x_2) - \varphi(a_1, x_2)}{(x_1 - a_1)(x_2 - a_2)}
  .\]
  by the mean value theorem,
  \[
    \varphi(x_1, x_2) - \varphi(a_1, x_2)
    = (x - a_1) \varphi_1'(\xi_1, x_2)
  \]
  for some $\xi_1$ between $x_1$ and $a_1$. Then
  \[
  \Delta x
  = \frac{\varphi_1'(\xi_1, x_2)}{x_2 - a_2}
  = \frac{f_1'(\xi_1, x_2) - f_1'(\xi_1, a_2)}{x_2 - a_2}
  = (f_1')_2'(\xi_1, \xi_2)
  \]
  by the mean value theorem again, for some
  $\xi_2$ between $a_2$ and $x_2$. In the same manner, we
  get \[\Delta x = (f_2')_1'(\xi_1', \xi_2'),\]
  where
  $(\xi_1', \xi_2')$ is in a neighborhood of $a$.
  Now let $(x_1, x_2) \to (a_1, a_2)$, then
  $(\xi_1, \xi_1), (\xi_1', \xi_2') \to (a_1, a_2)$.
  By assumption, $(f_2')_1'(\xi_1', \xi_2') \to (f_2')_1'(a)$ and
  $(f_1')_2'(\xi_1, \xi_2) \to (f_1')_2'(a)$ since
  $(f_2')_1'$ and $(f_1')_2'$ are continuous, so
  \[
    (f_1')_2'(a) = (f_2')_1'(a)
  \]
  since $(f_2')_1'(\xi_1', \xi_2') = (f_1')_2'(\xi_1, \xi_2) = \Delta x$.
\end{proof}

  \chapter{Mar.~12 --- More on the Implicit Function Theorem}

\section{Example of Unequal Mixed Partial Derivatives}

\begin{example}
  Consider the function.
  \[
    f(x, y) =
    \begin{cases}
      xy(x^2 - y^2) / (x^2 + y^2) & \text{if } x^2 + y^2 \ne 0 \\
      0 & \text{if } x^2 + y^2 = 0.
    \end{cases}
  \]
  We find
  \[
    f_x'(x, y) = \begin{cases}
      y((x^2 - y^2)(x^2 + y^2) + 4x^2 y^2 / (x^2 + y^2)^2) & \text{if } x^2 + y^2 \ne 0 \\
      0 & \text{if } x^2 + y^2 = 0.
    \end{cases}
  \]
  and
  \[
  f_y'(x, y) =
  \begin{cases}
    x((x^2 - y^2)(x^2 + y^2) + 4x^2 y^2 / (x^2 + y^2)^2) & \text{if } x^2 + y^2 \ne 0 \\
    0 & \text{if } x^2 + y^2 = 0.
  \end{cases}
  .\]
  Now $f_x'(0, y) = -y$ and so
  $f_{xy}''(0, 0) = -1$, but $f_y'(x, 0) = x$ and so
  $f_{yx}''(0, 0) = 1$.  Thus 
  $f_{xy}''(0, 0) \ne f_{yx}''(0, 0)$.
\end{example}

\section{Revisiting the Implicit Function Theorem}

\begin{theorem}[Mean value theorem in $\R^n$]
  Suppose $f : U \subseteq \R^n \to \R$ where $U$ is open,
  and that $f$ is differentiable. THen for any
  $a, b \in U$, there exists $\xi$ on the line segment
  connecting $a, b$ such that
  \[
    f(a) - f(b) = \nabla f(\xi) \cdot (a - b).
  \]
\end{theorem}

\begin{proof}
  Define $h(t) = f(ta + (1 - t)b)$ for $t \in [0, 1]$.
  Then
  \[
  f(a) - f(b) = h(1) - h(0) = h'(c)
  = \nabla f(ca + (1 - c)b) \cdot (a - b)
  \]
  for some $c \in (0, 1)$ by the usual mean value theorem.
  So we can pick $\xi = ca + (1 - c)b$.
\end{proof}

The following is a corollary of Theorem \ref{thm:implicit-higher-dim}:
\begin{corollary}
  If $f_1, \dots, f_n$ are continuously differentiable
  in a neighborhood of $(a, b)$, then $U, V$ can be
  chosen such that $\varphi : U \to V$ is continuously
  differentiable, i.e. $\varphi \in C^1(U, V)$.\footnote{In the notation of Theorem \ref{thm:implicit-higher-dim}, we had $f = (f_1, \dots, f_n)$, $(x_0, y_0)$ instead of $(a, b)$, and $U_0, V_0$ instead of $U, V$.}
\end{corollary}

\begin{proof}
  Recall that
  \[
    \begin{cases}
      f_i(x, \varphi(x)) = 0 \\
      \varphi(a) = b
    \end{cases}
  \]
  for all $1 \le i \le n$. Then we have
  \begin{align*}
    0 = f_i(x, \varphi(x)) - f_i(a, \varphi(a))
    &= \frac{\partial f_i}{\partial x_1} (z^i)(x_1 - a_1)
    + \dots + \frac{\partial f_i}{\partial x_m} (z^i)(x_m - a_m) \\
    & \quad + \frac{\partial f_i}{\partial y_1} (z^i)(\varphi_1(x) - b_1)
    + \dots + \frac{\partial f_i}{\partial y_n} (z^i)(\varphi_n(x) - b_n)
  \end{align*}
  where $z_i$ is between $(x, \varphi(x))$ and $(a, b)$.
  Now since
  $\det(\partial f / \partial y_j (z^i)) \ne 0$, we
  can solve this linear system to get
  \[
    \varphi_i(x) - b_i =
    A_{i 1}(x)(x_1 - a_1) + A_{i 2}(x)(x_2 - a_2) + \dots + A_{i m}(x)(x_m - a_m),
  \]
  where the coefficient functions $A_{i 1}, \dots, A_{i m}$
  are continuous in a neighborhood of $a$ by assumption.
  So
  $\varphi(x) = (\varphi_1(x), \dots, \varphi_n(x))$
  is continuously differentiable in a neighborhood of
  $a$, as desired.
\end{proof}

\begin{remark}
  We can improve this if we know $f$ is more smooth.
  In general, if $f \in C^k$, then $\varphi \in C^k$ also.
\end{remark}

\begin{example}[Implicit differentiation]
  Suppose $F(x, y) = 0$ and $y = \varphi(x)$. If
  $F \in C^1$ and $F_y' \ne 0$,
  \[
    \frac{d}{dx} F(x, \varphi(x)) = 0
    \implies F_x' + F_y' \varphi_x = 0
    \implies \varphi_x = -\frac{F_x'}{F_y'}.
  \]
  We can justify this rigorously via the implicit
  function theorem.
\end{example}

\section{Lagrange Multipliers}
Recall from multivariable calculus that we can use
the method of Lagrange multipliers to solve
the minimum (or maximum) of a function $f(x, y)$ subject
to a constraint $g(x, y) = 0$. The
\emph{Lagrange multiplier equation} is
\[
  \begin{cases}
    \nabla f = \lambda \nabla g \\
    g = 0,
  \end{cases}
\]
which is a system of three equations in $(x, y, \lambda)$.
Here $\lambda$ is called the \emph{Lagrange multiplier}.

\begin{prop}
Assume that at a local minimum (or maximum) point $(x_0, y_0)$ of $f$, we have
\[
  \nabla g(x_0, y_0) \ne \vec{0},
\]
i.e. $g_x^2(x_0, y_0) + g_y^2(x_0, y_0) \ne 0$. Then there
exists $\lambda$ such that $\nabla f = \lambda \nabla g$
at $(x_0, y_0)$.
\end{prop}

\begin{proof}
  Assume $g_y'(x_0, y_0) \ne 0$. Then by the implicit
  function theorem, there exists some $y = \varphi(x)$
  in a neighborhood of $y_0$ such that
  $g(x, \varphi(x)) = 0$. Then $h(x) = f(x, \varphi(x))$
  obtains a minimum (or maximum) at $x_0$. Then
  $h'(x_0) = 0$, so
  \[
    f_x'(x_0, y_0) + f_y'(x_0, y_0) \varphi_x'(x_0) = 0.
  \]
  Also, by implicit differentiation,
  \[
    \varphi_x'(x_0) = -\frac{g_x'(x_0, y_0)}{g_y'(x_0, y_0)}.
  \]
  So we can set
  \[
    \lambda = -\frac{f_y'(x_0, y_0)}{g_y'(x_0, y_0)}
    \implies
    \begin{cases}
      f_x'(x_0, y_0) + \lambda g_x'(x_0, y_0) = 0 \\
      f_y'(x_0, y_0) + \lambda g_y'(x_0, y_0) = 0.
    \end{cases}
  \]
  Thus we have
  $(\nabla f + \lambda \nabla g)(x_0, y_0) = 0$,
  as desired.
\end{proof}

\begin{remark}
  The condition $\nabla g(x_0, y_0) \ne \vec{0}$ is
  necessary to derive the Lagrange multiplier equation.
\end{remark}

\begin{example}
  Minimize $f(x, y) = x^2 + y^2$ subject to
  $g = (x - 1)^3 - y^2 = 0$. The distance of $g$ to
  the origin is clearly minimized at $(1, 0)$, but we have
  \[
    \nabla g(1, 0)
      = \left.(3(x - 1)^2, -2y)\right|_{(1, 0)}
      = (0, 0)
  \]
  and $\nabla f(1, 0) = (2, 0)$. So
  $\nabla f \ne \lambda \nabla g$ in this case since
  $\nabla g(1, 0) = \vec{0}$.
\end{example}

\begin{remark}
  So in general, we should also check the
  points where $\nabla g = \vec{0}$ separately.
\end{remark}

\begin{theorem}
  Suppose $f(x_1, \dots, x_n)$ obtains a local
  minimum (or maximum) at $p^0 = (x_1^{(0)}, \dots, x_n^{(0)})$
  subject to
  \[
  \begin{cases}
    g_1(x_1, \dots, x_n) = 0 \\
    \quad \vdots \\
    g_m(x_1, \dots, x_n) = 0,
  \end{cases}
  \]
  where $m < n$. Assume that $f, g_i \in C^1$ in a
  neighborhood of $p^0$ and the matrix
  \[
    \frac{\partial g}{\partial x}(p^0)
    = \begin{pmatrix}
      \partial g_1 / \partial x_1 & \dots & \partial g_1 / \partial x_n \\
      \vdots & \ddots & \vdots \\
      \partial g_m / \partial x_1 & \dots & \partial g_m / \partial x_n
    \end{pmatrix}(p^0)
  \]
  has rank $m$. Then there exist
  $\lambda_1, \dots, \lambda_m$ such that
  \[
    \begin{cases}
      (\nabla f + \lambda_1 \nabla g_1 + \dots + \lambda_m \nabla g_m)(p^0) = 0 \\
      g_i(p^0) = 0. \\
    \end{cases}
  \]
\end{theorem}

\begin{proof}
  We prove this in the case of four unknowns and
  two constraints (the proof can be generalized). Consider
  \[
    \text{minimize (or maximize) $f(x, y, z, t)$}
    \quad \text{subject to}
    \quad
    \begin{cases}
      \varphi(x, y, z, t) = 0 \\
      \psi(x, y, z, t) = 0.
    \end{cases}
  \]
  Suppose the minimum (or maximum) of $f$ is obtained
  at $p^0 = (x_0, y_0, z_0, t_0)$. Then by assumption,
  there are two variables (say $z, t$) such that
  \[
    \left.\det \frac{\partial (\varphi, \psi)}{\partial (z, t)}\right|_{p^0} \ne 0.
  \]
  Then by the implicit function theorem, we can find
  $z = g(x, y)$ and $t = h(x, y)$ such that the constraint
  equations are satisfied. Then
  $u(x, y) = f(x, y, z(x, y), t(x, y))$ obtains a minimum
  (or maximum) at $(x_0, y_0)$, so
  \[
    \nabla u(x_0, y_0) = 0
    \implies
    \begin{cases}
      f_x' + f_z' (\partial z / \partial x) + f_t' (\partial t / \partial x) = 0 \\
      f_y' + f_z' (\partial z / \partial y) + f_t' (\partial t / \partial y) = 0.
    \end{cases} \tag{1, 2}
  \]
  Then
  \[
    \begin{cases}
      \varphi_x' + \varphi_z' (\partial z / \partial x) + \varphi_t' (\partial t / \partial x) = 0 \\
      \psi_x' + \psi_z' (\partial z / \partial x) + \psi_t' (\partial t / \partial x) = 0.
    \end{cases}
    \text{and} \quad
    \begin{cases}
      \varphi_y' + \varphi_z' (\partial z / \partial y) + \varphi_t' (\partial t / \partial y) = 0 \\
      \psi_y' + \psi_z' (\partial z / \partial y) + \psi_t' (\partial t / \partial y) = 0.
    \end{cases} \tag{$\star$}
  \]
  Since $\det (\partial (\varphi, \psi) / \partial (z, t)) \ne 0$
  at $p^0$, we can find $\lambda, \mu$ such that
  \[
    \begin{cases}
      f_z' + \lambda \varphi_z' + \mu \psi_z' = 0 \\
      f_t' + \lambda \varphi_t' + \mu \psi_t' = 0,
    \end{cases}
  \]
  which is always possible since the coefficient
  matrix is nonsingular.
  Let $(A)$ and $(B)$ be the two systems in $(\star)$ and
  $A_1, A_2$ and $B_1, B_2$ be their first
  and second equations, respectively. Then look at
  the equation $\lambda(A_1) + \mu(A_2) + (1)$, which gives
  \[
  f_x' + \lambda \varphi_x' + \mu \psi_x' +
  \underbrace{(f_z' + \lambda \varphi_z' + \mu \psi_z')}_{= 0} \frac{\partial z}{\partial x}
  + \underbrace{(f_t' + \lambda \varphi_t' + \mu \psi_t')}_{= 0} \frac{\partial t}{\partial x} = 0
  .\]
  This implies that $f_x' + \lambda \varphi_x' + \mu \psi_x' = 0$, and
  we can similarly get that
  $f_y' + \lambda \varphi_y' + \mu \psi_y' = 0$. Thus
  we end up with
  $\nabla f + \lambda \nabla \varphi + \mu \nabla \psi = 0$
  at $p^0$, as desired.
\end{proof}

  \chapter{Mar.~14 --- The Riemann Integral in \texorpdfstring{$\R^n$}{Rn}}

\section{Defining the Riemann Integral in \texorpdfstring{$\R^n$}{Rn}}

\begin{definition}
  A \emph{closed interval}, or a \emph{rectangular domain},
  in $\R^n$ is a set of the form
  \[
    \{
      (x_1, \dots, x_n) : a_1 \le x_i \le b_i \text{ for } i = 1, 2, \dots, n
    \}.
  \]
  A \emph{partition} of $I$ is a partition of each
  $[a_i, b_i]$ for $i = 1, 2, \dots, n$, i.e.
  \[
    (x_1^0, x_1^1, \dots, x_1^{N_1}, \quad (x_2^0, x_2^1, \dots, x_2^{N_2}), \quad \dots, \quad (x_n^0, x_n^1, \dots, x_n^{N_n}).
  \]
  The \emph{width} of a partition $I$ is
  $\max\{x_i^{j} - x_{i}^{j - 1} : i = 1, \dots, n,\, j = 1, \dots, n\}$. A \emph{Riemann sum} is then
  \[
    S = \sum_{\substack{j_1 = 1, \dots, N_1 \\ j_2 = 1, \dots, N_2 \\ \dots \\ j_n = 1, \dots , N_n}} f(y_1^{j_1, \dots, j_n}, y_2^{j_1, \dots, j_n}, \dots, y_n^{j_1, \dots, j_n}) (x_1^{j_1} - x_1^{j_1 - 1}) \dots (x_n^{j_n} - x_n^{j_n - 1}).
  .\]
\end{definition}

\begin{definition}
  We say $f : I \subseteq \R^n \to \R$ is \emph{Riemann integrable}, where $I$ is a rectangular domain,
  if there exists $A \in \R$ such that for any
  $\epsilon > 0$, there exists $\delta > 0$ such that
  $|S - A| < \epsilon$ for any Riemann sum $S$ with partition
  width $< \delta$. In this case, we write
  \[
    A = \int_{I} f(x)\, dx_{1} \dots dx_{n}
    = \int_{I} f\, dx.
  \]
\end{definition}

\begin{remark}
  The Riemann integral in $\R^n$ is still uniquely defined,
  when it exists. If $A, A'$ both satisfy the definition,
  then for every $\epsilon > 0$, there exists a Riemann
  sum $S$ such that $|S - A| < \epsilon$ and $|S - A'| < \epsilon$.
  Then
  $|A - A'| < 2 \epsilon$ by the triangle inequality,
  which implies that
  $A = A'$ since $\epsilon$ was arbitrary.
\end{remark}

\begin{example}
  If $f(x) \equiv c$, then we have
  \[
    \int_I f\, dx = c (b_1 - a_1) \dots (b_n - a_n).
  \]
\end{example}

\begin{example}
  Consider $f(x) = 0$ if $x_1 \ne \xi_1$ for some $\xi_1 \in \R$, i.e. zero except on a single hyperplane.
  Then we have
  \[
    \int_I f\, dx = 0.
  \]
  This remains true if $x_i \ne \xi_i$ for any single
  fixed $1 \le i \le n$.
\end{example}

\begin{example}[Simple step functions]
  Let $\alpha_1, \dots, \alpha_n, \beta_1, \dots, \beta_n \in \R$
  such that $a_i \le \alpha_i \le \beta_i \le b_i$
  for each $i = 1, 2, \dots, n$. Then we can define
  $f : I \to \R$ by
  \[
  f(x_1, \dots, x_n) =
  \begin{cases}
    1 & \text{if } x_i \in (\alpha_i, \beta_i) \text{ for } i = 1, 2, \dots, n, \\
    0 & \text{otherwise}.
  \end{cases}
  .\]
  We call such a function a \emph{simple step function}.
  In this case, we have
  \[
    \int_I f\, dx = (\beta_1 - \alpha_1) \dots (\beta_n - \alpha_n).
  \]
\end{example}

\begin{example}
  Of course not all functions are Riemann integrable
  in $\R^n$ either. Let $I$ be a closed interval in
  $\R^n$ and define $f : I \to \R$ by
  \[
    f(x) =
    \begin{cases}
      1 & \text{if } x = (x_1, \dots, x_n) \in I \text{ and } x_1, \dots, x_n \text{ are rational}, \\
      0 & \text{otherwise}.
    \end{cases}
  \]
  This function $f$ is not Riemann integrable.
\end{example}

\section{Properties of the Riemann Integral in \texorpdfstring{$\R^n$}{Rn}}

\begin{prop}
  We have the following:
  \begin{enumerate}
    \item If $f, g \in \mathcal{R}(I)$, $I \subseteq \R^n$,
    then $f + g \in \mathcal{R}(I)$ and
    \[
      \int_{I} (f + g)\, dx = \int_{I} f\, dx + \int_{I} g\, dx.
    \]
  \item If $f \in \mathcal{R}(I)$ and $c \in \R$,
    then $cf \in \mathcal{R}(I)$ and
    \[
      \int_{I} cf\, dx = c \int_{I} f\, dx.
    \]
  \item If $f \ge 0$ on $I$ and $f \in \mathcal{R}(I)$,
    then
    \[
      \int_{I} f\, dx \ge 0.
    \]
  \item If $f \le g$ on $I$ and $f, g \in \mathcal{R}(I)$,
    then
    \[
      \int_{I} f\, dx \le \int_{I} g\, dx.
    \]
  \item If $m \le f \le M$ and $f \in \mathcal{R}(I)$, then
    \[
      m \Vol(I) \le \int_{I} f\, dx \le M \Vol(I),
    \]
    where $\Vol(I) = (b_1 - a_1) \dots (b_n - a_n)$.
  \end{enumerate}
\end{prop}

\begin{proof}
  Check these properties as an exercise.
\end{proof}

\section{Conditions for Riemann Integrability in \texorpdfstring{$\R^n$}{Rn}}

\begin{lemma}
  Let $I$ be a closed interval in $\R^n$. Then
  $f \in \mathcal{R}(I)$ if and only if for all
  $\epsilon > 0$, there exists $\delta > 0$ such that
  $|S_1 - S_2| < \epsilon$ whenever $S_1, S_2$ are
  two Riemann sums with partitions
  of width $< \delta$.
\end{lemma}

\begin{proof}
  Roughly the same idea as in $\R$, see Rosenlicht for
  details.
\end{proof}

\begin{example}[General step functions]
  Let $(x_1^0, x_1^1, \dots, x_1^{N_1}), \dots, (x_n^0, x_n^1, \dots, x_n^{N_n})$
  be a partition of $I$. A \emph{step function}
  $f : I \to \R$ is defined by
  \[
    f(x) =
    \begin{cases}
      c_{j_1, \dots, j_n} & \text{if } x_i^{j_1 - 1} < x_i < x_i^{j_i} \text{ for all } i = 1, 2, \dots, n \\
      0 & \text{otherwise}
    \end{cases}
  \]
  for some constants $c_{j_1, \dots, j_n} \in \R$.
  Observe that $f$ is a linear combination of simple step functions,
  so it is integrable. To be more precise, notice that
  we can write
  \[
    f = \sum c_{j_1, \dots, j_n} f_{j_1, \dots, j_n},
    \quad \text{where} \quad
    f_{j_1, \dots, j_n}(x) =
    \begin{cases}
      1 & \text{if } x_i^{j_i - 1} < x_i < x_i^{j_i} \text{ for all } i = 1, 2, \dots, n, \\
      0 & \text{otherwise}.
    \end{cases}
  \]
  Then we get
  \[
    \int_{I} f\, dx
    = \sum_{\substack{1 \le j_i \le N_i \\ 1 \le i \le n}} c_{j_1, \dots, j_n} (x_1^{j_1} - x_1^{j_1 - 1}) \dots (x_n^{j_n} - x_n^{j_n - 1}).
  \]
\end{example}

\begin{prop}
  \label{thm:step-integrable}
  Let $I \subseteq \R^n$ be a closed interval and
  $f : I \to \R$. Then
  $f \in \mathcal{R}(I)$ if and only if for any
  $\epsilon > 0$, there exist step functions
  $f_1, f_2$ on $I$ such that
  $f_1(x) \le f(x) \le f_2(x)$ for all $x \in I$
  and
  \[
    \int_{I} (f_2 - f_1)\, dx < \epsilon.
  \]
\end{prop}

\begin{proof}
  Similar to the proof in $\R$, see Rosenlicht for details.
\end{proof}

\begin{corollary}
  If $I$ is a closed interval $f \in \mathcal{R}(I)$,
  then $f$ is bounded on $I$.
\end{corollary}

\begin{proof}
  This follows from the proof of Proposition
  \ref{thm:step-integrable}, like in the case of $\R$.
\end{proof}

\begin{corollary}
  Let $I \subseteq J$ be closed intervals in $\R^n$ and
  $f : J \to \R$ such that $f(x) = 0$ if
  $x \in J \setminus I$. Then $f \in \mathcal{R}(J)$
  if and only if $f \in \mathcal{R}(I)$. Moreover,
  in this case we have
  \[
    \int_{I} f\, dx = \int_{J} f\, dx.
  \]
\end{corollary}

\begin{proof}
  $(\Leftarrow)$ Suppose $f \in \mathcal{R}(I)$. Then for
  any $\epsilon > 0$,
  there exist two step functions $f_1, f_2$ on $I$ such
  that $f_1(x) \le f(x) \le f_2(x)$ for all $x \in I$ and
  \[
    \int_{I} (f_2 - f_1)\, dx < \epsilon.
  \]
  Extend the step functions $f_1, f_2$ to $J$ by setting
  $f_1(x) = f_2(x) = 0$ if $x \in J \setminus I$. Note
  that $f_1, f_2$ become step functions on $J$ if we
  extend them in this manner. Since $f \equiv 0$ on
  $J \setminus I$, we have
  $f_1(x) \le f(x) \le f_2(x)$ for all $x \in J$, and
  \[
    \int_J (f_2 - f_1)\, dx = \int_I (f_2 - f_1)\, dx < \epsilon.
  \]
  So $f \in \mathcal{R}(J)$. To see that the two integrals
  are the same, observe that
  \[
    \int_{I} f_1 \le \int_{I} f \le \int_{I} f_2
    \quad \text{and} \quad
    \int_I f_1 = \int_{J} f_1 \le \int_{J} f \le \int_{J} f_2 = \int_I f_2,
  \]
  so we get
  \[
  \left|\int_I f - \int_J f\right|
  \le \int_I (f_2 - f_1) < \epsilon \implies
    \int_I f = \int_J f
  \]
  since $\epsilon$ was arbitrary.

  $(\Rightarrow)$ Suppose $f \in \mathcal{R}(J)$. Then
  for any $\epsilon > 0$, there exist step functions
  $f_1, f_2$ on $J$ such that
  \[
    f_1(x) \le f(x) \le f_2(x) \text{ for all } x \in J \quad \text{and} \quad
    \int_{J} (f_2 - f_1) < \epsilon.
  \]
  Define $g_1, g_2$ on $I$ by restricting $f_1, f_2$ to $I$.
  Then $g_1, g_2$ are step functions on $I$ and
  $g_1 \le f \le g_2$ on $I$. Now we also have
  \[
    \int_I (g_2 - g_1) = \int_I (f_2 - f_1)
    \le \int_J (f_2 - f_1) < \epsilon
  \]
  since $f_1 \le f_2$, so $f_2 - f_1 \ge 0$ on $J$.
  This gives $f \in \mathcal{R}(I)$.
\end{proof}

\section{Extending the Integral}

\begin{definition}
If $f : \R^n \to \R$ and $f \equiv 0$ outside
a bounded set (bounded support),\footnote{The \emph{support} of $f$ is the set on which its zero.} then there exists a closed
interval $I \subseteq \R^n$ such that
$f(x) = 0$ outside $I$. Then we say $f$ is
\emph{integrable on $\R^n$} if $f \in \mathcal{R}(I)$, and
we define
\[
  \int_{\R^n} f\, dx = \int_{I} f\, dx.
\]
\end{definition}

\begin{remark}
  The above definition is independent of the choice of
  $I$.
  This is because if $I, I'$ both contain the support
  of $f$, then there exists $J \supset I \cup I'$, so that
  \[
    \int_I f \text{ exists} \iff
    \int_J f \text{ exists} \iff
    \int_{I'} f \text{ exists}
    \quad \text{and} \quad
    \int_I f\, dx = \int_{I'} f\, dx = \int_J f\, dx.
  \]
\end{remark}

  \chapter{Mar.~26 --- Extending the Integral in \texorpdfstring{$\R^n$}{Rn}}

\section{Extending the Riemann Integral in \texorpdfstring{$\R^n$}{Rn}}
\begin{definition}
  If $f$ is defined on an arbitrary set
  $A \subseteq \R^n$, define $\overline{f} : \R^n \to \R$
  by zero extension, i.e.
  \[
    \overline{f}(x) = \begin{cases}
      f(x) & \text{if } x \in A, \\
      0 & \text{if } x \notin A.
    \end{cases}
  \]
  If the support of $f$ is a bounded subset of $\R^n$,
  then we define the \emph{integral of $f$ over $A$} as
  \[
    \int_A f = \int_{\R^n} \overline{f}.
  \]
\end{definition}

\begin{remark}
  We note once again that there exist functions which
  are not Riemann integrable on $\R^n$.
  Let $A$ be the points in a closed interval of
  $\R^n$ with all rational coordinates.
  Then $f = 1$ is not integrable on $A$.
\end{remark}

\begin{remark}
  If $A$ is bounded and $f = 1$, then
  \[
    \int_A f = \int_A 1 = \Vol(A)
  \]
  if $1$ is integrable on $A$. This volume is also called
  the \emph{Jordan measure} of $A$. The previous remark
  shows that not every set is Jordan measurable.
\end{remark}

\begin{prop}
  We have the following:
  \begin{enumerate}
    \item Let $A \subseteq \R^n$ and $f, g$ be two
      integrable functions on $A$. Then $f + g$ is
      also integrable and
      \[
        \int_A (f + g) = \int_A f + \int_A g.
      \]
    \item If $f$ is integrable on $A$ and $c \in \R$,
      then $cf$ is integrable on $A$ and
      \[
        \int_A cf = c \int_A f.
      \]
    \item If $f(x) \ge 0$ on $A$ and $f$ is integrable,
      then
      \[
        \int_A f \ge 0.
      \]
    \item If $f(x) \le g(x)$ on $A$ and $f, g$ are
      integrable, then
      \[
        \int_A f \le \int_A g.
      \]
    \item If $m \le f(x) \le M$ on $A$ and $A$ has a
      volume, and $f$ is integrable on $A$, then
      \[
      m \Vol(A) \le \int_A f \le M \Vol(A)
      .\]
  \end{enumerate}
\end{prop}

\begin{proof}
  Left as an exercise. Note that the proof in the case
  where $A$ is a box is already done.
\end{proof}

\section{Sets of Measure Zero}

\begin{prop}
  We have the following:
  \begin{enumerate}
    \item A set $A \subseteq \R^n$ has zero volume if and
      only if
      for any $\epsilon > 0$, there exist a finite number
      of closed intervals in $\R^n$ containing $A$ with
      the sum of their volumes less than $\epsilon$.
    \item Any subset of a subset of $\R^n$ of zero volume
      is of zero volume.
    \item If $A \subseteq \R^n$ has zero volume and
      $B \subseteq \R^n$ has volume, then
      \[
        \Vol(A \cup B) = \Vol(B)
        \quad \text{and} \quad
        \Vol(B \setminus A) = \Vol(B).
      \]
    \item The union of a finite number of zero volume
      sets is of zero volume.
    \item If $A$ has zero volume and $f : A \to \R$ is
      bounded, then $f$ is integrable on $A$ and
      \[
        \int_A f = 0
      .\]
    \item If $S \subseteq \R^{n - 1}$ is compact and
      $f : S \to \R$, then the graph of $f$ in
      $\R^n$, i.e. the set
      \[
        \{(x_1, \dots, x_n) \in \R^n : (x_1, \dots, x_{n - 1}) \in S,\, x_n = f(x_1, \dots, x_{n - 1})\},
      \]
      is of zero volume.
  \end{enumerate}
\end{prop}

\begin{proof}
  See the textbook (Rosenlicht).
\end{proof}

\begin{prop}
  Let $A, B$ be two subsets of $\R^n$ such that
  $\Vol(A \cap B) = 0$ and $f : A \cup B \to \R$ is
  integrable on both $A$ and $B$. Then
  \[
    \int_{A \cup B} f = \int_A f + \int_B f.
  \]
\end{prop}

\begin{proof}
  Define $f_1, f_2, f_3 : \R^n \to \R$ by
  \[
    f_1(x) = \begin{cases}
      f(x) & \text{if } x \in A, \\
      0 & \text{if } x \notin A,
    \end{cases}
    \quad
    f_2(x) = \begin{cases}
      f(x) & \text{if } x \in B, \\
      0 & \text{if } x \notin B,
    \end{cases}
    \quad \text{and} \quad
    f_3(x) = \begin{cases}
      f(x) & \text{if } x \in A \cap B, \\
      0 & \text{if } x \notin A \cap B.
    \end{cases}
  \]
  Then
  \[
    \int_{\R^n} f_1 = \int_A f
    \quad \text{and} \quad
    \int_{\R^n} f_2 = \int_B f. \tag{1}
  \]
  Since $f$ is integrable on $A$ and $B$, it must be
  bounded on $A$ and $B$. Then
  \[
    \int_{\R^n} f_3 = \int_{A \cap B} f = 0 \tag{2}
  \]
  since $\Vol(A \cap B) = 0$ and $f$ is bounded on
  $A \cap B$. Now if $x \in A \cup B$, then
  $f(x) = f_1 + f_2 - f_3$, and if
  $x \notin A \cup B$, then $f_1 + f_2 - f_3 = 0$.
  Then we get that
   \[
     \int_{A \cup B} f = \int_{\R^n} (f_1 + f_2 - f_3)
     = \int_{\R^n} f_1 + \int_{\R^n} f_2 - \int_{\R^n} f_3
     = \int_A f + \int_B f
  \]
  from $(1)$ and $(2)$, as required.
\end{proof}

\begin{corollary}
  If $A$ and $B$ have volume and $A \cap B$ has
  zero volume, then
  \[\Vol(A \cup B) = \Vol(A) + \Vol(B).\]
\end{corollary}

\begin{proof}
  Choose $f = 1$ in the previous proposition.
\end{proof}

\begin{theorem}[Lebesgue's criterion for Riemann integrability]
  Let $A \subseteq \R^n$ be a set with volume and
  let $f : A \to \R$ be a bounded function that is
  continuous except on a subset of $A$ with zero volume.
  Then $f$ is integrable on $A$.
\end{theorem}

\begin{proof}
  First we consider the case where $A$ is a closed
  interval in $\R^n$ and $f$ is continuous on $A$.
  Now $f$ is continuous on a compact set, so it is
  bounded, i.e. there exists $M \in \R$ such that
  $|f(x)| \le M$ on $A$. Also since $A$ is compact,
  in fact $f$ is uniformly continuous on $A$, so
  for any $\epsilon > 0$, there exists $\delta > 0$
  such that $|f(x) - f(y)| < \epsilon$ if
  $d(x, y) < \delta$. Choose a partition of $A = I$
  into closed subintervals $I_1, \dots, I_N$ such that
  $I = I_1 \cup I_2 \cup \dots \cup I_N$ and
  $\Vol(I_i \cap I_j) = 0$,\footnote{Cutting the box into little rectangles is sufficient to do this, for example.} and $d(x, y) < \delta$
  if $x, y \in I_j$. Define
  \[
    f_1(x) =
    \begin{cases}
      \min\{f(y) : y \in I_j\} & \text{if } x \in I_j \text{ and } x \notin I_k \text{ for } k \ne j \\
      -M & \text{otherwise}.
    \end{cases}
  \]
  Similarly define
  \[
    f_2(x) =
    \begin{cases}
      \max\{f(y) : y \in I_j\} & \text{if } x \in I_j \text{ and } x \notin I_k \text{ for } k \ne j \\
      M & \text{otherwise}.
    \end{cases}
  \]
  By construction we have $f_1 \le f \le f_2$, and\footnote{Here $\Int I_j$ denotes the \emph{interior} of $I_j$.}
  \[
    \int_I (f_2 - f_1)
    = \sum_{j = 1}^N \int_{\Int I_j} (f_2 - f_1)
    \le \sum_{j = 1}^N \epsilon \Vol(I_j)
    = \epsilon \Vol(I)
  \]
  by uniform continuity.
  So $f$ is integrable on $A$ in
  this case.
  Rest of the proof for next class.
\end{proof}

  \chapter{Apr.~2 --- Iterated Integrals}

\section{Double Integrals and Iterated Integrals}
\begin{theorem}
  Suppose $f(x, y)$ is integrable on
  $D = [a, b] \times [c, d]$ and for each $x \in [a, b]$,
  $f(x, y)$ is integrable on $[c, d]$.  Then
  \[
    \int_a^b  dx \left[\int_c^d f(x, y)\, dy\right]
    \text{ exists} \quad \text{and} \quad
    \int_a^b  dx \left[\int_c^d f(x, y)\, dy\right]
    = \iint_D f(x, y)\, dx dy
  \]
\end{theorem}

\begin{proof}
  Divide $[a, b]$ and $[c, d]$ by
  $a = x_0 < x_1 < \cdots < x_n = b$ and
  $c = y_0 < y_1 < \cdots < y_m = d$.
  Take $\xi_i \in [x_{i-1}, x_i]$ and
  $\eta_j \in [y_{j-1}, y_j]$ for each $1 \le i \le n$
  and $1 \le j \le m$. Denote
  \[
    \Delta = \max\{\Delta x_i, \Delta y_j\}, \quad
    \Delta x_i = x_i - x_{i-1},\, \Delta y_j = y_j - y_{j-1}.
  \]
  Then
  \[
    S = \sum_{i = 1}^n \sum_{j = 1}^m f(\xi_i, \eta_j) \Delta x_i \Delta y_j \to \iint_D f(x, y)\, dx dy
  \]
  as $\Delta \to 0$ since $f$ is integrable on $D$.
  Let $\lambda_1 = \max\{\Delta x_i\}$ and
  $\lambda_2 = \max\{\Delta y_j\}$. Note that
  $\Delta \to 0$ if and only if $\lambda_1, \lambda_2 \to 0$.
  Then we can write
  \[
    S = \sum_{i = 1}^n \left[\sum_{j = 1}^m f(\xi_i, \eta_j) \Delta y_j\right] \Delta x_i
    \to \sum_{i = 1}^n \left[\int_c^d f(\xi_i, y)\, dy\right] \Delta x_i
  \]
  as $\lambda_2 \to 0$ since $f(x, y)$ is integrable in
  $y$ for any fixed $x$. Now taking $\lambda_1 \to 0$,
  \[
    \int_a^b  dx \left[\int_c^d f(x, y)\, dy\right]
    = \lim_{\lambda_1 \to 0} \sum_{i = 1}^n \left[\int_c^d f(\xi_i, y)\, dy\right] \Delta x_i
    = \lim_{\Delta \to 0} S = \iint_D f(x, y)\, dx dy
  \]
  from before. So the left-hand side integral exists and
  equals the double integral.
\end{proof}

\begin{remark}
  Knowing that $f(x, y)$ is integrable on
  $D = [a, b] \times [c, d]$ does \emph{not} in
  general imply that
  $f(x, y)$ is integrable on $[c, d]$ for any fixed
  $x \in [a, b]$.
\end{remark}

\begin{example}
  Let $D = [0, 1] \times [0, 1]$ and
  \[
    f(x, y) =
    \begin{cases}
      1 / p & \text{if $x = r / p$ for $r, p$ coprime, $y$ is irrational} \\
      1 / q & \text{if $y = s / q$ for $s, q$ coprime, $x$ is irrational} \\
      0 & \text{if $x, y$ are both rational or irrational}.
    \end{cases}
  \]
  Observe that $f$ is integrable on $D$ since for any
  $\epsilon > 0$, $f(x, y) \ge \epsilon$ only on a finite
  number of straight lines. So there exists a partition
  such that the finite number of lines are contained in a
  union of small rectangular domains with area
  $< \epsilon$. Then the total oscillation amplitude
  of this partition for $f(x, y)$ is
  \[
    \sum_{i = 1}^n \omega_i(f) \Delta \sigma_i
    = \sum_{\text{oscillation $\le \epsilon$}} \omega_i(f) \Delta \sigma_i
    + \sum_{\text{contains lines}} \omega_i(f) \Delta \sigma_i
    \le \epsilon + \epsilon = 2\epsilon
  \]
  where $\Delta \sigma_i = \Delta x_i \Delta y_i$. So
  $f(x, y)$ is integrable on $D$. Now for a fixed
  $x = r / p$, we have
  \[
    f(r / p, y) =
    \begin{cases}
      0 & \text{if $y$ is rational} \\
      1 / p & \text{if $y$ is irrational}.
    \end{cases}
  \]
  This function is not integrable for $y \in [0, 1]$.
\end{example}

\begin{corollary}
  If $f$ is integrable on $D = [a, b] \times [c, d]$ 
  and for fixed $y \in [c, d]$, $f(x, y)$ is
  integrable on $[a, b]$. Then
  \[
    \iint_D f(x, y)\, dx dy = \int_c^d \left[\int_a^b f(x, y)\, dx\right] dy.
  \]
\end{corollary}

\begin{proof}
  Repeat the same proof but start by integrating in $x$.
\end{proof}

\begin{theorem}
  Let $\varphi_1, \varphi_2 : [a, b] \to \R$ be
  continuous and define the Type I region
  \[
    D = \{(x, y) \in \R^2 : a \le x \le b,\, \varphi_1(x) \le y \le \varphi_2(x)\}.
  \]
  Assume $f(x, y)$ is integrable on $D$ and for each
  $x \in [a, b]$,
  \[
    I(x) = \int_{\varphi_1(x)}^{\varphi_2(x)} f(x, y)\, dy
  \]
  exists. Then
  \[
    \iint_D f(x, y)\, dx dy = \int_a^b \left[\int_{\varphi_1(x)}^{\varphi_2(x)} f(x, y)\, dy\right] dx.
  \]
\end{theorem}

\begin{proof}
  Let $c = \inf_{a \le x \le b} \varphi_1(x)$ and
  $d = \sup_{a \le x \le b} \varphi_2(x)$, so that
  $D \subseteq R = [a, b] \times [c, d]$. Define
  \[
    f^*(x, y) =
    \begin{cases}
      f(x, y) & \text{if $(x, y) \in D$} \\
      0 & \text{if $(x, y) \in R \setminus D$}.
    \end{cases}
  \]
  Then $f^*$ is integrable on $R$ and
  \[
    \iint_R f^*(x, y)\, dx dy = \iint_D f(x, y)\, dx dy
  \]
  since
  \[
    \iint_R f^*(x, y)\, dx dy = \iint_D f^*(x, y)\, dx dy + \iint_{R \setminus D} f^*(x, y)\, dx dy = \iint_D f(x, y)\, dx dy.
  \]
  since $D$ is a measurable set. Now apply the
  previous theorem to $f^*$ on $R$ to get
  \[
    \iint_D f(x, y)\, dx dy
    \iint_R f^*(x, y)\, dx dy
    = \int_a^b \left[\int_c^d f^*(x, y)\, dy\right] dx
    = \int_a^b \left[\int_{\varphi_1(x)}^{\varphi_2(x)} f(x, y)\, dy\right] dx,
  \]
  which is the desired result.
\end{proof}

\begin{corollary}
  Let $\phi_1, \phi_2 : [c, d] \to \R$ be continuous and
  define the Type II region
  \[
    D = \{(x, y) \in \R^2 : c \le y \le d,\, \phi_1(y) \le x \le \phi_2(y)\}.
  \]
  Assume $f$ is integrable on $D$ and for each
  $y \in [c, d]$, $f(x, y)$ is integrable
  on $[\phi_1(y), \phi_2(y)]$. Then
  \[
    \iint_D f(x, y)\, dx dy = \int_c^d \left[\int_{\phi_1(y)}^{\phi_2(y)} f(x, y)\, dx\right] dy.
  \]
\end{corollary}

\begin{proof}
  Repeat the same proof.
\end{proof}

\section{Applications}
\begin{example}
  Assume $p(x)$ is integrable on $[a, b]$ and
  $p(x) \ge 0$, and $f(x), g(x)$ are increasing
  on $[a, b]$. Show
  \[
    \int_a^b p(x)f(x)\, dx \int_a^b p(x)g(x)\, dx
    \le \int_a^b p(x)\, dx \int_a^b p(x)f(x)g(x)\, dx.
  \]
\end{example}

\begin{proof}
  Let
  \[
  \Delta = \int_a^b p(x)f(x)g(x)\, dx \int_a^b p(x)\, dx - \int_a^b p(x)f(x)\, dx \int_a^b p(x)g(x)\, dx.
  \]
  Replace the integration variable from $x$ to $y$ in the
  two parts to get
  \[
    \Delta = \int_a^b p(x)f(x)g(x)\, dx \int_a^b p(y)\, dy
    - \int_a^b p(x)f(x)\, dx \int_a^b p(y)g(y)\, dy
  .\]
  Let $D = [a, b] \times [a, b]$, and by the previous
  theorem we get
  \[
    \Delta
    = \iint_D \left[p(x) f(x) g(x) p(y) - p(x) f(x) p(y) g(y)\right] dx dy
    = \iint_D p(x)p(y)f(x)\left[g(x) - g(y)\right] dx dy. \tag{1}
  \]
  By symmetry,
  \[
    \Delta = \iint_D p(x)p(y)f(y) \left[g(y) - g(x)\right] dx dy \tag{2}.
  \]
  also. Add $(1)$ and $(2)$ and divide by $2$ to get
  \[
    \Delta = \frac{1}{2} \iint_D p(x) p(y) [f(x) - f(y)][g(x) - g(y)] dx dy \ge 0
  \]
  since $p(x) \ge 0$, and $f(x) - f(y)$ and
  $g(x) - g(y)$ have the same sign since $f, g$ are
  increasing.
\end{proof}

\begin{example}
  To compute
  \[
    I = \int_\R e^{-x^2} \, dx,
  \]
  we can use a similar trick and write
  \[
    I^2 = \int_\R e^{-x^2} \, dx \int_\R e^{-y^2} \, dy
    = \iint_{\R^2} e^{-(x^2 + y^2)} \, dx dy
    = \int_0^\infty \int_0^{2\pi} e^{-r^2} r \, d\theta dr
    = 2\pi \int_0^\infty e^{-r^2} r \, dr
    = \pi,
  \]
  so we get that $I = \sqrt{\pi}$. Note that we
  get a factor for $u$-substitution after changing to
  polar coordinates.
\end{example}

\section{Triple Integrals and Iterated Integrals in 3D}
\begin{theorem}
  Let $D$ be a measurable region in the $xy$-plane and
  $\varphi_1, \varphi_2 : D \to \R$ be continuous, and
  define
  \[
    V = \{(x, y, z) \in \R^3 : (x, y) \in D,\, \varphi_1(x, y) \le z \le \varphi_2(x, y)\}.
  \]
  If $f(x, y, z)$ is integrable on $V$ and for each
  $(x, y) \in D$, $f(x, y, z)$ is integrable on
  $[\varphi_1(x, y), \varphi_2(x, y)]$, then
  \[
    \iiint_V f(x, y, z)\, dx dy dz
    = \iint_D dxdy \left[\int_{\varphi_1(x, y)}^{\varphi_2(x, y)} f(x, y, z)\, dz\right].
  \]
\end{theorem}

\begin{proof}
  Similar idea, zero-extend $f$ to a box in $\R^3$ and
  integrate.
\end{proof}

\begin{theorem}
  Suppose we have $D : [c, d] \to \R^2$ and a region of
  the form
  \[
    V = \{(x, y, z) \in \R^3 : c \le z \le d,\, (x, y) \in D(z)\}.
  \]
  If $f(x, y, z)$ is integrable on $V$ and for each
  $z \in [c, d]$, $D(z)$ is measurable and
  $f(x, y, z)$ is integrable on $D(z)$, then
  \[
    \iiint_V f(x, y, z)\, dx dy dz
    = \int_c^d dz \left[\iint_{D(z)} f(x, y, z)\, dx dy\right].
  \]
\end{theorem}

\begin{proof}
  Use a similar idea.
\end{proof}

  \chapter{Apr.~4 --- Change of Variables}

\section{Change of Variables for Double Integrals}
Consider a region $G$ in $uv$-coordinates, and define
a transformation $T$ on $G$ to $xy$-coordinates by
\[
  T :
  \begin{cases}
    x = x(u,v) \\
    y = y(u,v).
  \end{cases}
\]
Set $\Omega = T(G)$. Assume that $T$ is injective
and $T \in C^2(\overline{G})$.

\begin{definition}
  We define the \emph{Jacobian} of $T$ to be
  \[
    J(u, v) = \frac{\partial(x, y)}{\partial(u, v)}
    = \det
    \begin{pmatrix}
      \partial x / \partial u & \partial x / \partial v \\
      \partial y / \partial u & \partial y / \partial v
    \end{pmatrix}.
  \]
\end{definition}

\begin{remark}
  The Jacobian $J$ does not change sign in $G$.
\end{remark}

The goal now is to show the change of variables formula
\[
  \iint_{\Omega} f(x, y)\, dxdy
  = \iint_G f(x(u, v), y(u, v)) |J(u, v)|\, dudv.
\]
Notice the extra $|J(u, v)|$ factor on the right-hand side.

\begin{remark}
  Let $\Gamma$ be a closed curve in the counterclockwise
  orientation that encloses $\Omega$. If $\Gamma$ is
  piecewise continuous, then recall that
  by Green's theorem we have
  \[
    \area(\Omega) = \frac{1}{2} \oint_{\Gamma} x\, dy - y\, dx.
  \]
\end{remark}

\begin{lemma}
  Let $G = (u_0, u_0 + h) \times (v_0, v_0 + h)$ be a
  square region for some $h > 0$. Assume $T$ is
  injective and $T \in C^2(\overline{G})$. Then\footnote{Recall that $m(TG)$ denotes the measure of $TG$. Also $TG$ is the image of $G$ under $T$.}
  \[
    m(TG) = \iint_G |J(u, v)|\, dudv.
  \]
\end{lemma}

\begin{proof}
  When $J > 0$, $\partial(TG)$ is counterclockwise, so by the
  previous remark we can calculate
  \begin{align*}
    2 \area(TG)
    &= \int_{u_0}^{u_0 + h} \left[x(u, v_0) \frac{\partial y(u, v_0)}{\partial u} - y(u, v_0) \frac{\partial x(u, v_0)}{\partial u}\right] du \\
    &\quad + \int_{v_0}^{v_0 + h} \left[x(u_0 + h, v) \frac{\partial y(u_0 + h, v)}{\partial v} - y(u_0 + h, v) \frac{\partial x(u_0 + h, v)}{\partial v}\right] dv \\
    &\quad +\int_{u_0 + h}^{u_0} \left[x(u, v_0 + h) \frac{\partial y(u, v_0 + h)}{\partial u} - y(u, v_0 + h) \frac{\partial x(u, v_0 + h)}{\partial u}\right] du \\
    &\quad + \int_{v_0 + h}^{v_0} \left[x(u_0, v) \frac{\partial y(u_0, v)}{\partial v} - y(u_0, v) \frac{\partial x(u_0, v)}{\partial v}\right] dv \\
    &= I + II + III + IV.
  \end{align*}
  By the fundamental theorem of calculus, we get
  \begin{align*}
    I
    &= \int_{u_0}^{u_0 + h}
    \left[x(u, v_0 + h) \frac{\partial(u, v_0 + h)}{\partial u} - x(u, v_0) \frac{\partial y(u, v_0)}{\partial u}\right] du \\
    &= \int_{u_0}^{u_0 + h} du \left[\int_{v_0}^{v_0 + h} \frac{\partial x(u, v)}{\partial v} \frac{\partial y(u, v)}{\partial u} + x(u, v) \frac{\partial^2 y(u, v)}{\partial v \partial u}\right] dv
    = \iint_G \left[\frac{\partial x}{\partial v} \frac{\partial y}{\partial u} + x \frac{\partial^2 y}{\partial v \partial u}\right] dudv.
  \end{align*}
  Similarly apply this to the other three sides and
  combine to get (note that $T$ is $C^2$ on $\overline{G}$),
  \begin{align*}
    2\area(TG)
    &= \iint_G \left[\frac{\partial x}{\partial u}\frac{\partial y}{\partial v} + x \frac{\partial^2 y}{\partial u \partial v}\right] dudv
    - \iint_G \left[\frac{\partial x}{\partial v}\frac{\partial y}{\partial u} + x \frac{\partial^2 y}{\partial v \partial u}\right] dudv \\
    &\quad + \iint_G \left[\frac{\partial y}{\partial v}\frac{\partial x}{\partial u} + y \frac{\partial^2 x}{\partial v \partial u}\right] dudv
    - \iint_G \left[\frac{\partial y}{\partial u}\frac{\partial x}{\partial v} + y \frac{\partial^2 x}{\partial u \partial v}\right] dudv \\
    &= 2 \iint_G \left(\frac{\partial x}{\partial u} \frac{\partial y}{\partial v} - \frac{\partial x}{\partial v} \frac{\partial y}{\partial u}\right)
    = 2 \iint_G J(u, v)\, dudv,
  \end{align*}
  which is the desired result. The absolute
  value of $J$ is necessary when $J < 0$, since
  $\partial(TG)$ is clockwise.
\end{proof}

\begin{theorem}
  Let $T : G \to \Omega$, where $G$ and $\Omega$ are
  both measurable sets in $\R^2$. Suppose $T$ is
  bijective and $T \in \C^2(\overline{G})$, (and that
  $J(u, v) \ne 0$ in $G$).\footnote{This last condition is implied by the previous two.} If $f(x, y)$ is integrable
  on $\overline{\Omega}$, then
  \[
    \iint_{\Omega} f(x, y)\, dxdy
    = \iint_G f(x(u, v), y(u, v)) |J(u, v)|\, dudv.
  \]
\end{theorem}

\begin{proof}
  Use parallel lines with distance $h$ to cut
  $G$ into small ``rectangular'' (except
  near the boundary) sets $\Delta \sigma_i$. Let
  \[
    \Delta = \{\underbrace{\Delta \sigma_1, \Delta \sigma_2, \dots, \Delta \sigma_n}_{\text{contained in $\Int G$}}, \underbrace{\Delta \sigma_{n + 1}, \dots, \Delta \sigma_{n + p}}_{\text{intersects with $\partial G$}}\}.
  \]
  Let $(u_i, v_i) \in \Delta \sigma_i$ with
  $1 \le i \le n$. Then $T : \Delta \sigma_i \to T(\Delta \sigma_i)$,
  so that $x_i = x(u_i, v_i)$ and $y_i = y(u_i, v_i)$.
  By the lemma,
  \[
    m(T(\Delta \sigma_i)) = \iint_{\Delta \sigma_i} |J(u, v)|\, dudv
    = |J(\overline{u}_i, \overline{v}_i)| m(\Delta \sigma_i)
    = h^2 |J(\overline{u}_i, \overline{v}_i)|.
  \]
  for some $(\overline{u}_i, \overline{v}_i) \in \Delta \sigma_i$ by the middle value theorem
  (also note that $m(\Delta \sigma_i) = h^2$). So if
  $f$ is continuous on $G$, then
  \[
    \iint_G f\, dxdy = f(\xi) m(G)
  \]
  where $\xi$ is some point in $\overline{G}$. Now
  consider a limit of the Riemann sums. We have
  \[
    \lim_{h \to 0} \sum_{i = 1}^n f(x(u_i, v_i), y(u_i, v_i)) \underbrace{|J(\overline{u}_i, \overline{v}_i)| m(\Delta \sigma_i)}_{= m(T(\Delta \sigma_i))}
    = \iint_{\Omega} f(x, y)\, dxdy
  \]
  since for the sets near the boundary,
  \[
    \lim_{h \to 0} \sum_{i = n + 1}^{n + p} m(T(\Delta \sigma_i)) = 0 \implies
    \lim_{h \to 0} \sum_{i = n + 1}^{n + p} f(x_i, y_i) m(T(\Delta \sigma_i)) = 0.
  \]
  and $f$ is integrable and thus bounded on $\overline{\Omega}$.
  Now if $J(\overline{u}_i, \overline{v}_i) \to J(u_i, v_i)$,
  then
  \[
    \sum_{i = 1}^n f(x(u_i, v_i), y(u_i, v_i)) |J(u_i, v_i)| m(\Delta \sigma_i)
    \to \iint_G f(x(u, v), y(u, v)) |J(u, v)|\, dudv
  \]
  as $h \to 0$. To see that $J(\overline{u}_i, \overline{v}_i) \to J(u_i, v_i)$, observe that for
  any $\epsilon > 0$, since $|J(u, v)|$ is uniformly
  continuous on $\overline{G}$, there exists
  $\delta > 0$ such that
  \[
    d((u_i, v_i), (\overline{u}_i, \overline{v}_i)) < \delta
    \implies \left| |J(u_i, v_i)| - |J(\overline{u}_i, \overline{v}_i)| \right| < \frac{\epsilon}{M \cdot m(G)},
  \]
  where $|f| \le M$ on $\overline{G}$. So if
  $h < \delta / \sqrt{2}$, then we have
  \[
   \left| \sum_{i = 1}^n f(x_i, y_i) |J(\overline{u}_i, \overline{v}_i)| m(\Delta \sigma_i)
    - \sum_{i = 1}^n f(x_i, y_i) |J(u_i, v_i)| m(\Delta \sigma_i)\right|
    \le \frac{\epsilon}{M \cdot m(G)} \cdot M \cdot m(G) = \epsilon.
  \]
  This finishes the desired result.
\end{proof}

\begin{remark}
  The theorem actually also holds for $T \in C^1(\overline{G})$.
  This is because if
  \[
    \iint_{G} f(x(u, v), y(u, v)) |J(u, v)|\, dudv
    = \iint_{\Omega} f(x, y)\, dxdy
  \]
  for any $T \in C^2$, then we can choose $T^n \in C^2$
  such that $|T_n - T|_{C^1} \to 0$. Then we have
  \[
    \iint_G f(x_n(u, v), y_n(u, v)) |J_n(u, v)|\, dudv
    = \iint_{T_n(G)} f(x, y)\, dxdy.
  \]
  As $n \to \infty$, we have $T_n(G) \to T(G) = \Omega$,
  so that
  \[
    \text{LHS} \to \iint_G f(x(u, v), y(u, v)) |J(u, v)|\, dudv
    \quad \text{and} \quad
    \text{RHS} \to  \iint_{\Omega} f(x, y)\, dxdy
  \]
  since we have uniform convergence.
\end{remark}

\section{Change of Variables in 3D}

Let $G$ be a measurable set in $\R^3$ with $uvw$-coordinates
and
\[
  T :
  \begin{cases}
    x = x(u, v, w) \\
    y = y(u, v, w) \\
    z = z(u, v, w)
  \end{cases}.
\]
Set $D = T(G)$. Assume that $T$ is a homeomorphism and
$T \in C^1(\overline{G})$. Then define
\[
  J(u, v, w) = \frac{\partial(x, y, z)}{\partial(u, v, w)}
  = \det
  \begin{pmatrix}
    \partial x / \partial u & \partial x / \partial v & \partial x / \partial w \\
    \partial y / \partial u & \partial y / \partial v & \partial y / \partial w \\
    \partial z / \partial u & \partial z / \partial v & \partial z / \partial w
  \end{pmatrix}.
\]
If $f$ is integrable on $\overline{D}$, then
\[
  \iiint_D f(x, y, z)\, dxdydz
  = \iiint_G f(x(u, v, w), y(u, v, w), z(u, v, w)) |J(u, v, w)|\, dudvdw.
\]
The proof is similar to the 2D case (get a formula for
the cube and proceed similarly).

  \chapter{Apr.~9 --- Change of Variables, Part 2}

\section{Change of Variables Examples}
\begin{example}
  Compute
  \[
    I = \iint_D \sqrt{\sqrt{x} + \sqrt{y}}\, dxdy
  \]
  where $D$ is enclosed by $C : \sqrt{x} + \sqrt{y} = 1$
  and the coordinate axes.
\end{example}

\begin{proof}
  We would like $C : x = \cos^4 t, y = \sin^4 t$
  for $0 \le t \le \pi / 2$. So let
  \[
  T :
  \begin{cases}
    x = r \cos^4 t, & 0 \le r \le 1\\
    y = r \sin^4 t, & 0 \le t \le \pi / 2.
  \end{cases}
  \]
  This gives $T : [0, 1] \times [0, \pi / 2] \to D$
  which sends $(r, t) \to (x, y)$. Then
  \[
    J(r, t) = \det
    \begin{pmatrix}
      \partial x / \partial r & \partial x / \partial t\\
      \partial y / \partial r & \partial y / \partial t
    \end{pmatrix}
    = 4r \cos^3 t \sin^3 t.
  \]
  Now making the change of variables in the integral
  gives
  \[
    I = \int_0^{\pi / 2} \int_0^1 r^{1 / 4} 4r \cos^3 t \sin^3 t\, drdt
    = \int_0^1 4 r^{5 / 4}\, dr \int_0^{\pi / 2} \cos^3 t \sin^3 t\, dt
    = \frac{2}{15}.
  \]
  The latter integral can be calculated by writing
  $\sin^2 t = 1 - \cos^2 t$ and using the substitution
  $u = \cos t$.
\end{proof}

\begin{example}
  Let $h = \sqrt{\alpha^2 + \beta^2 + \gamma^2} > 0$
  and $f$ be continuous on $[-h, h]$. Show that
  \[
    \iiint_V f(\alpha x + \beta y + \gamma z)\, dx dy dz
    = \pi \int_{-1}^1 (1 - \zeta^2) f(h \zeta)\, d\zeta,
  \]
  where $V : x^2 + y^2 + z^2 \le 1$.
\end{example}

\begin{proof}
  Setting $\alpha x + \beta y + \gamma z = 0$ defines
  a plane passing through the origin in $\R^2$. Choose
  the $\zeta$-axis to be normal to this plane, and let
  $\xi, \eta$ be two orthogonal axes on the plane. Define
  the coordinate transformation
  $T^{-1} : (\xi, \eta, \zeta) \to (x, y, z)$ be
  given by
  \[
    \begin{cases}
      \xi = a_1 x + b_1 y + c_1 z \\
      \eta = a_2 x + b_2 y + c_2 z \\
      \zeta = (\alpha x + \beta y + \gamma z) / h,
    \end{cases}
  \]
  where $\{(a_1, b_1, c_1)^T, (a_2, b_2, c_2)^T\}$ is an
  orthonormal basis for the plane. Note that $(\alpha, \beta, \gamma)^T$ is the normal vector of the plane.
  Then
  \[
    \frac{\partial(\xi, \eta, \zeta)}{\partial(x, y, z)}
    = \det
    \begin{pmatrix}
      a_1 & b_1 & c_1 \\
      a_2 & b_2 & c_2 \\
      \alpha / h & \beta / h & \gamma / h
    \end{pmatrix} = 1
  \]
  since this matrix is orthogonal and preserves
  orientation by construction. This gives
  \[
    J(\xi, \eta, \zeta) =
    \frac{\partial(x, y, z)}{\partial(\xi, \eta, \zeta)}
    = \left(\frac{\partial(\xi, \eta, \zeta)}{\partial(x, y, z)}\right)^{-1} = 1.
  \]
  Now under the transformation $T^{-1}$, we have
  \[
    V : \{x^2 + y^2 + z^2 \le 1\} \xrightarrow{T^{-1}}
    G : \{\xi^2 + \eta^2 + \zeta^2 \le 1\}.
  \]
  Substituting this into the integral, we get
  \begin{align*}
    \iiint_V f(\alpha x + \beta y + \gamma z)\, dx dy dz
    &= \iiint_{\xi^2 + \eta^2 + \zeta^2 \le 1} f(h \zeta)\, d\xi d\eta d\zeta \\
    &= \int_{-1}^1 d\zeta \iint_{\xi^2 + \eta^2 \le 1 - \zeta^2} f(h \zeta)\, d\xi d\eta
    = \int_{-1}^1 f(h \zeta) \pi (1 - \zeta^2)\, d\zeta,
  \end{align*}
  which is the desired result.
\end{proof}

\section{The Cylindrical and Spherical Coordinate Systems}
The \emph{cylindrical} coordinate system is the change of
variables $(x, y, z) \to (r, \theta, z)$ given by
\[
  \begin{cases}
    x = r \cos \theta \\
    y = r \sin \theta \\
    z = z
  \end{cases}
  \text{with } J(r, \theta, z) = r.
\]
The \emph{spherical} coordinate system is the change of
variables $(x, y, z) \to (r, \varphi, \theta)$ given by
\[
\begin{cases}
  x = r \sin \varphi \cos \theta \\
  y = r \sin \varphi \sin \theta \\
  z = r \cos \varphi
\end{cases}
\text{with } J(r, \varphi, \theta) = r^2 \sin \varphi.
\]
For spherical coordinates we have
$r \ge 0$, $0 \le \varphi \le \pi$, and
$0 \le \theta \le 2\pi$.

\begin{example}
  Find
  \[
    \iiint_V (x^2 + y^2 + z^2)\, dx dy dz
  \]
  where $V : x^2 / a^2 + y^2 / b^2 + z^2 / c^2 \le 1$.
\end{example}

\begin{proof}
  Define the transformation
  \[
    T :
    \begin{cases}
      x = a r \sin \varphi \cos \theta \\
      y = b r \sin \varphi \sin \theta \\
      z = c r \cos \varphi.
    \end{cases}
  \]
  Then we have
  \[
    \frac{\partial(x, y, z)}{\partial(r, \varphi, \theta)}
    = abc r^2 \sin \varphi.
  \]
  Making the substitution, this gives
  \begin{align*}
    &\iiint_V (x^2 + y^2 + z^2)\, dx dy dz \\
    &\quad = \int_0^{2\pi} d\theta \int_0^\pi d\varphi \int_0^1 (a^2 \sin^2 \varphi \cos^2 \theta + b^2 \sin^2 \varphi \sin^2 \theta + c^2 \cos^2 \varphi) abc r^4 \sin \varphi\, dr \\
    &\quad = \frac{4}{15} abc(a^2 + b^2 + c^2) \pi
  \end{align*}
  as the desired result.
\end{proof}

\section{More Change of Variables Examples}
\begin{example}
  Find
  \[
    I = \int_0^\infty \frac{e^{-ax} - e^{-bx}}{x}\, dx
  \]
  for $a, b > 0$.
\end{example}

\begin{proof}
  First observe that for fixed $x > 0$, we can write
  \[
    \frac{e^{-ax} - e^{-bx}}{x} = \left.-\frac{e^{-yx}}{x}\right|_{y = a}^{y = b}
      = \int_a^b \frac{d}{dy} (-\frac{e^{-yx}}{x})\, dy
      = \int_a^b e^{-xy}\, dy
  \]
  by the fundamental theorem of calculus.
  This then gives
  \[
    I = \int_0^\infty dx \int_a^b e^{-xy}\, dy
    = \lim_{T \to \infty} \int_0^T dx \int_a^b e^{-xy}\, dy
  \]
  For the latter integral, we can switch the order of
  integration to get
  \begin{align*}
    \int_0^T dx \int_a^b e^{-xy}\, dy
    &= \int_a^b dy \int_0^T e^{-xy}\, dx
    = \int_a^b dy \left[-\frac{e^{-xy}}{y}\right]_{x = 0}^{x = T}
    = \int_a^b \frac{1 - e^{-Ty}}{y}\, dy \\
    &= \int_a^b \frac{1}{y}\, dy - \int_a^b \frac{e^{-Ty}}{y}\, dy
    = \ln(b / a) - \int_{Ta}{Tb} \frac{e^{-Ty'}}{y'}\, dy',
  \end{align*}
  where $y' = Ty$. Letting $T \to \infty$, we have
  \[
    I = \lim_{T \to \infty} \int_0^T dx \int_a^b e^{-xy}\, dy
    = \ln(b / a) - \lim_{T \to \infty} \int_{Ta}^{Tb} \frac{e^{-Ty'}}{y'}\, dy'
    = \ln(b / a) \tag{$*$}
  \]
  since
  \[
    \int_1^\infty \frac{e^{-y}}{y}\, dy
    \text{ converges} \quad \text{and} \quad
    h(A) = \int_0^A \frac{e^{-y}}{y}\, dy \text{ is Cauchy in $A$},
  \]
  so that the latter integral in $(*)$ vanishes as
  $T \to \infty$.
\end{proof}

\begin{remark}
  We might have been able to just switch orders at the
  very beginning with the improper integral, but we need
  to argue uniform
  convergence so that the exchange is permissible.
\end{remark}

\begin{exercise}
  Let $f$ be differentiable on $(0, \infty)$ and suppose
  that
  \[
    \int_1^\infty \frac{f(t)}{t}\, dt
  \]
  exists. Then for $a, b > 0$, show that
  \[
    I = \int_0^\infty \frac{f(ax) - f(bx)}{x}\, dx
    = f(0) \ln(b / a).
  \]
\end{exercise}

\begin{remark}
  The previous example is a special case of this exercise,
  with $f(x) = e^{-x}$.
\end{remark}

  \chapter{Apr.~11 --- Green's Theorem}

\section{Midterm 2 Problems}
The following was Problem 3 on Midterm 2:
\begin{exercise}
  For a given function $y(t) \in C[0, 1]$ (i.e. continuous
  functions in $[0, 1]$) and a constant $\lambda \in \R$
  with $|\lambda| < 1$, show that the integral equation
  \[
    x(t) - \lambda \int_0^1 e^{t - s} x(s)\, ds = y(t) \tag{$*$}
  \]
  has a unique solution $x(t) \in C[0, 1]$.
\end{exercise}

\begin{remark}
  A common idea was to define
  \[
    T : x(t) \mapsto y(t) + \lambda \int_0^1 e^{t - s} x(s)\, ds.
  \]
  and $\rho(u, v) = \sup_{t \in [0, 1]} |u(t) - v(t)|$.
  But we run into problems since
  \[
    \rho(Tu, Tv)
    \le |\lambda| \sup_{t \in [0, 1]} \left| \int_0^1 e^{t - s} (u(s) - v(s))\, ds \right|
    \le |\lambda| \rho(u, v) \sup_{t \in [0, 1]} e^t \int_0^1 e^{-s}\, ds
  \]
  But $\sup_{t \in [0, 1]} e^t = e$ and the integral
  evaluates to $1 - e^{-1}$, so we have
  \[
    \rho(Tu, Tv) \le |\lambda| e (1 - e^{-1}) \rho(u, v)
    \le |\lambda| (e - 1) \rho(u, v).
  \]
  This does not work since $(e - 1) > 1$.
\end{remark}

\begin{proof}
  A better idea is to first transform $(*)$ by
  multiplying by $e^{-t}$:
  \[
    e^{-t} x(t) - \lambda \int_0^1 e^{-s} x(s)\, ds = e^{-t} y(t)
  \]
  since $t$ is independent of the integral. Then set
  $z(t) = e^{-t} x(t)$ and define
  \[
    T : z(t) \mapsto e^{-t} y(t) + \lambda \int_0^1 z(s)\, ds.
  \]
  Using this we can get
  \[
    d(Tu, Tv) = \lambda \sup_{t \in [0, 1]} \left|\int_0^1 u(s) - v(s)\, ds\right|
    \le \lambda \rho(u, v),
  \]
  which is good enough. Then by the contraction
  mapping principle, $T$ has a unique fixed point, and
  multiplying by $e^t$ gives a unique solution to $(*)$.
\end{proof}

\section{Green's Theorem}
\begin{theorem}[Green's theorem]
  Let $D$ be a region in $\R^2$ enclosed by a finite
  number of measurable curves, and $P(x, y), Q(x, y)$
  be continuous on $D$ with continuous partial
  derivatives. Then
  \[
    \int_{\partial D} P\, dx + Q\, dy = \iint_D \left(\frac{\partial Q}{\partial x} - \frac{\partial P}{\partial y}\right) dx dy. \tag{$*$}
  \]
  Here $\partial D$ is oriented so that $\{\vec{n}, \vec{\tau}\}$ is a right-handed coordinate
  system, where $\vec{n}$ is the outward normal to the
  region and $\vec{\tau}$ is the tangent vector of
  the curve $\partial D$.\footnote{In other words, the curve $\partial D$ is oriented so that the region $D$ is to its \emph{left}.}
\end{theorem}

\begin{proof}
  First we consider Type I domains, i.e. regions of
  the form
  \[
    D = \{a \le x \le b,\, \varphi(x) \le y \le \psi(x)\}.
  \]
  Let $A, B, C, D$ be the four vertices of the
  region $D$, going counterclockwise and starting
  in the bottom-left, i.e. we have
  \[
    A = (a, \varphi(a)),\, B = (b, \varphi(b)),\, C = (b, \psi(b)),\, D = (a, \psi(a)).
  \]
  Then
  \[
    \int_{\widetilde{AB}} P\, dx = \int_a^b P(x, \varphi(x))\, dx,
    \quad
    \int_{\widetilde{CD}} P\, dx = -\int_a^b P(x, \psi(x))\, dx,
  \]
  and
  \[
    \int_{\widetilde{BC}} P\, dx = \int_{\widetilde{DA}} P\, dx = 0
  \]
  since $x$ is constant on $\widetilde{BC}$ and $\widetilde{DA}$. Thus
  \[
    \int_{\partial D} P\, dx = \int_a^b P(x, \varphi(x))\, dx - \int_a^b P(x, \psi(x))\, dx
    = \int_a^b [P(x, \varphi(x)) - P(x, \psi(x))]\, dx.
  \]
  By the fundamental theorem of calculus, we have
  \[
    P(x, \varphi(x)) - P(x, \psi(x)) = -\int_{\varphi(x)}^{\psi(x)} \frac{\partial P}{\partial y}(x, y)\, dy,
  \]
  so that
  \[
    \int_{\partial D} P\, dx = -\int_a^b \int_{\varphi(x)}^{\psi(x)} \frac{\partial P}{\partial y}(x, y)\, dy dx
    = -\iint_D \frac{\partial P}{\partial y}\, dx dy.
  \]
  For Type II domains, i.e. regions of the form
  \[
    D = \{c \le y \le d, \varphi(y) \le x \le \psi(y)\}.
  \]
  Using an identical argument,
  \[
    \int_{\partial D} Q\, dy = \iint_D \frac{\partial Q}{\partial x}\, dx dy.
  \]
  Now if $D$ is of both Type I and Type II, then we
  have both identities
  \[
    \int_{\partial D} P\, dx = -\iint_D \frac{\partial P}{\partial y}\, dx dy
    \quad \text{and} \quad
    \int_{\partial D} Q\, dy = \iint_D \frac{\partial Q}{\partial x}\, dx dy,
  \]
  and adding the two gives
  \[
    \int_{\partial D} P\, dx + Q\, dy = \iint_D \left(\frac{\partial Q}{\partial x} - \frac{\partial P}{\partial y}\right) dx dy,
  \]
  which is precisely Green's theorem for this special
  type of region.
  
  Now let $D$ be a simply connected (i.e. no holes\footnote{More formally, $D$ is \emph{simply connected} if it is path-connected and every loop can be contracted to a point.}) domain enclosed by a measurable curve. Then $D$ can be
  domain enclosed by a measurable curve. Then we try
  to divide $D$ by using parallel lines into
  $D_1, \dots, D_n$ such that each $D_1, \dots, D_n$
  is of both Type I and Type II. Each subdomain falls
  into the previous case, so
  \[
    \sum_{i = 1}^n \int_{\partial D_i} P\, dx + Q\, dy = \sum_{i = 1}^n \iint_{D_i} \left(\frac{\partial Q}{\partial x} - \frac{\partial P}{\partial y}\right) dx dy
    = \iint_D \left(\frac{\partial Q}{\partial x} - \frac{\partial P}{\partial y}\right) dx dy.
  \]
  Now notice that each inner part of the $\partial D_i$
  is traversed twice since it is part of the boundary
  of two $D_i$, so these cancel and we get
  \[
    \int_{\partial D} P\, dx + Q\, dy
    = \sum_{i = 1}^n \int_{\partial D_i} P\, dx + Q\, dy
    = \iint_D \left(\frac{\partial Q}{\partial x} - \frac{\partial P}{\partial y}\right) dx dy,
  \]
  which is Green's theorem for a simply connected
  domain.

  In the most general case, divide $D$ into a finite
  number of simply connected domains, and
  apply the previous case to each one. We still get
  cancellation of the inner parts of the boundaries.
\end{proof}

\end{document}
