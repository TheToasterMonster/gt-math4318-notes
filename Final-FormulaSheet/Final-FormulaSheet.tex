%%%%%%%%%%%%%%%%%%%%%%%%%%%%%%%%%%%%%%%%%%%%%%%%%%%%%%%%%%%%%%%%%%%%%%
% writeLaTeX Example: A quick guide to LaTeX
%
% Source: Dave Richeson (divisbyzero.com), Dickinson College
% 
% A one-size-fits-all LaTeX cheat sheet. Kept to two pages, so it 
% can be printed (double-sided) on one piece of paper
% 
% Feel free to distribute this example, but please keep the referral
% to divisbyzero.com
% 
%%%%%%%%%%%%%%%%%%%%%%%%%%%%%%%%%%%%%%%%%%%%%%%%%%%%%%%%%%%%%%%%%%%%%%
% How to use writeLaTeX: 
%
% You edit the source code here on the left, and the preview on the
% right shows you the result within a few seconds.
%
% Bookmark this page and share the URL with your co-authors. They can
% edit at the same time!
%
% You can upload figures, bibliographies, custom classes and
% styles using the files menu.
%
% If you're new to LaTeX, the wikibook is a great place to start:
% http://en.wikibooks.org/wiki/LaTeX
%
%%%%%%%%%%%%%%%%%%%%%%%%%%%%%%%%%%%%%%%%%%%%%%%%%%%%%%%%%%%%%%%%%%%%%%

\documentclass[10pt,landscape]{article}
\usepackage{amssymb,amsmath,amsthm,amsfonts}
\usepackage{multicol,multirow}
\usepackage{calc}
\usepackage{ifthen}
\usepackage[landscape]{geometry}
\usepackage[colorlinks=true,citecolor=blue,linkcolor=blue]{hyperref}

\newtheorem{theorem}{Theorem}
\newtheorem{proposition}{Proposition}
\theoremstyle{definition}
\newtheorem{definition}{Definition}

\newcommand{\R}{\mathbb{R}}
\newcommand{\N}{\mathbb{N}}
\newcommand{\Z}{\mathbb{Z}}
\newcommand{\C}{\mathbb{C}}
\newcommand{\Q}{\mathbb{Q}}
\newcommand{\F}{\mathbb{F}}

\DeclareMathOperator{\Vol}{Vol}
\DeclareMathOperator{\Int}{int}
\DeclareMathOperator{\id}{id}

\ifthenelse{\lengthtest { \paperwidth = 11in}}
    { \geometry{top=.5in,left=.5in,right=.5in,bottom=.5in} }
	{\ifthenelse{ \lengthtest{ \paperwidth = 297mm}}
		{\geometry{top=1cm,left=1cm,right=1cm,bottom=1cm} }
		{\geometry{top=1cm,left=1cm,right=1cm,bottom=1cm} }
	}
\pagestyle{empty}
\makeatletter
\renewcommand{\section}{\@startsection{section}{1}{0mm}%
                                {-1ex plus -.5ex minus -.2ex}%
                                {0.5ex plus .2ex}%x
                                {\normalfont\large\bfseries}}
\renewcommand{\subsection}{\@startsection{subsection}{2}{0mm}%
                                {-1explus -.5ex minus -.2ex}%
                                {0.5ex plus .2ex}%
                                {\normalfont\normalsize\bfseries}}
\renewcommand{\subsubsection}{\@startsection{subsubsection}{3}{0mm}%
                                {-1ex plus -.5ex minus -.2ex}%
                                {1ex plus .2ex}%
                                {\normalfont\small\bfseries}}
\makeatother
\setcounter{secnumdepth}{0}
\setlength{\parindent}{0pt}
\setlength{\parskip}{0pt plus 0.5ex}
% -----------------------------------------------------------------------

\title{MATH 4318 Final Formula Sheet}

\begin{document}

\raggedright
\footnotesize

\begin{center}
     \Large{\textbf{MATH 4318 Final Formula Sheet}} \\
\end{center}
\begin{multicols}{3}
\setlength{\premulticols}{1pt}
\setlength{\postmulticols}{1pt}
\setlength{\multicolsep}{1pt}
\setlength{\columnsep}{2pt}

\section{Differentiation}

\begin{theorem}[Quotient rule]
  If $f, g : U \to \R$ are differentiable and $g(x_0) \ne 0$, then
  \[
    (f / g)'(x_0) = \frac{f'(x_0) g(x_0) - f(x_0) g'(x_0)}{g(x_0)^2}.
  \]
\end{theorem}

\begin{theorem}[Cauchy's mean value theorem]
  Let $f, g \in C([a, b])$ be differentiable in $(a, b)$.
  If $g'(x) \ne 0$ for any $x \in (a, b)$, then there
  exists $x_0 \in (a, b)$ such that
  \[
  \frac{f'(x_0)}{g'(x_0)} = \frac{f(b) - f(a)}{g(b) - g(a)}.
  \]
\end{theorem}

\begin{theorem}
   Let $R \in (0, \infty)$ and
   $f \in C^\infty(x_0 - R, x_0 + R)$. If there
   exists $M > 0$ such that for all $x \in (x_0 - R, x_0 + R)$,
   $|f^{(n)}(x)| \le M$ for all $n \in \N$, then
   \[
     f(x) = \sum_{n = 0}^\infty \frac{f^{(n)}(x_0)}{n!} (x - x_0)^n
   \]
   for all $x \in (x_0 - R, x_0 + R)$.
\end{theorem}

\begin{theorem}[Lagrange remainder]
  Let $f \in C^n([a, b])$ and assume that
  $f$ is $(n + 1)$-times differentiable in $(a, b)$. Then
  \[
    f(x) - \sum_{k = 0}^n \frac{f^{(k)}(a)}{k!} (x - a)^k
    = \frac{f^{(n + 1)}(\xi)}{(n + 1)!} (x - a)^{n + 1}
  \]
  for some $\xi \in [a, x]$.
\end{theorem}

\section{Integration}
\begin{definition}[Riemann integrability]
  A function $f : [a, b] \to \R$ is \emph{Riemann integrable}
  on $[a, b]$ if there exists $A \in \R$ such that for
  all $\epsilon > 0$, there exists $\delta > 0$ such
  that $|S - A| < \epsilon$ whenever $S$ is any
  Riemann sum on a partition of width $< \delta$. We
  call $A$ the \emph{Riemann integral} of $f$ on $[a, b]$.
\end{definition}

\begin{theorem}[Cauchy criterion for integrability]
  A function $f : [a, b] \to \R$ is integrable if
  and only if for any $\epsilon > 0$, there exists
  $\delta > 0$ such that $|S_1 - S_2| < \epsilon$
  whenever $S_1$ and $S_2$ are Riemann sums for
  partitions of width $< \delta$.
\end{theorem}

\begin{definition}[Step function]
  We say $f : [a, b] \to \R$ is a \emph{step function}
  if there exists a partition $a = x_0 < x_1 < \cdots < x_n = b$
  such that $f$ is constant on each interval $(x_{i - 1}, x_i)$.
\end{definition}

\begin{theorem}
  A function $f : [a, b] \to \R$ is integrable if and
  only if for any $\epsilon > 0$, there exist
  step functions $f_1, f_2$ such that
  $f_1(x) \le f(x) \le f_2(x)$ for all $x \in [a, b]$ and
  \[
    \int_a^b (f_2 - f_1)\, dx < \epsilon.
  \]
\end{theorem}

\begin{definition}[Oscillation amplitude]
  Let $f : [a, b] \to \R$ be bounded and
  $a = x_0 < x_1 < \cdots < x_n = b$ be a partition.
  Then
  \[
    \omega_i(f) = \sup\{|f(x) - f(y)| : x, y \in [x_{i - 1}, x_i)\}
  \]
  is the \emph{oscillation amplitude} of $f$ on $[x_{i - 1}, x_i)$.
\end{definition}

\begin{theorem}
  A function $f : [a, b] \to \R$ is integrable if and
  only if for any $\epsilon > 0$, there exists
  a partition such that
  \[
    \sum_{n = 1}^n \omega_i(f) (x_i - x_{i - 1}) < \epsilon.
  \]
\end{theorem}

\begin{theorem}[Du Bois-Reymond]
  Let $f$ be bounded on $[a, b]$. Then $f \in \mathcal{R}([a, b])$ 
  if and only if for any $\epsilon, a > 0$, there
  exists a partition such that the total length of
  subintervals with $\omega_i(f) \ge \epsilon$ is $< a$.
\end{theorem}

\section{Exchange of Limit Operations}

\begin{theorem}
  Let $\{f_n\}$ be a sequence of functions on an
  open interval $U \subseteq \R$ such that
  each $f_n$ has a continuous derivative. Suppose
  $\{f_n'\}$ converges uniformly on $U$ and for some
  $a \in U$, $\{f_n'(a)\}$ converges. Then
  $\lim_{n \to \infty} f_n = f$ exists and
  $f$ is differentiable. Furthermore, we have
  $f' = \lim_{n \to \infty} f_n'$.
\end{theorem}

\begin{theorem}
  Let $f_n : [a, b] \to \R$ be continuously differentiable.
  Suppose that $\sum_{n = 1}^\infty f_n(x)$ converges
  pointwise on $[a, b]$ and $\sum_{n = 1}^\infty f_n'(x)$ converges
  uniformly on $[a, b]$. Then
  $\sum_{n = 1}^\infty f_n(x)$ converges uniformly on $[a, b]$
  and
  \[
    \frac{d}{dx} \sum_{n = 1}^\infty f_n(x)
    = \sum_{n = 1}^\infty f_n'(x).
  \]
\end{theorem}

\section{Infinite Series}

\begin{theorem}
  If $a_n \ge 0$, then $\sum_{n = 1}^\infty a_n$
  either converges or diverges to $\infty$.
\end{theorem}

\begin{theorem}[Alternating series test]
  Let $\{a_n\}$ be a decreasing sequence with
  $a_n \to 0$ as $n \to \infty$. Then the series
  $\sum_{n = 1}^\infty (-1)^{n + 1} a_n$ converges (to $S$, say), and
  the partial sums $S_n$ have error
  $|S_n - S| \le a_{n + 1}$.
\end{theorem}

\begin{theorem}
  Let $f_n \in C([a, b])$. If $\sum_{n = 1}^\infty f_n(x)$
  converges uniformly on $(a, b)$, then it converges
  uniformly on $[a, b]$.
\end{theorem}

\section{Power Series}
\begin{theorem}
  We have the following:
  \begin{enumerate}
    \item If $\sum_{n = 0}^\infty a_n x^n$ converges
      at $x = x_1 \ne 0$, then it converges absolutely
      for all $x$ with $|x| < |x_1|$.
    \item If $\sum_{n = 0}^\infty a_n x^n$ diverges
      at $x = x_2 \ne 0$, then it diverges for all
      $x$ with $|x| > |x_2|$.
  \end{enumerate}
\end{theorem}

\begin{theorem}[Hadamard's formula]
  For a power series $\sum_{n = 0}^\infty a_n x^n$,
  let $L = \limsup_{n \to \infty} |a_n|^{1 / n}$.
  Then its radius of convergence is $R = 1 / L$.
\end{theorem}

\begin{theorem}
  For a series $\sum_{n = 0}^\infty a_n x^n$ with $a_n \ne 0$,
  if $\lim_{n \to \infty} |a_{n + 1} / a_n| = L$,
  then its radius of convergence is $R = 1 / L$.
\end{theorem}

\begin{theorem}
  If $\sum_{n = 0}^\infty a_n x^n$ has radius of convergence
  $R > 0$, then for any $0 < r < R$, the power
  series $\sum_{n = 0}^\infty a_n r^n$ converges
  uniformly on $[-r, r]$. Moreover, if
  $\sum_{n = 0}^\infty a_n x^n$ converges at
  $x = R < \infty$ (or $x = -R$), then $\sum_{n = 0}^\infty a_n x^n$
  converges uniformly on $[0, R]$ (or $[-R, 0]$).
\end{theorem}

\begin{theorem}
  If $\sum_{n = 0}^\infty a_n x^n$ has radius of
  convergence $R > 0$, then
  $f(x) = \sum_{n = 0}^\infty a_n x^n \in C^\infty(-R, R)$.
\end{theorem}

\begin{theorem}
  Suppose $f(x) = \sum_{n = 0}^\infty a_n x^n$ has
  radius of convergence $R > 0$. Then for any
  $x \in (-R, R)$, $f \in \mathcal{R}([0, x])$
  and
  \[
    \int_0^x f(t)\, dt = \sum_{n = 0}^\infty \frac{a_n}{n + 1} x^{n + 1}.
  \]
\end{theorem}

\section{Differentiation in \texorpdfstring{$\R^n$}{Rn}}
\begin{theorem}
  Let $f : U \subseteq \R^n \to \R$ where $U$ is open.
  Then $f$ is differentiable at $x = a$ if and only
  if there exist functions $A_1, \dots, A_n$ on $U$,
  continuous at $x = a$, such that
  \[
    f(x) - f(a) = A_1(x) (x_1 - a_1) + \cdots + A_n(x) (x_n - a_n)
  \]
  for all $x \in U$. In this case, $\partial f / \partial x_i(a) = A_i(a)$.
\end{theorem}

\begin{theorem}
  Let $U$ be an open set in $\R^n$ and suppose that
  $f : U \to \R$ has partial derivatives $f_1', \dots, f_n'$
  on $U$ which are continuous on $x = a$. Then $f$ is
  differentiable at $x = a$.
\end{theorem}

\begin{theorem}[Implicit function theorem]
  Let $f : \R^n \times \R^m \to \R^m$ and
  $U \times V \subseteq \R^n \times \R^m$ be a
  neighborhood of $(x_0, y_0)$. Suppose $f$ and
  $\partial f / \partial y$ are continuous $U \times V$,
  and $f(x_0, y_0) = 0$ and
  \[
    \det\left(\frac{\partial f}{\partial y}(x_0, y_0)\right)
    \ne 0.
  \]
  Then there exists a neighborhood $U_0 \times V_0 \subseteq U \times V$
  of $(x_0, y_0)$ and a unique continuous function
  $\varphi : U_0 \to V_0$ satisfying
  \[
    \begin{cases}
      f(x, \varphi(x)) = 0, \\
      \varphi(x_0) = y_0.
    \end{cases}
  \]
\end{theorem}

\begin{theorem}
  Let $f : [a, b] \times [c, d] \to \R$ be continuous
  and suppose that $\partial f / \partial y$ exists
  and is continuous on $[a, b] \times [c, d]$. Then
  \[
    \frac{d}{dy} \int_a^b f(x, y)\, dx = \int_a^b \frac{\partial f}{\partial y}(x, y)\, dx.
  \]
\end{theorem}

\section{Integration in \texorpdfstring{$\R^n$}{Rn}}
\begin{theorem}[Lebesgue's criterion for Riemann integrability]
  Let $A \subseteq \R^n$ be a set with volume and
  let $f : A \to \R$ be a bounded function that is
  continuous except on a subset of $A$ with zero volume.
  Then $f$ is integrable on $A$.
\end{theorem}

\begin{theorem}
  Suppose $f(x, y)$ is integrable on
  $D = [a, b] \times [c, d]$ and for each $x \in [a, b]$,
  $f(x, y)$ is integrable on $[c, d]$. Then
  \[
    \int_a^b dx \left[\int_c^d f(x, y)\, dy\right]
    = \iint_D f(x, y)\, dxdy
  \]
\end{theorem}

\section{Reverse Triangle Inequality}
\begin{proposition}
  For all $x, y \in \R$, we have
  $\left| |x| - |y| \right| \le |x - y|$.
\end{proposition}

\vfill
\hrule
~\\
Frank Qiang, Georgia Institute of Technology, Spring 2024
\end{multicols}

\end{document}

