%%%%%%%%%%%%%%%%%%%%%%%%%%%%%%%%%%%%%%%%%%%%%%%%%%%%%%%%%%%%%%%%%%%%%%
% writeLaTeX Example: A quick guide to LaTeX
%
% Source: Dave Richeson (divisbyzero.com), Dickinson College
% 
% A one-size-fits-all LaTeX cheat sheet. Kept to two pages, so it 
% can be printed (double-sided) on one piece of paper
% 
% Feel free to distribute this example, but please keep the referral
% to divisbyzero.com
% 
%%%%%%%%%%%%%%%%%%%%%%%%%%%%%%%%%%%%%%%%%%%%%%%%%%%%%%%%%%%%%%%%%%%%%%
% How to use writeLaTeX: 
%
% You edit the source code here on the left, and the preview on the
% right shows you the result within a few seconds.
%
% Bookmark this page and share the URL with your co-authors. They can
% edit at the same time!
%
% You can upload figures, bibliographies, custom classes and
% styles using the files menu.
%
% If you're new to LaTeX, the wikibook is a great place to start:
% http://en.wikibooks.org/wiki/LaTeX
%
%%%%%%%%%%%%%%%%%%%%%%%%%%%%%%%%%%%%%%%%%%%%%%%%%%%%%%%%%%%%%%%%%%%%%%

\documentclass[10pt,landscape]{article}
\usepackage{amssymb,amsmath,amsthm,amsfonts}
\usepackage{multicol,multirow}
\usepackage{calc}
\usepackage{ifthen}
\usepackage[landscape]{geometry}
\usepackage[colorlinks=true,citecolor=blue,linkcolor=blue]{hyperref}

\newtheorem{theorem}{Theorem}

\newcommand{\R}{\mathbb{R}}
\newcommand{\N}{\mathbb{N}}
\newcommand{\Z}{\mathbb{Z}}
\newcommand{\C}{\mathbb{C}}
\newcommand{\Q}{\mathbb{Q}}
\newcommand{\F}{\mathbb{F}}

\DeclareMathOperator{\Vol}{Vol}
\DeclareMathOperator{\Int}{int}

\ifthenelse{\lengthtest { \paperwidth = 11in}}
    { \geometry{top=.5in,left=.5in,right=.5in,bottom=.5in} }
	{\ifthenelse{ \lengthtest{ \paperwidth = 297mm}}
		{\geometry{top=1cm,left=1cm,right=1cm,bottom=1cm} }
		{\geometry{top=1cm,left=1cm,right=1cm,bottom=1cm} }
	}
\pagestyle{empty}
\makeatletter
\renewcommand{\section}{\@startsection{section}{1}{0mm}%
                                {-1ex plus -.5ex minus -.2ex}%
                                {0.5ex plus .2ex}%x
                                {\normalfont\large\bfseries}}
\renewcommand{\subsection}{\@startsection{subsection}{2}{0mm}%
                                {-1explus -.5ex minus -.2ex}%
                                {0.5ex plus .2ex}%
                                {\normalfont\normalsize\bfseries}}
\renewcommand{\subsubsection}{\@startsection{subsubsection}{3}{0mm}%
                                {-1ex plus -.5ex minus -.2ex}%
                                {1ex plus .2ex}%
                                {\normalfont\small\bfseries}}
\makeatother
\setcounter{secnumdepth}{0}
\setlength{\parindent}{0pt}
\setlength{\parskip}{0pt plus 0.5ex}
% -----------------------------------------------------------------------

\title{MATH 4318 Exam 2 Formula Sheet}

\begin{document}

\raggedright
\footnotesize

\begin{center}
     \Large{\textbf{MATH 4318 Exam 2 Formula Sheet}} \\
\end{center}
\begin{multicols}{3}
\setlength{\premulticols}{1pt}
\setlength{\postmulticols}{1pt}
\setlength{\multicolsep}{1pt}
\setlength{\columnsep}{2pt}

\section{Exchange of Limit Operations}
\begin{theorem}
  Let $\{f_n\}$ be a uniformly convergent sequence
  of Riemann integrable functions on $[a, b]$. Then
  \[
    \int_a^b \lim_{n \to \infty} f_n(x)\, dx
    = \lim_{n \to \infty} \int_a^b f_n(x)\, dx.
  \]
\end{theorem}

\begin{theorem}
  Let $\{f_n\}$ be a sequence of functions on an
  open interval $U \subseteq \R$ such that
  each $f_n$ has a continuous derivative. Suppose
  $\{f_n'\}$ converges uniformly on $U$ and for some
  $a \in U$, $\{f_n'(a)\}$ converges. Then
  $\lim_{n \to \infty} f_n = f$ exists and
  $f$ is differentiable. Furthermore, we have
  $f' = \lim_{n \to \infty} f_n'$.
\end{theorem}

\begin{theorem}
  Let $f_n \in \mathcal{R}([a, b])$ for
  $n \in \N$. If $\sum_{n = 1}^\infty f_n(x)$
  converges uniformly on $[a, b]$, then
  $\sum_{n = 1}^\infty f_n(x) \in \mathcal{R}([a, b])$ and
  \[
    \sum_{n = 1}^\infty \int_a^b f_n(x)\, dx
    = \int_a^b \sum_{n = 1}^\infty f_n(x)\, dx.
  \]
\end{theorem}

\begin{theorem}
  Let $f_n : [a, b] \to \R$ be continuously differentiable.
  Suppose that $\sum_{n = 1}^\infty f_n(x)$ converges
  pointwise on $[a, b]$ and $\sum_{n = 1}^\infty f_n'(x)$ converges
  uniformly on $[a, b]$. Then
  $\sum_{n = 1}^\infty f_n(x)$ converges uniformly on $[a, b]$
  and
  \[
    \frac{d}{dx} \sum_{n = 1}^\infty f_n(x)
    = \sum_{n = 1}^\infty f_n'(x).
  \]
\end{theorem}

\section{Infinite Series Basics}
\begin{theorem}
  A series $\sum_{n = 1}^\infty a_n$ converges if and
  only if for every $\epsilon > 0$, there exists
  $N \in \N$ such that if $n > m \ge N$, then
  $|a_{m + 1} + \cdots + a_n| < \epsilon$.
\end{theorem}

\begin{theorem}
  If $\sum_{n = 1}^\infty a_n$ converges, then
  $a_n \to 0$ as $n \to \infty$.
\end{theorem}

\begin{theorem}
  If $a_n \ge 0$, then $\sum_{n = 1}^\infty a_n$
  either converges or diverges to $\infty$.
\end{theorem}

\begin{theorem}
  If $\sum_{n = 1}^\infty a_n$ converges absolutely,
  then $\sum_{n = 1}^\infty a_n$ converges and
  $\left|\sum_{n = 1}^\infty a_n\right| \le \sum_{n = 1}^\infty |a_n|$.
\end{theorem}

\section{Convergence Tests}
\begin{theorem}[Comparison test]
  If $\sum_{n = 1}^\infty a_n$ and $\sum_{n = 1}^\infty b_n$
  are two series such that $|a_n| \le b_n$ and
  $\sum_{n = 1}^\infty b_n$ converges, then
  $\sum_{n = 1}^\infty a_n$ converges and
  $\left|\sum_{n = 1}^\infty a_n\right| \le \sum_{n = 1}^\infty b_n$.
\end{theorem}

\begin{theorem}[Limit comparison]
  Let $\sum_{n = 1}^\infty a_n$ and $\sum_{n = 1}^\infty b_n$
  be two positive series and suppose that
  $\lim_{n \to \infty} a_n / b_n = \ell > 0$.
  Then $\sum_{n = 1}^\infty a_n$ converges if and only if
  $\sum_{n = 1}^\infty b_n$ converges.
\end{theorem}

\begin{theorem}[Root test]
  Let $\sum_{n = 1}^\infty a_n$ be a positive
  series and suppose that
  $\limsup_{n \to \infty} \sqrt[n]{a_n} = \ell$.
  Then
  \begin{enumerate}
    \item if $\ell < 1$, then $\sum_{n = 1}^\infty a_n$ converges;
    \item if $\ell > 1$, then $\sum_{n = 1}^\infty a_n$ diverges.
  \end{enumerate}
\end{theorem}

\begin{theorem}[Integral test]
  Let $\{a_n\}$ be a positive decreasing sequence.
  If there exists a continuous decreasing $f$ on
  $[1, \infty)$ such that $a_n = f(n)$, then
  $\sum_{n = 1}^\infty a_n$ converges if and only if
  $\int_1^\infty f(x)\, dx$ converges.
\end{theorem}

\begin{theorem}
  Let $\{a_n\}$ be a decreasing sequence with
  $a_n \to 0$ as $n \to \infty$. Then the series
  $\sum_{n = 1}^\infty (-1)^{n + 1} a_n$ converges (to $S$, say), and
  the partial sums $S_n$ have error
  $|S_n - S| \le a_{n + 1}$.
\end{theorem}

\section{Series of Functions}

\begin{theorem}[Cauchy criterion]
  A series $\sum_{n = 1}^\infty f_n(x)$ converges uniformly
  on $I \subseteq \R$ if and only if for every $\epsilon > 0$,
  there exists $N$ such that whenever $n \ge N$,
  for any $x \in I$ and $p \in \N$ we have
  $|f_{n + 1}(x) + \dots + f_{n + p}(x)| < \epsilon$.
\end{theorem}

\begin{theorem}
  If $\sum_{n = 1}^\infty f_n(x)$ converges uniformly
  on $I$, then $f_n \to 0$ uniformly on $I$.
\end{theorem}

\begin{theorem}
  Let $f_n \in C([a, b])$. If $\sum_{n = 1}^\infty f_n(x)$
  converges uniformly on $(a, b)$, then it converges
  uniformly on $[a, b]$.
\end{theorem}

\begin{theorem}[Weierstrass $M$-test]
  If there exists a nonnegative and convergent
  series such that
  $|f_n(x)| \le M_n$ for all $x \in I$, then
  $\sum_{n = 1}^\infty f_n(x)$ converges uniformly
  on $I$.
\end{theorem}

\section{Power Series}
\begin{theorem}
  We have the following:
  \begin{enumerate}
    \item If $\sum_{n = 0}^\infty a_n x^n$ converges
      at $x = x_1 \ne 0$, then it converges absolutely
      for all $x$ with $|x| < |x_1|$.
    \item If $\sum_{n = 0}^\infty a_n x^n$ diverges
      at $x = x_2 \ne 0$, then it diverges for all
      $x$ with $|x| > |x_2|$.
  \end{enumerate}
\end{theorem}

\begin{theorem}[Hadamard's formula]
  For a power series $\sum_{n = 0}^\infty a_n x^n$,
  let $L = \limsup_{n \to \infty} |a_n|^{1 / n}$.
  Then its radius of convergence is $R = 1 / L$.
\end{theorem}

\begin{theorem}
  For a series $\sum_{n = 0}^\infty a_n x^n$ with $a_n \ne 0$,
  if $\lim_{n \to \infty} |a_{n + 1} / a_n| = L$,
  then its radius of convergence is $R = 1 / L$.
\end{theorem}

\begin{theorem}
  If $\sum_{n = 0}^\infty a_n x^n$ has radius of convergence
  $R > 0$, then for any $0 < r < R$, the power
  series $\sum_{n = 0}^\infty a_n r^n$ converges
  uniformly on $[-r, r]$. Moreover, if
  $\sum_{n = 0}^\infty a_n x^n$ converges at
  $x = R < \infty$ (or $x = -R$), then $\sum_{n = 0}^\infty a_n x^n$
  converges uniformly on $[0, R]$ (or $[-R, 0]$).
\end{theorem}

\begin{theorem}
  If $\sum_{n = 0}^\infty a_n x^n$ has radius of
  convergence $R > 0$, then
  $f(x) = \sum_{n = 0}^\infty a_n x^n \in C^\infty(-R, R)$.
\end{theorem}

\begin{theorem}
  Suppose $f(x) = \sum_{n = 0}^\infty a_n x^n$ has
  radius of convergence $R > 0$. Then for any
  $x \in (-R, R)$, $f \in \mathcal{R}([0, x])$
  and
  \[
    \int_0^x f(t)\, dt = \sum_{n = 0}^\infty \frac{a_n}{n + 1} x^{n + 1}.
  \]
\end{theorem}

\section{Taylor Series}
\begin{theorem}
   Let $R \in (0, \infty)$ and
   $f \in C^\infty(x_0 - R, x_0 + R)$. If there
   exists $M > 0$ such that for all $x \in (x_0 - R, x_0 + R)$,
   $|f^{(n)}(x)| \le M$ for all $n \in \N$, then
   \[
     f(x) = \sum_{n = 0}^\infty \frac{f^{(n)}(x_0)}{n!} (x - x_0)^n
   \]
   for all $x \in (x_0 - R, x_0 + R)$.
\end{theorem}

\begin{theorem}[Lagrange remainder]
  Let $f \in C^n([a, b])$ and assume that
  $f$ is $(n + 1)$-times differentiable in $(a, b)$. Then
  \[
    f(x) - \sum_{k = 0}^n \frac{f^{(k)}(a)}{k!} (x - a)^k
    = \frac{f^{(n + 1)}(\xi)}{(n + 1)!} (x - a)^{n + 1}
  \]
  for some $\xi \in [a, x]$.
\end{theorem}

\section{Contraction Mapping}
\begin{theorem}[Contraction mapping]
  Let $(X, d)$ be a complete metric space and
  $T : X \to X$ a contraction mapping, i.e.
  there exists $0 < k < 1$ such that
  $d(Tx, Ty) \le k d(x, y)$ for all $x, y \in X$.
  Then $T$ admits a unique fixed point in $X$.
\end{theorem}

\begin{theorem}[Newton's method]
  Let $f \in C^2([a, b])$ and $\hat{x} \in (a, b)$
  such that $f(\hat{x}) = 0$ and $f'(\hat{x}) \ne 0$.
  Then there exists a neighborhood $U(\hat{x})$ of
  $\hat{x}$ such that for all $x_0 \in U(\hat{x})$,
  the sequence
  \[
    x_{n + 1} = x_n - \frac{f(x_n)}{f'(x_n)}
  \]
  converges to $\hat{x}$ as $n \to \infty$.
\end{theorem}

\begin{theorem}[Picard-Lindel\"of]
  Let $f(t, x) : [a, b] \times [c, d] \to \R$ be continous
  in $t$ and locally Lipschitz in $x$, i.e. it is
  Lipschitz in $x$ for $|t| \le h$ with $h$ small enough,
  then the initial value problem
  \[
    \begin{cases}
      x'(t) = f(t, x) \\
      x(0) = \xi,
    \end{cases}
  \]
  has a unique local solution. (Note: Consider
  \[
    x(t) = \xi + \int_0^t f(\tau, x(\tau))\, d\tau
  \]
  by integrating. This is called a \emph{Picard iteration}.)
\end{theorem}

\begin{theorem}[Implicit function theorem]
  Let $f : \R^n \times \R^m \to \R^m$ and
  $U \times V \subseteq \R^n \times \R^m$ be a neighorhood
  of $(x_0, y_0)$. Suppose $f$ and $\partial f / \partial y$
  are continuous on $U \times V$, and $f(x_0, y_0) = 0$,
  \[
    \det\left(\frac{\partial f}{\partial y}(x_0, y_0)\right)
    \ne 0.
  \]
  Then there exists a neighborhood $U_0 \times V_0 \subseteq U \times V$ of
  $(x_0, y_0)$ and a unique continuous function
  $\varphi : U_0 \to V_0$ satisfying
  \[
  \begin{cases}
    f(x, \varphi(x)) = 0,
    \varphi(x_0) = y_0.
  \end{cases}
  \]
\end{theorem}

\vfill
\hrule
~\\
Frank Qiang, Georgia Institute of Technology, Spring 2024
\end{multicols}

\end{document}

