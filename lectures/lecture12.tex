\chapter{Feb.~20 --- Series of Functions}

\section{Rearrangements of Series}

\begin{definition}
  Let $\sigma : \N \to \N$ be a bijection, i.e. it is
  one-to-one and onto. Given an
  infinite series $\sum_{n = 1}^\infty a_n$, we define
  the \emph{rearranged series} to be $\sum_{n = 1}^\infty a_{\sigma(n)}$.
\end{definition}

\begin{theorem}[Dirichlet]
  If $(A) = \sum_{n = 1}^\infty a_n$ converges absolutely
  and $(B) = \sum_{n = 1}^\infty a_{\sigma(n)}$ is any
  rearrangement of $(A)$, then $(B)$ also converges and
  \[
    \sum_{n = 1}^\infty a_{\sigma(n)} = \sum_{n = 1}^\infty a_n.
  \]
\end{theorem}

\begin{proof}
  First assume $a_n \ge 0$ and let $S = \sum_{n = 1}^\infty a_n$
  with partial sums $S_n = \sum_{k = 1}^n a_k$. Let
  $\sigma_n$ be the partial sums of $(B)$. Then
  \[
    \sigma_m \le \sum_{n = 1}^\infty a_n = S \tag{$*$}\]
  since $a_n \ge 0$. Then $\sigma_m$ is bounded,
  so $(B)$ converges by the monotone convergence theorem.
  Also by $(*)$, we can take $m \to \infty$ to get
  \[
    \lim_{m \to \infty} \sigma_m \le S
    \implies \sum_{n = 1}^\infty a_{\sigma(n)} \le S.
  \]
  We also get the reverse inequality by thinking
  of $(A)$ as a rearrangement of $(B)$, so the two
  are equal.

  Now for the general case, suppose that
  $\sum_{n = 1}^\infty a_n$ is a series with differing
  signs. Let $a_n^+, a_n^- \ge 0$ be the positive
  and negative parts of $a_n$, respectively, so
  that $|a_n| = a_n^+ + a_n^-$. Now $b_n = a_{\sigma(n)}^+$,
  and $|b_n| = b_n^+ + b_n^-$, so
  $b_n^{\pm} = a_{\sigma(n)}^{\pm}$. By the previous
  part, $\sum_{n = 1}^\infty b_n^{\pm}$ converges and
  \[
    \sum_{n = 1}^\infty b_n^{\pm} = \sum_{n = 1}^\infty a_n^{\pm}.
  \]
  Then we have
  \[
    \sum_{n = 1}^\infty b_n
    = \sum_{n = 1}^\infty (b_n^+ - b_n^-)
    = \sum_{n = 1}^\infty (a_n^+ - a_n^-)
    = \sum_{n = 1}^\infty a_n,
  \]
  which is the desired conclusion.
\end{proof}

\begin{theorem}[Riemann]
  If $(A) = \sum_{n = 1}^\infty a_n$ converges
  conditionally, then there exists a rearrangement
  $(B) = \sum_{n = 1}^\infty a_{\sigma(n)}$ such that
  any one of the following occurs:
  \begin{enumerate}
    \item $(B)$ converges to any number $\sigma$,
    \item $(B)$ diverges to $\infty$,
    \item $(B)$ diverges to $-\infty$,
    \item $(B)$ oscillates in an unbounded manner,
    \item or $(B)$ oscillates in a bounded manner.
  \end{enumerate}
\end{theorem}

\begin{proof}
  Look this up.
\end{proof}

\begin{remark}
  This shows that when a series only converges
  conditionally, then pretty much anything can happen
  when rearranging, which is not the case for
  absolutely convergent series.
\end{remark}

\section{Series of Functions}
\begin{definition}
  A \emph{series of functions} is a series of the form
  $\sum_{n = 1}^\infty f_n(x)$, where each $f_n(x)$
  is defined on an interval $I$. Define its
  \emph{partial sums} to be $S_n(x) = \sum_{k = 1}^n f_k(x)$.
  We say that
  \[
    \sum_{n = 1}^\infty f_n(x) = S(x),
  \]
  i.e. the series \emph{converges} to $S(x)$, if
  $\lim_{n \to \infty} S_n(x) = S(x)$, which can be
  \emph{pointwise} or \emph{uniform}.
\end{definition}

\begin{theorem}[Cauchy criterion for uniform convergence]
  A series $\sum_{n = 1}^\infty f_n(x)$ converges
  uniformly on $I$ if and only if for every
  $\epsilon > 0$, there exists $N$ such that whenever
  $n \ge N$, for any $x \in I$ and integer $p$ we have
  \[
    \left| \sum_{k = n + 1}^{n + p} f_k(x) \right| < \epsilon.
  \]
\end{theorem}

\begin{proof}
  This is because $\{S_n(x)\}$ is a Cauchy sequence, so
  for any $\epsilon > 0$, there exists $N$ such that
  when $n \ge N$,
  \[
    \sup_{x \in I} |S_{n + p}(x) - S_{n}(x)| < \epsilon,
  \]
  which is precisely the given condition.
\end{proof}

\begin{corollary}
  If $\sum_{n = 1}^\infty f_n(x)$ converges uniformly
  on $I$, then $f_n(x)$ converges uniformly to $0$
  on $I$.
\end{corollary}

\begin{corollary}
  Let $f_n(x) \in C([a, b])$. If $\sum_{n = 1}^\infty f_n(x)$
  converges uniformly on $(a, b)$, then
  it converges uniformly on $[a, b]$.
\end{corollary}

\begin{proof}
  Since $\sum_{n = 1}^\infty f_n(x)$ converges uniformly,
  for any $\epsilon > 0$ there exists $N$ such that
  when $n \ge N$, we have
  \[
    \left| \sum_{k = n + 1}^{n + p} f_k(x) \right| < \epsilon
  \]
  for all $x \in (a, b)$. Let $x \to a^+$ and
  $x \to b^-$ to get
  \[
    \left| \sum_{k = n + 1}^{n + p} f_k(a) \right| \le \epsilon
    \quad \text{and} \quad
    \left| \sum_{k = n + 1}^{n + p} f_k(b) \right| \le \epsilon.
  \]
  So for all $x \in [a, b]$, we have
  \[
    \left| \sum_{k = n + 1}^{n + p} f_k(x) \right| \le \epsilon,
  \]
  which gives uniform convergence on $[a, b]$.
\end{proof}

\begin{example}
  The series
  \[
    \sum_{n = 1}^\infty \frac{1}{n^x}
  \]
  does not converge uniformly on $(1, \infty)$.
\end{example}

\begin{proof}
  Note that each $1 / n^x$ is continuous
  but at $x = 1$, the series becomes the harmonic
  series
  $\sum_{n = 1}^\infty 1 / n$, which diverges.
  So by the previous
  corollary, this series cannot converge uniformly on
  $(1, \infty)$.
\end{proof}

\begin{theorem}[Weierstrass $M$-test]
  If there exists a nonnegative and convergent
  series $\sum_{n = 1}^\infty M_n$ such that for all
  $x \in I$, we have
  \[
    |f_n(x)| \le M_n,
  \]
  then $\sum_{n = 1}^\infty f_n(x)$ converges
  uniformly on $I$.
\end{theorem}

\begin{proof}
  Since $\sum_{n = 1}^\infty M_n$ converges uniformly,
  for any $\epsilon > 0$ there exists $N$ such that
  \[
    \left| \sum_{k = n + 1}^{n + p} M_k \right| < \epsilon.
  \]
  This directly implies
  \[
    \left| \sum_{k = n + 1}^{n + p} f_k(x) \right|
    \le \sum_{k = n + 1}^{n + p} |f_k(x)|
    < \epsilon
  \]
  since each $M_k$ is nonnegative. Then
  $\sum_{n = 1}^\infty f_n(x)$ converges uniformly on $I$
  by the Cauchy criterion.
\end{proof}

\begin{example}
  If $\sum_{n = 1}^\infty a_n$ converges absolutely,
  then
  \[
    \sum_{n = 1}^\infty \frac{a_n x^n}{1 + x^{2n}}
  \]
  converges uniformly on $(-\infty, \infty)$.
\end{example}

\begin{proof}
  This is because
  \[
    \left| \frac{a_n x^n}{1 + x^{2n}} \right|
    \le |a_n| \underbrace{\frac{|x|^n}{1 + |x|^{2n}}}_{\le 1} \le |a_n|,
  \]
  so by the Weierstrass $M$-test, we get uniform
  convergence on $(-\infty, \infty)$.
\end{proof}

\section{Exchange of the Limit and Infinite Series}

\begin{theorem}
  Let $f_n \in \mathcal{R}([a, b])$ for each $n = 1, 2, \dots$.
  If $\sum_{n = 1}^\infty f_n(x)$ converges uniformly
  to $S(x)$ on $[a, b]$, then $S(x) \in \mathcal{R}([a, b])$ and
  \[
    \sum_{n = 1}^\infty \int_a^b f_n(x) \, dx
    = \int_a^b S(x)\, dx
    = \int_a^b \sum_{n = 1}^\infty f_n(x) \, dx.
  \]
\end{theorem}

\begin{proof}
  Let $S_n(x) = \sum_{k = 1}^n f_k(x)$. Then
  $S_n \to S$ uniformly on $[a, b]$, so we can apply
  Theorem \ref{thm:exchange-limit-integral} to get
  $S(x) \in \mathcal{R}([a, b])$ and
  \[
    \sum_{n = 1}^\infty \int_a^b f_n(x) \, dx
    = \lim_{n \to \infty} \int_a^b S_n(x) \, dx
    = \int_a^b S(x) \, dx,
  \]
  which is the desired conclusion.\footnote{This is just applying the previous theorem on exchanging the limit and the integral.}
\end{proof}

\begin{theorem}
  Let $f_n(x)$ be continuously differentiable on
  $[a, b]$. Suppose that
  \[
    \sum_{n = 1}^\infty f_n(x) = S(x)
  \]
  pointwise on $[a, b]$, and $\sum_{n = 1}^\infty f_n'(x)$
  converges uniformly to $G(x)$ on $[a, b]$. Then
  $\sum_{n = 1}^\infty f_n(x)$ converges uniformly
  to $S(x)$ and $S'(x) = G(x)$. In other words, in this
  case we have
  \[
    \left(\sum_{n = 1}^\infty f_n(x)\right)'
    = \sum_{n = 1}^\infty f_n'(x).
  \]
\end{theorem}

\begin{proof}
  Similarly apply Theorem \ref{thm:exchange-limit-derivative},
  our previous theorem on
  exchanging the limit and the derivative.
\end{proof}

\begin{example}
  Show that
  \[
    S(x) = \sum_{n = 1}^\infty \frac{\sin nx}{n^3}
  \]
  is continuously differentiable on $(-\infty, \infty)$.
\end{example}

\begin{proof}
  For each fixed $n$, we see that
  $(\sin nx) / n^3$ is continuously
  differentiable, and we at least have pointwise
  convergence since
  \[
    \left| \frac{\sin nx}{n^3} \right| \le \frac{1}{n^3}.
  \]
  Now the series of derivatives
  \[
    \sum_{n = 1}^\infty \frac{n\cos nx}{n^3}
    = \sum_{n = 1}^\infty \frac{\cos nx}{n^2}
  \]
  converges uniformly on $(-\infty, \infty)$ by the
  Weierstrass $M$-test since
  \[
    \left| \frac{\cos nx}{n^2} \right| \le \frac{1}{n^2}.
  \]
  So we can apply the previous theorem to see that
  $S$ is continuously differentiable on
  $(-\infty, \infty)$.
\end{proof}

\begin{remark}
  Changing the $n^3$ to $n^2$ in the previous example
  will cause this argument to fail, since we get
  \[
    \left| \frac{\cos nx}{n} \right| \le \frac{1}{n},
  \]
  which does not converge. So we cannot use the
  Weierstrass $M$-test.
\end{remark}

\begin{example}
  Let
  \[
    S(x) = \sum_{n = 1}^\infty \frac{|x|}{n^2 + x^2}.
  \]
  Study its differentiability on $(-\infty, \infty)$.
\end{example}

\begin{proof}
  Observe that
  \[
    \frac{|x|}{n^2 + x^2} \le \frac{|x|}{n^2},
  \]
  so the series converges pointwise on
  $(-\infty, \infty)$. Now let $f_n(x) = |x| / (n^2 + x^2)$,
  and we see that
  \[
    \sum_{n = 1}^\infty f_n'(x)
    = \sum_{n = 1}^\infty \frac{\frac{(n^2 + x^2)|x|}{x} - 2x |x|}{(n^2 + x^2)^2}
    = \sum_{n = 1}^\infty \frac{\frac{n^2|x|}{x} - x |x|}{(n^2 + x^2)^2}
  \]
  for any $0 < |x| < A$.
  Now
  \[
    \left| \frac{n^2 |x|}{x} - x |x| \right|
    \le n^2 + A^2,
  \]
  so $\sum_{n = 1}^\infty f_n'(x)$ converges uniformly
  for any $x$ away from $0$. So $S(x)$ is differentiable
  when $x \ne 0$.
  At $x = 0$, we use the definition of the derivative
  to see that
  \[
    \lim_{\Delta x \to 0} \frac{S(\Delta x) - S(0)}{\Delta x}
    = \lim_{\Delta x \to 0} \frac{|\Delta x|}{\Delta x} \sum_{n = 1}^\infty \frac{1}{n^2 + (\Delta x)^2}.
  \]
  From this we get
  \[
    S'_-(0) = -\sum_{n = 1}^\infty \frac{1}{n^2}
    \ne \sum_{n = 1}^\infty \frac{1}{n^2}
    = S'_+(0),
  \]
  so $S(x)$ is not differentiable at $x = 0$.
\end{proof}
