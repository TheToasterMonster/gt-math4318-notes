\chapter{Jan.~18 --- Taylor Polynomials}

\section{Common Taylor Polynomials}
We have
\begin{align*}
  e^x &= 1 + x + \frac{x^2}{2!} + \dots + \frac{x^n}{n!} + o(x^n), \\
  \sin x &= x - \frac{x^3}{3!} + \frac{x^5}{5!} - \dots
  + \frac{(-1)^{n - 1} x^{2n - 1}}{(2n - 1)!} + o(x^{2n}), \\
  \cos &= 1 - \frac{x^2}{2!} + \frac{x^4}{4!} - \dots
  + \frac{(-1)^n x^{2n}}{(2n)!} + o(x^{2n + 1}), \\
  (1 + x)^\alpha &= 1 + \alpha x + \frac{\alpha(\alpha - 1)}{2!} x^2 + \dots + \frac{\alpha(\alpha - 1) \dots (\alpha - n + 1)}{n!} x^n + o(x^n), \\
  \ln (1 + x) &= x - \frac{x^2}{2} + \frac{x^3}{3}
  + \dots + (-1)^{n - 1} \frac{x^n}{n} + o(x^n).
\end{align*}

\section{Combining Taylor Polynomials}
\begin{remark}
  If $a = 0$ and $f(x)$ is even in $(-b, b)$, then
  \[
    f(x) = \sum_{k = 0}^{n / 2} a_k x^{2k} + o(x^n).
  \]
  Similarly if $f(x)$ is odd in $(-b, b)$, then
  \[
    f(x) = \sum_{k = 0}^{n / 2} a_k x^{2k + 1} + o(x^{n + 1}).
  \]
\end{remark}

\begin{remark}
To create new Taylor polynomials from known ones,
we can observe that if
$f(x) = P_n(x) + o((x - a)^n)$ and $g(x) = Q_n(x) + o((x - a)^n)$, then
\[
  f(x) + g(x) = (P_n(x) + Q_n(x)) + o((x - a)^n)
  \quad \text{and} \quad
  f(x) g(x) = \underbrace{(P_n(x) Q_n(x))}_{\text{take first $n$ terms}} + o((x - a)^n).
\]
If $P_n(x) = \sum_{k = 0}^n a_k (x - a)^k$ and
$Q_n(x) = \sum_{k = 0}^n b_k (x - a)^k$, then
$f(x)g(x)$ has Taylor polynomial
$\sum_{k = 0}^n c_k (x - a)^k$ where
\[c_k = \sum_{i = 0}^k a_i b_{k - i}.\]
If $h(x) = f(x) / g(x)$ and $g(x) \ne 0$ near $x = a$,
then $f(x) = h(x) g(x)$. Let $h(x) = \sum_{k = 0}^n c_k (x - a)^k + o((x - a)^n)$, then
\[
  a_k = \sum_{i = 0}^k c_i b_{k - i}
\]
for $0 \le k \le n$, after which we can solve for the
$c_k$.
\end{remark}

\begin{example}
Find the Taylor polynomial for $\tan x$ up to $n = 5$.
\end{example}

\begin{proof}
Note that $\tan x$ is odd, so we can write
\[
  \tan x = x + a_3 x^3 + a_5 x^5 + o(x^5).
\]
Now since $\tan x = \sin x / \cos x$, we have
$\sin x = \tan x \cos x$, so
\[
  x - \frac{x^3}{6} + \frac{x^5}{5!} + o(x^5)
  = (x + a_3 x^3 + a_5x^5)(1 - \frac{x^2}{2!} + \frac{x^4}{4!})
\]
We can solve to get
\[
  \begin{cases}
    -\frac{1}{6} = -\frac{1}{2} + a_3 \\
    \frac{1}{5!} = \frac{1}{4!} - \frac{a_3}{2!} + a_5
  \end{cases}
  \implies \quad
  a_3 = -\frac{1}{3}, \quad
  a_5 = \frac{2}{15}
\]
as the coefficients for the Taylor polynomial.
\end{proof}

\begin{remark}
  If
  \[
    f'(x) = \sum_{k = 0}^n b_k (x - a)^k + o((x - a)^n),
  \]
  then the anti-derivative of $f(x)$ has
  \[f(x) = f(x_0) + \sum_{k = 0}^n a_{k + 1} (x - a){k + 1} + o((x - a)^{n + 1}),\]
  where $a_{k + 1} = b_k / (k + 1)$ for $0 \le k \le n$.
  This is because
  \[
    b_k = \frac{(f')^{(k)}(a)}{k!} = \frac{f^{(k + 1)}(a)}{k!}
    \quad \text{and} \quad
    a_{k + 1} = \frac{f^{(k + 1)}(a)}{k + 1}
    = \frac{1}{k + 1} \frac{f^{(k + 1)}(a)}{k!}
    = \frac{b_k}{k + 1}.
  \]
\end{remark}

\begin{example}
  Find the Taylor polynomial for $f(x) = \arctan x$.
\end{example}

\begin{proof}
  Recall that
  \[f'(x) = \frac{1}{1 + x^2} = \sum_{k = 0}^n (-1)^k x^{2k}.\]
  Using the above we get
  \[
    f(x) = \arctan x = \sum_{k = 0}^n (-1)^k \frac{x^{2k + 1}}{2k + 1} + o(x^{2n + 2})
  \]
  as the Taylor polynomial.
\end{proof}

\section{Applications for Taylor Polynomials}
\subsection{Finding Limits}
\begin{remark}
Let $f(x) = ax^n + o(x^n)$ as $x \to 0$
and $g(x) = bx^n + o(x^n)$ where $b \ne 0$.
Then
\[
  \lim_{x \to 0} \frac{f(x)}{g(x)} = \frac{a}{b}.
\]
\end{remark}

\begin{remark}
  For the polynomial of $f(g(x))$, we can do
  \[
    f(u) = \sum_{k = 0}^n a_k (u - g(a))^k + o((u - g(a))^n),
    \quad \text{where} \quad
    u = g(x) = \sum_{k = 0}^n b_k(x - a)^k + o((x - a)^n).
  \]
  Then we can substitute  in $u = g(x)$ to find the
  overall polynomial.
\end{remark}

\begin{example}
  Find
  \[
    \lim_{x \to 0} \frac{\sqrt{1 + 2 \tan x} - e^x + x^2}{\arcsin x - \sin x}.
  \]
\end{example}

\begin{proof}
  Note that
  \begin{align*}
    \sqrt{1 + 2\tan x} - e^x + x^2 &= \frac{2x^3}{3} + o(x^3), \\
    \arcsin x - \sin x &= \frac{x^3}{3} + o(x^3).
  \end{align*}
  So the desired limit is $2$.
\end{proof}

\begin{remark}
  If $f(x) = ax^n + o(x^n)$ and $g(x) = bx^m + o(x^m)$
  for $a, b \ne 0$, then
  \[
    \lim_{x \to 0} \frac{f(x)}{g(x)} = \begin{cases}
      a / b & \text{if $m = n$}, \\
      0 & \text{if $m < n$}, \\
      \infty & \text{if $m > n$}.
    \end{cases}
  \]
\end{remark}

\begin{example}
  Assume $f(x) = 1 + ax^n + o(x^n)$ where
  $a \ne 0$ and
  \[
    g(x) = \frac{1}{bx^n + o(x^n)}, \quad \text{i.e.} \quad
    \frac{1}{g(x)} = bx^n + o(x^n).
  \]
  for $b \ne 0$. Then
  \[
    \lim_{x \to 0} f(x)^{g(x)} = e^{a / b}.
  \]
  Let $y(x) = f(x)^{g(x)}$, then
  $\ln y(x) = g(x) \ln f(x)$. Note that
  \[
    \ln f(x) = \ln(1 + ax^n + o(x^n))
    = ax^n + o(x^n),
  \]
  so that
  \[
    \frac{\ln f(x)}{1 / g(x)} = \frac{ax^n + o(x^n)}{bx^n + o(x^n)} \to \frac{a}{b}
  \]
  as $x \to 0$. Thus $\ln y(x) \to a / b$ and
  $y(x) \to e^{a / b}$ as $x \to 0$.
\end{example}

\begin{example}
  Find
  \[
    \lim_{x \to 0} \left[\cos(xe^x) - \ln (1 - x) - x\right]^{\cot x^3}.
  \]
\end{example}

\begin{proof}
  Here we have
  \[
    f(x) = \cos(xe^x) - \ln(1 - x) - x
    = 1 - \frac{2}{3} x^3 + o(x^3) \quad
    \text{and} \quad \frac{1}{g(x)} = \tan x^3 = x^3 + o(x^3).
  \]
  Thus the limit is $e^{-2 / 3}$.
\end{proof}

\subsection{Estimation}
\begin{example}
  Let $f(x)$ be twice differentiable in $[0, 1]$ and
  $f(0) = f(1)$. Further assume $|f''(x)| \le M$
  for $0 \le x \le 1$. Prove that
  $|f'(x)| \le M / 2$ for $0 \le x \le 1$.
\end{example}

\begin{proof}
  Recall that Lagrange's form of Taylor's theorem says
  \[
    f(x) = f(a) + f'(a) (x - a) + \frac{f''(\xi)}{2!}(x - a)^2
  \]
  for some $\xi$ between $a$ and $x$. Thus for any
  $x \in (0, 1)$, we have
  \[
    f(1) = f(x) + f'(x)(1 - x) + \frac{f''(\xi_1)}{2}(1 - x)^2.
  \]
  Similarly, we have
  \[
    f(0) = f(x) + f'(x)(-x) + \frac{f''(\xi_2)}{2}x^2.
  \]
  Here $x \le \xi_1 \le 1$ and $0 \le \xi_2 \le x$.
  Since $f(1) = f(2)$, we can solve for $f'(x)$ to get
  \[
    f'(x) = \frac{f''(\xi_2)x^2 - f''(\xi_1)(1 - x)^2}{2}.
  \]
  Then taking absolute values yields
  \[
    |f'(x)| \le M \left(\frac{x^2 + (1 - x)^2}{2}\right) \le \frac{M}{2} \max_{0 \le x \le 1} \left[x^2 + (1 - x)^2\right]
    = \frac{M}{2},
  \]
  as desired.
\end{proof}

\begin{example}
  Let $f(x)$ be twice differentiable in $[0, 1]$ and
  $f'(a) = f'(b) = 0$. Then there exists $\xi \in (a, b)$
  such that
  \[
    |f''(\xi)| \ge 4 \frac{|f(a) - f(b)|}{(b - a)^2}.
  \]
\end{example}

\begin{proof}
  Note that this is equivalent to
  \[
    |f(b) - f(a)| \le f''(\xi)\left(\frac{b - a}{2}\right)^2.
  \]
  Then we have
  \[
    f\left(\frac{b + a}{2}\right) = f(a) + \frac{f''(\xi_1)}{2}\left(\frac{b - a}{2}\right)^2
    = f(b) - \frac{f''(\xi_2)}{2}\left(\frac{b - a}{2}\right)^2,
  \]
  so that
  \[
    f(b) - f(a) = \frac{f''(\xi_2) + f''(\xi_1)}{2} \left(\frac{b - a}{2}\right)^2.
  \]
  From here we have
  \[
    |f(b) - f(a)| \le \underbrace{\frac{|f''(\xi_1)| + |f''(\xi_2)|}{2}}_{= |f''(\xi)|} \left(\frac{b - a}{2}\right)^2
  \]
  for some $\xi \in (a, b)$ by Darboux's lemma, as
  desired.
\end{proof}
