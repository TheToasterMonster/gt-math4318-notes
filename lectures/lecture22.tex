\chapter{Apr.~4 --- Change of Variables}

\section{Change of Variables for Double Integrals}
Consider a region $G$ in $uv$-coordinates, and define
a transformation $T$ on $G$ to $xy$-coordinates by
\[
  T :
  \begin{cases}
    x = x(u,v) \\
    y = y(u,v).
  \end{cases}
\]
Set $\Omega = T(G)$. Assume that $T$ is injective
and $T \in C^2(\overline{G})$.

\begin{definition}
  We define the \emph{Jacobian} of $T$ to be
  \[
    J(u, v) = \frac{\partial(x, y)}{\partial(u, v)}
    = \det
    \begin{pmatrix}
      \partial x / \partial u & \partial x / \partial v \\
      \partial y / \partial u & \partial y / \partial v
    \end{pmatrix}.
  \]
\end{definition}

\begin{remark}
  The Jacobian $J$ does not change sign in $G$.
\end{remark}

The goal now is to show the change of variables formula
\[
  \iint_{\Omega} f(x, y)\, dxdy
  = \iint_G f(x(u, v), y(u, v)) |J(u, v)|\, dudv.
\]
Notice the extra $|J(u, v)|$ factor on the right-hand side.

\begin{remark}
  Let $\Gamma$ be a closed curve in the counterclockwise
  orientation that encloses $\Omega$. If $\Gamma$ is
  piecewise continuous, then recall that
  by Green's theorem we have
  \[
    \area(\Omega) = \frac{1}{2} \oint_{\Gamma} x\, dy - y\, dx.
  \]
\end{remark}

\begin{lemma}
  Let $G = (u_0, u_0 + h) \times (v_0, v_0 + h)$ be a
  square region for some $h > 0$. Assume $T$ is
  injective and $T \in C^2(\overline{G})$. Then\footnote{Recall that $m(TG)$ denotes the measure of $TG$. Also $TG$ is the image of $G$ under $T$.}
  \[
    m(TG) = \iint_G |J(u, v)|\, dudv.
  \]
\end{lemma}

\begin{proof}
  When $J > 0$, $\partial(TG)$ is counterclockwise, so by the
  previous remark we can calculate
  \begin{align*}
    2 \area(TG)
    &= \int_{u_0}^{u_0 + h} \left[x(u, v_0) \frac{\partial y(u, v_0)}{\partial u} - y(u, v_0) \frac{\partial x(u, v_0)}{\partial u}\right] du \\
    &\quad + \int_{v_0}^{v_0 + h} \left[x(u_0 + h, v) \frac{\partial y(u_0 + h, v)}{\partial v} - y(u_0 + h, v) \frac{\partial x(u_0 + h, v)}{\partial v}\right] dv \\
    &\quad +\int_{u_0 + h}^{u_0} \left[x(u, v_0 + h) \frac{\partial y(u, v_0 + h)}{\partial u} - y(u, v_0 + h) \frac{\partial x(u, v_0 + h)}{\partial u}\right] du \\
    &\quad + \int_{v_0 + h}^{v_0} \left[x(u_0, v) \frac{\partial y(u_0, v)}{\partial v} - y(u_0, v) \frac{\partial x(u_0, v)}{\partial v}\right] dv \\
    &= I + II + III + IV.
  \end{align*}
  By the fundamental theorem of calculus, we get
  \begin{align*}
    I
    &= \int_{u_0}^{u_0 + h}
    \left[x(u, v_0 + h) \frac{\partial(u, v_0 + h)}{\partial u} - x(u, v_0) \frac{\partial y(u, v_0)}{\partial u}\right] du \\
    &= \int_{u_0}^{u_0 + h} du \left[\int_{v_0}^{v_0 + h} \frac{\partial x(u, v)}{\partial v} \frac{\partial y(u, v)}{\partial u} + x(u, v) \frac{\partial^2 y(u, v)}{\partial v \partial u}\right] dv
    = \iint_G \left[\frac{\partial x}{\partial v} \frac{\partial y}{\partial u} + x \frac{\partial^2 y}{\partial v \partial u}\right] dudv.
  \end{align*}
  Similarly apply this to the other three sides and
  combine to get (note that $T$ is $C^2$ on $\overline{G}$),
  \begin{align*}
    2\area(TG)
    &= \iint_G \left[\frac{\partial x}{\partial u}\frac{\partial y}{\partial v} + x \frac{\partial^2 y}{\partial u \partial v}\right] dudv
    - \iint_G \left[\frac{\partial x}{\partial v}\frac{\partial y}{\partial u} + x \frac{\partial^2 y}{\partial v \partial u}\right] dudv \\
    &\quad + \iint_G \left[\frac{\partial y}{\partial v}\frac{\partial x}{\partial u} + y \frac{\partial^2 x}{\partial v \partial u}\right] dudv
    - \iint_G \left[\frac{\partial y}{\partial u}\frac{\partial x}{\partial v} + y \frac{\partial^2 x}{\partial u \partial v}\right] dudv \\
    &= 2 \iint_G \left(\frac{\partial x}{\partial u} \frac{\partial y}{\partial v} - \frac{\partial x}{\partial v} \frac{\partial y}{\partial u}\right)
    = 2 \iint_G J(u, v)\, dudv,
  \end{align*}
  which is the desired result. The absolute
  value of $J$ is necessary when $J < 0$, since
  $\partial(TG)$ is clockwise.
\end{proof}

\begin{theorem}
  Let $T : G \to \Omega$, where $G$ and $\Omega$ are
  both measurable sets in $\R^2$. Suppose $T$ is
  bijective and $T \in \C^2(\overline{G})$, (and that
  $J(u, v) \ne 0$ in $G$).\footnote{This last condition is implied by the previous two.} If $f(x, y)$ is integrable
  on $\overline{\Omega}$, then
  \[
    \iint_{\Omega} f(x, y)\, dxdy
    = \iint_G f(x(u, v), y(u, v)) |J(u, v)|\, dudv.
  \]
\end{theorem}

\begin{proof}
  Use parallel lines with distance $h$ to cut
  $G$ into small ``rectangular'' (except
  near the boundary) sets $\Delta \sigma_i$. Let
  \[
    \Delta = \{\underbrace{\Delta \sigma_1, \Delta \sigma_2, \dots, \Delta \sigma_n}_{\text{contained in $\Int G$}}, \underbrace{\Delta \sigma_{n + 1}, \dots, \Delta \sigma_{n + p}}_{\text{intersects with $\partial G$}}\}.
  \]
  Let $(u_i, v_i) \in \Delta \sigma_i$ with
  $1 \le i \le n$. Then $T : \Delta \sigma_i \to T(\Delta \sigma_i)$,
  so that $x_i = x(u_i, v_i)$ and $y_i = y(u_i, v_i)$.
  By the lemma,
  \[
    m(T(\Delta \sigma_i)) = \iint_{\Delta \sigma_i} |J(u, v)|\, dudv
    = |J(\overline{u}_i, \overline{v}_i)| m(\Delta \sigma_i)
    = h^2 |J(\overline{u}_i, \overline{v}_i)|.
  \]
  for some $(\overline{u}_i, \overline{v}_i) \in \Delta \sigma_i$ by the middle value theorem
  (also note that $m(\Delta \sigma_i) = h^2$). So if
  $f$ is continuous on $G$, then
  \[
    \iint_G f\, dxdy = f(\xi) m(G)
  \]
  where $\xi$ is some point in $\overline{G}$. Now
  consider a limit of the Riemann sums. We have
  \[
    \lim_{h \to 0} \sum_{i = 1}^n f(x(u_i, v_i), y(u_i, v_i)) \underbrace{|J(\overline{u}_i, \overline{v}_i)| m(\Delta \sigma_i)}_{= m(T(\Delta \sigma_i))}
    = \iint_{\Omega} f(x, y)\, dxdy
  \]
  since for the sets near the boundary,
  \[
    \lim_{h \to 0} \sum_{i = n + 1}^{n + p} m(T(\Delta \sigma_i)) = 0 \implies
    \lim_{h \to 0} \sum_{i = n + 1}^{n + p} f(x_i, y_i) m(T(\Delta \sigma_i)) = 0.
  \]
  and $f$ is integrable and thus bounded on $\overline{\Omega}$.
  Now if $J(\overline{u}_i, \overline{v}_i) \to J(u_i, v_i)$,
  then
  \[
    \sum_{i = 1}^n f(x(u_i, v_i), y(u_i, v_i)) |J(u_i, v_i)| m(\Delta \sigma_i)
    \to \iint_G f(x(u, v), y(u, v)) |J(u, v)|\, dudv
  \]
  as $h \to 0$. To see that $J(\overline{u}_i, \overline{v}_i) \to J(u_i, v_i)$, observe that for
  any $\epsilon > 0$, since $|J(u, v)|$ is uniformly
  continuous on $\overline{G}$, there exists
  $\delta > 0$ such that
  \[
    d((u_i, v_i), (\overline{u}_i, \overline{v}_i)) < \delta
    \implies \left| |J(u_i, v_i)| - |J(\overline{u}_i, \overline{v}_i)| \right| < \frac{\epsilon}{M \cdot m(G)},
  \]
  where $|f| \le M$ on $\overline{G}$. So if
  $h < \delta / \sqrt{2}$, then we have
  \[
   \left| \sum_{i = 1}^n f(x_i, y_i) |J(\overline{u}_i, \overline{v}_i)| m(\Delta \sigma_i)
    - \sum_{i = 1}^n f(x_i, y_i) |J(u_i, v_i)| m(\Delta \sigma_i)\right|
    \le \frac{\epsilon}{M \cdot m(G)} \cdot M \cdot m(G) = \epsilon.
  \]
  This finishes the desired result.
\end{proof}

\begin{remark}
  The theorem actually also holds for $T \in C^1(\overline{G})$.
  This is because if
  \[
    \iint_{G} f(x(u, v), y(u, v)) |J(u, v)|\, dudv
    = \iint_{\Omega} f(x, y)\, dxdy
  \]
  for any $T \in C^2$, then we can choose $T^n \in C^2$
  such that $|T_n - T|_{C^1} \to 0$. Then we have
  \[
    \iint_G f(x_n(u, v), y_n(u, v)) |J_n(u, v)|\, dudv
    = \iint_{T_n(G)} f(x, y)\, dxdy.
  \]
  As $n \to \infty$, we have $T_n(G) \to T(G) = \Omega$,
  so that
  \[
    \text{LHS} \to \iint_G f(x(u, v), y(u, v)) |J(u, v)|\, dudv
    \quad \text{and} \quad
    \text{RHS} \to  \iint_{\Omega} f(x, y)\, dxdy
  \]
  since we have uniform convergence.
\end{remark}

\section{Change of Variables in 3D}

Let $G$ be a measurable set in $\R^3$ with $uvw$-coordinates
and
\[
  T :
  \begin{cases}
    x = x(u, v, w) \\
    y = y(u, v, w) \\
    z = z(u, v, w)
  \end{cases}.
\]
Set $D = T(G)$. Assume that $T$ is a homeomorphism and
$T \in C^1(\overline{G})$. Then define
\[
  J(u, v, w) = \frac{\partial(x, y, z)}{\partial(u, v, w)}
  = \det
  \begin{pmatrix}
    \partial x / \partial u & \partial x / \partial v & \partial x / \partial w \\
    \partial y / \partial u & \partial y / \partial v & \partial y / \partial w \\
    \partial z / \partial u & \partial z / \partial v & \partial z / \partial w
  \end{pmatrix}.
\]
If $f$ is integrable on $\overline{D}$, then
\[
  \iiint_D f(x, y, z)\, dxdydz
  = \iiint_G f(x(u, v, w), y(u, v, w), z(u, v, w)) |J(u, v, w)|\, dudvdw.
\]
The proof is similar to the 2D case (get a formula for
the cube and proceed similarly).
