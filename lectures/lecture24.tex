\chapter{Apr.~11 --- Green's Theorem}

\section{Midterm 2 Problems}
The following was Problem 3 on Midterm 2:
\begin{exercise}
  For a given function $y(t) \in C[0, 1]$ (i.e. continuous
  functions in $[0, 1]$) and a constant $\lambda \in \R$
  with $|\lambda| < 1$, show that the integral equation
  \[
    x(t) - \lambda \int_0^1 e^{t - s} x(s)\, ds = y(t) \tag{$*$}
  \]
  has a unique solution $x(t) \in C[0, 1]$.
\end{exercise}

\begin{remark}
  A common idea was to define
  \[
    T : x(t) \mapsto y(t) + \lambda \int_0^1 e^{t - s} x(s)\, ds.
  \]
  and $\rho(u, v) = \sup_{t \in [0, 1]} |u(t) - v(t)|$.
  But we run into problems since
  \[
    \rho(Tu, Tv)
    \le |\lambda| \sup_{t \in [0, 1]} \left| \int_0^1 e^{t - s} (u(s) - v(s))\, ds \right|
    \le |\lambda| \rho(u, v) \sup_{t \in [0, 1]} e^t \int_0^1 e^{-s}\, ds
  \]
  But $\sup_{t \in [0, 1]} e^t = e$ and the integral
  evaluates to $1 - e^{-1}$, so we have
  \[
    \rho(Tu, Tv) \le |\lambda| e (1 - e^{-1}) \rho(u, v)
    \le |\lambda| (e - 1) \rho(u, v).
  \]
  This does not work since $(e - 1) > 1$.
\end{remark}

\begin{proof}
  A better idea is to first transform $(*)$ by
  multiplying by $e^{-t}$:
  \[
    e^{-t} x(t) - \lambda \int_0^1 e^{-s} x(s)\, ds = e^{-t} y(t)
  \]
  since $t$ is independent of the integral. Then set
  $z(t) = e^{-t} x(t)$ and define
  \[
    T : z(t) \mapsto e^{-t} y(t) + \lambda \int_0^1 z(s)\, ds.
  \]
  Using this we can get
  \[
    d(Tu, Tv) = \lambda \sup_{t \in [0, 1]} \left|\int_0^1 u(s) - v(s)\, ds\right|
    \le \lambda \rho(u, v),
  \]
  which is good enough. Then by the contraction
  mapping principle, $T$ has a unique fixed point, and
  multiplying by $e^t$ gives a unique solution to $(*)$.
\end{proof}

\section{Green's Theorem}
\begin{theorem}[Green's theorem]
  Let $D$ be a region in $\R^2$ enclosed by a finite
  number of measurable curves, and $P(x, y), Q(x, y)$
  be continuous on $D$ with continuous partial
  derivatives. Then
  \[
    \int_{\partial D} P\, dx + Q\, dy = \iint_D \left(\frac{\partial Q}{\partial x} - \frac{\partial P}{\partial y}\right) dx dy. \tag{$*$}
  \]
  Here $\partial D$ is oriented so that $\{\vec{n}, \vec{\tau}\}$ is a right-handed coordinate
  system, where $\vec{n}$ is the outward normal to the
  region and $\vec{\tau}$ is the tangent vector of
  the curve $\partial D$.\footnote{In other words, the curve $\partial D$ is oriented so that the region $D$ is to its \emph{left}.}
\end{theorem}

\begin{proof}
  First we consider Type I domains, i.e. regions of
  the form
  \[
    D = \{a \le x \le b,\, \varphi(x) \le y \le \psi(x)\}.
  \]
  Let $A, B, C, D$ be the four vertices of the
  region $D$, going counterclockwise and starting
  in the bottom-left, i.e. we have
  \[
    A = (a, \varphi(a)),\, B = (b, \varphi(b)),\, C = (b, \psi(b)),\, D = (a, \psi(a)).
  \]
  Then
  \[
    \int_{\widetilde{AB}} P\, dx = \int_a^b P(x, \varphi(x))\, dx,
    \quad
    \int_{\widetilde{CD}} P\, dx = -\int_a^b P(x, \psi(x))\, dx,
  \]
  and
  \[
    \int_{\widetilde{BC}} P\, dx = \int_{\widetilde{DA}} P\, dx = 0
  \]
  since $x$ is constant on $\widetilde{BC}$ and $\widetilde{DA}$. Thus
  \[
    \int_{\partial D} P\, dx = \int_a^b P(x, \varphi(x))\, dx - \int_a^b P(x, \psi(x))\, dx
    = \int_a^b [P(x, \varphi(x)) - P(x, \psi(x))]\, dx.
  \]
  By the fundamental theorem of calculus, we have
  \[
    P(x, \varphi(x)) - P(x, \psi(x)) = -\int_{\varphi(x)}^{\psi(x)} \frac{\partial P}{\partial y}(x, y)\, dy,
  \]
  so that
  \[
    \int_{\partial D} P\, dx = -\int_a^b \int_{\varphi(x)}^{\psi(x)} \frac{\partial P}{\partial y}(x, y)\, dy dx
    = -\iint_D \frac{\partial P}{\partial y}\, dx dy.
  \]
  For Type II domains, i.e. regions of the form
  \[
    D = \{c \le y \le d, \varphi(y) \le x \le \psi(y)\}.
  \]
  Using an identical argument,
  \[
    \int_{\partial D} Q\, dy = \iint_D \frac{\partial Q}{\partial x}\, dx dy.
  \]
  Now if $D$ is of both Type I and Type II, then we
  have both identities
  \[
    \int_{\partial D} P\, dx = -\iint_D \frac{\partial P}{\partial y}\, dx dy
    \quad \text{and} \quad
    \int_{\partial D} Q\, dy = \iint_D \frac{\partial Q}{\partial x}\, dx dy,
  \]
  and adding the two gives
  \[
    \int_{\partial D} P\, dx + Q\, dy = \iint_D \left(\frac{\partial Q}{\partial x} - \frac{\partial P}{\partial y}\right) dx dy,
  \]
  which is precisely Green's theorem for this special
  type of region.
  
  Now let $D$ be a simply connected (i.e. no holes\footnote{More formally, $D$ is \emph{simply connected} if it is path-connected and every loop can be contracted to a point.}) domain enclosed by a measurable curve. Then $D$ can be
  domain enclosed by a measurable curve. Then we try
  to divide $D$ by using parallel lines into
  $D_1, \dots, D_n$ such that each $D_1, \dots, D_n$
  is of both Type I and Type II. Each subdomain falls
  into the previous case, so
  \[
    \sum_{i = 1}^n \int_{\partial D_i} P\, dx + Q\, dy = \sum_{i = 1}^n \iint_{D_i} \left(\frac{\partial Q}{\partial x} - \frac{\partial P}{\partial y}\right) dx dy
    = \iint_D \left(\frac{\partial Q}{\partial x} - \frac{\partial P}{\partial y}\right) dx dy.
  \]
  Now notice that each inner part of the $\partial D_i$
  is traversed twice since it is part of the boundary
  of two $D_i$, so these cancel and we get
  \[
    \int_{\partial D} P\, dx + Q\, dy
    = \sum_{i = 1}^n \int_{\partial D_i} P\, dx + Q\, dy
    = \iint_D \left(\frac{\partial Q}{\partial x} - \frac{\partial P}{\partial y}\right) dx dy,
  \]
  which is Green's theorem for a simply connected
  domain.

  In the most general case, divide $D$ into a finite
  number of simply connected domains, and
  apply the previous case to each one. We still get
  cancellation of the inner parts of the boundaries.
\end{proof}
