\chapter{Apr.~16 --- The Divergence Theorem}

\section{Green's Theorem for Flux}

\begin{remark}
  For the orientation of the curve $\partial D$, we
  can also think of defining the tangent direction
  $\vec{\tau}$ by rotating the outward normal $\vec{n}$
  by $90^\circ$ counterclockwise.
\end{remark}

\begin{remark}
  Another notation for line integrals is to set
  $\vec{v} = \langle P, Q \rangle$ and write
  \[
    \int_{C} P\, dx + Q\, dy
    = \int_{C} \vec{v} \cdot ds
    = \int_{C} \vec{v} \cdot \vec{\tau}\, d\ell.
  \]
  Here $\vec{v} = \langle P, Q \rangle$ is called a
  \emph{vector field}.
\end{remark}

\begin{corollary}[Green's theorem for flux]
  Assume the same hypotheses as in Green's theorem.
  Then
  \[
    \iint_D \left( \frac{\partial P}{\partial x} + \frac{\partial Q}{\partial y} \right) \, dxdy
    = \iint_{\partial D} \langle P, Q \rangle \cdot \vec{n}\, d\ell,
  \]
  where $\vec{n}$ is the outward normal vector to $D$.
\end{corollary}

\begin{proof}
  Set $P \to -Q$ and $Q \to P$ in the original
  statement of Green's theorem to get
  \[
    \iint_D \left( \frac{\partial P}{\partial x} + \frac{\partial Q}{\partial y} \right) \, dxdy = \int_{\partial D} -Q\, dx + P\, dy
    = \iint_{\partial D} \langle -Q, P \rangle \cdot ds
    = \iint_{\partial D} \langle -Q, P \rangle \cdot \vec{\tau}\, d\ell.
  \]
  Now observe that $\langle -Q, P \rangle$ is $\langle P, Q \rangle$
  rotated $90^\circ$ clockwise, so that
  $\langle -Q, P \rangle \cdot \vec{\tau} = \langle P, Q \rangle \cdot \vec{n}$.
\end{proof}

\section{An Application of Green's Theorem}
\begin{example}
  Let $C$ be a measurable closed curve enclosing the
  origin. Show that
  \[
    I = \int_C \frac{x\, dy - y\, dx}{x^2 + y^2} = 2\pi.
  \]
\end{example}

\begin{proof}
  Set
  \[
    P = - \frac{y}{x^2 + y^2} \quad \text{and} \quad
    Q = \frac{x}{x^2 + y^2}.
  \]
  Since $P, Q$ have singularities at the origin, let
  $C_\epsilon = \{x^2 + y^2 = \epsilon^2\}$ be a circle
  of radius $\epsilon$ for $\epsilon$ small. Let
  \[
    D_\epsilon = D \setminus B_\epsilon,
  \]
  where $D$ is the region enclosed by $C$ and
  $B_\epsilon = \{x^2 + y^2 \le \epsilon^2\}$. Then by
  Green's theorem, we have
  \[
    \int_{\partial D_\epsilon}
    \frac{x\, dy - y\, dx}{x^2 + y^2}
  = \iint_{D_\epsilon} \underbrace{\left( \frac{\partial Q}{\partial x} - \frac{\partial P}{\partial y} \right)}_{= 0} \, dxdy = 0.
  \]
  Now observe
  \[
    \int_{\partial D_\epsilon} \frac{x\, dy - y\, dx}{x^2 + y^2}
    = \int_C \frac{x\, dy - y\, dx}{x^2 + y^2}
    - \int_{C_\epsilon} \frac{x\, dy - y\, dx}{x^2 + y^2},
  \]
  where $C$ and $C_\epsilon$ is oriented counterclockwise.
  This gives
  \[
    \int_C \frac{x\, dy - y\, dx}{x^2 + y^2}
    = \int_{C_\epsilon} \frac{x\, dy - y\, dx}{x^2 + y^2}.
  \]
  Now parametrize $C_\epsilon$ by
  $x = \epsilon \cos t$ and $y = \epsilon \sin t$ for
  $0 \le t \le 2\pi$. Then
  \[
    \int_{C_\epsilon} \frac{x\, dy - y\, dx}{x^2 + y^2}
    = \int_0^{2\pi} \frac{\epsilon \cos t \cdot \epsilon \cos t - \epsilon \sin t \cdot (-\epsilon \sin t)}{\epsilon^2} \, dt
    = \int_0^{2\pi} 1\, dt = 2\pi,
  \]
  which is the desired result.
\end{proof}

\section{The Divergence Theorem}
\begin{remark}
  The divergence theorem is also sometimes called
  \emph{Gauss's formula}.
\end{remark}

\begin{remark}
  Recall that for a surface $S : z = f(x, y)$. This is
  called the \emph{graph} of $f$ and we have
  \[
    \vec{n} = \frac{1}{\sqrt{1 + f_x^2 + f_y^2}} \langle -f_x, -f_y, 1 \rangle.
  \]
  Then we get
  \[
    \iint_S \vec{v} \cdot \vec{n}\, dS
    = \iint_S \vec{v} \cdot \langle -f_x, -f_y, 1\rangle \, dxdy,
  \]
  where $dS = \sqrt{1 + f_x^2 + f_y^2} \, dxdy$.
\end{remark}

\begin{theorem}[Divergence theorem]
  Let $D$ be a bounded region in $\R^3$ enclosed by a
  finite number of piecewise smooth orientable surfaces.\footnote{A surface is \emph{orientable} if it has a consistent (and continuous) choice of normal vector everywhere on the surface. The \emph{M\"obius strip} is an example of a non-orientable surface.}
  Suppose we have a vector field
  $\vec{v} = \langle P, Q, R \rangle$ that is
  continuous with continuous partial
  derivatives in $D$. Then
  \[
    \iiint_D \nabla \cdot \vec{v} \, dxdydz
    = \iiint_D \left( \frac{\partial P}{\partial x} + \frac{\partial Q}{\partial y} + \frac{\partial R}{\partial z} \right) \, dxdydz
    = \iint_{\partial D} \vec{v} \cdot \vec{n}\, dS,
  \]
  where $\vec{n}$ is the outward normal vector to $D$.
\end{theorem}

\begin{proof}
  First consider Type I solids of the form
  \[
    D = \{(x, y) \in \Omega,\, \varphi(x, y) \le z \le \psi(x, y)\}
  \]
  and let $\vec{v} = \langle 0, 0, R \rangle$. In this
  case we wish to show that
  \[
    \iiint_D \frac{\partial R}{\partial z} \, dxdydz
    = \iint_{\partial D} \vec{v} \cdot \vec{n}\, dS,
  \]
  where we note that $\nabla \cdot \vec{v} = \partial R / \partial z$.
  Now let $\partial D = S_1 \cup S_2 \cup S_3$ where
  \begin{align*}
    S_1 &= \{z = \psi(x, y),\, (x, y) \in \Omega\}, \\
    S_2 &= \{z = \varphi(x, y),\, (x, y) \in \Omega\}, \\
    S_3 &= \{(x, y) \in \partial \Omega,\, \varphi(x, y) \le z \le \psi(x, y)\}.
  \end{align*}
  Note that $\vec{n}$ is upwards on $S_1$, downwards
  on $S_2$, and outwards radially on $S_3$. Then
  \[
    \iint_{S_3} \vec{v} \cdot \vec{n}\, dS
    = 0
  \]
  since $\vec{v} \perp \vec{n}$ on $S_3$. So we get that
  \begin{align*}
    \iint_{\partial D} \vec{v} \cdot \vec{n}\, dS
    &= \iint_{S_1} \vec{v} \cdot \vec{n}\, dS
    + \iint_{S_2} \vec{v} \cdot \vec{n}\, dS \\
    &= \iint_\Omega R(x, y, \psi(x, y))\, dxdy
    - \iint_\Omega R(x, y, \varphi(x, y))\, dxdy \\
    &= \iint_D \left[\int_{\varphi(x, y)}^{\psi(x, y)} \frac{\partial R}{\partial z}\, dz \right] dxdy
    = \iiint_D \frac{\partial R}{\partial z}\, dxdydz
  \end{align*}
  by the fundamental theorem of calculus. This is
  the desired result in this case where $D$ is a Type
  I solid. Also note that similar
  results hold for Type II solids of the form
  (set $\vec{v} = \langle 0, Q, 0 \rangle$)
  \[
    D = \{(x, z) \in \Omega,\, \varphi(x, z) \le y \le \psi(x, z)\}
    \implies \iiint_D \frac{\partial Q}{\partial y}\, dxdydz
    = \iint_{\partial D} \vec{v} \cdot \vec{n}\, dS
  \]
  and Type III solids of the form (set $\vec{v} = \langle P, 0, 0 \rangle$)
  \[
  D = \{(y, z) \in \Omega,\, \varphi(y, z) \le x \le \psi(y, z)\}
  \implies \iiint_D \frac{\partial P}{\partial x}\, dxdydz
  = \iint_{\partial D} \vec{v} \cdot \vec{n}\, dS
  .\]
  The proofs follow similarly as in the Type I
  solid case.

  Now if a solid $D$ is of all three types, then
  for any vector field $\vec{v} = \langle P, Q, R \rangle$,
  we can write
  \[
    \vec{v} = \vec{v}_1 + \vec{v}_2 + \vec{v}_3
    = \langle 0, 0, R \rangle + \langle 0, Q, 0 \rangle + \langle P, 0, 0 \rangle .
  \]
  Now since $D$ is of all three types, we can apply the
  previous special cases to get
  \[
    \iiint_D \nabla \cdot \vec{v}\, dxdydz
    = \sum_{i=1}^3 \iiint_{D} \nabla \cdot \vec{v}_i\, dxdydz
    = \sum_{i=1}^3 \iint_{\partial D} \vec{v}_i \cdot {n} \, dS
    = \iint_{\partial D} \vec{v} \cdot \vec{n}\, dS.
  \]
  This verifies the divergence theorem in this case where
  $D$ is of all three types.

  Now check the case where $D$ is a tetrahedron and
  $\vec{v}$ is arbitrary. For
  each component of $\vec{v}$, project into the
  corresponding plane to break $D$ into pieces of
  a specific type of solid. Then apply the special
  cases we had before. Note the cancellation on inner
  surface pieces. We can also get a similar result for
  polyhedra in general by breaking them into tetrahedra
  and applying the tetrahedron case.

  Now for a general solid, approximate it by polyhedra
  and apply the previous cases.
\end{proof}

\begin{remark}
  This is a three-dimensional generalization of the flux
  version of Green's theorem.
\end{remark}
