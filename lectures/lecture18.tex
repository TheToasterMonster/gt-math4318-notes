\chapter{Mar.~12 --- More on the Implicit Function Theorem}

\section{Example of Unequal Mixed Partial Derivatives}

\begin{example}
  Consider the function.
  \[
    f(x, y) =
    \begin{cases}
      xy(x^2 - y^2) / (x^2 + y^2) & \text{if } x^2 + y^2 \ne 0 \\
      0 & \text{if } x^2 + y^2 = 0.
    \end{cases}
  \]
  We find
  \[
    f_x'(x, y) = \begin{cases}
      y((x^2 - y^2)(x^2 + y^2) + 4x^2 y^2 / (x^2 + y^2)^2) & \text{if } x^2 + y^2 \ne 0 \\
      0 & \text{if } x^2 + y^2 = 0.
    \end{cases}
  \]
  and
  \[
  f_y'(x, y) =
  \begin{cases}
    x((x^2 - y^2)(x^2 + y^2) + 4x^2 y^2 / (x^2 + y^2)^2) & \text{if } x^2 + y^2 \ne 0 \\
    0 & \text{if } x^2 + y^2 = 0.
  \end{cases}
  .\]
  Now $f_x'(0, y) = -y$ and so
  $f_{xy}''(0, 0) = -1$, but $f_y'(x, 0) = x$ and so
  $f_{yx}''(0, 0) = 1$.  Thus 
  $f_{xy}''(0, 0) \ne f_{yx}''(0, 0)$.
\end{example}

\section{Revisiting the Implicit Function Theorem}

\begin{theorem}[Mean value theorem in $\R^n$]
  Suppose $f : U \subseteq \R^n \to \R$ where $U$ is open,
  and that $f$ is differentiable. THen for any
  $a, b \in U$, there exists $\xi$ on the line segment
  connecting $a, b$ such that
  \[
    f(a) - f(b) = \nabla f(\xi) \cdot (a - b).
  \]
\end{theorem}

\begin{proof}
  Define $h(t) = f(ta + (1 - t)b)$ for $t \in [0, 1]$.
  Then
  \[
  f(a) - f(b) = h(1) - h(0) = h'(c)
  = \nabla f(ca + (1 - c)b) \cdot (a - b)
  \]
  for some $c \in (0, 1)$ by the usual mean value theorem.
  So we can pick $\xi = ca + (1 - c)b$.
\end{proof}

The following is a corollary of Theorem \ref{thm:implicit-higher-dim}:
\begin{corollary}
  If $f_1, \dots, f_n$ are continuously differentiable
  in a neighborhood of $(a, b)$, then $U, V$ can be
  chosen such that $\varphi : U \to V$ is continuously
  differentiable, i.e. $\varphi \in C^1(U, V)$.\footnote{In the notation of Theorem \ref{thm:implicit-higher-dim}, we had $f = (f_1, \dots, f_n)$, $(x_0, y_0)$ instead of $(a, b)$, and $U_0, V_0$ instead of $U, V$.}
\end{corollary}

\begin{proof}
  Recall that
  \[
    \begin{cases}
      f_i(x, \varphi(x)) = 0 \\
      \varphi(a) = b
    \end{cases}
  \]
  for all $1 \le i \le n$. Then we have
  \begin{align*}
    0 = f_i(x, \varphi(x)) - f_i(a, \varphi(a))
    &= \frac{\partial f_i}{\partial x_1} (z^i)(x_1 - a_1)
    + \dots + \frac{\partial f_i}{\partial x_m} (z^i)(x_m - a_m) \\
    & \quad + \frac{\partial f_i}{\partial y_1} (z^i)(\varphi_1(x) - b_1)
    + \dots + \frac{\partial f_i}{\partial y_n} (z^i)(\varphi_n(x) - b_n)
  \end{align*}
  where $z_i$ is between $(x, \varphi(x))$ and $(a, b)$.
  Now since
  $\det(\partial f / \partial y_j (z^i)) \ne 0$, we
  can solve this linear system to get
  \[
    \varphi_i(x) - b_i =
    A_{i 1}(x)(x_1 - a_1) + A_{i 2}(x)(x_2 - a_2) + \dots + A_{i m}(x)(x_m - a_m),
  \]
  where the coefficient functions $A_{i 1}, \dots, A_{i m}$
  are continuous in a neighborhood of $a$ by assumption.
  So
  $\varphi(x) = (\varphi_1(x), \dots, \varphi_n(x))$
  is continuously differentiable in a neighborhood of
  $a$, as desired.
\end{proof}

\begin{remark}
  We can improve this if we know $f$ is more smooth.
  In general, if $f \in C^k$, then $\varphi \in C^k$ also.
\end{remark}

\begin{example}[Implicit differentiation]
  Suppose $F(x, y) = 0$ and $y = \varphi(x)$. If
  $F \in C^1$ and $F_y' \ne 0$,
  \[
    \frac{d}{dx} F(x, \varphi(x)) = 0
    \implies F_x' + F_y' \varphi_x = 0
    \implies \varphi_x = -\frac{F_x'}{F_y'}.
  \]
  We can justify this rigorously via the implicit
  function theorem.
\end{example}

\section{Lagrange Multipliers}
Recall from multivariable calculus that we can use
the method of Lagrange multipliers to solve
the minimum (or maximum) of a function $f(x, y)$ subject
to a constraint $g(x, y) = 0$. The
\emph{Lagrange multiplier equation} is
\[
  \begin{cases}
    \nabla f = \lambda \nabla g \\
    g = 0,
  \end{cases}
\]
which is a system of three equations in $(x, y, \lambda)$.
Here $\lambda$ is called the \emph{Lagrange multiplier}.

\begin{prop}
Assume that at a local minimum (or maximum) point $(x_0, y_0)$ of $f$, we have
\[
  \nabla g(x_0, y_0) \ne \vec{0},
\]
i.e. $g_x^2(x_0, y_0) + g_y^2(x_0, y_0) \ne 0$. Then there
exists $\lambda$ such that $\nabla f = \lambda \nabla g$
at $(x_0, y_0)$.
\end{prop}

\begin{proof}
  Assume $g_y'(x_0, y_0) \ne 0$. Then by the implicit
  function theorem, there exists some $y = \varphi(x)$
  in a neighborhood of $y_0$ such that
  $g(x, \varphi(x)) = 0$. Then $h(x) = f(x, \varphi(x))$
  obtains a minimum (or maximum) at $x_0$. Then
  $h'(x_0) = 0$, so
  \[
    f_x'(x_0, y_0) + f_y'(x_0, y_0) \varphi_x'(x_0) = 0.
  \]
  Also, by implicit differentiation,
  \[
    \varphi_x'(x_0) = -\frac{g_x'(x_0, y_0)}{g_y'(x_0, y_0)}.
  \]
  So we can set
  \[
    \lambda = -\frac{f_y'(x_0, y_0)}{g_y'(x_0, y_0)}
    \implies
    \begin{cases}
      f_x'(x_0, y_0) + \lambda g_x'(x_0, y_0) = 0 \\
      f_y'(x_0, y_0) + \lambda g_y'(x_0, y_0) = 0.
    \end{cases}
  \]
  Thus we have
  $(\nabla f + \lambda \nabla g)(x_0, y_0) = 0$,
  as desired.
\end{proof}

\begin{remark}
  The condition $\nabla g(x_0, y_0) \ne \vec{0}$ is
  necessary to derive the Lagrange multiplier equation.
\end{remark}

\begin{example}
  Minimize $f(x, y) = x^2 + y^2$ subject to
  $g = (x - 1)^3 - y^2 = 0$. The distance of $g$ to
  the origin is clearly minimized at $(1, 0)$, but we have
  \[
    \nabla g(1, 0)
      = \left.(3(x - 1)^2, -2y)\right|_{(1, 0)}
      = (0, 0)
  \]
  and $\nabla f(1, 0) = (2, 0)$. So
  $\nabla f \ne \lambda \nabla g$ in this case since
  $\nabla g(1, 0) = \vec{0}$.
\end{example}

\begin{remark}
  So in general, we should also check the
  points where $\nabla g = \vec{0}$ separately.
\end{remark}

\begin{theorem}
  Suppose $f(x_1, \dots, x_n)$ obtains a local
  minimum (or maximum) at $p^0 = (x_1^{(0)}, \dots, x_n^{(0)})$
  subject to
  \[
  \begin{cases}
    g_1(x_1, \dots, x_n) = 0 \\
    \quad \vdots \\
    g_m(x_1, \dots, x_n) = 0,
  \end{cases}
  \]
  where $m < n$. Assume that $f, g_i \in C^1$ in a
  neighborhood of $p^0$ and the matrix
  \[
    \frac{\partial g}{\partial x}(p^0)
    = \begin{pmatrix}
      \partial g_1 / \partial x_1 & \dots & \partial g_1 / \partial x_n \\
      \vdots & \ddots & \vdots \\
      \partial g_m / \partial x_1 & \dots & \partial g_m / \partial x_n
    \end{pmatrix}(p^0)
  \]
  has rank $m$. Then there exist
  $\lambda_1, \dots, \lambda_m$ such that
  \[
    \begin{cases}
      (\nabla f + \lambda_1 \nabla g_1 + \dots + \lambda_m \nabla g_m)(p^0) = 0 \\
      g_i(p^0) = 0. \\
    \end{cases}
  \]
\end{theorem}

\begin{proof}
  We prove this in the case of four unknowns and
  two constraints (the proof can be generalized). Consider
  \[
    \text{minimize (or maximize) $f(x, y, z, t)$}
    \quad \text{subject to}
    \quad
    \begin{cases}
      \varphi(x, y, z, t) = 0 \\
      \psi(x, y, z, t) = 0.
    \end{cases}
  \]
  Suppose the minimum (or maximum) of $f$ is obtained
  at $p^0 = (x_0, y_0, z_0, t_0)$. Then by assumption,
  there are two variables (say $z, t$) such that
  \[
    \left.\det \frac{\partial (\varphi, \psi)}{\partial (z, t)}\right|_{p^0} \ne 0.
  \]
  Then by the implicit function theorem, we can find
  $z = g(x, y)$ and $t = h(x, y)$ such that the constraint
  equations are satisfied. Then
  $u(x, y) = f(x, y, z(x, y), t(x, y))$ obtains a minimum
  (or maximum) at $(x_0, y_0)$, so
  \[
    \nabla u(x_0, y_0) = 0
    \implies
    \begin{cases}
      f_x' + f_z' (\partial z / \partial x) + f_t' (\partial t / \partial x) = 0 \\
      f_y' + f_z' (\partial z / \partial y) + f_t' (\partial t / \partial y) = 0.
    \end{cases} \tag{1, 2}
  \]
  Then
  \[
    \begin{cases}
      \varphi_x' + \varphi_z' (\partial z / \partial x) + \varphi_t' (\partial t / \partial x) = 0 \\
      \psi_x' + \psi_z' (\partial z / \partial x) + \psi_t' (\partial t / \partial x) = 0.
    \end{cases}
    \text{and} \quad
    \begin{cases}
      \varphi_y' + \varphi_z' (\partial z / \partial y) + \varphi_t' (\partial t / \partial y) = 0 \\
      \psi_y' + \psi_z' (\partial z / \partial y) + \psi_t' (\partial t / \partial y) = 0.
    \end{cases} \tag{$\star$}
  \]
  Since $\det (\partial (\varphi, \psi) / \partial (z, t)) \ne 0$
  at $p^0$, we can find $\lambda, \mu$ such that
  \[
    \begin{cases}
      f_z' + \lambda \varphi_z' + \mu \psi_z' = 0 \\
      f_t' + \lambda \varphi_t' + \mu \psi_t' = 0,
    \end{cases}
  \]
  which is always possible since the coefficient
  matrix is nonsingular.
  Let $(A)$ and $(B)$ be the two systems in $(\star)$ and
  $A_1, A_2$ and $B_1, B_2$ be their first
  and second equations, respectively. Then look at
  the equation $\lambda(A_1) + \mu(A_2) + (1)$, which gives
  \[
  f_x' + \lambda \varphi_x' + \mu \psi_x' +
  \underbrace{(f_z' + \lambda \varphi_z' + \mu \psi_z')}_{= 0} \frac{\partial z}{\partial x}
  + \underbrace{(f_t' + \lambda \varphi_t' + \mu \psi_t')}_{= 0} \frac{\partial t}{\partial x} = 0
  .\]
  This implies that $f_x' + \lambda \varphi_x' + \mu \psi_x' = 0$, and
  we can similarly get that
  $f_y' + \lambda \varphi_y' + \mu \psi_y' = 0$. Thus
  we end up with
  $\nabla f + \lambda \nabla \varphi + \mu \nabla \psi = 0$
  at $p^0$, as desired.
\end{proof}
