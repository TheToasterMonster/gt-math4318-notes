\chapter{Feb.~1 --- Riemann Integrability, Part 3}

\section{Even More Conditions for Integrability}
\begin{example}
  If $f(x)$ is monotone on $[a, b]$, then
  $f \in \mathcal{R}([a, b])$.
\end{example}

\begin{proof}
  Suppose $f(x)$ is monotone increasing on
  $[a, b]$ and $f(x)$ is not constant (since the result
  is trivial if $f$ is constant). Then
  $f(a) \le f(x) \le f(b)$. For any $\epsilon > 0$,
  for any partition $x_0 < \dots < x_n$ with width
  \[
    \delta < \frac{\epsilon}{f(b) - f(a)},
  \]
  we have on $[x_{i - 1}, x_i]$ that
  $M_i = f(x_i)$ and $f(x_{i - 1}) = m_i$ since
  $f$ is monotone. Then
  \[
    \omega_i(f) = f(x_i) - f(x_{i - 1}) = M_i - m_i.
  \]
  Thus
  \[
    \sum_{i = 1}^n \omega_i(f) \Delta x_i
    = \sum_{i = 1}^n (f(x_i) - f(x_{i - 1})) \Delta x_i
    < \frac{\epsilon}{f(b) - f(a)} \sum_{i = 1}^n (f(x_i) - f(x_{i - 1})) = \epsilon
  \]
  since the sum telescopes and comes out to
  $f(b) - f(a)$. Thus $f$ is integrable.
\end{proof}

\begin{theorem}[Du Bois-Reymond]
  Let $f$ be bounded on $[a, b]$. Then $f \in \mathcal{R}([a, b])$
  if and only if for any $\epsilon, a > 0$,
  there exists a partition such that
  the total length of subintervals with
  $\omega_i(f) \ge \epsilon$ is $< a$.
\end{theorem}

\begin{proof}
  For any partition $x_0 < \dots < x_n$, split
  \[
    \sum_{i = 1}^n \omega_i(f) \Delta x_i =
    \sum_{(A)} \omega_i(f) \Delta x_i +
    \sum_{(B)} \omega_i(f) \Delta x_i
  \]
  where $(A)$ is over subintervals with width
  $\omega_i(f) < \epsilon$ and $(B)$ is over
  subintervals with width $\omega_i(f) \ge \epsilon$.

  $(\Rightarrow)$ Let
  \[
    \Omega = \sup_{x, y \in [a, b]} |f(x) - f(y)|.
  \]
  For any $\epsilon > 0$, for
  \[
    \epsilon_1 = \frac{\epsilon}{2(b - a)} \quad \text{and} \quad a = \frac{\epsilon}{2\Omega},
  \]
  by assumption there exists a partition
  $x_0 < \dots < x_n$ such that
  \begin{align*}
    \sum_{i = 1}^n \omega_i(f) \Delta x_i
    &= \sum_{(A)} \omega_i(f) \Delta x_i +
    \sum_{(B)} \omega_i(f) \Delta x_i \\
    &< \frac{\epsilon}{2(b - a)} \sum_{(a)} \Delta x_i
    + \Omega \sum_{(B)} \Delta x_i
    < \frac{\epsilon}{2(b - a)}(b - a) + \Omega \frac{\epsilon}{2\Omega} = \epsilon.
  \end{align*}
  So we see that $f \in \mathcal{R}([a, b])$ as desired.

  $(\Rightarrow)$ If $f \in \mathcal{R}([a, b])$, then
  for any $\epsilon , a > 0$, there exists a partition
  $x_0 < \dots < x_n$ such that
  \[
    \sum_{i = 1}^n \omega_i(f) \Delta x_i < a\epsilon.
  \]
  Then we have
  \[
    \epsilon \sum_{(B)} \Delta x_i
    \le \sum_{(B)} \omega_i(f) \Delta x_i
    < a\epsilon
    \implies \sum_{(B)} \Delta x_i < a,
  \]
  which shows the desired result.
\end{proof}

\begin{corollary}
  If $f : [a, b] \to \R$ is bounded and has
  only finitely many discontinuity points, then
  $f \in \mathcal{R}([a, b])$.
\end{corollary}

\begin{proof}
  Suppose $f(x)$ has $p$ discontinuity points on
  $[a, b]$ and $m \le f(x) \le M$ for all $x \in [a, b]$.
  Then for any $\epsilon > 0$, first (1) we construct $p$
  small open intervals on $[a, b]$ containing the $p$
  discontinuity points with
  \[
    \text{total length} < \frac{\epsilon}{2(M - m)}.
  \]
  Next (2) for any subintervals in $[a, b]$ excluding the above
  $p$ subintervals, $f$ is continuous on them, so
  there exists a partition such that
  \[
    \sum_{(2)} \omega_i(f) \Delta x_i < \frac{\epsilon}{2}.
  \]
  Now combine (1) and (2) to get
  \[
    \sum_{i = 1}^n \omega_i(f) \Delta
    = \sum_{(1)} \omega_i(f) \Delta x_i
    + \sum_{(2)} \omega_i(f) \Delta x_i
    < (M - m) \frac{\epsilon}{2(M - m)} + \frac{\epsilon}{2} = \epsilon.
  \]
  Thus $f \in \mathcal{R}([a, b])$, as desired.
\end{proof}

\begin{example}
  Consider
  \[
    f(x) =
    \begin{cases}
      \sin(1 / x) & \text{if } x \ne 0 \\
      A & \text{if } x = 0
    \end{cases}
  \]
  for any constant $A \in \R$. Then by the previous
  corollary, $f \in \mathcal{R}([0, 1])$.
\end{example}

\begin{theorem}
  If $f, g \in \mathcal{R}([a, b])$, then
  $fg \in \mathcal{R}([a, b])$.
\end{theorem}

\begin{proof}
  Since $f, g$ are integrable, they are bounded.
  So assume $|f|, |g| \le M$. Then for any $\epsilon > 0$,
  there exists $\delta > 0$ such that for any
  partition of width $< \delta$, we have
  \[
    \sum_{i = 1}^n \omega_i(f) \Delta x_i <
    \frac{\epsilon}{2M}, \quad
    \sum_{i = 1}^n \omega_i(g) \Delta x_i <
    \frac{\epsilon}{2M}.
  \]
  Notice
  \[
    \omega_i(fg) \le M(\omega_i(f) + \omega_i(g))
  \]
  because
  \begin{align*}
    |f(x) g(x) - f(y) g(y)
    &\le |g(x)| |f(x) - f(y)| + |f(y)| |g(x) - g(y)| \\
    &\le M(|f(x) - f(y)| + |g(x) - g(y)|).
  \end{align*}
  Taking suprememes over $x, y \in [x_{i - 1}, x_i]$ from
  here
  gives $\omega_i(fg) \le M(\omega_i(f) + \omega_i(g))$.
  Then
  \[
    \sum_{i = 1}^n \omega_i(fg) \Delta x_i
    \le M \left(\sum_{i = 1}^n \omega_i(f) \Delta x_i + \sum_{i = 1}^n \omega_i(g) \Delta x_i\right)
    < M\left(\frac{\epsilon}{2M} + \frac{\epsilon}{2M}\right) = \epsilon.
  \]
  Thus $fg \in \mathcal{R}([a, b])$ as desired.
\end{proof}

\begin{theorem}
  If $f \in \mathcal{R}([a, b])$, then
  $|f| \in \mathcal{R}([a, b])$ and
  \[
    \left| \int_a^b f(x) \, dx \right| \le \int_a^b |f(x)| \, dx.
  \]
\end{theorem}

\begin{proof}
  Since $f \in \mathcal{R}([a, b])$, for any
  $\epsilon > 0$ there exists a partition
  $x_0 < \dots < x_n$ such that
  \[
    \sum_{i = 1}^n \omega_i(f) \Delta x_i < \epsilon.
  \]
  Since
  \[
    \big||f(x)| - |f(y)|\big| \le |f(x) - f(y)|,
  \]
  taking supremums over $x, y \in [x_{i - 1}, x_i]$
  gives $\omega_i(|f|) \le \omega_i(f)$. Then
  \[
    \sum_{i = 1}^n \omega_i(|f|) \Delta x_i
    \le \sum_{i = 1}^n \omega_i(f) \Delta x_i < \epsilon.
  \]
  So we indeed have $|f| \in \mathcal{R}([a, b])$.
  Now observe that $-|f| \le f \le |f|$. After
  integrating, we get
  \[
    -\int_a^b |f(x)| \, dx \le \int_a^b f(x) \, dx \le \int_a^b |f(x)| \, dx.
  \]
  This immediately implies the desired result.
\end{proof}

\begin{example}[Cauchy-Schwarz]
  If $f, g \in \mathcal{R}([a, b])$, then
  \[
    \left| \int_a^b f(x) g(x) \, dx \right|
    \le \left( \int_a^b f(x)^2 \, dx \right)^{1/2}
    \left( \int_a^b g(x)^2 \, dx \right)^{1/2}. \tag{$*$}
  \]
\end{example}

\begin{proof}
  Let
  \[
    A = \int_a^b f^2 \, dx, \quad
    B = \int_a^b |fg| \, dx,\quad
    C = \int_a^b g^2 \, dx.
  \]
  Note that it suffices to
  show that $B^2 \le AC$, which will imply $(*)$
  by the previous theorem. Then
  \[
    0 \le \int_a^b (t|f| - |g|)^2\, dx
    = A t^2 - 2Bt + C
  \]
  for any $t \in \R$. So the discriminant must satisfy
  $(2B)^2 - 4AC \le 0$, which gives $B^2 \le AC$
  as desired.
\end{proof}

\begin{example}[Riemann-Lebesgue lemma]
  If $f \in \mathcal{R}([a, b])$, then
  \[
    \lim_{\lambda \to \infty} \int_a^b f(x) \sin(\lambda x) \, dx = 0.
  \]
\end{example}

\begin{proof}
  Since $f \in \mathcal{R}([a, b])$, for any
  $\epsilon > 0$ there exists a partition
  $x_0 < \dots < x_n$ of $[a, b]$ such that
  \[
    \sum_{i = 1}^n \omega_i(f) \Delta x_i < \frac{\epsilon}{2}.
  \]
  Also assume $|f| \le M$ on $[a, b]$ since $f$ is
  integrable. Then we choose
  \[
    \lambda > \frac{4nM}{\epsilon}.
  \]
  We can estimate
  \begin{align*}
    \left| \int_a^b f(x) \sin(\lambda x) \, dx \right|
    &= \left|\sum_{i = 1}^n \int_{x_{i - 1}}^{x_i} (f(x) - f(x_{i}) + f(x_i)) \sin(\lambda x)\, dx\right| \\
    &\le
    \sum_{i = 1}^n |f(x_i)| \left|\int_{x_{i - 1}}^{x_i} \sin(\lambda x)\, dx\right|
    + \sum_{i = 1}^n \int_{x_{i - 1}}^{x_i} \underbrace{|f(x) - f(x_i)|}_{\le \omega_i(f)} \underbrace{|\sin(\lambda x)|}_{\le 1}\, dx \\
    &\le
    M \sum_{i = 1}^n \frac{\overbrace{|\cos(\lambda x_{i}) - \cos(\lambda x_{i - 1})|}^{\le 2}}{\lambda}
    + \sum_{i = 1}^n \int_{x_{i - 1}}^{x_i} \omega_i(f)\, dx \\
    &\le M \frac{2n}{\lambda} + \sum_{i = 1}^n \omega_i(f) \Delta x_i < \frac{\epsilon}{2} + \frac{\epsilon}{2} = \epsilon.
  \end{align*}
  So as $\lambda \to \infty$, the integral goes to $0$.
\end{proof}

\begin{remark}
  Recall that
  \[
    f(x) =
    \begin{cases}
      0 & \text{if } x \text{ is irrational} \\
      1 & \text{if } x \text{ is rational}
    \end{cases}
  \]
  is not Riemann integrable, but we might expect that
  this
  should integrate to $0$. The Lebesgue integral
  will fix this, which was discovered much later.
\end{remark}
