\chapter{Apr.~18 --- Stokes's Theorem}

\section{Stokes's Theorem}

\begin{definition}
  The \emph{curl} of a vector field $\vec{v} = \langle P, Q, R \rangle$,
  denoted $\curl \vec{v}$ or $\nabla \times \vec{v}$, is
  \[
    \begin{vmatrix}
      i & j & k \\
      \partial_x & \partial_y & \partial_z \\
      P & Q & R
    \end{vmatrix}
    = \langle \partial_y R - \partial_z Q, \partial_z P - \partial_x R, \partial_x Q - \partial_y P \rangle.
  \]
\end{definition}

\begin{theorem}
  Let $B$ be a smooth surface in $\R^3$ and
  $\partial B$ consist of a finite number of continuous
  curves such that the orientation of $\partial B$
  and $\vec{n}$ satisfy the right hand rule.
  Then for $\vec{v} = \langle P, Q, R \rangle \in \C^1(B)$,
  \[
    \int_{\partial B} \vec{v} \cdot ds
    = \iint_B (\nabla \times \vec{v}) \cdot \vec{n} \, dS.
  \]
\end{theorem}

\begin{proof}
  First suppose $B$ is a graph surface, i.e.
  $z = f(x, y)$ for $(x, y) \in D$. Assume that
  $f \in C^1(D)$ and
  $\vec{v} = \langle P, 0, 0 \rangle$. Then by Green's
  theorem, we have
  \begin{align*}
    \text{LHS}
    = \int_{\partial B} P(x, y, z)\, dx
    = \int_{\partial D} P(x, y, f(x, y))\, dx
    &= - \iint_D \frac{\partial}{\partial y} P(x, y, f(x, y))\, dxdy \\
    &= -\iint_D \left(\frac{\partial P}{\partial y} + \frac{\partial P}{\partial z} \frac{\partial f}{\partial y}\right) dxdy.
  \end{align*}
  Now for the right hand side,
  \[
    \nabla \times \vec{v} = \langle 0, \partial_z P, - \partial_y P \rangle,
  \]
  so that
  \[
    \iint_B (\nabla \times \vec{v}) \cdot \vec{n} \, dS
    = \iint_D (\nabla \times \vec{v}) \cdot \langle -f_x, -f_y, 1 \rangle \, dxdy
    = \iint_D (-\partial_z P f_y - \partial_y P)\, dxdy.
  \]
  Comparing this with the LHS, we see that the two agree
  and thus the theorem holds in this case.
  Observe that applying the same argument when
  $\vec{v} = \langle 0, Q, 0 \rangle$ gives the similar
  result that
  \[
    \int_{\partial B} Q\, dy
    = \iint_B (\nabla \times \vec{v}) \cdot \vec{n} \, dS,
  \]
  where $\nabla \times \vec{v} = \langle -\partial_z Q, 0, \partial_x Q \rangle$.
  Now if $B$ is a graph surface of the form
  $x = f(y, z)$ for $(y, z) \in D$, then for
  $\vec{v}_1 = \langle 0, Q, 0 \rangle$ and
  $\vec{v}_2 = \langle 0, 0, R \rangle$, we have
  \[
    \int_{\partial B} Q\, dy
    = \iint_B (\nabla \times \vec{v}_1) \cdot \vec{n} \, dS
    \quad \text{and} \quad
    \int_{\partial B} R\, dz
    = \iint_B (\nabla \times \vec{v}_2) \cdot \vec{n} \, dS.
  \]
  Similarly if $B$ is a graph surface of the form
  $y = f(z, x)$  for $(z, x) \in D$, then for
  $\vec{v}_1 = \langle 0, 0, R \rangle$ and
  $\vec{v}_2 = \langle P, 0, 0 \rangle$, we have
  \[
    \int_{\partial B} R\, dz
    = \iint_B (\nabla \times \vec{v}_1) \cdot \vec{n} \, dS
    \quad \text{and} \quad
    \int_{\partial B} P\, dx
    = \iint_B (\nabla \times \vec{v}_2) \cdot \vec{n} \, dS.
  \]
  The proofs for these special cases are identical to
  the first.

  Now if $B$ is a triangle in $\R^3$, then $B$ is
  of all three of the previous types. So for
  $\vec{v} = \langle P, Q, R \rangle$, we have
  \[
    \int_{\partial B} P\, dx + Q\, dy + R\, dz
    = \iint_B (\nabla \times \vec{v}) \cdot \vec{n} \, dS,
  \]
  which follows by breaking $\vec{v}$ into its components
  and applying the previous cases on each component.

  For a general surface $B$, use a union of polygons
  to approximate $B$, and furthermore each polygon
  can be broken down into triangles. So we can approximate
  the surface by triangles. Then since Stokes's theorem
  holds for triangles, apply this to each triangle
  and note the internal edges cancel out. Now taking
  the limit to approach $B$ and $\partial B$, we
  get Stokes's theorem for general surfaces.
\end{proof}

\begin{remark}
  Green's theorem is a special case of Stokes's
  theorem when the surface lies in the $xy$-plane.
  Here $\vec{n} = \vec{k}$ and $\vec{v} = \langle P, Q, 0 \rangle$, so
  \[
    \nabla \times \vec{v} = (\partial_x Q - \partial_y P) \vec{k},
  \]
  which gives
  \[
    (\nabla \times \vec{v}) \cdot \vec{n} = (\partial_x Q - \partial_y P) \vec{k} \cdot \vec{k} = \partial_x Q - \partial_y P,
  \]
  which is precisely the integrand in the statement of
  Green's theorem.
\end{remark}

\section{The Generalized Stokes's Theorem}
The generalized Stokes's theorem is a generalization of
Stokes's theorem to \emph{differential forms} on \emph{manifolds}.
Letting $\omega = P\, dx + Q\, dy + R\, dz$ be a
$1$-form on $\R^3$, we can write
\[
  \int_{\partial B} \omega = \int_B d\omega,
\]
where $d \omega$ is the \emph{exterior derivative} of $\omega$, given by
\begin{align*}
  d\omega
  &= \left(\frac{\partial P}{\partial x}\, dx + \frac{\partial P}{\partial y}\, dy + \frac{\partial P}{\partial z}\, dz\right) \land dx
  + \left(\frac{\partial Q}{\partial x}\, dx + \frac{\partial Q}{\partial y}\, dy + \frac{\partial Q}{\partial z}\, dz\right) \land dy \\
  &\hspace{16em} + \left(\frac{\partial R}{\partial x}\, dx + \frac{\partial R}{\partial y}\, dy + \frac{\partial R}{\partial z}\, dz\right) \land dz \\
  &= \left(-\frac{\partial P}{\partial y} + \frac{\partial Q}{\partial x}\right) dx \land dy
  + \left(\frac{\partial P}{\partial z} - \frac{\partial R}{\partial x}\right) dz \land dx
  + \left(\frac{\partial Q}{\partial z} - \frac{\partial R}{\partial y}\right) dz \land dy.
\end{align*}
Note that the \emph{wedge product} $\land$ is
alternating and multilinear, so that
$dx \land dx = 0$ and $dx \land dy = -dy \land dx$.
Here $d\omega$ is a $2$-form,\footnote{In general, if $\omega$ is a $k$-form, then $d\omega$ is a $(k + 1)$-form.} and we can integrate
$2$-forms by writing
\[
  u_1\, dy \land dz + u_2\, dz \land dx + u_3 \, dx \land dy
  \implies \vec{u} = \langle u_1, u_2, u_3 \rangle
  = \nabla \times \vec{v}
\]
where $\vec{v} = \langle P, Q, R \rangle$. Using this,
we see that
\[
  \int_B d\omega = \iint_B (\nabla \times \vec{v}) \cdot \vec{n} \, dS,
\]
which recalls the familiar form of Stokes's theorem.

Recall the divergence theorem, which says that
\[
  \iiint_D \nabla \cdot \vec{v} \, dV = \iint_{\partial D} \vec{v} \cdot \vec{n} \, dS.
\]
Here we have the $2$-form
\[
  \omega = v_1\, dy \land dz + v_2\, dz \land dx + v_3 \, dx \land dy
\]
on $\partial D$. Taking the exterior derivative
of $\omega$ gives the $3$-form
\begin{align*}
  d\omega
  &= \left(\frac{\partial v_1}{\partial x}\, dx + \frac{\partial v_1}{\partial y}\, dy + \frac{\partial v_1}{\partial z}\, dz\right) \land dy \land dz.
  + \left(\frac{\partial v_2}{\partial x}\, dx + \frac{\partial v_2}{\partial y}\, dy + \frac{\partial v_2}{\partial z}\, dz\right) \land dz \land dx \\
  &\hspace{19em}+ \left(\frac{\partial v_3}{\partial x}\, dx + \frac{\partial v_3}{\partial y}\, dy + \frac{\partial v_3}{\partial z}\, dz\right) \land dx \land dy \\
  &= \frac{\partial v_1}{\partial x} dx \land dy \land dz
  + \frac{\partial v_2}{\partial y} dy \land dz \land dx
  + \frac{\partial v_3}{\partial z} dz \land dx \land dy
  = \left(\frac{\partial v_1}{\partial x} + \frac{\partial v_2}{\partial y} + \frac{\partial v_3}{\partial z}\right) dx \land dy \land dz,
\end{align*}
which recovers the divergence of $\vec{v}$. So the
divergence theorem is a special case of the generalized
Stokes's theorem.

In general, for a $(k - 1)$-form $\omega$
on a $k$-manifold $B$, the generalized Stokes's theorem
asserts that
\[
  \int_B d\omega = \int_{\partial B} \omega.
\]
All three of the vector calculus theorems are special
cases of this.
