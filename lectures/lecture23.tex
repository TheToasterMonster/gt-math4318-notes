\chapter{Apr.~9 --- Change of Variables, Part 2}

\section{Change of Variables Examples}
\begin{example}
  Compute
  \[
    I = \iint_D \sqrt{\sqrt{x} + \sqrt{y}}\, dxdy
  \]
  where $D$ is enclosed by $C : \sqrt{x} + \sqrt{y} = 1$
  and the coordinate axes.
\end{example}

\begin{proof}
  We would like $C : x = \cos^4 t, y = \sin^4 t$
  for $0 \le t \le \pi / 2$. So let
  \[
  T :
  \begin{cases}
    x = r \cos^4 t, & 0 \le r \le 1\\
    y = r \sin^4 t, & 0 \le t \le \pi / 2.
  \end{cases}
  \]
  This gives $T : [0, 1] \times [0, \pi / 2] \to D$
  which sends $(r, t) \to (x, y)$. Then
  \[
    J(r, t) = \det
    \begin{pmatrix}
      \partial x / \partial r & \partial x / \partial t\\
      \partial y / \partial r & \partial y / \partial t
    \end{pmatrix}
    = 4r \cos^3 t \sin^3 t.
  \]
  Now making the change of variables in the integral
  gives
  \[
    I = \int_0^{\pi / 2} \int_0^1 r^{1 / 4} 4r \cos^3 t \sin^3 t\, drdt
    = \int_0^1 4 r^{5 / 4}\, dr \int_0^{\pi / 2} \cos^3 t \sin^3 t\, dt
    = \frac{2}{15}.
  \]
  The latter integral can be calculated by writing
  $\sin^2 t = 1 - \cos^2 t$ and using the substitution
  $u = \cos t$.
\end{proof}

\begin{example}
  Let $h = \sqrt{\alpha^2 + \beta^2 + \gamma^2} > 0$
  and $f$ be continuous on $[-h, h]$. Show that
  \[
    \iiint_V f(\alpha x + \beta y + \gamma z)\, dx dy dz
    = \pi \int_{-1}^1 (1 - \zeta^2) f(h \zeta)\, d\zeta,
  \]
  where $V : x^2 + y^2 + z^2 \le 1$.
\end{example}

\begin{proof}
  Setting $\alpha x + \beta y + \gamma z = 0$ defines
  a plane passing through the origin in $\R^2$. Choose
  the $\zeta$-axis to be normal to this plane, and let
  $\xi, \eta$ be two orthogonal axes on the plane. Define
  the coordinate transformation
  $T^{-1} : (\xi, \eta, \zeta) \to (x, y, z)$ be
  given by
  \[
    \begin{cases}
      \xi = a_1 x + b_1 y + c_1 z \\
      \eta = a_2 x + b_2 y + c_2 z \\
      \zeta = (\alpha x + \beta y + \gamma z) / h,
    \end{cases}
  \]
  where $\{(a_1, b_1, c_1)^T, (a_2, b_2, c_2)^T\}$ is an
  orthonormal basis for the plane. Note that $(\alpha, \beta, \gamma)^T$ is the normal vector of the plane.
  Then
  \[
    \frac{\partial(\xi, \eta, \zeta)}{\partial(x, y, z)}
    = \det
    \begin{pmatrix}
      a_1 & b_1 & c_1 \\
      a_2 & b_2 & c_2 \\
      \alpha / h & \beta / h & \gamma / h
    \end{pmatrix} = 1
  \]
  since this matrix is orthogonal and preserves
  orientation by construction. This gives
  \[
    J(\xi, \eta, \zeta) =
    \frac{\partial(x, y, z)}{\partial(\xi, \eta, \zeta)}
    = \left(\frac{\partial(\xi, \eta, \zeta)}{\partial(x, y, z)}\right)^{-1} = 1.
  \]
  Now under the transformation $T^{-1}$, we have
  \[
    V : \{x^2 + y^2 + z^2 \le 1\} \xrightarrow{T^{-1}}
    G : \{\xi^2 + \eta^2 + \zeta^2 \le 1\}.
  \]
  Substituting this into the integral, we get
  \begin{align*}
    \iiint_V f(\alpha x + \beta y + \gamma z)\, dx dy dz
    &= \iiint_{\xi^2 + \eta^2 + \zeta^2 \le 1} f(h \zeta)\, d\xi d\eta d\zeta \\
    &= \int_{-1}^1 d\zeta \iint_{\xi^2 + \eta^2 \le 1 - \zeta^2} f(h \zeta)\, d\xi d\eta
    = \int_{-1}^1 f(h \zeta) \pi (1 - \zeta^2)\, d\zeta,
  \end{align*}
  which is the desired result.
\end{proof}

\section{The Cylindrical and Spherical Coordinate Systems}
The \emph{cylindrical} coordinate system is the change of
variables $(x, y, z) \to (r, \theta, z)$ given by
\[
  \begin{cases}
    x = r \cos \theta \\
    y = r \sin \theta \\
    z = z
  \end{cases}
  \text{with } J(r, \theta, z) = r.
\]
The \emph{spherical} coordinate system is the change of
variables $(x, y, z) \to (r, \varphi, \theta)$ given by
\[
\begin{cases}
  x = r \sin \varphi \cos \theta \\
  y = r \sin \varphi \sin \theta \\
  z = r \cos \varphi
\end{cases}
\text{with } J(r, \varphi, \theta) = r^2 \sin \varphi.
\]
For spherical coordinates we have
$r \ge 0$, $0 \le \varphi \le \pi$, and
$0 \le \theta \le 2\pi$.

\begin{example}
  Find
  \[
    \iiint_V (x^2 + y^2 + z^2)\, dx dy dz
  \]
  where $V : x^2 / a^2 + y^2 / b^2 + z^2 / c^2 \le 1$.
\end{example}

\begin{proof}
  Define the transformation
  \[
    T :
    \begin{cases}
      x = a r \sin \varphi \cos \theta \\
      y = b r \sin \varphi \sin \theta \\
      z = c r \cos \varphi.
    \end{cases}
  \]
  Then we have
  \[
    \frac{\partial(x, y, z)}{\partial(r, \varphi, \theta)}
    = abc r^2 \sin \varphi.
  \]
  Making the substitution, this gives
  \begin{align*}
    &\iiint_V (x^2 + y^2 + z^2)\, dx dy dz \\
    &\quad = \int_0^{2\pi} d\theta \int_0^\pi d\varphi \int_0^1 (a^2 \sin^2 \varphi \cos^2 \theta + b^2 \sin^2 \varphi \sin^2 \theta + c^2 \cos^2 \varphi) abc r^4 \sin \varphi\, dr \\
    &\quad = \frac{4}{15} abc(a^2 + b^2 + c^2) \pi
  \end{align*}
  as the desired result.
\end{proof}

\section{More Change of Variables Examples}
\begin{example}
  Find
  \[
    I = \int_0^\infty \frac{e^{-ax} - e^{-bx}}{x}\, dx
  \]
  for $a, b > 0$.
\end{example}

\begin{proof}
  First observe that for fixed $x > 0$, we can write
  \[
    \frac{e^{-ax} - e^{-bx}}{x} = \left.-\frac{e^{-yx}}{x}\right|_{y = a}^{y = b}
      = \int_a^b \frac{d}{dy} (-\frac{e^{-yx}}{x})\, dy
      = \int_a^b e^{-xy}\, dy
  \]
  by the fundamental theorem of calculus.
  This then gives
  \[
    I = \int_0^\infty dx \int_a^b e^{-xy}\, dy
    = \lim_{T \to \infty} \int_0^T dx \int_a^b e^{-xy}\, dy
  \]
  For the latter integral, we can switch the order of
  integration to get
  \begin{align*}
    \int_0^T dx \int_a^b e^{-xy}\, dy
    &= \int_a^b dy \int_0^T e^{-xy}\, dx
    = \int_a^b dy \left[-\frac{e^{-xy}}{y}\right]_{x = 0}^{x = T}
    = \int_a^b \frac{1 - e^{-Ty}}{y}\, dy \\
    &= \int_a^b \frac{1}{y}\, dy - \int_a^b \frac{e^{-Ty}}{y}\, dy
    = \ln(b / a) - \int_{Ta}{Tb} \frac{e^{-Ty'}}{y'}\, dy',
  \end{align*}
  where $y' = Ty$. Letting $T \to \infty$, we have
  \[
    I = \lim_{T \to \infty} \int_0^T dx \int_a^b e^{-xy}\, dy
    = \ln(b / a) - \lim_{T \to \infty} \int_{Ta}^{Tb} \frac{e^{-Ty'}}{y'}\, dy'
    = \ln(b / a) \tag{$*$}
  \]
  since
  \[
    \int_1^\infty \frac{e^{-y}}{y}\, dy
    \text{ converges} \quad \text{and} \quad
    h(A) = \int_0^A \frac{e^{-y}}{y}\, dy \text{ is Cauchy in $A$},
  \]
  so that the latter integral in $(*)$ vanishes as
  $T \to \infty$.
\end{proof}

\begin{remark}
  We might have been able to just switch orders at the
  very beginning with the improper integral, but we need
  to argue uniform
  convergence so that the exchange is permissible.
\end{remark}

\begin{exercise}
  Let $f$ be differentiable on $(0, \infty)$ and suppose
  that
  \[
    \int_1^\infty \frac{f(t)}{t}\, dt
  \]
  exists. Then for $a, b > 0$, show that
  \[
    I = \int_0^\infty \frac{f(ax) - f(bx)}{x}\, dx
    = f(0) \ln(b / a).
  \]
\end{exercise}

\begin{remark}
  The previous example is a special case of this exercise,
  with $f(x) = e^{-x}$.
\end{remark}
