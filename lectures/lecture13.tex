\chapter{Feb.~22 --- Power Series}

\section{Power Series}

\begin{definition}
  Given a point $x_0$, a \emph{power series} around
  $x_0$ is a series of the form
  \[
    \sum_{n=0}^\infty a_n (x - x_0)^n
    = a_0 + a_1(x - x_0) + a_2(x - x_0)^2 + \dots.
  \]
\end{definition}

\begin{remark}
  The question is: For what $x$ does the series converge?
\end{remark}

\begin{example}
  Consider the series\footnote{Later we will see that this is the Taylor series for the exponential function $e^x$.}
  \[
    \sum_{n = 0}^\infty \frac{x^n}{n!}.
  \]
  In fact this series converges for all $x \in \R$. Let
  $a_n = x^n / n!$. By
  the ratio test, for $x \ne 0$, we have
  \[
    \lim_{n \to \infty} \left|\frac{a_{n + 1}}{a_n}\right|
    = \lim_{n \to \infty} \frac{|x|}{n + 1} = 0,
  \]
  so the series converges at $x \ne 0$. Of course, if
  $x = 0$, then every term past $n = 0$ is zero, so the
  series also converges there. Thus the series
  converges everywhere on $\R$.
\end{example}

\begin{example}
  The series
  \[
    \sum_{n = 0}^\infty n! x^n
  \]
  converges only at $x = 0$. For $x \ne 0$, we
  can similarly apply the ratio test to find that
  \[
    \lim_{n \to \infty} \left|\frac{a_{n + 1}}{a_n}\right|
    = \lim_{n \to \infty} |n + 1| |x| = \infty,
  \]
  so the series diverges for $x \ne 0$.
\end{example}

\begin{example}
  Recall that the geometric series
  \[
    \sum_{n = 0}^\infty x^n
  \]
  converges if and only if $|x| < 1$.
\end{example}

\section{Radius of Convergence}
\begin{lemma}
  \label{lem:converge-at}
  We have the following:
  \begin{enumerate}
    \item If $(A) = \sum_{n = 0}^\infty a_n x^n$ converges
      at $x = x_1 \ne 0$, then it converges absolutely
      for all $x$ with $|x| < |x_1|$.
    \item If $(A)$ diverges at $x = x_2 \ne 0$, then it
      diverges for all $x$ with $|x| > |x_2|$.
  \end{enumerate}
\end{lemma}

\begin{proof}
  (1) If $\sum_{n = 0}^\infty a_n x_1^n$ converges, then
  $a_n x_1^n \to 0$ as $n \to \infty$. In particular,
  the terms are bounded, so there exists $M > 0$
  such that $|a_n x_1^n| \le M$ for every $n \in \N$.
  For any $|x| < |x_1|$, let
  \[
    q = \left|\frac{x}{x_1}\right| < 1.
  \]
  Then we have
  \[
    |a_n x^n| = \left|a_n x_1^n \left(\frac{x}{x_1}\right)^n\right|
    \le Mq^n.
  \]
  Comparing with the geometric series, we get that
  $\sum_{n = 0}^\infty a_n x^n$ converges absolutely
  when $|x| < |x_1|$.

  (2) Suppose otherwise that there exists $x_3$ such that
  $|x_3| > |x_2|$ and $\sum_{n = 0}^\infty a_n x_3^n$
  converges. Then by (1), we see that the power series
  converges at $x = x_2$. Contradiction.
\end{proof}

\begin{corollary}
  \label{cor:radius-of-convergence}
  If $(A) = \sum_{n = 0}^\infty a_n x^n$ converges at
  some $x_1 \ne 0$ and diverges at $x_2 \ne 0$, then
  there exists $R > 0$ such that $(A)$ converges
  for $|x| < R$ and diverges for $|x| > R$.
\end{corollary}

\begin{proof}
  Let $E$ be the set of all convergence points of $(A)$.
  By Lemma \ref{lem:converge-at}, we have
  $E \subseteq \{|x| \le x_2\}$ since $(A)$ diverges
  for all $|x| > |x_2|$. Let
  \[R = \sup \{|x| : x \in E\},\]
  which exists since
  $E$ is nonempty and bounded above by $|x_2|$.
  Also $R > 0$ since $x_1 \in E$ and $x_1 \ne 0$. Now if
  $|x| < R$, then
  there exists $x_1 \in E$ such that
  $|x| < |x_1| < R$ and $x = x_1$ is a convergence point.\footnote{This is by definition of the supremum.}
  By Lemma \ref{lem:converge-at}, we get that
  $(A)$ converges at $x$. If $|x| > R$, then there
  exists $x_2$ such that $|x| > |x_2| > R$ and
  $(A)$ diverges at $x = x_2$. Then by Lemma
  \ref{lem:converge-at}, $(A)$ diverges at $x$.
\end{proof}

\begin{remark}
  The $R$ in Corollary \ref{cor:radius-of-convergence}
  is called the \emph{radius of convergence} of the
  power series. If $(A) = \sum_{n = 0}^\infty a_n x^n$
  converges for all $x \in \R$, then by convention
  we use $R = \infty$. If $(A)$ converges only
  at $x = 0$, we use $R = 0$. At $x = \pm R$, the
  convergence or divergence of $(A)$ needs to be checked
  separately.
\end{remark}

\begin{theorem}
  For a power series $\sum_{n = 0}^\infty a_n x^n$, let
  \[
    \limsup_{n \to \infty} \sqrt[n]{|a_n|} = \rho.
  \]
  Then
  \begin{enumerate}
    \item if $\rho = 0$, then $R = \infty$,
    \item if $\rho = \infty$, then $R = 0$,
    \item and if $\rho \in (0, \infty)$, then $R = 1 / \rho$.
  \end{enumerate}
\end{theorem}

\begin{proof}
  We use the root test for
  $\sum_{n = 0}^\infty |a_n x^n|$. Let $A_n = |a_n x^n|$.
  Then
  \[
    \limsup_{n \to \infty} \sqrt[n]{A_n}
    = \limsup_{n \to \infty} \sqrt[n]{|a_n|} \cdot |x|
    = \rho |x|.
  \]
  First suppose $0 < \rho < \infty$. Then $\sum_{n = 0}^\infty |a_n x^n|$
  converges if $|x| \rho < 1$ and diverges if
  $|x| \rho > 1$. This gives $R = 1 / \rho$. Now
  if $\rho = 0$, then
  \[
    \limsup_{n \to \infty} \sqrt[n]{A_n} = 0 < 1,
  \]
  so $\sum_{n = 0}^\infty |a_n x^n|$ regardless of the
  choice of $x$, i.e. $R = \infty$. Finally if $\rho = \infty$, then
  for $x_0 \ne 0$,
  \[
    \limsup_{n \to \infty} \sqrt[n]{A_n} = \infty > 1.
  \]
  By the root test, this implies that $\sum_{n = 0}^\infty |a_n x^n|$
  diverges for all $x \ne 0$, i.e. $R = 0$.
\end{proof}

\begin{corollary}
  For $\sum_{n = 0}^\infty a_n x^n$ with $a_n \ne 0$,
  if
  \[
    \lim_{n \to \infty} \left|\frac{a_{n + 1}}{a_n}\right|
    = \rho,
  \]
  then the radius of convergence is $R = 1 / \rho$.
\end{corollary}

\begin{proof}
  Left as an exercise.
\end{proof}

\begin{example}
  Find the convergence intervals for
  \begin{enumerate}
    \item $\displaystyle \sum_{n = 1}^\infty \frac{2^n(x + 1)^n}{n \ln^2(n + 1)}$,
    \item and $\displaystyle \sum_{n = 1}^\infty n^n x^{n^2}$.
  \end{enumerate}
\end{example}

\begin{proof}
  (1) Let $a_n$ be the summand. Then
  \[
    \lim_{n \to \infty} \left|\frac{a_{n + 1}}{a_n}\right|
    = \lim_{n \to \infty} \frac{2n \ln^2(n + 1)}{(n + 1)\ln^2(n + 2)}
    = 2,
  \]
  so $R = 1 / 2$. So we get convergence for
  $|x + 1| < 1 / 2$, i.e. $x \in (-3 / 2, -1 / 2)$.
  At $x = -3 / 2, -1 / 2$, we get
  \[
    \sum_{n = 1}^\infty \frac{1}{n \ln^2(n + 1)}
  \]
  after taking absolute values, which converges by
  the integral test. So the interval
  is $[-3 / 2, -1 / 2]$.

  (2) Let the general term be
  \[
    a_k =
    \begin{cases}
      n^n & \text{if $k = n^2$ for some $n \in \N$},\\
      0 & \text{otherwise}.
    \end{cases}
  \]
  Then we see that
  \[
    \limsup_{k \to \infty} \sqrt[k]{|a_k|}
    = \lim_{n \to \infty} \sqrt[n^2]{n^n}
    = \lim_{n \to \infty} \sqrt[n]{n} = 1,
  \]
  so $R = 1$. At $x = \pm 1$, the series diverges since
  the general term
  $n^n x^{n^2}$ does not go to $0$ when $|x| = 1$. So
  we conclude that the interval of convergence is
  $(-1, 1)$.
\end{proof}

\begin{theorem}
  If $(A) = \sum_{n = 0}^\infty a_n x^n$ has radius of
  convergence $R > 0$ (including $R = \infty$), then
  for any $0 < r < R$, the power series $(A)$ converges
  uniformly on $[-r, r]$. Moreover, if $(A)$ converges
  at $x = R < \infty$ (or $x = -R$), then $(A)$
  converges uniformly on $[0, R]$ (or $[-R, 0]$).
\end{theorem}

\begin{proof}
  For $x \in [-r, r]$, we have $|a_n x^n| \le |a_n| r^n$,
  and $\sum_{n = 0}^\infty |a_n| r^n$ converges since
  $r < R$. So by the Weierstrass $M$-test, we get
  uniform convergence on $[-r, r]$. Second part left
  as an exercise.
\end{proof}

\begin{corollary}
  \label{thm:power-series-analytic}
  For a power series $(A) = \sum_{n = 0}^\infty a_n x^n$,
  we have the following:
  \begin{enumerate}
    \item if $(A)$ has
      radius of convergence $R > 0$, then
      $f(x) = \sum_{n = 0}^\infty a_n x^n$ is continuous
      on $(-R, R)$,
    \item $f(x)$ is differentiable on $(-R, R)$,
    \item and $f(x) \in C^\infty(-R, R)$, i.e. it is
      infinitely differentiable.
  \end{enumerate}
\end{corollary}

\begin{proof}
  (1) We get
  \[
    \lim_{x \to x_0} f(x)
    = \lim_{x \to x_0} \sum_{n = 0}^\infty a_n x^n
    = \sum_{n = 0}^\infty \lim_{x \to x_0} a_n x^n
    = f(x_0)
  \]
  since we have uniform convergence.

  (2) One can verify that $\sum_{n = 0}^\infty a_n n x^{n - 1}$
  also has radius of convergence $R$. In particular, the
  derivative series also converges uniformly, so
  by Theorem \ref{thm:exchange-limit-derivative}, we can
  differentiate term by term.

  (3) We can repeat (2) as many times as we want.
\end{proof}

\begin{theorem}
  Suppose $f(x) = \sum_{n = 0}^\infty a_n x^n$ has radius
  of convergence $R > 0$. Then for any $x \in (-R, R)$,
  we have $f \in \mathcal{R}([0, x])$ and
  \[
    \int_0^x f(t)\, dt = \int_0^x \sum_{n = 0}^\infty a_n t^n\, dt
    = \sum_{n = 0}^\infty a_n \int_0^x t^n\, dt
    = \sum_{n = 0}^\infty \frac{a_n}{n + 1} x^{n + 1}.
  \]
\end{theorem}

\begin{proof}
  As $\sum_{n = 0}^\infty a_n x^n$ converges
  uniformly in $[-r, r]$, by Theorem \ref{thm:exchange-limit-integral}
  we can integrate term by term.
\end{proof}

\begin{example}
  Show that
  \[
    \sum_{n = 1}^\infty \frac{x^n}{n} = -\ln(1 - x)
  \]
  for $-1 < x < 1$.
\end{example}

\begin{proof}
  By the previous theorem, we can write
  \[
    \sum_{n = 1}^\infty \frac{x^n}{n}
    = \int_0^x \sum_{n = 1}^\infty t^{n - 1}\, dt
    = \int_0^x \frac{1}{1 - t} \, dt
    = -\ln(1 - x),
  \]
  as desired. Note that we used the geometric
  series in the second step.
\end{proof}

\begin{example}
  Find the sum
  \[
    \sum_{n = 0}^\infty \frac{(-1)^n}{3n + 1}.
  \]
\end{example}

\begin{proof}
  Let
  \[
    S(x) = \sum_{n = 0}^\infty \frac{(-1)^n}{3n + 1} x^{3n + 1}.
  \]
  In particular, $S(1)$ is the desired sum, so $S$
  converges at $x = 1$
  by the alternating series test. Then
  \begin{align*}
    S(1)
    &= \sum_{n = 0}^\infty (-1)^n \int_0^1 x^{3n}\, dx
    = \int_0^1 \sum_{n = 0}^\infty (-x^3)^n \, dt
    = \int_0^1 \frac{1}{1 + x^3}\, dx \\
    &= \frac{1}{3} \left[\ln \frac{1 + x}{\sqrt{1 - x + x^2}} + \sqrt{3} \arctan \frac{2x - 1}{\sqrt{3}}\right]_{x = 0}^{x = 1}
    = \frac{1}{3} \left(\ln 2 + \frac{\pi}{\sqrt{3}}\right)
  \end{align*}
  since we have uniform convergence on $[0, 1]$.
\end{proof}
