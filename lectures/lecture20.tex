\chapter{Mar.~26 --- Extending the Integral in \texorpdfstring{$\R^n$}{Rn}}

\section{Extending the Riemann Integral in \texorpdfstring{$\R^n$}{Rn}}
\begin{definition}
  If $f$ is defined on an arbitrary set
  $A \subseteq \R^n$, define $\overline{f} : \R^n \to \R$
  by zero extension, i.e.
  \[
    \overline{f}(x) = \begin{cases}
      f(x) & \text{if } x \in A, \\
      0 & \text{if } x \notin A.
    \end{cases}
  \]
  If the support of $f$ is a bounded subset of $\R^n$,
  then we define the \emph{integral of $f$ over $A$} as
  \[
    \int_A f = \int_{\R^n} \overline{f}.
  \]
\end{definition}

\begin{remark}
  We note once again that there exist functions which
  are not Riemann integrable on $\R^n$.
  Let $A$ be the points in a closed interval of
  $\R^n$ with all rational coordinates.
  Then $f = 1$ is not integrable on $A$.
\end{remark}

\begin{remark}
  If $A$ is bounded and $f = 1$, then
  \[
    \int_A f = \int_A 1 = \Vol(A)
  \]
  if $1$ is integrable on $A$. This volume is also called
  the \emph{Jordan measure} of $A$. The previous remark
  shows that not every set is Jordan measurable.
\end{remark}

\begin{prop}
  We have the following:
  \begin{enumerate}
    \item Let $A \subseteq \R^n$ and $f, g$ be two
      integrable functions on $A$. Then $f + g$ is
      also integrable and
      \[
        \int_A (f + g) = \int_A f + \int_A g.
      \]
    \item If $f$ is integrable on $A$ and $c \in \R$,
      then $cf$ is integrable on $A$ and
      \[
        \int_A cf = c \int_A f.
      \]
    \item If $f(x) \ge 0$ on $A$ and $f$ is integrable,
      then
      \[
        \int_A f \ge 0.
      \]
    \item If $f(x) \le g(x)$ on $A$ and $f, g$ are
      integrable, then
      \[
        \int_A f \le \int_A g.
      \]
    \item If $m \le f(x) \le M$ on $A$ and $A$ has a
      volume, and $f$ is integrable on $A$, then
      \[
      m \Vol(A) \le \int_A f \le M \Vol(A)
      .\]
  \end{enumerate}
\end{prop}

\begin{proof}
  Left as an exercise. Note that the proof in the case
  where $A$ is a box is already done.
\end{proof}

\section{Sets of Measure Zero}

\begin{prop}
  We have the following:
  \begin{enumerate}
    \item A set $A \subseteq \R^n$ has zero volume if and
      only if
      for any $\epsilon > 0$, there exist a finite number
      of closed intervals in $\R^n$ containing $A$ with
      the sum of their volumes less than $\epsilon$.
    \item Any subset of a subset of $\R^n$ of zero volume
      is of zero volume.
    \item If $A \subseteq \R^n$ has zero volume and
      $B \subseteq \R^n$ has volume, then
      \[
        \Vol(A \cup B) = \Vol(B)
        \quad \text{and} \quad
        \Vol(B \setminus A) = \Vol(B).
      \]
    \item The union of a finite number of zero volume
      sets is of zero volume.
    \item If $A$ has zero volume and $f : A \to \R$ is
      bounded, then $f$ is integrable on $A$ and
      \[
        \int_A f = 0
      .\]
    \item If $S \subseteq \R^{n - 1}$ is compact and
      $f : S \to \R$, then the graph of $f$ in
      $\R^n$, i.e. the set
      \[
        \{(x_1, \dots, x_n) \in \R^n : (x_1, \dots, x_{n - 1}) \in S,\, x_n = f(x_1, \dots, x_{n - 1})\},
      \]
      is of zero volume.
  \end{enumerate}
\end{prop}

\begin{proof}
  See the textbook (Rosenlicht).
\end{proof}

\begin{prop}
  Let $A, B$ be two subsets of $\R^n$ such that
  $\Vol(A \cap B) = 0$ and $f : A \cup B \to \R$ is
  integrable on both $A$ and $B$. Then
  \[
    \int_{A \cup B} f = \int_A f + \int_B f.
  \]
\end{prop}

\begin{proof}
  Define $f_1, f_2, f_3 : \R^n \to \R$ by
  \[
    f_1(x) = \begin{cases}
      f(x) & \text{if } x \in A, \\
      0 & \text{if } x \notin A,
    \end{cases}
    \quad
    f_2(x) = \begin{cases}
      f(x) & \text{if } x \in B, \\
      0 & \text{if } x \notin B,
    \end{cases}
    \quad \text{and} \quad
    f_3(x) = \begin{cases}
      f(x) & \text{if } x \in A \cap B, \\
      0 & \text{if } x \notin A \cap B.
    \end{cases}
  \]
  Then
  \[
    \int_{\R^n} f_1 = \int_A f
    \quad \text{and} \quad
    \int_{\R^n} f_2 = \int_B f. \tag{1}
  \]
  Since $f$ is integrable on $A$ and $B$, it must be
  bounded on $A$ and $B$. Then
  \[
    \int_{\R^n} f_3 = \int_{A \cap B} f = 0 \tag{2}
  \]
  since $\Vol(A \cap B) = 0$ and $f$ is bounded on
  $A \cap B$. Now if $x \in A \cup B$, then
  $f(x) = f_1 + f_2 - f_3$, and if
  $x \notin A \cup B$, then $f_1 + f_2 - f_3 = 0$.
  Then we get that
   \[
     \int_{A \cup B} f = \int_{\R^n} (f_1 + f_2 - f_3)
     = \int_{\R^n} f_1 + \int_{\R^n} f_2 - \int_{\R^n} f_3
     = \int_A f + \int_B f
  \]
  from $(1)$ and $(2)$, as required.
\end{proof}

\begin{corollary}
  If $A$ and $B$ have volume and $A \cap B$ has
  zero volume, then
  \[\Vol(A \cup B) = \Vol(A) + \Vol(B).\]
\end{corollary}

\begin{proof}
  Choose $f = 1$ in the previous proposition.
\end{proof}

\begin{theorem}[Lebesgue's criterion for Riemann integrability]
  Let $A \subseteq \R^n$ be a set with volume and
  let $f : A \to \R$ be a bounded function that is
  continuous except on a subset of $A$ with zero volume.
  Then $f$ is integrable on $A$.
\end{theorem}

\begin{proof}
  First we consider the case where $A$ is a closed
  interval in $\R^n$ and $f$ is continuous on $A$.
  Now $f$ is continuous on a compact set, so it is
  bounded, i.e. there exists $M \in \R$ such that
  $|f(x)| \le M$ on $A$. Also since $A$ is compact,
  in fact $f$ is uniformly continuous on $A$, so
  for any $\epsilon > 0$, there exists $\delta > 0$
  such that $|f(x) - f(y)| < \epsilon$ if
  $d(x, y) < \delta$. Choose a partition of $A = I$
  into closed subintervals $I_1, \dots, I_N$ such that
  $I = I_1 \cup I_2 \cup \dots \cup I_N$ and
  $\Vol(I_i \cap I_j) = 0$,\footnote{Cutting the box into little rectangles is sufficient to do this, for example.} and $d(x, y) < \delta$
  if $x, y \in I_j$. Define
  \[
    f_1(x) =
    \begin{cases}
      \min\{f(y) : y \in I_j\} & \text{if } x \in I_j \text{ and } x \notin I_k \text{ for } k \ne j \\
      -M & \text{otherwise}.
    \end{cases}
  \]
  Similarly define
  \[
    f_2(x) =
    \begin{cases}
      \max\{f(y) : y \in I_j\} & \text{if } x \in I_j \text{ and } x \notin I_k \text{ for } k \ne j \\
      M & \text{otherwise}.
    \end{cases}
  \]
  By construction we have $f_1 \le f \le f_2$, and\footnote{Here $\Int I_j$ denotes the \emph{interior} of $I_j$.}
  \[
    \int_I (f_2 - f_1)
    = \sum_{j = 1}^N \int_{\Int I_j} (f_2 - f_1)
    \le \sum_{j = 1}^N \epsilon \Vol(I_j)
    = \epsilon \Vol(I)
  \]
  by uniform continuity.
  So $f$ is integrable on $A$ in
  this case.
  Rest of the proof for next class.
\end{proof}
