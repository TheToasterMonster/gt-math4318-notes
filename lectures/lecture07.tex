\chapter{Jan.~30 --- Riemann Integrability, Part 2}

\section{Conditions for an Anti-Derivative}
\begin{lemma}
  \label{lem:split-integral}
  Let $c \in (a, b)$. Then $f \in \mathcal{R}([a, b])$ if and only if $f \in \mathcal{R}([a, c])$ and $f \in \mathcal{R}([c, b])$. Moreover,
  \[
    \int_a^b f(x) \, dx = \int_a^c f(x) \, dx + \int_c^b f(x) \, dx. \tag{$*$}
  \]
\end{lemma}

\begin{proof}
  $(\Rightarrow)$ If $f \in \mathcal{R}([a, b])$, then
  for any $\epsilon > 0$, there exist two step
  functions $f_1, f_2$ such that $f_1 \le f \le f_2$
  and
  \[
    \int_a^b (f_2 - f_1) \, dx < \epsilon.
  \]
  Let $f_1, f_2$ be the restrictions to $[a, c]$.
  Then still $f_1 \le f \le f_2$ on $[a, c]$ and
  \[
    \int_a^c (f_2 - f_1)\, \le
    \int_a^b (f_2 - f_1) < \epsilon
  \]
  since $f_2 - f_1$ is a nonnegative step function. (Note
  that the desired result is easy to verify for step
  functions.)
  So $f \in \mathcal{R}([a, c])$, and the same
  argument works to show that $f \in \mathcal{R}([c, b])$.

  $(\Leftarrow)$ If $f \in \mathcal{R}([a, c])$ and
  $f \in \mathcal{R}([c, b])$, then for any
  $\epsilon > 0$, there exist step functions
  $g_1, g_2, h_1, h_2$ such that
  $g_1 \le f \le g_2$ on $[a, c]$,
  $h_1 \le f \le h_2$ on $[c, b]$, and
  \[
    \int_a^c (g_2 - g_1)\, dx < \epsilon,
    \quad
    \int_c^b (h_2 - h_1)\, dx < \epsilon.
  \]
  Now define
  \[
    f_i = \begin{cases}
      g_i & \text{if } x \in [a, c) \\
      h_i & \text{if } x \in [c, b]
    \end{cases}
  \]
  for $i = 1, 2$. Then $f_1 \le f \le f_2$ on $[a, b]$,
  and
  \[
    \int_a^b (f_2 - f_1)\, dx
    = \int_a^c (g_2 - g_1)\, dx + \int_c^b (h_2 - h_1)\, dx
    < 2\epsilon,
  \]
  so $f \in \mathcal{R}([a, b])$. Now to prove $(*)$,
  note that $f \in \mathcal{R}([a, c])$, so for
  any $\epsilon > 0$ there exist Riemann sums $S_1$
  on $[a, c]$ and $S_2$ on $[c, b]$ such that
  \[
    |S_1 - \int_a^c f(x)\, dx| < \frac{\epsilon}{3},
    \quad
    |S_2 - \int_c^b f(x)\, dx| < \frac{\epsilon}{3}.
  \]
  Now choose $\delta > 0$ such that if the Riemann
  sum $S$ has partition with
  width $< \delta$, then
  \[
    |S - \int_a^c f(x)\, dx| < \frac{\epsilon}{3},
    \quad
    |S - \int_c^b f(x)\, dx| < \frac{\epsilon}{3},
    \quad
    |S - \int_a^b f(x)\, dx| < \frac{\epsilon}{3}.
  \]
  Now combine $S_1, S_2$ on $[a, b]$ to be a Riemann
  sum $S = S_1 + S_2$, so that
  \[
    |S - \int_a^b f(x)\, dx| < \frac{\epsilon}{3}.
  \]
  By the triangle inequality on the previous results,
  \[
    \left| \int_a^b f(x)\, dx - \left(\int_a^c f(x)\, dx + \int_c^b f(x)\, dx\right)\right| < \epsilon.
  \]
  Since $\epsilon$ is arbitrarily small, we conclude that
  \[
    \int_a^b f(x)\, dx = \int_a^c f(x)\, dx + \int_c^b f(x)\, dx
  \]
  as desired.
\end{proof}

\begin{remark}
  The formula $(*)$ is true for any three numbers
  $a, b, c$, as long as $f$ is integrable. This is
  because by convention, if $a > b$, then
  \[
    \int_a^b f(x)\, dx = -\int_b^a f(x)\, dx.
  \]
\end{remark}

\begin{theorem}
  If $f : [a, b] \to \R$ is continuous, then\footnote{Note that this integral is well-defined since any continuous function is integrable, and a continuous function restricted to a subset of its domain, i.e. $[a, x] \subseteq [a, b]$, remains continuous.}
  \[
    F(x) = \int_a^x f(\xi) \, d\xi
  \]
  is an anti-derivative of $f$.
\end{theorem}

\begin{proof}
  For any $x_0 \in (a, b)$, we check that
  $F'(x_0) = f(x_0)$. We can compute using Lemma
  \ref{lem:split-integral} that
  \begin{align*}
    \left| \frac{F(x_0 + h) - F(x_0)}{h} - f(x_0) \right|
    &= \left| \frac{1}{h} \left( \int_a^{x_0 + h} f(x)\, dx - \int_a^{x_0} f(x)\, dx \right) - f(x_0) \right| \\
    &= \left| \frac{1}{h} \int_{x_0}^{x_0 + h} f(x)\, dx - f(x_0) \right|
    = \left| \frac{1}{h} \int_{x_0}^{x_0 + h} (f(x) - f(x_0))\, dx \right|.
  \end{align*}
  The last step is from observing
  \[
    f(x_0) = \frac{1}{h} \int_{x_0}^{x_0 + h} f(x_0)\, dx.
  \]
  Since $f$ is continuous, for any $\epsilon > 0$, there
  exists $\delta$
  such that if $|x_0 - x| < \delta$, then
  $|f(x) - f(x_0)| < \epsilon$. This gives
  \[
    \left| \frac{1}{h} \int_{x_0}^{x_0 + h} (f(x) - f(x_0))\, dx \right|
    \le \frac{1}{h} \int_{x_0}^{x_0 + h} |f(x) - f(x_0)|\, dx
    \le \frac{\epsilon h}{h} = \epsilon
  \]
  if $|h| < \delta$. Thus,
  \[
    \lim_{h \to 0} \frac{F(x_0 + h) - F(x_0)}{h} - f(x_0) = f(x_0),
  \]
  so we indeed have $F'(x_0) = f(x_0)$.
\end{proof}

\section{More Conditions for Integrability}
\begin{definition}
  Let $f : [a, b] \to \R$ be bounded and
  $x_0 < x_1 < \cdots < x_n$ be a partition of $[a, b]$.
  Define
  \[
    \omega_i = \sup\{|f(x) - f(y)| : x, y \in [x_{i - 1}, x_i)\}
  \]
  for $i = 1, 2, \dots, n$. We will call this
  the \emph{oscillation amplitude} of $f$.
\end{definition}

\begin{theorem}
  A function $f : [a, b] \to \R$ is integrable if and only
  if for any $\epsilon > 0$, there exists $\delta > 0$
  such that for any partition with width $< \delta$,
  we have
  \[
    \sum_{i = 1}^n \omega_i \Delta x_i < \epsilon,
  \]
  where $\Delta x_i = x_i - x_{i - 1}$.
\end{theorem}

\begin{proof}
  $(\Leftarrow)$ For any $\epsilon > 0$, choose any
  two Riemann sums $S_1, S_2$ over partitions with
  width $< \delta$. Assume without loss of generality
  that $S_1$ and $S_2$ are defined over the same
  (maybe refined) partition. Let
  \[
    S_1 = \sum_{i = 1}^n f(x_i') (x_i - x_{i - 1}),
    \quad
    S_2 = \sum_{i = 1}^n f(x_i'') (x_i - x_{i - 1}).
  \]
  Then we have
  \[
    |S_1 - S_2| \le \sum_{i = 1}^n |f(x_i') - f(x_i'')| \Delta x_i
    \le \sum_{i = 1}^n \omega_i \Delta x_i < \epsilon.
  \]
  Then by Lemma \ref{lem:first-integrable}, we conclude
  that $f$ is integrable.

  $(\Rightarrow)$ Since $f$ is integrable, by Lemma
  \ref{lem:first-integrable} we have that for any
  $\epsilon > 0$, there eixsts $\delta > 0$ such that
  for any two Riemann sums $S_1, S_2$ over partitions
  of with $< \delta$, we have $|S_1 - S_2| < \epsilon$.
  In the interval $[x_{i - 1}, x_i]$, let
  \[
    M_i = \sup_{x \in [x_{i - 1}, x_i]} |f(x)|, \quad
    m_i = \inf_{x \in [x_{i - 1}, x_i]} |f(x)|.
  \]
  In particular note that $\omega_i = M_i - m_i$.
  Now for any $\eta > 0$, there exist
  $x_i', x_i'' \in [x_{i - 1}, x_i]$ such that
  \[
    f(x_i') > M_i - \eta, \quad f(x_i'') < m_i + \eta.
  \]
  Let
  \[
    S_1 = \sum_{i = 1}^n f(x_i') \Delta x_i,
    \quad
    S_2 = \sum_{i = 1}^n f(x_i'') \Delta x_i.
  \]
  Then we have
  \[
    |S_1 - S_2| \le \left| \sum_{i = 1}^n (f(x_i') - f(x_i'')) \Delta x_i \right|
  \]
  Note that $f(x_i') - f(x_i'') \ge M_i - m_i - 2\eta$
  for $\eta$ sufficiently small. Thus
  \[
    |S_1 - S_2| \ge \sum_{i = 1}^n \omega_i \Delta x_i
    - 2\eta \sum_{i = 1}^n \Delta x_i,
  \]
  so that
  \[
    \sum_{i = 1}^n \omega_i \Delta x_i
    \le |S_1 - S_2| + 2\eta (b - a)
    < \epsilon + 2\eta (b - a).
  \]
  From here letting $\eta \to 0$ gives the desired result.
\end{proof}

\begin{theorem}
  A function $f : [a, b] \to \R$ is integrable if and only
  if for any $\epsilon > 0$, there exists a partition
  such that
  \[
    \sum_{i = 1}^n \omega_i(f) \Delta x_i < \epsilon.
  \]
\end{theorem}

\begin{proof}
  $(\Rightarrow)$ This is immediate from the
  previous theorem.

  $(\Leftarrow)$ We show that for any $\epsilon > 0$,
  there exists $\delta > 0$ such that for any partition
  with width $< \delta$, we have
  \[
    \sum_{i = 1}^n \omega_i(f) \Delta x_i < \epsilon.
  \]
  This will imply that $f$ is integrable by the
  previous theorem.
  Let $y_0 < y_1 < \dots < y_m$ be the partition
  satisfying
  \[
    \sum_{j = 1}^m \omega_j(f) \Delta y_j < \epsilon,
  \]
  and choose
  \[
    \delta < \frac{1}{4}\min_{j = 1, \dots, m} \Delta y_j.
  \]
  For any partition $x_0 < \dots < x_n$ with
  width $< \delta$. We divide the intervals
  $[x_{i - 1}, x_i]$ into two classes. The first case
  (1) is where $[x_{i - 1}, x_i]$ is contained in one
  of the $[y_{j - 1}, y_j]$, and the second case (2)
  is where
  $[x_{i - 1}, x_i]$ contains an interior point $y_j$.
  In the first case, we have
  \[
    \sum_{(1)} \omega_i(f) \Delta x_i
    \le \sum_{j = 1}^m \omega_j(f) \Delta y_j.
  \]
  For the second case, since $[x_{i - 1}, x_i]$
  contains an interior point $y_j$
  but $\delta < \Delta y_j, \Delta y_{j + 1}$, we must
  have
  \[
    y_{j - 1} < x_{i - 1} < y_j < x_i < y_{j + 1},
  \]
  so that
  \[
    \omega_i(f) \Delta x_i \le \frac{1}{2} (\omega_j(f) \Delta y_j + \omega_{j + 1}(f) \Delta y_{j + 1}).
  \]
  This implies
  \[
    \sum_{(2)} \omega_i(f) \Delta x_i \le
    \frac{1}{2} \sum_{j = 1}^m \omega_j(f) \Delta y_j.
  \]
  Thus
  \[
    \sum_{i = 1}^n \omega_i(f) \Delta x_i
    = \sum_{(1)} \omega_i(f) \Delta x_i + \sum_{(2)} \omega_i(f) \Delta x_i
    \le 2 \sum_{j = 1}^m \omega_j(f) \Delta y_j
    < 2\epsilon,
  \]
  so that $f$ is integrable.
\end{proof}
