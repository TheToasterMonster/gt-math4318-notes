\chapter{Jan.~25 --- Riemann Integrability}

\section{Conditions for Integrability}
\begin{lemma}
  \label{lem:first-integrable}
  A function $f : [a, b] \to \R$ is integrable if and only if
  for any $\epsilon > 0$, there exists $\delta > 0$
  such that $|S_1 - S_2| < \epsilon$ whenever $S_1$
  and $S_2$ are Riemann sums for partitions of
  width less than $\delta$.
\end{lemma}

\begin{proof}
  $(\Rightarrow)$ If $f$ is integrable, then for
  any $\epsilon > 0$, there exists $\delta > 0$ such that
  \[
    \left|S - \int_a^b f(x)\, dx\right| < \frac{\epsilon}{2}
  \]
  for any Riemann sum $S$ of a partition with width
  less than $\delta$. Then
  \[
    |S_1 - S_2| \le
    \left|S_1 - \int_a^b f(x)\, dx\right| +
    \left|S_2 - \int_a^b f(x)\, dx\right| <
    \frac{\epsilon}{2} + \frac{\epsilon}{2} = \epsilon,
  \]
  as desired.

  $(\Leftarrow)$ Take the special partition into intervals
  of equal length, with width $(a - b) / n$. Pick
  the middle point in each interval, and let
  \[
    S_n = \sum_{i = 1}^n f(x_i') (x_i - x_{i - 1})
  \]
  be the corresponding Riemann sum. Now we check that
  $\{S_n\}_{n = 1}^\infty$ is a Cauchy sequence. This is
  because for any $\epsilon > 0$, if $N$ is large
  enough, then for any $n, m \ge N$, we have
  $|S_n - S_m| < \epsilon$ if $1 / N < \delta$. Then
  $\{S_n\}_{n = 1}^\infty$ converges, so let
  $\lim_{n \to \infty} S_n = A$. Now for any
  $\epsilon > 0$, there exists $\delta > 0$ such that
  for any Riemann sum $S$ with width $< \delta$,
  if $1 / n < \delta$, then $|S_n - S| < \epsilon / 2$.
  So
  \[
    |S - A| \le |S_n - S| + |S_n - A|
    < \frac{\epsilon}{2} + \frac{\epsilon}{2} = \epsilon,
  \]
  if $n$ is large enough. Thus
  \[
    A = \int_a^b f(x)\, dx
  \]
  exists and is the Riemann integral of $f$.
\end{proof}

\begin{remark}
Recall the step function $f : [a, b] \to \R$ given by
\[
  f(x) =
  \begin{cases}
    1 & \text{if } x \in (\alpha, \beta) \subseteq [a, b] \\
    0 & \text{if } x \notin (\alpha, \beta).
  \end{cases}
\]
Last time we saw that $f$ is integrable and that
\[
  \int_a^b f(x)\, dx = \beta - \alpha.
\]
Now let us consider a more general step function. We
call $f$ a \emph{step function} on $[a, b]$ if there
exists
a partition $x_0 < x_1 < \dots < x_n$ of $[a, b]$ such
that $f(x)$ is constant on each subinterval
$(x_{i - 1}, x_i)$.
\end{remark}

\begin{lemma}
  If $f : [a, b] \to \R$ is a step function for
  a partition $x_0 < x_1 < \dots < x_n$ and
  $f(x) = c_i$ when $x \in (x_{i - 1}, x_i)$, then
  $f$ is integrable and
  \[
    \int_a^b f(x)\, dx = \sum_{i = 1}^n c_i (x_i - x_{i - 1}).
  \]
\end{lemma}

\begin{proof}
  Define
  \[
    \varphi_i(x) = \begin{cases}
      1 & \text{if } x \in (x_{i - 1}, x_i) \\
      0 & \text{otherwise}.
    \end{cases}
  \]
  Now let
  \[
    h = f - \sum_{i = 1}^n c_i \varphi_i.
  \]
  Then $h(x)$ is nonzero only at $\{x_i\}_{i = 0}^n$.
  Each $\varphi_i$ is integrable and $h$ is integrable
  with
  \[
    \int_a^b h(x)\, dx = 0,
  \]
  so $f$ is also integrable and
  \[
    \int_a^b f(x)\, dx
    = \sum_{i = 1}^{n} c_i \int_a^b \varphi_i(x)\, dx
    = \sum_{i = 1}^n c_i (x_i - x_{i - 1})
  \]
  by linearity and the integral of a simple
  step function that we calculated before.
\end{proof}

\begin{prop}
  A function $f : [a, b] \to \R$ is integrable if
  and only if for any $\epsilon > 0$, there exist
  step functions $f_1, f_2$ such that
  $f_1(x) \le f(x) \le f_2(x)$ for all $x \in [a, b]$
  and
  \[
    \int_a^b (f_2 - f_1)\, dx < \epsilon.
  \]
\end{prop}

\begin{proof}
  $(\Leftarrow)$ For any $\epsilon > 0$, choose
  step functions $f_1, f_2$ such that
  \[
    \int_a^b (f_2 - f_1)\, dx < \frac{\epsilon}{3}.
  \]
  Then there exists $\delta > 0$ such that for any
  partition with width $< \delta$, the Riemann
  sums $S_1, S_2$ for $f_1, f_2$ satisfy
  \[
    |S_1 - \int_a^b f_1(x)\, dx| < \frac{\epsilon}{3}
    \quad\text{and}\quad
    |S_2 - \int_a^b f_2(x)\, dx| < \frac{\epsilon}{3}.
  \]
  So for any partition width $< \delta$, the Riemann sum
  of $f$ is
  \[
    S = \sum_{i = 1}^n f(x_i') (x_i - x_{i - 1}),
  \]
  and $S_1 \le S \le S_2$ since
  \[
    S_1 = \sum_{i = 1}^n f_1(x_i') (x_i - x_{i - 1})
    \quad \text{and} \quad
    S_2 = \sum_{i = 1}^n f_2(x_i') (x_i - x_{i - 1}).
  \]
  So $S$ is in the interval $(S_1, S_2)$, which has
  length $< \epsilon$ by the triangle inequality on the
  previous results. For any two Riemann sums of
  $f$ with partitions of width $< \delta$, we have
  $|S' - S''| < \epsilon$. Thus $f$ is integrable.

  $(\Rightarrow)$ First we show that $f$ is bounded
  in $[a, b]$. This is because for any $\epsilon > 0$,
  there exists $\delta > 0$ such that any two Riemann
  sums $S_1, S_2$ corresponding to partitions of
  width $< \delta$.
  satisfy $|S_1 - S_2| < \epsilon$. Let
  \[
    S_1 = \sum_{i = 1}^n f(x_i') (x_i - x_{i - 1}),
  \]
  and replace $x_{i_0}' \in (x_{i_0 - 1}, x_{i_0})$
  with $x_{i_0}'' \in (x_{i_0 - 1}, x_{i_0})$. Keep
  $x_i'$ for $i \ne i_0$. Define this new Riemann sum
  to be $S_2$. Then
  \[
    |S_2 - S_1| \le |f(x_{i_0}'') - f(x_{i_0}')| |x_{i_0} - x_{i_0 - 1}|
    < \epsilon,
  \]
  so that
  \[
    |f(x_{i_0}'')| \le |f(x_{i_0}')| + \frac{\epsilon}{x_{i_0} - x_{i_0 - 1}},
  \]
  i.e. $f$ is bounded in $(x_{i_0 - 1}, x_{i_0})$ since
  $x_{i_0}''$ was arbitrary. Since we also picked $i_0$
  arbitrarily, we can repeat this for any
  interval to conclude that $f$ is bounded in $[a, b]$.

  Now for any partition $x_0 < x_1 < \dots < x_n$
  with width $< \delta$, define
  \[
    m_i = \inf \{f(x) : x \in (x_{i - 1}, x_i)\}
    \quad \text{and} \quad
    M_i = \sup \{f(x) : x \in (x_{i - 1}, x_i)\}.
  \]
  Define the step function
  \[
    f_1(x) =
    \begin{cases}
      m_i & \text{if } x \in (x_{i - 1}, x_i) \\
      \min\{m_1, \dots, m_n\} & \text{if } x = x_i \text{ for } i = 0, \dots, n.
    \end{cases}
  \]
  Similarly define
  \[
    f_2(x) =
    \begin{cases}
      M_i & \text{if } x \in (x_{i - 1}, x_i) \\
      \max\{M_1, \dots, M_n\} & \text{if } x = x_i \text{ for } i = 0, \dots, n.
    \end{cases}
  \]
  Observe that $f_1(x) \le f(x) \le f_2(x)$ for
  any $x \in [a, b]$ by construction. Now we verify
  that
  \[
    \int_a^b (f_2 - f_1)\, dx < \epsilon
  \]
  if $\delta > 0$ is small enough. This is because
  for any $\eta > 0$, there exists $x_i', x_i'' \in [x_{i - 1}, x_i]$
  such that $f(x_i') < m_i + \eta$ and
  $f(x_i'') > M_i - \eta$. Then
  \[
    \sum_{i = 1}^n (f(x_i'') - f(x_i'))(x_i - x_{i - 1})
    > \sum_{i = 1}^n (M_i - m_i - 2\eta)(x_i - x_{i - 1})
    = \int_a^b (f_2 - f_1)\, dx - 2\eta(b - a).
  \]
  If $\delta > 0$ is small enough, then
  \[
    \sum_{i = 1}^n (f(x_i'') - f(x_i'))(x_i - x_{i - 1})
    < \epsilon
  \]
  since this a difference of two Riemann sums with
  partitions of width $< \delta$. Thus
  \[
    \int_a^b (f_2 - f_1)\, dx < \epsilon + 2\eta(b - a).
  \]
  But $\eta$ was arbitrary, so taking $\eta \to 0$
  gives the desired result.
\end{proof}

\begin{corollary}
  If $f : [a, b] \to \R$ is integrable, then it is
  bounded.
\end{corollary}

\begin{proof}
  This was shown in the proof of the previous proposition.
\end{proof}

\begin{theorem}
  If $f : [a, b] \to \R$ is continuous, then $f$ is
  integrable.
\end{theorem}

\begin{proof}
  Since $f$ is continuous on the compact set $[a, b]$,
  it is uniformly continuous. So for any $\epsilon > 0$,
  there exists $\delta > 0$ such that for any
  $x', x'' \in [a, b]$, we have
  $|f(x') - f(x'')| < \epsilon$ whenever
  $|x' - x''| < \delta$. Now let $S_1, S_2$
  be two Riemann sums
  with partitions of width $< \delta$. Assume without
  loss of generality that $S_1, S_2$ are defined
  over the same partition (we can always combine two
  partitions to give a finer partition, if necessary).
  Let
  \[
    S_1 = \sum_{i = 1}^n f(x_i') (x_i - x_{i - 1})
    \quad \text{and} \quad
    S_2 = \sum_{i = 1}^n f(x_i'') (x_i - x_{i - 1}).
  \]
  Then
  \[
    |S_1 - S_2| \le \sum_{i = 1}^n |f(x_i') - f(x_i'')| (x_i - x_{i - 1})
    < \epsilon \sum_{i = 1}^n (x_i - x_{i - 1})
    = \epsilon (b - a).
  \]
  Since $\epsilon > 0$ was arbitrary, we conclude that
  that $f$ is integrable by Lemma \ref{lem:first-integrable}.
\end{proof}

\section{The Fundamental Theorem of Calculus}
\begin{theorem}[Fundamental theorem of calculus]
  If $f : [a, b] \to \R$ has anti-derivative
  $F : [a, b] \to \R$ and $f \in \mathcal{R}([a, b])$,\footnote{Here $\mathcal{R}([a, b])$ is the class of Riemann integrable functions on $[a, b]$.}
  then
  \[
    \int_a^b f(x)\, dx = F(b) - F(a).
  \]
\end{theorem}

\begin{proof}
  Since $f$ is integrable, let
  \[
    A = \int_a^b f(x)\, dx.
  \]
  For any $\epsilon > 0$, there exists $\delta > 0$
  such that for any Riemann sum $S$ with partition of
  width $< \delta$, we have $|S - A| < \epsilon$. Let
  $x_0 < x_1 < \dots < x_n$ be a partition of
  width $< \delta$. Then by telescoping,
  \[
    F(b) - F(a) = \sum_{i = 1}^n (F(x_i) - F(x_{i - 1}))
    = \sum_{i = 1}^n f(x_i') (x_i - x_{i - 1})
  \]
  by Lagrange's mean value theorem, where
  $x_i' \in (x_{i - 1}, x_i)$. Then
  \[
    |F(b) - F(a) - A| = |S - A| < \epsilon,
  \]
  so letting $\epsilon \to 0$ gives $F(b) - F(a) = A$.
\end{proof}

\begin{remark}
  The fundamental theorem of calculus requires both
  being Riemann integrable and having an anti-derivative,
  which do not always overlap. In fact, neither is
  a subset of the other.
\end{remark}

\begin{example}
  The step function
  \[
    f(x) =
    \begin{cases}
      -1 & \text{if } 0 \le x \le 1 \\
      1 & \text{if } 1 < x \le 2
    \end{cases}
  \]
  is integrable but has no anti-derivative.
\end{example}

\begin{example}
  Define
  \[
    F(x) =
    \begin{cases}
      0 & \text{if } x = 0 \\
      x^2 \sin(1/x) & \text{if } x \ne 0.
    \end{cases}
  \]
  Then we have
  \[
    F'(x) = f(x) =
    \begin{cases}
      0 & \text{if } x = 0 \\
      (-2 / x)(\cos(1 / x^2)) + 2x \sin(1 / x^2) & \text{if } x \ne 0.
    \end{cases}
  \]
  We can check that $F'(0) = 0$ via the definition
  of the derivative. Note that $f$ has an anti-derivative,
  namely $F$. However, $f$ is not integrable since
  it is not bounded near $x = 0$.
\end{example}
